\subsection*{Trunkerings- og afrundingsproblemer}
Mange af de metoder, vi bruger til at udregne, det gyldne
snits position eller størrelsen af en margin, udregnes med brøker. Hvorimod et
billede opbygges af pixels. Dette gør, at vi bliver nødt til at tage
approksimationer af udregningerne for at få dem tilbage til et helt tal.
Men hvor meget af data er gået tabt, og hvor mange af resultaterne kan
man stole på?

\subsubsection{Acceptabel afvigelse}
\note{Nogle referanse}Som beskrevet i afsnit \ref{mange_tal}, udregnes
det gyldne snit med mange decimaler. En kunstner, hvor god han end er,
har ingen chance for at male så præcist at man kan sige at strøget
ligger nøjagtigt oven på snittet selv om hans ententioner er at ramme
snittet. Vi kommer derfor til at have en vis uprecis hed på de data vi
få fra billedet selv. Vi starter med at se på alle de ting, som kan
skabe en usikkerhed fra malerens side. Man kan gå ud fra at den
procentvise afvigelse ikke er særlig stor, da vi ikke har bestræbelser
på at arbejde på abstrakt malerier, dog har vi sat den procentvise
afvigelse til 0.5 \%. Det vil sige at en maler med et lærred på 100
cm,maksimalt vil male $0.5$ cm forkert.

Når maleren vælger en ramme og et lærred, har vi igen problematikken,
selv om maleren spicifikt gå efter at bygge maleriet op efter det gyldne
snit, kan snittets placering i maleriet have forskubbet sig, ved dårlige
valg at ramme eller lærred. Derfor sætter vi den afvigelse til $1\%$. Da
vi igen mener at dette er den maksimale afvigelse, der kan opstå.

Når maleren maler en region i et maleri, forekommerer der normalt en
lille kant rundt om objektet, et omrids. Dette omrids kan vores
algoritmer ikke tage højde for, og vi må derfor modregne omridset, så vi
er sikre på at vi ser på regionen og ikke dens omrids. Da et omrids ikke
er særligt stort, har vi sat denne procentsats til $0.5\%$. 

Alt i alt giver det en afvigelse på vores aktuelle udtrækning af data
fra malerierne på $2\%$ det vil sige at finder vi en region, som ligger
på pixel 200, i et billedet, der er $500$ cm bredt, befinder den sig
faktisk i intervallet [190,210]. Måden hvorpå vi tager højde for den
forskel, er ved hjælp af marginer, som nævnt i afsnit
\ref{section_naiv}.

\subsection{Inddeling af billede efter snit}

I et billede betegnes højden og bredden som hhv. $H$ og $B$, se figur
\ref{cut}. Der er 4 gyldne snit, 2 vertikale og 2 horisontale, som vist
i figur \ref{lenasnit2}. For at finde ud af, hvor de 4 snit skal ligge i
billedet, multipliceres B og H med $\varPhi$,og man får to tal. Disse
betegner, hvor mange pixels, det gyldne snit befinder sig fra hhv. $H$
og $B$ f.eks vil de 2 horisontale snit, i et billedet, som har $B =
4000$ pixel, ligge hhv. $4000 \cdot \varPhi \approx 2472$ pixels fra
billedets øvre og nedre kant.

\begin{figure}[h]
	\begin{center}
		\includegraphics[scale=0.42,angle=0]{afsnit/vores_implementation/billeder/naiv_algoritme/Lenagolden}
	\end{center}
	\caption[]{Billedet som har indtegnet de fire gyldne snit}
	\label{lenasnit2}
\end{figure}

\begin{figure}[h]
	\begin{center}
		\includegraphics[scale=0.42,angle=0]{afsnit/vores_implementation/billeder/naiv_algoritme/Cut}
	\end{center}
	\caption[]{Billedets højde og bredde betegnes hvv. H og B. De 4 snit er navngivet.}
	\label{cut}
\end{figure}

De 4 snit tildeles hvert deres Id, "snit 0,1,2 og 3" så vi kan kende
forskel på de individuelde snit, Id'erne placering kan ses i figur
\ref{cut}. Vi vil i resten af rapporten kalde snittene efter deres Id.
Hvis vi gerne vil finde snittet som ligger i miden kommer der kun 2
snit, med vær deres Id "snit 0 og 1" som kan ses i figur \ref{Cut2}

\begin{figure}[h]
	\begin{center}
		\includegraphics[scale=0.42,angle=0]{afsnit/vores_implementation/billeder/naiv_algoritme/2Cut}
	\end{center}
	\caption[]{Billedet skæres her kun af 2 snit}
	\label{2Cut}
\end{figure}

\subsubsection{Heltal i det gyldne snit}

I eksemplet med 4000 pixels ovenfor, approksimerer vi antal pixels ved
at afrunde resultatet $2472.13595 \approx 2472$, se udregning
\ref{afrundning}. Det betyder at vi mister 0.13595 pixels i præction,
hvilket svarer til en misvisning af punktet på 0.00339875 $\%$ i forholdt
til $B$ på billedet. Se udregning \ref{afrundning2}.

\begin{equation}
	4000 \cdot \varPhi = 4000(\sqrt{5}-1)/2 = 2472.13595 \approx 2472 \label{afrundning}
\end{equation}

\begin{equation}
	0.13595/4000 \cdot 100 = 0.00339875 \label{afrundning2}
\end{equation}

Det er en meget lille del af selve billedet og skulle ikke give nogle
misvisninger i forhold til udregningen. For at gøre det lidt mere
generelt, sætter vi trunkeringsfejlen til $0.5$, da det er den maksimale
afrundingsfacktor som kan forekomme. Hvis billedet har en størrelse på
500 pixels, hvilket er det mindste billedet vi har, giver dette en fejlmargin
på $0.1 \%$. Dette tal bliver adderet til fejlsatsen ovenfor, og giver
en samlet afvigelse på $2.1\%$.

\subsubsection{Snitratio}
End til vider har vi kun arbejder med det gyldne snit, men andre snit i
billedet kan godt optræde, derfor indføre vi en nu betegnelse
snitratio, som betegner en procents sats for hvor lang inde i billedet
snittet befinder sig. Det vil sige at hvis en snitration er på $0.2$. Et
billedet har $B$ på 4000 vil et snit befinde sig i pixel $4000*0.2 =
800$.

\subsubsection{Heltal ved udregning af Margin}
Når vi har 2 forskellige snitratioer, f.eks. $\varPhi$ og $\frac{2}{3}$,
som ligger meget tæt på hinanden, og vi gerne vil sammenligne hvilken
regioner der ligger i snitratioernens snit, er det vigtigt at margin for
vært af de 2 snitratioers snit ikke krydser hinanden. 

Hvis margin krydser. Vil det indebære, at den samme region bliver fundet
af begge snit. Dette vil give et skævt billedet af forskellen på de to.
Derfor må vi sørge for at marginerne ikke krydser. Hvis $x$ betegner
antal pixels i $B$ eller $H$, og vi vil se på, hvor mange pixels, der er
mellem snitratio $\frac{2}{3}$ og $\varPhi$, multiplicerer vi $x$ med de
to snitratioen for at finde deres placering. Derefter subtraheres vi et
af de snit som befinder sig tetest på hinanden i vær snitratio med
hinanden.

\begin{eqnarray}
	\frac{x2}{3} - \frac{x2}{\sqrt{5}+1} & = & x(\frac{2}{3} - \frac{2}{\sqrt{5} + 1}) \nonumber \\
	& = & x(0.666667-0.618034) \\ \nonumber
	& = & x(0.048633)
\end{eqnarray}

Vi har nu fundet antal pixels mellem de to snit. Vi vil gerne undgå at
de to marginens ikke krydser hinanden, så der dividere vi med 2 og
afrunder værdien.

\begin{equation}
	\left\lfloor \frac{0.048633x}{2}\right\rfloor = \left\lfloor0.024316x \right\rfloor
\end{equation}

Tallet $2.4316$ er altså den minimale procentvise størrelse som vores
margin må have, nå vi sammen liner det gyldne snit og $\frac{2}{3}$.
Det betyder også at vi ikke må sammen line snit som ligger særlig
meget tætter på hinanden, da 0.021 er den minimale procent margin som vi må have.
$\lfloor 0.024316x \rfloor$ giver os et antal pixels som skildre de to snit. For at vise
hvor stort marginen egentlige kan være, bruger jeg denne formel på to
billeder, et som svare til vores mindste billedet, 500 pixels, og et
som svare til vores største billedet, 4000 pixels. Ved 500 pixels
bliver resultatet

\begin{equation}
	 \lfloor 500(0.024316)\rfloor = 12
\end{equation}

Det er en fint margin, da vores fejl på udregningerne ligger på 2.1 \%,
som svare til $\lceil 500*0.021 \rceil = 11$ pixels, som er 1 pixels fra vores
margin

Ved 4000 pixels giver det.

\begin{equation}
	 \left\lfloor 4000(0.024316)\right\rfloor = 97
\end{equation}

Som også er god nok da $4000*0.021 = 84$ pixels.
% vim: set tw=72 spell spelllang=da:
