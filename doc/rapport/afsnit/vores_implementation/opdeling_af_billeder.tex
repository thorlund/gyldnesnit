\subsection*{Heltal Problematikken i vores udregninger}
\subsection*{Trunkerings og afrundings fejl}
\subsection*{Fejl ø}
\subsection*{Mallers pracition og afrundings fejl}
\subsection*{Heltal, er det et problem?}
\subsection*{Kan heltal også være ikke hele}
Mange af de metoder som vi bruger til at udregne, position af det gyldne
snit eller størrelsen af en margin, udregnes ved hjælp af brøker.
Til gengæld, er et billedet opbygget af pixels, Dette gør at vi må nød
til at tage approksimationer af udregningerne for at få dem tilbage til et
helt tal, men hvor meget af dataenden er tabt, og hvor maget af
resultaterne kan man stole på.

\subsubsection{Acceptabel afvigelse}
(Nogle referancer)
Det gyldne snit er et meget fast defineret matematisk begreb, som kan
udregnes med mange decimaler. En kunstner, selv hvor god han er, har
ingen chance for at male så præcist at man kan sige at strøet ligge
nøjagtigt oven på snittet. Men hvor stor afvigelse fra snittet er
acceptabelt 0.1 \%, 1 \% eller mere. hvis vi starter med at se på alle
de ting som kan skabe en usikkerhed for hvad kunstarnen maler. Man kan
gå ud fra at den \% afvigelse ikke er særlig stor, da vi taler om kunst
malerier fra dygtige kunstner, dog har vi sat den til \% afvigelse til
0.5 \%. Det vil sige at en maller med et lærred på 100 cm, kan male
maksimalt 0.5 cm forkert.

Nå maleren vælger en ramme og et lærred, har vi igen problemtikken. Hvor
præcis er maleren i sit valg og hvor godt passer rammen til billedet,
Dette giver vi en afvigelse på 1 \%. Da vi igen mener at dette er den
maksimalle afvigelse der kan opstå.

Nå en maller, maller en figur eller en ting i et maleri, forekommerer der normalt en
lille kant rundt og objektet, som et omrids. Dette omrids kan vores
algoritmer ikke tage højte for, og vi må derfor modregne omridset. Nå vi
er sikre at vi ser på objektet og ikke den omrids, Da et omrids ikke er
særligt stor, har vi sat denne procent sats til 0.5\%. Alt i alt giver
det en afvigelse på vores aktuelde data på 2\% det vil sige at nå vi
finder en region i en billedet som ligger på pixel 200, i et billedet
som er 500 bredt, befinder den sig faktisk i intervallet mellem 190 og
210. Måden vi tager højte for den forskel, er ved hjælp af marginer, som
er beskravet i opgave. Der befinder sig også nogle afrunding fejl i
vores metoder, som beskravet neden for.

Som man nok kan se, regner vi med procenter, som måske kan være lidt
misvisende, da det afhænger af billedet strøelse, så ved et billedet som
er meget stort, kan malleret male meget ved siden af, i forhold til en
lille billedet.

\subsection{Hvordan deles billedet efter det snit}
I et billedet befinder der sig 4 gyldne snit. 2 i det vertikale plan med
brede B, se figur \ref{box} og 2 i det horisontale plan med højte H.
placeringen af de 4 snit udregnes i forhold til udledningen i afsnittet
om det gyldne snit, se figur \ref{lenasnit2}, hvor konstanten $\varPhi =
0.618$ bliver fundet. $\varPhi$ bliver også betegnet som et cutRatio i
denne opgave.

\begin{figure}[h]
	\begin{center}
		\includegraphics[scale=0.42,angle=0]{afsnit/vores_implementation/billeder/naiv_algoritme/path2407}
	\end{center}
	\caption[]{Billedet opbygning}
	\label{box}
\end{figure}

B og H bliver divideret med $\varPhi$ som giver to tal som betegner, hvor
mange pixels fra begge kanter, det gyldne snit befinder sig f.eks ved et
billedet, som har B = 4000 pixel, vil punktet ligge ved, $4000/\varPhi = 2472$.
\begin{figure}[h]
	\begin{center}
		\includegraphics[scale=0.42,angle=0]{afsnit/vores_implementation/billeder/naiv_algoritme/Lenagolden}
	\end{center}
	\caption[]{Billedet som har indtegnet de fire gyldne snit}
	\label{lenasnit2}
\end{figure}

I vores implementering af dette har vi give de 4 forskelige cut vær
deres id 'cut 0 - 3' se figur \ref{cut}, vi vil i reste af raporten
referet til dem ud fra deres id. Hvis cutRatio er pracics 0.5 kommer cuttet til at ligge oven i hinanden,
og kun skabe 2 cut i steden for firere. id for disse 2 snit er cut 0 og
cut 1. ilustret i figur \ref{2Cut}

\begin{figure}[h]
	\begin{center}
		\includegraphics[scale=0.42,angle=0]{afsnit/vores_implementation/billeder/naiv_algoritme/Cut}
	\end{center}
	\caption[]{Billedet hvor de 4 cut er navngivet}
	\label{cut}
\end{figure}



\begin{figure}[h]
	\begin{center}
		\includegraphics[scale=0.42,angle=0]{afsnit/vores_implementation/billeder/naiv_algoritme/2Cut}
	\end{center}
	\caption[]{cutRation er 0.5, og skær derfor kun 2 cut}
	\label{2Cut}
\end{figure}

\subsubsection*{Heltal i det gyldne snit}



Eksemplet med 4000 pixels ovenfor, approksimationen vi antal pixels ved at
runde resultatet se udregning \ref{afrundning}. Det gør at vi mister 0.13595 nøjagtighed, dette svare
til en misvisning af punktet på 0.00339875 $\%$ i forholdt til bredden på
billedet. se udregnig \ref{afrundning2}. 

\begin{equation}
	4000/\varPhi = 4000(\sqrt{5}-1)/2 = 2472.13595 \approx 2472 \label{afrundning}
\end{equation}

\begin{equation}
	0.13595/4000*100 = 0.00339875 \label{afrundning2}
\end{equation}

Det er en maget lille del af selve billedet og skulle ikke give nogle
misvisninger i forhold til udregningen. For at gøre det lidt mere
generat, sætter vi den mistet nøjagtighed til 0.5, det er den maximale
afrundings factor som kan forekommer. Sætter billedet størrelse til 500
pixels, som er det miste billedet vi har, dette giver en fejl margen på
0.1 $\%$, Dette tal bliver adderet til fejl satsen ovenfor og vi kommer
til at at se at det er en ok afrundning af snittet, som ikke kommmer til
at give nogle problemer. 

\subsubsection{Heltal ved udregning af Margin}
Nå vi har 2 snit, F.eks. $\varPhi$ og $\frac{2}{3}$ som skal sammenligne og
som ligger meget tæt på hinanden, er det vigtigt at margin for vært
snittet ikke krydser hinanden. Da dette vil indebære at de samme region
vil blive fundet af begge snit, og vil give et skævt billedet af
forskellen på de to snit. Derfor må vi sørge for at de margin ikke
krydser. hvis x betegner antal pixels i B eller H, og vi vil se
forskellen mellem 2/3 og $\varPhi$, multiplicere vi x med snittet for at
finde dens placering og subtrahere dem fra hinanden.

\begin{eqnarray}
	\frac{x2}{3}-\frac{x2}{\sqrt{5}+1} &=& x(\frac{2}{3}-\frac{2}{\sqrt{5}+1}) \\ \nonumber
	&=& x(0.666667-0.618034) \\ \nonumber
	&=& x(0.048633) \\
\end{eqnarray}

Vi har nu fundet antal pixels mellem de to snit. Vi vil gerne undgå at
de to marginens ikke krydser hinanden, så der dividere med 2 og tager en
floor på helle funktionen.

\begin{equation}
	\lfloor(0.048633x/2)\rfloor = 0.024316x
\end{equation}

Talet $2.4316$ er så den minimale procent visse størelse som vores
margin må have, nå vi sammen liner det gyldne snit og $\frac{2}{3}$,
Dette betyder også at vi ikke må sammen line snit som ligger særlige
meget tætter på hinanden, da 0.021 er den minimal procent margin som vi må have.
$0.024316x$ giver os et antal pixels som skildre de to snit. For at vise
hvor stort marginen egentlige kan være, bruger jeg denne formel på to
billeder, et som svare til vores mindste billedet, 500 pixels, og et
som svare til vores største billedet, 4000 pixels. Ved 500 pixels
bliver resultatet

\begin{equation}
	 \lfloor 500(0.024316)\rfloor = 12
\end{equation}

Dette er en ok margin, da vores fejl på udregningerne ligger på 2.1 \%,
som svare til $500*0.021 = 10.5$ pixels, som er 1.5 pixels fra vores
margin

Ved 4000 pixels giver det.

\begin{equation}
	 \lfloor 4000(0.024316)\rfloor = 97
\end{equation}

Som også er god nok da $4000*0.021 = 84$ pixels. Vi har valt at bruge
$2.4\%$ som vores minimum margin procentiv og ikke 2.1 da vi gerne
vil have en margin som er lidt støre en den teoratiske fejl margin.
% vim: set tw=72 spell spelllang=da:
