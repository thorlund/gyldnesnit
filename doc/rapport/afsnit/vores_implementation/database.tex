{
Når vi har trukket regioner ud af billedet og vurderet dem efter den
naive algoritme givet i afsnit \ref{naiv_algoritme} kan vi stå tilbage
med et egentligt resultat. Vi ønsker at gemme dette resultat i en
database så vi på et senere tidspunkt kan bruge det i en samlet analyse
af resultaterne.  Vi så i afsnit \ref{section_opbv_billeder} på et
databaseskema til opbevaring af maleriers metadata. Vi bygger nu videre
på dette skema så vi kan tilknytte resultater fra en analyse til de
enkelte malerier. Det fulde databaseskema ses nedenfor og bliver
efterfølgende forklaret.

\begin{table}[!h]
    \centering
    \begin{tabular}{|l||c|c|c|c|c|c|}
        \hline
        \bf{artist} \hspace{0.5cm} & \underline{artistId} & name & born & died & school & timeline \\\hline
    \end{tabular}
    \caption{Databasetabel for kunstner}
    \label{artistTable}
\end{table}

\begin{table}[!h]
    \centering
    \begin{tabular}{|l||c|c|c|c|c|c}
        \hline
        \bf{painting} \hspace{0.5cm} & \underline{paintingId} & artistId & title & date & paint & $\cdots$ \\\hline
    \end{tabular}\\ \vspace{0.2cm}\hspace{1.2cm}
    \begin{tabular}{c|c|c|c|c|c|c}
        \hline
        $\cdots$ & material & location & url & form & type & $\cdots$ \\\hline
    \end{tabular}\\ \vspace{0.2cm}\hspace{1.4cm}
    \begin{tabular}{c|c|c|c|c|c|}
        \hline
        $\cdots$ & realHeight & realWidth & height & width & filepath \\\hline
    \end{tabular}
    \caption{Databasetabel for malerier}
    \label{paintingTable}
\end{table}

\begin{table}[!h]
    \centering
    \begin{tabular}{|l||c|c|c|c|c|c|c|}
        \hline
        \bf{run} \hspace{0.5cm} & \underline{runId} & trsh1 & trsh2 & lo & up & marginPercentage & method \\\hline
    \end{tabular}
    \caption{Databasetabel for en kørsel}
    \label{runTable}
\end{table}

\begin{table}[!h]
    \centering
    \begin{tabular}{|l||c|c|c|c|c|c|}
        \hline
        \bf{result} \hspace{0.5cm} & \underline{resultId} & runId & paintingId & cutRatio & cutNo & numberOfRegions \\\hline
    \end{tabular}
    \caption{Databasetabel for resultater}
    \label{resultTable}
\end{table}

\begin{table}[!h]
    \centering
    \begin{tabular}{|l||c|c|c|c|c|c|c|}
        \hline
        \bf{region} \hspace{0.5cm} & \underline{regionId} & resultId & x & y & height & width & area \\\hline
    \end{tabular}
    \caption{Databasetabel for regioner}
    \label{regionTable}
\end{table}

Tabellerne \ref{artistTable} og \ref{paintingTable} er de samme som vist
i afsnit \ref{section_opbv_billeder}, men er taget med her blot for at
vise databaseskemaet i sin helhed. Det øvrige databaseskema lægger vægt
på at minimere redundans, at kunne gemme data fra flere forskellige
kørsler med forskellige parametre og mulighed for at genskabe kørte
analyser. Vi vil nu kigge på betydningen af de ovenstående tabeller i
databasen og deres relation til analysen på et billede.

Hvis vi vil genskabe et fundet resultat, har vi brug for at vide hvilke
parametre vi har brugt for at komme frem til resultatet. Til dette
formål har vi tabellen \texttt{run} (tabel \ref{runTable}) som beskriver
en kørsel. Denne tabel holder de parametre som er fælles for alle
billeder i en kørsel. Her har vi adgang til de tærskelparametre som
bruges til kantdetektion (\texttt{trsh1} og \texttt{trsh2}) samt nedre
og øvre grænse for floodfill (\texttt{lo} og \texttt{up}). Her holdes
også en procentsats for hvor stort et margin vi bruger i floodfill og
udvælgelse af regioner. Endelig har vi et felt der angiver hvilken
metode der er blevet brugt til at finde resultatet. Dette er en
tekststreng og vil i tilfældet af den naive algoritme være sat til
\texttt{'naive'}. Felterne \texttt{trsh1}, \texttt{trsh2} og
\texttt{marignPercentage} er repræsenteret som floats i databasen,
mens \texttt{lo} og \texttt{up} står som heltal. Afslutningsvis har
hver kørsel en unik id. Således vi kan tilknytte indgange i tabellen
\texttt{result} (tabel \ref{resultTable}) til et sæt af parametre.

Vi beskriver et resultat, som det antal af regioner vi får ud fra
analysen af et snit på et givet billede. Givet en snitratio vil en
analyse typisk have fire snit hvor vi vil finde regioner i nærheden af.
En undtagelse er hvis snitratioen deler billedet i to lige store dele.
Vi vil i dette tilfælde kun have to snit vi kigger på. Tabellen
\texttt{result} fortæller os hvilket af de mulige snit vi har med at
gøre, hvilket billede resultatet er tilknyttet, hvilke parametre der er
blevet brugt samt hvor mange regioner vi har fundet.

Tabellen \texttt{region} (tabel \ref{regionTable}) holder alle de fundne
regioner fra vores analyse. Hver region henviser til det resultat som
denne tilhører. Vi kan således skelne de enkelte regioner fra hinanden
og afgøre i hvilket snit af billedetde ligger.  På grund af
begrænsninger i \emph{OpenCV} kan vi ikke gemme regionens præcise form, men kun
dennes begrænsende rektangel og regionens areal.

\subsection{Oprettelse af resultater}
Denne sektion afhænger til dels af en færdig implementation af den
automatiserede analyse. Hvordan lægger vi resultater fra en kørsel ind i
databasen? Forklar de (smarte) metoder vi har til rådighed. Vi kan blot
kaste klasser ind i databasen. Vores settings-klasse bruges til at
oprette runs, dictionary til at smække resultatet fra analysen ind.

\subsection{Genskabelse af parametre og resultater}
At kunne genskabe de fundne resultater fra en analyse har meget stor
betydning, dels for at kunne udtage stikprøver i udviklingen af hele
programmet, men også for at kunne fremvise grafiske resultater. Vi har
allerede været inde på, at man for at kunne genskabe et resultat, skal
vide hvilke parametre der oprindeligt har været brugt. Ovenstående
databaseskema gør det let at hente disse parametre ud. Hvis vi får et
resultat med overraskende mange regioner og gerne vil undersøge dette
tilfælde, har vi metoder til rådighed der giver os lige nøjagtig de
informationer vi har brug for at vise dette grafisk. Helt konkret har vi
metoderne vist i listing \ref{rekonst_koersel} til rådighed.

\vspace{0.5cm}
\begin{lstlisting}[caption={Metoder til rekonstruktion af kørsler},captionpos=b,label={rekonst_koersel},numbers=none]
def getSettingsForRunId(runId):
    """Return the settings instance for a given run"""
    pass

def getCutRatiosForRunId(runId):
    """Return the list of cut ratios for a given run"""
    pass

def getSettingsForResultId(resultId):
    """Return the settings instance for a given result"""
    pass

def getSettingsForRegionId(regionId):
    """Return the settings instance for a given region"""
    pass

def getCutRatioForRegionId(regionId):
    """Return the list of cut ratios for a given region"""
    pass

def getCutNoForRegionId(regionId):
    """Return the cut number for a given region"""
    pass

def getRegionsForResultId(resultId):
    """Return the list of regions for a given result"""
    pass
\end{lstlisting}

Selvom metoderne i listing \ref{rekonst_koersel} ikke viser noget
egentlig kode, bør det ud fra sammenhængen være klart hvad disse metoder
gør. Alle metoder der starter med \texttt{getSettings} returnerer
klassen \texttt{Settings} som vist i listing \ref{settings_klassen} med
indstillinger tilpasset den enkelte forespørgelse\footnote{Det kan godt tænkes
at denne klasse allerede bliver introduceret i afsnittet inden}.

\vspace{0.5cm}
\begin{lstlisting}[caption={Settings-klassen med standardindstillinger},captionpos=b,label={settings_klassen},numbers=none]
class Settings:
    """These are the default settings for the analysis"""
    edgeThreshold1 = 78
    edgeThreshold2 = 2.5 * edgeThreshold1
    lo = 4
    up = 4
    cutRatios = None
    marginPercentage = 0.009
    method = 'naive'
    ...
\end{lstlisting}

Det ses at vi har mulighed for at trække de fundne regioner ved et
snit ud og vi behøver derfor ikke at køre nogen analyse på billedet hvis
vi blot ønsker at få de fundne regioners begrænsende areal vist. I dette
tilfælde kan vi nøjes med at forespørge databasen om de regioner der er
tilknyttet et snit vi gerne vil undersøge og traversere gennem den liste
af regioner vi får tilbage. Hver region er repræsenteret som en klasse
hvor vi kan trække rektanglet ud og vi bruger da \emph{OpenCV} til at
tegne rektanglet på det tilknyttede billede.

\subsection{Opsamling (Diskussion, Vurdering, Udvidelser?)}
Vi startede med at sige at databaseskemaet var udviklet med blandt andet
det formål at mindske redundans. Vi har dog feltet
\texttt{numberOfRgions} i tabellen \texttt{result} som på sin vis er
redundant, da vi blot kan finde antallet af tilknyttede regioner ved at
bruge simple SQL-sætninger. At have antallet stående direkte i databasen
giver dog et umiddelbart bedre overblik over resultaterne end ved at
bruge SQL-sætninger. Når databasen vokser, ved at ændre på parametrene,
vil SQL-sætningerne skulle søge meget af databasen igennem for at
returnere en meget simpel forespørgsel. Derfor har vi valgt at gemme
antallet at fundne regioner direkte i databasen.

Databaseskemaet er let at udvide, hvis vi i udviklingen af mere
avancerede metoder, skulle få brug for flere indgange. Specielt kunne
tabellen \texttt{region} udvides med en vægt som et mål for ``hvor
meget'' denne region ligger i snittet.

}
% vim: set tw=72 spell spelllang=da:
