{
Når vi har trukket regioner ud af billedet og vurderet dem efter
ovenstående simple algoritme kan vi stå tilbage med et egentligt
resultat. Vi ønsker at gemme dette resultat i en database så vi på et
senere tidspunkt kan bruge det i en samlet analyse af resultaterne.
Vi så i afsnit \ref{section_opbv_billeder} på et databaseskema til
opbevaring af maleriers metadata. Vi bygger nu videre på dette skema så
vi kan tilknytte resultater fra en analyse til de enkelte malerier. Det
fulde databaseskema ses nedenfor.

\begin{table}[!h]
    \centering
    \begin{tabular}{|l||c|c|c|c|c|c|}
        \hline
        \bf{artist} \hspace{0.5cm} & \underline{artistId} & name & born & died & school & timeline \\\hline
    \end{tabular}
    \caption{Databasetabel for kunstner}
    \label{artistTable}
\end{table}

\begin{table}[!h]
    \centering
    \begin{tabular}{|l||c|c|c|c|c|c}
        \hline
        \bf{painting} \hspace{0.5cm} & \underline{paintingId} & artistId & title & date & paint & $\cdots$ \\\hline
    \end{tabular}\\ \vspace{0.2cm}\hspace{1.2cm}
    \begin{tabular}{c|c|c|c|c|c|c}
        \hline
        $\cdots$ & material & location & url & form & type & $\cdots$ \\\hline
    \end{tabular}\\ \vspace{0.2cm}\hspace{1.4cm}
    \begin{tabular}{c|c|c|c|c|c|}
        \hline
        $\cdots$ & realHeight & realWidth & height & width & filepath \\\hline
    \end{tabular}
    \caption{Databasetabel for malerier}
    \label{paintingTable}
\end{table}

\begin{table}[!h]
    \centering
    \begin{tabular}{|l||c|c|c|c|c|c|c|}
        \hline
        \bf{run} \hspace{0.5cm} & \underline{runId} & trsh1 & trsh2 & lo & up & marginPercentage & method \\\hline
    \end{tabular}
    \caption{Databasetabel for en kørsel}
    \label{runTable}
\end{table}

\begin{table}[!h]
    \centering
    \begin{tabular}{|l||c|c|c|c|c|c|}
        \hline
        \bf{result} \hspace{0.5cm} & \underline{resultId} & runId & paintingId & cutRatio & cutNo & numberOfRegions \\\hline
    \end{tabular}
    \caption{Databasetabel for resultater}
    \label{resultTable}
\end{table}

\begin{table}[!h]
    \centering
    \begin{tabular}{|l||c|c|c|c|c|c|c|}
        \hline
        \bf{region} \hspace{0.5cm} & \underline{regionId} & resultId & x & y & height & width & area \\\hline
    \end{tabular}
    \caption{Databasetabel for regioner}
    \label{regionTable}
\end{table}

Tabellerne \ref{artistTable} og \ref{paintingTable} er de samme som vist
i afsnit \ref{section_opbv_billeder}, men er taget med her blot for at
vise databaseskemaet i sin helhed. Det øvrige databaseskema lægger vægt
på at minimere redundans og mulighed for at genskabe kørte analyser.
Sidstnævnte er vigtig osv\dots

}
% vim: set tw=72 spell spelllang=da:
