{
{\sffamily Vi præsenterer to udvidede fremgangsmåder til at afgøre, om
et billede har interessante regioner i det gyldne snit. Bemærk, at vi
ikke vil forbedre metoden som trækker regioner ud af et billede eller
ændre på definition af en interessant region. Vi forbedrer udelukkende
metoden, der skal afgøre, om en interessant region er placeret i det
gyldne snit, og i det følgende præsenteres to metoder til dette formål.
Den første metode, bygger videre på den naive algoritme, hvor den
ideelle placering i det gyldne snit redefineres. Den anden metode tager
afstand fra den binære klassifikation, og søger i stedet at tildele
interessante regioner en værdi, der angiver, i hvor høj grad de ligger i
det gyldne snit.

Vi antager i det følgende, at vi har perfekt udtrækning af regioner i et
maleri, og at den digitale gengivelse af maleriet således bliver
segmenteret helt som ønsket.
}

\subsection{Udvidelse til den naive metode\label{subsec_udvidet_massemidtpunkt}}
{
Denne udvidelse sigter mod at forbedre den eksisterende bedømmelse af
interessante regioner i forhold til det gyldne snit. Specielt er der
en type af interessante regioner som bliver fravalgt af den naive
metode, nemlig regioner med massemidtpunkt i det gyldne snit, men med et
afgrænsende rektangel uden for margin. Et eksempel på dette kan ses i
figur \ref{hus}, hvor den sorte region ikke anses som liggende i det
gyldne snit af den naive fremgangsmåde. Vi vil gerne have at sådanne
regioner klassificeres som en positiv interessant region.

\begin{figure}[h]
	\begin{center}
		\includegraphics[scale=0.3,angle=0]{afsnit/vores_implementation/billeder/udvidet_loesning/husworks.png}
	\end{center}
	\caption[]{Et hus som bliver skåret over så midten ligger inde for snittet margin}
	\label{hus}
\end{figure}


\subsubsection{Opdeling af ragion med et grid}
% Original
%\subsubsection{Opdeling af ragion med et grid}
%Måden vi deller ragionen op på, er hved hjælp af et gridt som vi
%tilføjer vær baunding box, se figur \ref{grid}. Da vi ved hvilken farve
%baunding boxens ragion er, tager vi alle de punkter i grittet som har
%denne farve, og få der ved en bedre beskrivels af hvordan ragionen ser
%ud. For at gøre algoritmen lidt hurtiger, er vores gridt punkter lavet
%med 1 pixels mellemrum. 

\begin{figure}[h]
	\centering
	\includegraphics[scale=0.76,angle=0]{afsnit/vores_implementation/billeder/udvidet_loesning/udvidetloesninglayer.png}
	\caption[]{Et grid over en ragion i baunding box}
	\label{grid}
\end{figure}

\subsubsection{Bedømmelse med hensyn til massemidtpunkt}
Givet en region $R$ betegner vi antallet af punkter i regionen med
$|R|$. Vi antager at vi betragter et vertikalt snit $G$ i et billede.
Det gælder for alle punkter $p \in R$ at de kan befinde sig ovenpå, til
højre eller til venstre for $G$. I denne sammenhæng lader vi $R_r$ og
$R_l$ beskrive punkter henholdsvis til højre og venstre for snittet $G$.
Afstanden fra en punkt til kanten af et billede kaldes $D_p$, hvor
kanten er origo i billedet. Med disse informationer kan vi afgøre om
snitte deler regionen to lige store dele, samt om store dele af regionen
befinder sig langt væk fra snittet. Vi vil gerne kigge på en regions
massemidtpunkt og se om dette ligger inden for margin. Vi beregner
regionens massemidtpunkt ved funktionen


\begin{eqnarray}
    m(R) & = & \frac{\sum_{p \in R}{D_p}}{|R|} \label{masssemidpunkt}
    \label{MPunkt}
\end{eqnarray}

hvor m(R) giver os en værdi for hvad for et lige linie der delle ragionen
op i 2 delle, som er mest ens.

\begin{figure}[h]
	\begin{center}
		\includegraphics[scale=0.5,angle=0]{afsnit/vores_implementation/billeder/udvidet_loesning/cOMCutMargin.png}
	\end{center}
	\caption[]{Regioner hvor massemidtpunktet, snit og margin er tegnet ind, som man kan se ligger midtpunktet inde for marginen}
	\label{cOMCutMargin}
\end{figure}

Hvis $m(R)$ ligger inden for margin, se figur \ref{cOMCutMargin}. Bliver
regionen godtaget som liggende i snittet. Det er dog ikke helt nok kun
at bedømme regionerne efter massemidtpunkt, da man kan konstruere en
region som vi ikke vil godtage, men som har massemidtpunkt inde for
marginen, se figur \ref{dontwork}. 

\begin{figure}[h]
	\begin{center}
		\includegraphics[scale=0.5,angle=0]{afsnit/vores_implementation/billeder/udvidet_loesning/dontWork.png}
	\end{center}
	\caption[]{Figur som har et massemidtpunkt indenfor margin, men vi ikke vil have med i godtaget regioner, på grund af den lange stang}
	\label{dontwork}
\end{figure}

Måden vi få løst problemet på er ved at lave en nyt tjek som kan sortere
de regioner fra som vi ikke vil have godkende. Det gør vi hjælp af
funktion

\begin{eqnarray}
    f(R) & = & \frac{|R_{l}| - |R_{r}|}{|R|}
    \label{Fordeling}
\end{eqnarray}

Som sammenligner antal pixels på begge sider af snittet og giver en
procent på hvor stor forskel der er, det vil sige at $f(R) \in [-1,1]$.
Hvis $f(R)$ er positivt, er der $f(R)$ procent flere pixels på $|R_l|$
side og vise verser. hvis den abselutte værdig af $f(R)$ er over $0.75$
betyder det at regionen er få skævt fordelt og den sortere den fra. Få
at give et eksempel på hvornår vores naive algoritme virker, se figur \ref{centerOfMass}
\begin{figure}[h]
	\begin{center}
		\includegraphics[scale=0.35,angle=0]{afsnit/vores_implementation/billeder/udvidet_loesning/centerOfMass.png}
	\end{center}
	\caption[]{Eksempel på hvordan den udvidet algoritme virker på et billedet hvor den naive algoritme ikke vil have fundet det}
	\label{centerOfMass}
\end{figure}


}

% vim: set tw=72 spell spelllang=da:


\subsection{Topografisk kort til omkostninger}
{

Hvad er et topografisk kort? Eksempler på topografiske kort kan ses i
figur \ref{topography_plus} og \ref{topography_times}.

Vi vil nu beskrive en metode hvorved regioner bliver tildelt en
omkostning. Denne omkostning beregnes ud fra regionens placering i
billedet på baggrund af et topografisk kort. Således vil regioner ikke
blive vurderet efter hvorvidt de ligger i det gyldne snit eller ej, men
efter \emph{hvor meget} de ligger i det gyldne snit.

\subsubsection*{Generation af topografisk kort}

Givet en afbildning af et maleri ved matricen $\mathbf{I}$ med dimensioner
$N \times{} M$, kan man opstille et topografisk kort $\mathbf{T}$ med dimensioner
$N \times{} M$.

Det topografiske kort generes ud fra to vektorer $\mathbf{X}$ og
$\mathbf{Y}$ med dimensioner på henholdsvis $N \times 1$ og $1 \times M$.
\begin{equation}
    \mathbf{X} = \left(
    \begin{array}{c}
        x_1     \\
        x_2     \\
        \vdots  \\
        x_n
    \end{array} \right)
    \label{x_vector}
\end{equation}
og
\begin{equation}
    \mathbf{Y}^{t} = \left(
    \begin{array}{c}
        y_1     \\
        y_2     \\
        \vdots  \\
        y_m
    \end{array} \right)
\end{equation}

Vi betragter vektoren $\mathbf{X}$ som liniestykket givet ved $AB$ i
figur \ref{topograph_line}. Længden af liniestykket betegnes ved $|AB|$.
Det følger af ligning \ref{x_vector} at $|AB| = n$. På liniestykket $AB$
er det gyldne snit placeret ved $G$ hvilket betegner indeks $\lfloor
n\varPhi \rfloor$ i $\mathbf{X}$. Et margin er angivet ved punkterne $(G
- \delta) = m$ og $(G + \delta) = m'$, hvor $\delta$ er størrelsen på
margin. Ligeledes betegner $m$ og $m'$ indeks $\lfloor n \varPhi \pm
\delta \rfloor$ i $\mathbf{X}$.  Endvidere har vi at $|Ap| = \lfloor
\frac{1}{2}|AB| \rfloor \leq |pB|$ og $|m'q| = \lfloor \frac{1}{2}|qB|
\rfloor \leq |qB|$. I figur \ref{topograph_line} er kun liniestykket
$pB$ segmenteret, men $Ap$ segmenteres symmetrisk.

Vi placerer nu et nyt punkt $x$ på $pB$. I det et-dimensionelle plan kan
længden $|Gx|$ bruges som mål for hvor tæt $x$ er på det gyldne snit.
Punktet $x$ har dog ingen udstrækning, hvorfor vi ikke blot kan bruge
længden i praksis. Vi betragter nu en region $R \in \mathbb{Z}^{+}$.
Regionen $R$ er en liniestykke i det et-dimensionelle plan. Vi kan da
udregne placeringen af $R$ i forhold til punktet $G$ ved at summere alle
afstandene fra punkterne $x$ i $R$ til $G$.  Dette medfører at lange
liniestykker bliver tildelt højere værdi end små. Vi udregner således
$\frac{\sum_{x \in R}{|Gx|}}{|R|} = |G(\frac{|R|}{2})|$, hvor $|R|$ er
længden, dvs. antallet af punkter i $R$. Dette svarer til at beregne
afstanden fra regionen midtpunkt til $G$. Dette er ikke ønskværdigt, da
vi kan have to regioner med forskellig længde, men med samme midtpunkt.
Hvis vi lader $R_{max}$ betegne den større region og $R_{min}$ være den
mindre, hvor $ \frac{|R_{max}|}{2} = \frac{|R_{min}|}{2}$, da må
$R_{max}$ nødvendigvis have et ekstrema tættere på $G$ end $R_{min}$.
Det er derfor ikke retfærdigt at give begge regioner den samme værdi.

Vi ønsker at belønne punkter der ligger i eller tæt ved det gyldne snit,
men også omvendt give stor omkostning til punkter der ikke ligger i det
gyldne snit. Til dette bruges vektoren $\mathbf{X}$ som vil angive
omkostningen for hvert punkt på $AB$. Da vi ønsker at belønne punkter i
det gyldne snit er der ingen omkostning til punkter der ligger
i det gyldne snit. Vi sætter derfor $\mathbf{X}_{|AG|}$ til $0$.

\begin{figure}[!h]
    \centering
    \begin{picture}(240,30)
        \put(0, 10){$A$}
        \put(3, -5){\line(0, 1){10}}

        \put(116, 10){$p$}
        \put(118, -5){\line(0, 1){10}}

        \put(131, 10){$m$}
        \put(134, -4){\line(0, 1){8}}

        \put(144, 10){$G$}
        \put(147, -4){\line(0, 1){8}}

        \put(157, 10){$m'$}
        \put(160, -4){\line(0, 1){8}}

        \put(195, 10){$q$}
        \put(198, -4){\line(0, 1){8}}

        \put(233, 10){$B$}
        \put(236, -5){\line(0, 1){10}}

        \put(182, 10){$x$}
        \put(185, 0){\circle*{3}}

        \put(3, 0){\line(1, 0){233}}
    \end{picture}
    \caption[]{Liniestykke}
    \label{topograph_line}
\end{figure}

%Vektorerne angiver omkostningen for at have et punkt i den givne
%dimension. Hvis vi betragter det gyldne snit vil vi gerne have at
%værdierne i $\mathbf{X}_{\lfloor n \varPhi \rfloor}$ og
%$\mathbf{X}_{\lfloor n(1 - \varPhi) \rfloor}$ sættes til $0$, da det
%ikke skal have nogen omkostning at have et punkt i det gyldne snit.
%Ligeledes sættes $\mathbf{Y}_{\lfloor m \varPhi \rfloor}$ og
%$\mathbf{Y}_{\lfloor m(1 - \varPhi) \rfloor}$ også til $0$. Vi bruger
%også her et margin $\sigma$ hvor omkostningen for punkter er lav. Indeks
%$\lfloor n \varPhi \pm \sigma \rfloor$ sættes da til at have
%omkostningen $1$. Dette gøres helt analogt for indeks $\lfloor n(1 -
%\varPhi) \rfloor$ og for vektoren $\mathbf{Y}$.

\begin{figure}[h]
    \setlength\fboxsep{0pt}
    \setlength\fboxrule{0.5pt}
    \begin{center}
        \fbox{\includegraphics[width=0.8\textwidth]{afsnit/vores_implementation/billeder/udvidet_loesning/topographic_plus.png}}
    \end{center}
    \caption[]{Topografisk kort over omkostninger for regioner ud fra det
    gyldne snit. Værdien $T_{(x, y)}$ er givet ved funktionen
    $t(i, j) = C^{x}_{i} + C^{y}_{j}$ hvor $C^{x}$ og $C^{y}$ angiver
    omkostningsvektorerne.}
    \label{topography_plus}
\end{figure}

\begin{figure}[h]
    \setlength\fboxsep{0pt}
    \setlength\fboxrule{0.5pt}
    \begin{center}
        \fbox{\includegraphics[width=0.8\textwidth]{afsnit/vores_implementation/billeder/udvidet_loesning/topographic_times.png}}
    \end{center}
    \caption[]{Topografisk kort over omkostninger for regioner ud fra det
    gyldne snit. Her er værdierne udregnet ved at multiplicere værdierne
    fra omkostningsvektorene.}
    \label{topography_times}
\end{figure}

\subsubsection*{Omkostningsfunktionen}

Givet en mængde $R \in \{\mathbb{Z}^{+}\times\mathbb{Z}^{+}\}$, som
angiver punkterne i en given region, kan man finde omkostningen
$C$ ved
\begin{equation}
    C(R) = \sum_{(i, j) \in R}{\frac{T_{ij}}{nm}}
\end{equation}

}

% vim: set tw=72 spell spelllang=da:


\subsection{Implementering af udvidelser}
Vi har valgt at implementere den først nævnte udvidelse til bedømmelse
af interessante regioner. Denne er valgt, da den er en simpel
modifikation af den eksisterende metode, hvilket gør, at vi også kan
afprøve metoden på vores billedekorpus. Strukturen i metoden med topografiske
kort er meget langt væk fra den eksisterende metode, hvorfor vi har
vurderet, at denne ikke skulle implementeres. Topografiske kort indfører
et mål på regionerne, hvilket vi ellers ikke benytter os af i den
eksisterende metode.

}

% vim: set tw=72 spell spelllang=da:

