I den udvidet løsning har vi valt, at se på den måde, vi vælger ragioner
ud på. Hvor vi i den naiveMethode valgte at se på ragionens bounding
box, vil vi i den udviden metode valje at se på mindre snit af ragionen,
vi har også valt at udvide måde vi bestemmer hvad en ragion er. En
ragion var før, ænden med eller ikke med i en snit. kommer den udvidelsen
til at indeholde en kateroesering af ragionerne, så en ragion kan have
en vis katerogiering i forhold til et snit. 

\subsection{Udvidelse af ragioner plasering}
I denne udvidelse vil vi gerne indføre mere udføliger defination af
hvornår en ragion ligger i et snit. hvor vi før kun arbejdet med
ragionen som tagenteret inde i snittet, vil vi nu også gerne have
ragioner med som skæres i midden af snittet se figur
\ref{hus}.

\begin{figure}[h]
	\begin{center}
		\includegraphics[scale=0.3,angle=0]{afsnit/vores_implementation/billeder/udvidet_loesning/husworks.png}
	\end{center}
	\caption[]{Et hus som bliver skåret over i midden af snittet}
	\label{hus}
\end{figure}

\subsubsection{Forklaring at vi ikke laver om på vores featuer detector}


\subsubsection{Opdeling af ragion med et grid}
Måden vi deller ragionen op på, er hved hjælp af et gridt som vi
tilføjer vær baunding box, se figur \ref{grid}. Da vi ved hvilken farve
baunding boxens ragion er, tager vi alle de punkter i grittet som har
denne farve, og få der ved en bedre beskrivels af hvordan ragionen ser
ud. For at gøre algoritmen lidt hurtiger, er vores gridt punkter lavet
med 2 pixels mellemrum. XXX(dette rette vi nå vi pracis ved hvor stort
vores gridt er) XXX(har vi skravet at vi altid laver ragionerne
forskelige farver.)

\begin{figure}[h]
	\begin{center}
		\includegraphics[scale=0.76,angle=0]{afsnit/vores_implementation/billeder/udvidet_loesning/udvidetloesninglayer.png}
	\end{center}
	\caption[]{Et grid over en ragion i baunding box}
	\label{grid}
\end{figure}

\subsubsection{Metode forklaring}
Vi har nu x antal punkter i en ragion og vi andtager at det er et af den
vertikale snit vi ser på. Så kan punkterne befinde sig oven på snittet,
til venstre eller til højre fra snittet. Alle punkterne har også en
afstand d til snittet og en afstand D til kanten af billedet. Med de 3
information kan vi sige en masse om ragionen, f.eks. skære snitte
ragionen i 2 lige store delle eller befinder næste helle figuren sig
lang væk fra snittet osv. Få at få nogle konkrate matematik ned, har vi
valt at se på disse 2 formler
\ref{MPunkt}, \ref{Fordeling}.

\begin{eqnarray} 
 MPunkt &=& \frac {\sum{D}}{x}\label{MPunkt}
\end{eqnarray}

\begin{eqnarray} 
 Fordeling &=& \frac{x_{venstre~side}-x_{højre~side}}{x}\label{Fordeling}
\end{eqnarray}

Formel \ref{MPunkt} finder et mid punktet af ragionen i forhold til
venstre side for vertikale snit og toppen ved højesuntale snit. se figur
\ref{midpunkt}

\begin{figure}[h]
	\begin{center}
		\includegraphics[scale=0.5,angle=0]{afsnit/vores_implementation/billeder/udvidet_loesning/centerOfmass.png}
	\end{center}
	\caption[]{ragion hvor masse midpunktet er tegnet in i forhold til venster side af billedet}
	\label{midpunkt}
\end{figure}

Vi sammen linier nu midpunktet med snittet og få en afstand, hvis denne
afstand er mindre en magien, så er ragione godtaget, se figur
\ref{cOMCutMargin}

\begin{figure}[h]
	\begin{center}
		\includegraphics[scale=0.5,angle=0]{afsnit/vores_implementation/billeder/udvidet_loesning/cOMCutMargin.png}
	\end{center}
	\caption[]{ragion hvor masse midpunktet, snit og margin er teget ind, som man kan se ligger midpunktet inde for marginen}
	\label{cOMCutMargin}
\end{figure}

Nu skulle man tror at alting er godt men desværre kan vi kommer ud for
nogle figure som formel \ref{MPunkt} ikke tager højte for.

\begin{figure}[h]
	\begin{center}
		\includegraphics[scale=0.5,angle=0]{afsnit/vores_implementation/billeder/udvidet_loesning/dontWork.png}
	\end{center}
	\caption[]{Figur som overhold farmel \ref{MPunkt} men ikke \ref{Fordeling}}
	\label{dontwork}
\end{figure}

Som man kan se i figur \ref{Fordeling} er det fundene midpunkt inde for
marginen, men figur \ref{dontwork} har en lang stang som ænder med at
være lang væk fra snittet. Dette vil vi helt undgå da denne figur ikke
ligger i snittet. Men da de fleste pixels ligger på højre sider af
figuren kan vi, kan vi sotere denne figur væk ved at sammen line pixel
andtal på begge sider og se om den reletive pixeltal kommer under en vis
granse. Formel \ref{Fordeling} bruges til dette formål. Denne metode gør
også at vi få soteret de figure væk hvor snittet gå i gemmen side af
figuren, f.eks. et andsigt hvor snittet gå i gemmen kinden og ikke
næsen. se figur \ref{dontwork2}

\begin{figure}[h]
	\begin{center}
		\includegraphics[scale=0.5,angle=0]{afsnit/vores_implementation/billeder/udvidet_loesning/dontwork2.png}
	\end{center}
	\caption[]{Figur som overhold farmel \ref{MPunkt} men ikke \ref{Fordeling}}
	\label{dontwork2}
\end{figure}

ud fra disse opsavatione, har vi indført de 3 kategoriera neden for.
kategori 1: Alle de ragioner som vi fand før med bounding box og $ |snit - MPunkt| \leq Q \wedge |Fordeling| \leq P_1$ \\
kategori 2: $|snit - MPunkt| \leq Q \wedge |Fordeling| \leq P_2 \wedge (snit - MPunkt)*fordeling \geq 0$ \\
kategori 3: Resten\\

Hvor Q er antal pixels mellem snittet og margien og $P_1$ og $P_2$ er procentvis forskel på de to sider.
Som man kan se vægter vi formlel \ref{MPunkt} højre en \ref{Fordeling}


\subsection{brugbarhed}
\subsubsection{Fordele vs ulember}

\subsubsection{Implemation}

\subsection{Topografisk kort til omkostninger}
{

Hvad er et topografisk kort? Eksempler på topografiske kort kan ses i
figur \ref{topography_plus} og \ref{topography_times}.

Vi vil nu beskrive en metode hvorved regioner bliver tildelt en
omkostning. Denne omkostning beregnes ud fra regionens placering i
billedet på baggrund af et topografisk kort. Således vil regioner ikke
blive vurderet efter hvorvidt de ligger i det gyldne snit eller ej, men
efter \emph{hvor meget} de ligger i det gyldne snit.

\subsubsection*{Generation af topografisk kort}

Givet en afbildning af et maleri ved matricen $\mathbf{I}$ med dimensioner
$N \times{} M$, kan man opstille et topografisk kort $\mathbf{T}$ med dimensioner
$N \times{} M$.

Det topografiske kort generes ud fra to vektorer $\mathbf{X}$ og
$\mathbf{Y}$ med dimensioner på henholdsvis $N \times 1$ og $1 \times M$.
\begin{equation}
    \mathbf{X} = \left(
    \begin{array}{c}
        x_1     \\
        x_2     \\
        \vdots  \\
        x_n
    \end{array} \right)
    \label{x_vector}
\end{equation}
og
\begin{equation}
    \mathbf{Y}^{t} = \left(
    \begin{array}{c}
        y_1     \\
        y_2     \\
        \vdots  \\
        y_m
    \end{array} \right)
\end{equation}

Vi betragter vektoren $\mathbf{X}$ som liniestykket givet ved $AB$ i
figur \ref{topograph_line}. Længden af liniestykket betegnes ved $|AB|$.
Det følger af ligning \ref{x_vector} at $|AB| = n$. På liniestykket $AB$
er det gyldne snit placeret ved $G$ hvilket betegner indeks $\lfloor
n\varPhi \rfloor$ i $\mathbf{X}$. Et margin er angivet ved punkterne $(G
- \delta) = m$ og $(G + \delta) = m'$, hvor $\delta$ er størrelsen på
margin. Ligeledes betegner $m$ og $m'$ indeks $\lfloor n \varPhi \pm
\delta \rfloor$ i $\mathbf{X}$.  Endvidere har vi at $|Ap| = \lfloor
\frac{1}{2}|AB| \rfloor \leq |pB|$ og $|m'q| = \lfloor \frac{1}{2}|qB|
\rfloor \leq |qB|$. I figur \ref{topograph_line} er kun liniestykket
$pB$ segmenteret, men $Ap$ segmenteres symmetrisk.

Vi placerer nu et nyt punkt $x$ på $pB$. I det et-dimensionelle plan kan
længden $|Gx|$ bruges som mål for hvor tæt $x$ er på det gyldne snit.
Punktet $x$ har dog ingen udstrækning, hvorfor vi ikke blot kan bruge
længden i praksis. Vi betragter nu en region $R \in \mathbb{Z}^{+}$.
Regionen $R$ er en liniestykke i det et-dimensionelle plan. Vi kan da
udregne placeringen af $R$ i forhold til punktet $G$ ved at summere alle
afstandene fra punkterne $x$ i $R$ til $G$.  Dette medfører at lange
liniestykker bliver tildelt højere værdi end små. Vi udregner således
$\frac{\sum_{x \in R}{|Gx|}}{|R|} = |G(\frac{|R|}{2})|$, hvor $|R|$ er
længden, dvs. antallet af punkter i $R$. Dette svarer til at beregne
afstanden fra regionen midtpunkt til $G$. Dette er ikke ønskværdigt, da
vi kan have to regioner med forskellig længde, men med samme midtpunkt.
Hvis vi lader $R_{max}$ betegne den større region og $R_{min}$ være den
mindre, hvor $ \frac{|R_{max}|}{2} = \frac{|R_{min}|}{2}$, da må
$R_{max}$ nødvendigvis have et ekstrema tættere på $G$ end $R_{min}$.
Det er derfor ikke retfærdigt at give begge regioner den samme værdi.

Vi ønsker at belønne punkter der ligger i eller tæt ved det gyldne snit,
men også omvendt give stor omkostning til punkter der ikke ligger i det
gyldne snit. Til dette bruges vektoren $\mathbf{X}$ som vil angive
omkostningen for hvert punkt på $AB$. Da vi ønsker at belønne punkter i
det gyldne snit er der ingen omkostning til punkter der ligger
i det gyldne snit. Vi sætter derfor $\mathbf{X}_{|AG|}$ til $0$.

\begin{figure}[!h]
    \centering
    \begin{picture}(240,30)
        \put(0, 10){$A$}
        \put(3, -5){\line(0, 1){10}}

        \put(116, 10){$p$}
        \put(118, -5){\line(0, 1){10}}

        \put(131, 10){$m$}
        \put(134, -4){\line(0, 1){8}}

        \put(144, 10){$G$}
        \put(147, -4){\line(0, 1){8}}

        \put(157, 10){$m'$}
        \put(160, -4){\line(0, 1){8}}

        \put(195, 10){$q$}
        \put(198, -4){\line(0, 1){8}}

        \put(233, 10){$B$}
        \put(236, -5){\line(0, 1){10}}

        \put(182, 10){$x$}
        \put(185, 0){\circle*{3}}

        \put(3, 0){\line(1, 0){233}}
    \end{picture}
    \caption[]{Liniestykke}
    \label{topograph_line}
\end{figure}

%Vektorerne angiver omkostningen for at have et punkt i den givne
%dimension. Hvis vi betragter det gyldne snit vil vi gerne have at
%værdierne i $\mathbf{X}_{\lfloor n \varPhi \rfloor}$ og
%$\mathbf{X}_{\lfloor n(1 - \varPhi) \rfloor}$ sættes til $0$, da det
%ikke skal have nogen omkostning at have et punkt i det gyldne snit.
%Ligeledes sættes $\mathbf{Y}_{\lfloor m \varPhi \rfloor}$ og
%$\mathbf{Y}_{\lfloor m(1 - \varPhi) \rfloor}$ også til $0$. Vi bruger
%også her et margin $\sigma$ hvor omkostningen for punkter er lav. Indeks
%$\lfloor n \varPhi \pm \sigma \rfloor$ sættes da til at have
%omkostningen $1$. Dette gøres helt analogt for indeks $\lfloor n(1 -
%\varPhi) \rfloor$ og for vektoren $\mathbf{Y}$.

\begin{figure}[h]
    \setlength\fboxsep{0pt}
    \setlength\fboxrule{0.5pt}
    \begin{center}
        \fbox{\includegraphics[width=0.8\textwidth]{afsnit/vores_implementation/billeder/udvidet_loesning/topographic_plus.png}}
    \end{center}
    \caption[]{Topografisk kort over omkostninger for regioner ud fra det
    gyldne snit. Værdien $T_{(x, y)}$ er givet ved funktionen
    $t(i, j) = C^{x}_{i} + C^{y}_{j}$ hvor $C^{x}$ og $C^{y}$ angiver
    omkostningsvektorerne.}
    \label{topography_plus}
\end{figure}

\begin{figure}[h]
    \setlength\fboxsep{0pt}
    \setlength\fboxrule{0.5pt}
    \begin{center}
        \fbox{\includegraphics[width=0.8\textwidth]{afsnit/vores_implementation/billeder/udvidet_loesning/topographic_times.png}}
    \end{center}
    \caption[]{Topografisk kort over omkostninger for regioner ud fra det
    gyldne snit. Her er værdierne udregnet ved at multiplicere værdierne
    fra omkostningsvektorene.}
    \label{topography_times}
\end{figure}

\subsubsection*{Omkostningsfunktionen}

Givet en mængde $R \in \{\mathbb{Z}^{+}\times\mathbb{Z}^{+}\}$, som
angiver punkterne i en given region, kan man finde omkostningen
$C$ ved
\begin{equation}
    C(R) = \sum_{(i, j) \in R}{\frac{T_{ij}}{nm}}
\end{equation}

}

% vim: set tw=72 spell spelllang=da:

