I den udvidet løsning har vi valt, at se på den måde, vi vælger ragioner
ud på. Hvor vi i den naiveMethode valgte at se på ragionens bounding
box, vil vi i den udviden metode valje at se på mindre snit af ragionen,
vi har også valt at udvide måde vi bestemmer hvad en ragion er. En
ragion var før, ænden med eller ikke med i en snit. kommer den udvidelsen
til at indeholde en kateroesering af ragionerne, så en ragion kan have
en vis katerogiering i forhold til et snit. 

\subsection{Udvidelse af ragioner plasering}
I denne udvidelse vil vi gerne indføre mere udføliger defination af hvornår en ragion ligger i et snit. hvor vi før kun arbejdet med ragionen som tagenteret inde i snittet, vil vi nu også gerne have ragioner med som skæres i midden af snittet se figur \ref{PCirkel}.(Vidre med billeder)

\subsubsection{Forklaring at vi ikke laver om på vores featuer detector}


\subsubsection{Opdeling af ragion med et grid}
Måden vi deller ragionen op på, er hved hjælp af et gridt som vi
tilføjer vær baunding box. Da vi ved hvilken farve baunding boxens
ragion har, tager vi alle de punkter i grittet som har denne farve, og
få et bedre beskrivels af hvordan ragionen ser ud. For at gøre
algoritmen lidt hurtiger, er vores gridt punkter lavet med 2 pixels
mellemrum. XXX(dette rette vi nå vi pracis ved hvor stort vores gridt er)
XXX(har vi skravet at vi altid laver ragionerne forskelige farver.)

\subsubsection{Metode forklaring}
Vi har nu x antal punkter i en ragion og vi andtager at det er et af den
vertikale snit vi ser på. Så kan punkterne befinde sig oven på snittet
XXX(margin), til venstre eller til højre fra snittet. Alle punkterne har
også en afstand d til snittet og en afstand D til kanten af billedet. Med de 3 information kan vi sige en masse
om ragionen, f.eks. skære snitte ragionen i 2 lige store delle eller
befinder næste helle figuren sig lang væk fra snittet osv. Få at få
nogle konkrate matematik ned, har vi valt at se på disse 2 formler
\ref{MPunkt}, \ref{Fordeling}.

\begin{eqnarray} 
 MPunkt &=& \frac {\sum{D}}{x}\label{MPunkt}
\end{eqnarray}

\begin{eqnarray} 
 Fordeling &=& \frac{x_{venstre~side}-x_{højre~side}}{x}\label{Fordeling}
\end{eqnarray}

Formel \ref{MPunkt} finder et mid punktet af ragionen i forhold til venstre side for vertikale snit og toppen ved højesuntale snit. Formel \ref{Fordeling} beskriver den reletive pixel forskel på vær side af snittet. Vi har også indførst 3 kategoriera.
kategori 1: Alle de ragioner som vi fand før med bounding box og $ |snit - MPunkt| \leq Q \wedge |Fordeling| \leq P_1$ \\
kategori 2: $|snit - MPunkt| \leq Q \wedge |Fordeling| \leq P_2 \wedge (snit - MPunkt)*fordeling \geq 0$ \\
kategori 3: Resten\\

Hvor Q er antal pixels mellem snittet og margien og $P_1$ og $P_2$ er procentvis forskel på de to sider.
(forklaring hvorfor man kan få de rigtige ragioner, hvis man lige sætte disse 3 kategorier op)
\subsection{brugbarhed}
\subsubsection{Fordele vs ulember}

\subsubsection{Implemation}
