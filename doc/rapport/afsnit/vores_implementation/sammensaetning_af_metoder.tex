{

Vi vil nu, uden at gå alt for meget i detajler, give en intuitiv
forklaring på hvordan ovenstående metoder bruges til at trække regioner
ud af et billede.  Regioner trækkes ud i forhold til et snit i billedet,
således at vi kun finder regioner inden for margin relevant for dette
snit. Vi går gennem fire skridt, når vi trækker regioner ud, hvoraf de
tre første kan betragtes som forberedelse af billedet inden den endelige
segmentering segmentering ved floodfill:

\paragraph{Kantdetektion}
Først finder vi kanter i billedet. Vi skal bruge kanterne i senere
skridt.

\begin{figure}[!h]
    \setlength\fboxsep{0pt}
    \setlength\fboxrule{0.5pt}
    \begin{center}
        \fbox{Kantdetektion}
    \end{center}
    \caption[]{}
    \label{sammen_kanter}
\end{figure}

\paragraph{Sløring}
Billedet sløres således at vi kan male større regioner når vi senere
bruger floodfill.

\begin{figure}[!h]
    \setlength\fboxsep{0pt}
    \setlength\fboxrule{0.5pt}
    \begin{center}
        \fbox{Sløring}
    \end{center}
    \caption[]{}
    \label{sammen_slør}
\end{figure}

\paragraph{Fremhævelse af kanter}
Da vi slørede billedet brugte vi den simple sløringsmetode som kan
ødelægge kanterne i billedet. Vi tager nu de originale kanter og lægger
oveni billedet således at kanterne bevares.

\begin{figure}[!h]
    \setlength\fboxsep{0pt}
    \setlength\fboxrule{0.5pt}
    \begin{center}
        \fbox{Fremhævede kanter}
    \end{center}
    \caption[]{}
    \label{sammen_frem_kant}
\end{figure}

\paragraph{Segmentering}
Vi har nu et sløret billede, men med de originale kanter fremhævet.
Dette billede segmenteres nu ved at bruge floodfill. De originale kanter
bruges i dette skridt også til at fortælle floodfill hvor nye regioner
starter. Vi bruger således floodfill ned langs et snit i billedet og
bruger en ny farve når vi passerer en kant i det originale billede.

\begin{figure}[!h]
    \setlength\fboxsep{0pt}
    \setlength\fboxrule{0.5pt}
    \begin{center}
        \fbox{Floodfill}
    \end{center}
    \caption[]{}
    \label{sammen_floodfill}
\end{figure}

}

% vim: set tw=72 spell spelllang=da:
