{
{\sffamily Dette afsnit fungerer indtil videre som holdeplads for tekst som bør
stå et andet sted$^{\textsf{TM}}$. Titlen er derfor dybt misvisende.
}

\subsection{Implementation af databasen}
Vi bruger SQLite til selve databasen, hovedsageligt fordi der ikke kræves
nogen videre konfiguration af en sådan database. Den underliggende
database er dog underordnet, da vi bruger Python-pakken \emph{SQLObject}
som giver et abstraktionslag til en bred vifte af databaser. Vi opretter
blot de tabeller vi ønsker at have i databasen som klasser gennem
Python. Ligeledes får vi en sådan klasse tilbage når der laves
forespørgsler til databasen. Da \emph{SQLObject} klarer al kommunikation
med databasen er det derfor muligt at skifte den underliggende database
ud hvis man ønsker det. Vi ser dog ikke nogen grund til at bruge en
anden foreløbig, da SQLite opfylder vores behov. Endvidere har SQLite
den umiddelbare fordel at selve databasen lægges i en fil i filsystemet.
Det er derfor en let sag at tage sikkerhedskopier af databasen uden alt
for meget besvær.

\subsection{Crawler til wga.hu}
I den kommaseparerede fil fra \cite{wgahu} er
malerier opstillet efter kunstneren, hvilket gør det let først at
oprette denne i databasen og derefter oprette de efterfølgende malerier
tilknyttet denne kunstner.

Vores korpus består af billeder hentet fra hjemmesiden \cite{wgahu} som
indeholder europæiske kunstartikler fra år 1001 -- 1900. I
kunstartiklerne, hvor det samlede antal er omkring 23.000, indgår
møbler, skulpturer, mosaikker og malerier, hvor sidstnævnte vil være
vores fokus. Fra \cite{wgahu} tilbydes en kommasepareret fil over hele
deres database som vi har brugt til at populere vores egen database med.
Der gives mange oplysninger om den enkelte artikel samt dennes kunstner.
Vi har konstrueret en parser som trækker disse informationer ud fra
filen og lægger dem ind i tabellerne \ref{artistTable0} og
\ref{paintingTable0} i databasen. Da vi primært vil beskæftige os med
malerier vil vi nu blot omtale kunstartikler som malerier.

Den konstruerede parser til den kommaseparerede fil er dog ret grov, da
folkene bag \cite{wgahu} ikke har lagt meget vægt på at være konsistente
i deres formulering af en kunstners fødsels- og dødsår eller en
genstands dimensioner. En følge deraf er, at nogle kunstnere, hvor
\cite{wgahu} ikke har en klar indikation af dennes levealder, ikke bliver
registreret i databasen. Vi kan dog stadig slå kunstneren op ved at
bruge feltet ``timeline'' i tabel \ref{artistTable0} som angiver hvilken
periode kunstneren tilhører. Vi har i enkelte tilfælde set os nødsaget
til at rette i den kommaseparerede fil i tilfælde hvor der er blevet
indsat tegn der helt umuliggør korrekt parsing, såsom ekstra komma
eller semikolon.

Givet den kommaseparerede fil fra \cite{wgahu} er det en smal sag at
konstruere en crawler som henter alle billederne fra hjemmesiden ned. I
filen gives nemlig en henvisning til hvor man kan finde et billede af
genstanden på deres side.

\subsection{Oprettelse af resultater (database)}
Denne sektion afhænger til dels af en færdig implementation af den
automatiserede analyse. Hvordan lægger vi resultater fra en kørsel ind i
databasen? Forklar de (smarte) metoder vi har til rådighed. Vi kan blot
kaste klasser ind i databasen. Vores settings-klasse bruges til at
oprette runs, dictionary til at smække resultatet fra analysen ind.

På grund af begrænsninger i \emph{OpenCV} kan vi ikke gemme regionens
præcise form, men kun dennes begrænsende rektangel og regionens areal.

\subsection{Genskabelse af parametre og resultater (database)}
At kunne genskabe de fundne resultater fra en analyse har meget stor
betydning, dels for at kunne udtage stikprøver i udviklingen af hele
programmet, men også for at kunne fremvise grafiske resultater. Vi har
allerede været inde på, at man for at kunne genskabe et resultat, skal
vide hvilke parametre der oprindeligt har været brugt. Ovenstående
databaseskema gør det let at hente disse parametre ud. Hvis vi får et
resultat med overraskende mange regioner og gerne vil undersøge dette
tilfælde, har vi metoder til rådighed der giver os lige nøjagtig de
informationer vi har brug for at vise dette grafisk. Helt konkret har vi
metoderne vist i listing \ref{rekonst_koersel} til rådighed.

\vspace{0.5cm}
\begin{lstlisting}[caption={Metoder til rekonstruktion af kørsler},captionpos=b,label={rekonst_koersel},numbers=none]
def getSettingsForRunId(runId):
    """Return the settings instance for a given run"""
    pass

def getCutRatiosForRunId(runId):
    """Return the list of cut ratios for a given run"""
    pass

def getSettingsForResultId(resultId):
    """Return the settings instance for a given result"""
    pass

def getSettingsForRegionId(regionId):
    """Return the settings instance for a given region"""
    pass

def getCutRatioForRegionId(regionId):
    """Return the list of cut ratios for a given region"""
    pass

def getCutNoForRegionId(regionId):
    """Return the cut number for a given region"""
    pass

def getRegionsForResultId(resultId):
    """Return the list of regions for a given result"""
    pass
\end{lstlisting}

Selvom metoderne i listing \ref{rekonst_koersel} ikke viser noget
egentlig kode, bør det ud fra sammenhængen være klart hvad disse metoder
gør. Alle metoder der starter med \texttt{getSettings} returnerer
klassen \texttt{Settings} som vist i listing \ref{settings_klassen} med
indstillinger tilpasset den enkelte forespørgelse\footnote{Det kan godt tænkes
at denne klasse allerede bliver introduceret i afsnittet inden}.

\vspace{0.5cm}
\begin{lstlisting}[caption={Settings-klassen med standardindstillinger},captionpos=b,label={settings_klassen},numbers=none]
class Settings:
    """These are the default settings for the analysis"""
    edgeThreshold1 = 78
    edgeThreshold2 = 2.5 * edgeThreshold1
    lo = 4
    up = 4
    cutRatios = None
    marginPercentage = 0.009
    method = 'naive'
    ...
\end{lstlisting}

Det ses at vi har mulighed for at trække de fundne regioner ved et
snit ud og vi behøver derfor ikke at køre nogen analyse på billedet hvis
vi blot ønsker at få de fundne regioners begrænsende areal vist. I dette
tilfælde kan vi nøjes med at forespørge databasen om de regioner der er
tilknyttet et snit vi gerne vil undersøge og traversere gennem den liste
af regioner vi får tilbage. Hver region er repræsenteret som en klasse
hvor vi kan trække rektanglet ud og vi bruger da \emph{OpenCV} til at
tegne rektanglet på det tilknyttede billede.
}

% vim: set tw=72 spell spelllang=da:

