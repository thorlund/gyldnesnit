{
{\sffamily I billedbehandling findes begrebet \emph{feature detection}.
En \emph{feature} oversættes bedst som \emph{et træk} eller et
\emph{kendetegn}. Man har da med \emph{detektion af træk} i et billede
at gøre. Nøjagtig \emph{hvad} et træk \emph{er}, er ikke klart
defineret, men skal tilpasses den enkelte opgave. En andet område
indenfor detektion af træk kaldes for \emph{blob detection}, hvor en
\emph{blob} kan kort beskrives som en ensfarvet region i billedet. Vi
vil derfor fremover referere til en blob som en region. Det er også
detekteringen af regioner vi hovedsageligt vil koncentrere os om.

Opgaven i billedbehandling består nu i at finde de \emph{interessante}
regioner i billedet. Vi vil i det følgende komme frem til præcis hvad vi
forstår ved region. Senere i kapitlet, i afsnit \ref{section_naiv}, vil
vi præsentere en simpel fremgangsmåde som fortæller os hvornår en region
kan betegnes som interessant. Vi vil dog først se på hvordan digitale billeder
repræsenteres.
}

\subsection{Repræsentation af digitale billeder}
Et digitalt billede er sammensat af pixels som er et punkt i et
koordinatsystem hvortil der er knyttet en værdi. Et gråtonebillede er
givet ved relationen
\begin{equation}
    \mathbb{Z}^{+}\times{} \mathbb{Z}^{+} \rightarrow \mathbb{Z}^{+}
\end{equation}
hvor to positive heltal afbilledes over i et positivt heltal.
Funktionen $G(x, y) \in \mathbb{Z}^{+}$ angiver mængden af hvid farve i
en pixel med koordinaterne $(x, y)$.

Som et simpelt eksempel siger vi nu, at en pixel i et billede kan antage
to værdier: $0$ og $1$. Vi har altså relationen $\mathbb{Z}^{+}\times{}
\mathbb{Z}^{+} \rightarrow [0, 1]$. En pixel med værdi $0$ vil ikke
blive farvet, dvs. den forbliver sort, mens en pixel med værdien $1$ vil
blive farvet hvid.  Figur \ref{billede_pixels} viser et sådan simpelt,
binært billede.

% Please, PLEASE, do not try this at home!
% This is the UGLY way to tex things :P
\begin{figure}[!h]
    \renewcommand{\arraystretch}{1.5}
    \centering
    \begin{tabular}{cc|c|c|c|}
        % Start crying
           & \multicolumn{4}{c}{\hspace{1.5em}$y$}\\
           & \multicolumn{4}{c}{\hspace{1.6em}0\hspace{1.2em}1\hspace{1.2em}2} \\\cline{3-5}
           &  0 & 1                                     & \cellcolor{black}\textcolor{white}{0} & 1                                     \\\cline{3-5}
      $x$  &  1 & \cellcolor{black}\textcolor{white}{0} & \cellcolor{black}\textcolor{white}{0} & \cellcolor{black}\textcolor{white}{0} \\\cline{3-5}
           &  2 & 1                                     & \cellcolor{black}\textcolor{white}{0} & 1                                     \\\cline{3-5}
    \end{tabular}
    \caption[]{Et simpelt $3 \times 3$ billede vist som pixels.}
    \label{billede_pixels}
\end{figure}

  Bemærk at koordinatsystemet starter i øverste
venstre hjørne stigende mod nedre højre hjørne.

Et billede kan da opstilles som en $N \times M$ matrix som vist i
ligning \ref{billede_matrix}.

\begin{equation}
    P = \left ( \begin{array}{ccc}
        1 & 0 & 1 \\
        0 & 0 & 0 \\
        1 & 0 & 1
    \end{array} \right )
    \label{billede_matrix}
\end{equation}

At bruge billedet som en matrix har nogle beregningsmæssige fordele, men
det falder udenfor denne introduktions formål at gå ind i dette.

\subsection{Regioner i vilkårlige malerier}
Vi betegner en region som en samling af pixels i et billede, som har en
farve indenfor en vis afvigelse af en givet startpixel. Hvis nu antager
at vi sætter denne afvigelse til $0$, vil billedet i figur
\ref{billede_pixels} have en sort region formet som et kryds. Alt efter
hvordan man definerer afvigelsen, kan man da få regioner ud fra digitale
gengivelser af malerier såsom en himmel, i et maleri af et landskab,
eller et ansigt. En regionen kan altså være hvad som helst i billedet.

\subsection{Antagelser}
Vi kan med sikkerhed sige, at der ikke kan opstilles nogen globale
regler for hvad der er interessant i et maleri.  Interessante regioner i
billeder er fremhævet, har en vis størrelse og er ``tydeligt''
afgrænset.

}

% vim: set tw=72 spell spelllang=da:
