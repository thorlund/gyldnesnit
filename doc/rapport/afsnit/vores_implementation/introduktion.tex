{
{\sffamily I billedbehandling tales der ofte om begrebet \emph{feature
detection}. En \emph{feature} oversættes bedst som \emph{et træk} eller
et \emph{kendetegn}. Man har da med \emph{detektion af træk} i et
billede at gøre. Nøjagtig \emph{hvad} et træk \emph{er}, er ikke klart
defineret, men skal tilpasses den enkelte opgave. En andet område
indenfor detektion af træk kaldes for \emph{blob detection}, hvor en
\emph{blob} kan kort beskrives som en ensfarvet region i billedet. Vi
vil derfor fremover referere til en blob som en region. Det er også
detekteringen af regioner vi hovedsageligt vil koncentrere os om.

Opgaven i billedbehandling består nu i at finde de \emph{interessante}
regioner i billedet. Vi vil i det følgende komme frem til præcis hvad vi
forstår ved region. Senere i kapitlet, i afsnit \ref{section_naiv}, vil
vi præsentere en simpel fremgangsmåde som fortæller os hvornår en region
kan betegnes som interessant. Vi vil dog først se på hvordan billeder
bliver repræsenteret internt i computeren og ud fra dette se hvorfor det
i grunden er så svært at definere hvad der er kan betragtes som
interessant i malerier.
}

\subsection{Intern repræsentation af billeder}
Et digitalt billede er sammensat af det der kaldes pixels. En pixel kan
ses som et punkt i et koordinatsystem hvortil der er knyttet en værdi.
Som et simpelt eksempel siger vi nu, at en pixel i et billede kan antage
to værdier: $0$ og $1$. En pixel med værdi $0$ vil ikke blive farvet,
dvs. den forbliver sort, mens en pixel med værdien $1$ vil blive farvet
hvid. Figur \ref{billede_pixels} viser et sådan simpelt, binært billede.

% Please, PLEASE, do not try this at home!
% This is the UGLY way to tex things :P
\begin{figure}[!h]
    \renewcommand{\arraystretch}{1.5}
    \centering
    \begin{tabular}{cc|c|c|c|}
        % Start crying
           & \multicolumn{4}{c}{\hspace{1.5em}$y$}\\
           & \multicolumn{4}{c}{\hspace{1.6em}0\hspace{1.2em}1\hspace{1.2em}2} \\\cline{3-5}
           &  0 & 1                                     & \cellcolor{black}\textcolor{white}{0} & 1                                     \\\cline{3-5}
      $x$  &  1 & \cellcolor{black}\textcolor{white}{0} & \cellcolor{black}\textcolor{white}{0} & \cellcolor{black}\textcolor{white}{0} \\\cline{3-5}
           &  2 & 1                                     & \cellcolor{black}\textcolor{white}{0} & 1                                     \\\cline{3-5}
    \end{tabular}
    \caption[]{Et simpelt $3 \times 3$ billede vist som pixels.}
    \label{billede_pixels}
\end{figure}

Vi bemærker at en pixels koordinater kun kan antage heltallige værdier.
Dette skyldes at billedet i sidste ende skal vises på en computerskærm.
Skærmen er delt ind i mange små felter svarende til pixels og man kan
derfor ikke have værdier mellem heltallene.  Endvidere ser vi at
koordinatsystemet starter i øverste venstre hjørne stigende mod nedre
højre hjørne.

Et billede kan da opstilles som en $N \times M$ matrix som vist i
ligning \ref{billede_matrix}.

\begin{equation}
    P = \left ( \begin{array}{ccc}
        1 & 0 & 1 \\
        0 & 0 & 0 \\
        1 & 0 & 1
    \end{array} \right )
    \label{billede_matrix}
\end{equation}

At bruge billedet som en matrix har nogle beregningsmæssige fordele, men
det falder udenfor denne introduktions formål at gå ind i dette.
Endeligt skal det siges at pixels godt kan tage andre værdier end $0$ og
$1$. Vi arbejder med billeder hvor værdien for hver pixel er
repræsenteret ved tre 8 bit størrelser, hver især med værdier i mængden
$\{0, 1, 2, \cdots, 254, 255\}$. Sammensætningen af de tre værdier, som
beskrives som kanaler eller farvebånd, kaldes for en RGB-farve, hvor
tallene repræsenteret ved $(R,G,B)$ henholdsvis angiver mængden af rød,
grøn og blå farve i en pixel. Et sådan billede kaldes for et
RGB-billede. For uddybende information omkring billeders repræsentation,
se da \cite{SIOlsen}.

\subsection{Regioner i vilkårlige malerier}
Vi betegner en region som en samling af pixels i et billede, som har en
farve indenfor en vis afvigelse af en givet startpixel. Hvis nu antager
at vi sætter denne afvigelse til $0$, vil billedet i figur
\ref{billede_pixels} have en sort region formet som et kryds. Alt efter
hvordan man definerer afvigelsen, kan man da få regioner ud fra digitale
gengivelser af malerier såsom en himmel, i et maleri af et landskab,
eller et ansigt.

\subsubsection*{Computeren som betragter}
Giver man tre forskellige mennesker den opgave at afgøre hvad der er det
interessante i et givet maleri, kan man meget vel få tre forskellige
svar. Mere kompliceret bliver det når man spørger ind til \emph{hvordan}
de er kommet frem til deres svar. Én begrunder måske sit valg med en
viden om netop det givne maleri, en anden med viden om maleriets
kunstner eller periode, mens en tredie begrunder det med æstetiske
virkemidler eller subjektive holdninger. Her er alle muligheder åbne for
at lave fejl, da man som regel ikke har kunstneren til rådighed til at
give det rigtige svar, hvis han da overhovedet selv kender det.
Maleriets motiv kan lede én på sporet af hvad der er interessant, men
dette kræver måske en viden om motivets bagvedliggende historie.
Mennesket tager altså en lang række overvejelser og baggrundsviden i
betragtning når ovenstående opgave skal løses.

Som allerede nævnt afhænger de interessante træk af den enkelte opgave.
Hvis vi forestiller os en læge der kigger på et røntgenbillede af en
patient som har brækket armen, er det indlysende hvad lægen betragter
som det interessante i billedet, nemlig der hvor bruddet sidder. Når vi
har med tilfældige malerier at gøre, så står vi dog stadig tilbage med
spørgsmålet: \emph{``Hvad er egentlig \emph{interessant} i et maleri?''}
Vi lader dette spørgsmål stå åbent for at se på hvordan computeren
betragter et maleri.

[Det er som om at det er lettest bare at spille fallit her]

Når computeren ser på et billede ser den pixels. Computeren ser
endvidere disse pixels \emph{én ad gangen}. Det svarer altså til at de
enkelte pixelværdier bliver læst op for en person der ikke kan se selve
billedet. For at køre eksemplet helt ud, så har vi at en person får
følgende at vide:
\begin{quote}
    \emph{``Pixel med koordinater $(0, 0)$ har værdien $1$. Pixel med
    koordinater $(1, 0)$ har værdien $0$.''} etc.
\end{quote}
Dette giver ingen egentlig information om \emph{hvad} billedet
forestiller. Computeren kan ikke se billedet i sin helhed og har som
udgangspunkt ikke nogen baggrundsviden at basere en vurdering på. Det
eneste computeren kan gøre er at gå det igennem pixel for pixel og
sammenligne dem. Dette betyder faktisk, at man for hver pixel bliver nød
til at tage beslutningen om denne er del af noget interessant.

\subsection{Antagelser}
Vi kan med sikkerhed sige, at der ikke kan opstilles nogen globale
regler for hvad der er interessant i et maleri.  Interessante regioner i
billeder er fremhævet, har en vis størrelse og er ``tydeligt''
afgrænset.

}

% vim: set tw=72 spell spelllang=da:
