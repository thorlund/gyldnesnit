{
{\sffamily Inden for billedbehandling dækker begrebet \emph{feature
detection} over den opgave at detektere \emph{features} i et givet
billede.  \emph{Feature} oversættes bedst med \emph{træk} eller
\emph{kendetegn}, men nøjagtig \emph{hvad} et træk \emph{er}, er ikke
klart defineret og skal tilpasses den enkelte
opgave\cite{SIOlsen,WikiFeatureDetection}. Et andet område inden
for detektion af træk kaldes for \emph{blob detection}, hvor en
\emph{blob} kan beskrives som en ensfarvet region i billedet. Vi
vil derfor fremover referere til en blob som en region, og i det
følgende vil vi komme ind på præcis hvad vi forstår ved en sådan. Da det
imidlertid kun er de interessante regioner, der ønskes udvalgt i
billederne, præsenteres i afsnit \ref{section_naiv}, en simpel
fremgangsmåde, som fortæller os, hvornår en region kan betegnes som
interessant. Vi vil dog først se på, hvordan digitale billeder
repræsenteres. For en mere dybdegående indledning til billedbehandling
henvises til \cite{SIOlsen}.
}

\subsection{Repræsentation af digitale billeder}
Et digitalt billede er, som nævnt i afsnit
\ref{section_computer_betragter}, sammensat af pixels. En pixel er et punkt i et
koordinatsystem, hvortil der er knyttet en værdi. Et gråtonebillede er
givet ved afbildningen
\begin{equation}
    \mathbb{Z}^{+}\times{} \mathbb{Z}^{+} \rightarrow \mathbb{Z}^{+}
\end{equation}
fra to positive heltal over i et positivt heltal.  Funktionen $G(x, y)
\in \mathbb{Z}^{+}$ angiver mængden af hvid farve i en pixel med
koordinaterne $(x, y)$.

Som et simpelt eksempel siger vi nu, at en pixel i et billede kan antage
to værdier: $0$ og $1$. Vi har altså afbildningen $\mathbb{Z}^{+}\times{}
\mathbb{Z}^{+} \rightarrow \{0, 1\}$. En pixel med værdi $0$ vil ikke
blive farvet, dvs. den forbliver sort, mens en pixel med værdien $1$ vil
blive farvet hvid.  Figur \ref{billede_pixels} viser et sådant simpelt,
binært billede.

% Please, PLEASE, do not try this at home!
% This is the UGLY way to tex things :P
\begin{figure}[!h]
    \renewcommand{\arraystretch}{1.5}
    \centering
    \begin{tabular}{cc|c|c|c|}
        % Start crying
           & \multicolumn{4}{c}{\hspace{1.5em}$y$}\\
           & \multicolumn{4}{c}{\hspace{1.6em}0\hspace{1.2em}1\hspace{1.2em}2} \\\cline{3-5}
           &  0 & 1                                     & \cellcolor{black}\textcolor{white}{0} & 1                                     \\\cline{3-5}
      $x$  &  1 & \cellcolor{black}\textcolor{white}{0} & \cellcolor{black}\textcolor{white}{0} & \cellcolor{black}\textcolor{white}{0} \\\cline{3-5}
           &  2 & 1                                     & \cellcolor{black}\textcolor{white}{0} & 1                                     \\\cline{3-5}
    \end{tabular}
    \caption[]{Et simpelt $3 \times 3$ billede vist som pixels.}
    \label{billede_pixels}
\end{figure}

Bemærk, at koordinatsystemet starter i øverste venstre hjørne, med
stigende $x$- og $y$-værdier mod nedre højre hjørne.

Et billede kan da opstilles som en $N \times M$ matrix som vist i
ligning \ref{billede_matrix}.

\begin{equation}
    \mathbf{I} = \left ( \begin{array}{ccc}
        1 & 0 & 1 \\
        0 & 0 & 0 \\
        1 & 0 & 1
    \end{array} \right )
    \label{billede_matrix}
\end{equation}

At bruge billedet som en matrix har nogle beregningsmæssige fordele, men
det falder uden for denne introduktions formål at gå ind i dette.

\subsection{Regioner i vilkårlige malerier}
Billedet i figur \ref{billede_pixels} har en sort region formet som et
kryds. Dette leder os til følgende definition:
\begin{definition}
    En \textbf{region} er en sammenhængende, ensfarvet gruppe pixels i
    et billede.
    \label{def_region}
\end{definition}
Egentlig har vi også fire små regioner, i hjørnerne af figur
\ref{billede_pixels}, hver enkelt på én pixel. I praksis tillades dog en
vis afvigelse i farven. Alt efter hvordan man definerer afvigelsen, kan
man da få regioner ud fra digitale gengivelser af malerier såsom en
himmel, i et maleri af et landskab, eller et ansigt.  En region kan
altså være hvad som helst i billedet.

\subsection{Antagelser}
Det er meget svært at opstille nogen regler for, hvad der er interessant
i et arbitrært maleri. Vi må dog antage, at det, som af kunstneren
betragtes som interessant, fremstår tydeligt i billedet. At noget
fremstår tydeligt betyder, at den region, som udgør det interessante
område, er klart afgrænset i maleriet. Endvidere antager vi, at den
interessante region har en vis størrelse.

}

% vim: set tw=72 spell spelllang=da:
