{
Udtrykket \emph{topografisk kort} er hentet fra geografi og er en
fremstilling af terrænforskelle i et landskab. Traditionelle
topografiske kort bruger forskellige farver til at repræsentere de
relative højder over havet. Vi kan bruge denne metode og udskifte højden
med en omkostning.  Et simpelt eksempel på et topografisk kort er vist i
figur \ref{topography_pixel}.

\begin{figure}[!h]
    \begin{center}
        \includegraphics[width=0.8\textwidth]{afsnit/vores_implementation/billeder/udvidet_loesning/topographic_pixelize.png}
    \end{center}
    \caption[]{Simpelt topografisk kort med dimensionerne $20 \times
    31$. Hver pixel har en værdi angivet ved mængden af hvid farve. Helt
    hvid er dyrest, mens sort er billigst.}
    \label{topography_pixel}
\end{figure}

Vi vil nu beskrive en metode, hvorved regioner bliver tildelt en
omkostning beregnet ud fra dens placering og udstrækning på baggrund af
et topografisk kort. Således vil regioner ikke blive vurderet efter,
hvorvidt de ligger i det gyldne snit eller ej, men efter \emph{hvor
meget} de ligger i det gyldne snit.

\subsubsection{Generering af topografiske kort}
Givet en afbildning af et maleri ved matricen $\mathbf{I}$ med
dimensioner $N \times{} M$, kan man opstille et topografisk kort
$\mathbf{T}$ med dimensioner $N \times{} M$. Det topografiske kort
genereres ud fra to vektorer $\mathbf{X} = \left(x_1, x_2, \cdots,
x_n\right)$ og $\mathbf{Y} = \left(y_1, y_2, \cdots, y_m\right)$ med
dimensioner på henholdsvis $N$ og $M$.

\begin{figure}[!h]
    \centering
    \begin{picture}(240,30)
        \put(0, 10){$A$}
        \put(3, -5){\line(0, 1){10}}

        \put(116, 10){$p$}
        \put(118, -5){\line(0, 1){10}}

        \put(131, 10){$m$}
        \put(134, -4){\line(0, 1){8}}

        \put(144, 10){$G$}
        \put(147, -4){\line(0, 1){8}}

        \put(157, 10){$m'$}
        \put(160, -4){\line(0, 1){8}}

        \put(195, 10){$q$}
        \put(198, -4){\line(0, 1){8}}

        \put(233, 10){$B$}
        \put(236, -5){\line(0, 1){10}}

        \put(182, 10){$x$}
        \put(185, 0){\circle*{3}}

        \put(3, 0){\line(1, 0){233}}
    \end{picture}
    \caption[]{Liniestykket repræsenteret af vektoren $X$.}
    \label{topograph_line}
\end{figure}

Vi betragter vektoren $\mathbf{X}$ som linjestykket givet ved $AB$ i
figur \ref{topograph_line}. Længden af linjestykket betegnes ved $|AB|$.
Det følger af definitionen på $\mathbf{X}$, at $|AB| = n$. På
linjestykket $AB$ er det gyldne snit placeret ved $G$, hvilket betegner
indeks $\lfloor n\varPhi \rfloor$ i $\mathbf{X}$. Et margin er angivet
ved punkterne $(G - \delta) = m$ og $(G + \delta) = m'$, hvor $\delta$
er størrelsen på margin. Ligeledes betegner $m$ og $m'$ indeks $\lfloor
n \varPhi \pm \delta \rfloor$ i $\mathbf{X}$.  Endvidere har vi at $|Ap|
= \lfloor \frac{1}{2}|AB| \rfloor \leq |pB|$ og $|m'q| = \lfloor
\frac{1}{2}|qB| \rfloor \leq |qB|$. I figur \ref{topograph_line} er kun
linjestykket $pB$ segmenteret, men $Ap$ segmenteres symmetrisk.

Vi placerer nu et nyt punkt $x$ på $pB$. I det et-dimensionelle plan kan
længden $|Gx|$ bruges som mål for, hvor tæt $x$ er på det gyldne snit.
Punktet $x$ har dog ingen udstrækning, hvorfor vi ikke blot kan bruge
længden i praksis. Vi betragter nu en region $R \in \mathbb{Z}^{+}$.
Regionen $R$ er et linjestykke i det et-dimensionelle plan. Vi kan da
udregne placeringen af $R$, i forhold til punktet $G$, ved at summere
alle afstandene fra punkterne $x$ i $R$ til $G$.  Dette medfører, at
lange linjestykker bliver tildelt højere værdi end små. Vi udregner
således $\frac{\sum_{x \in R}{|Gx|}}{|R|} = |G(\frac{|R|}{2})|$, hvor
$|R|$ er længden, dvs. antallet af punkter i $R$. Dette svarer til at
beregne afstanden fra regionens midtpunkt til $G$, hvilket ikke er
ønskværdigt, da vi kan have to regioner med forskellig længde, men med
samme midtpunkt.  Hvis vi lader $R_{max}$ betegne den større region og
$R_{min}$ den mindre, hvor $ \frac{|R_{max}|}{2} = \frac{|R_{min}|}{2}$,
da må $R_{max}$ nødvendigvis have et ekstrema tættere på $G$ end
$R_{min}$.  Det er derfor ikke retfærdigt at give begge regioner den
samme værdi.

Vi ønsker at belønne punkter, der ligger i eller tæt ved det gyldne
snit, men give stor omkostning til punkter, som ikke ligger i det gyldne
snit.  Til dette bruges vektoren $\mathbf{X}$, der angiver omkostningen
for hvert punkt på linjestykket $AB$. Da vi ønsker at belønne punkter i
det gyldne snit, tildeles disse ingen omkostning. Vi sætter derfor
$\mathbf{X}_{|AG|}$ til $0$. Vi ønsker heller ikke at straffe regioner,
som ligger inden for margin alt for meget, og vi sætter derfor
$\mathbf{X}_{|Am|} = \mathbf{X}_{|Am'|} = 1$, hvilket angiver
omkostningen for at ligge på margin. Værdierne, mellem det gyldne snit
og margin, interpoleres således, at vi har en lineær overgang. Passende
værdier vælges til $\mathbf{X}_{|Ap|}$, $\mathbf{X}_{|Aq|}$ og
$\mathbf{X}_{0} = \mathbf{X}_{|AB|}$, hvor der ligeledes interpoleres
mellem punkterne.  Liniestykket bliver således delt ind i nogle
sektorer, som har en vis omkostning alt efter hvor tæt, man ligger på
snittet. Derved kan vi undgå ovenstående eksempel, hvor to regioner
gives samme omkostning på trods af, at de har forskellig størrelse.

\paragraph{Omkostningsfunktionen}
Vi vil nu overføre det ovenstående til to dimensioner, hvor det er let
at opdele vektoren $\mathbf{Y}$ på samme måde som $\mathbf{X}$. Vi
ønsker at beregne en omkostning ud fra værdierne i vektorerne ved en
given koordinat, og til dette defineres en funktion $t : \mathbb{Z}^{+}
\times \mathbb{Z}^{+} \rightarrow \mathbb{R}_{0}$ ved
\begin{equation}
    t(x, y) = \mathbf{X}_x + \mathbf{Y}_y
    \label{topo_plus}
\end{equation}
Det topografiske kort, som angivet ved funktionen \ref{topo_plus}, kan
ses i figur \ref{topography_plus}, hvor omkostningerne er illustreret
ved mængden af hvid farve. Kortet viser, at der ikke er nogen omkostning
i punktet $(G_{x}, G_{y})$. Endvidere ses det, at omkostningen er høj
for punkter ved hjørnerne og ved kanterne generelt. Dog kan man ret
tydeligt se grænserne mellem regionerne, da der interpoleres lineært.

Vi kan lave en mere flydende overgang mellem regionerne ved at definere
en ny funktion $u : \mathbb{Z}^{+} \times \mathbb{Z}^{+} \rightarrow
\mathbb{R}_{0}$ ved
\begin{equation}
    u(x, y) = \mathbf{X}_x\mathbf{Y}_y
    \label{topo_multiply}
\end{equation}
hvor vi multiplicerer omkostningerne i stedet for at addere dem. Det
resulterende topografiske kort ses i figur \ref{topography_times}. Da vi
nu bruger multiplikation vil vi gange med $0$ i det gyldne snit, hvorfor
dette er meget mere fremstående. Det skal dog nævnes, at det ikke er
helt hensigtsmæssigt, at omkostningen er meget lav ude ved ekstremerne.
Funktionen $u$ mangler altså den gode egenskab fra $t$, hvor ekstremerne
har store omkostninger.  Vi ser dog en eksponentiel stigning i
omkostningerne, når vi bevæger os længere væk fra det gyldne snit.
Optimalt ville man have den samme omkostning ved ekstremerne som i $t$,
men med den eksponentielle stigning som i $u$.

Vi kan nu beregne omkostningen for en interessant region. Givet en
mængde $R \in \mathbb{Z}^{+2}$, som indeholder punkterne regionen, kan
man finde omkostningen $C$ ved
\begin{equation}
    C(R) = \sum_{(x, y) \in R}{\frac{\tau(x, y)}{|R|}}
\end{equation}
hvor $\tau$ er en funktion, fra $\mathbb{Z}^{+}\times\mathbb{Z}^{+}$ ind
i $\mathbb{R}_0$, som beregner omkostningen for et punkt ud fra et
topografisk kort. Vi har foreslået funktionerne $t$ og $u$. Jo lavere
omkostning en region har, jo bedre er den placeret i forhold til det
gyldne snit. I praksis er det oplagt at approksimere regionens størrelse
og udstrækning, ved at bruge et gitter, ligesom i det foregående afsnit.

\begin{figure}[h]
    \setlength\fboxsep{0pt}
    \setlength\fboxrule{0.5pt}
    \begin{center}
        \fbox{\includegraphics[width=0.8\textwidth]{afsnit/vores_implementation/billeder/udvidet_loesning/topographic_plus.png}}
    \end{center}
    \caption[]{Topografisk kort angivet ved funktionen $t(x, y) =
    \mathbf{X}_x + \mathbf{Y}_y$. Mængden af hvid farve reflekterer
    omkostningen.}
    \label{topography_plus}
\end{figure}

\begin{figure}[h]
    \setlength\fboxsep{0pt}
    \setlength\fboxrule{0.5pt}
    \begin{center}
        \fbox{\includegraphics[width=0.8\textwidth]{afsnit/vores_implementation/billeder/udvidet_loesning/topographic_times.png}}
    \end{center}
    \caption[]{Topografisk kort angivet ved funktionen $u(x, y) =
    \mathbf{X}_x\mathbf{Y}_y$. Vi har igen, at hvid farve reflekterer
    omkostningen. Bemærk hvordan det gyldne snit er fremhævet.}
    \label{topography_times}
\end{figure}

\subsubsection{Fastsættelse af omkostninger}
Det er let at se, at udformningen af det topografiske kort ikke blot
afhænger af funktionen $\tau$, men snarere af de værdier, man tildeler
omkostningsvektorerne. Værdierne, som er tildelt i det ovenstående, er
valgt arbitrært efter det bedste grafiske resultat. Vi har tidligere
nævnt, at hvis værdierne interpoleres lineært, er det ikke helt
hensigtsmæssigt at addere dem. Det er dog oplagt at gøre brug af
normalfordelingen til at fastsætte omkostninger, men dette skal ses med
omvendt fortegn, hvilket betyder, at vi nu tildeler regioner points. Man
kan altså basere antallet af points for et punkt i det et-dimensionelle
plan, ved at bruge en normalfordeling $N(\mu, \sigma)$ med middelværdi
$n\varPhi$ og en passende varians til margin. Umiddelbart ville man
sætte standardafvigelsen til $\delta$, vores margin, hvilket giver os en
varians på $\sigma^{2} = \delta^{2}$. Dette giver os en meget spids
normalfordeling, hvor punkter bliver tildelt mange points for at ligge
inden for margin, men hvor antal points hurtigt aftager, når man bevæger
sig længere væk.

%\subsubsection*{Fordele og ulemper}
%Brug af omkostninger ved topografiske kort har den umiddelbare fordel at
%kunne tildele regioner en værdi og derved være mere nuanceret i
%bedømmelsen af regioner. Metoden afhænger dog af fornuftige værdier i
%omkostningsvektorerne.

}

% vim: set tw=72 spell spelllang=da:
