% Denne fil er inkluderet i udtraekning_af_regioner.tex
{
Sløring, som kommer fra det engelske ord ``blur'', er en gruppering af
filtre som bruges til at fjerne støj og uregelmæssigheder i billeder. Vi
så i afsnit \ref{subsec_floodfill}, at metoden floodfill nogle gange kan
have svært ved at fylde hele regionen ud. Specielt i figur
\ref{dot_ff_var_7_7} ses at himlen har små huller. En sløring af
billedet kan hjælpe med at glatte farverne ud, således at vi dækker mere
af regionen. Sløring af billedet kan også hjælpe til at fjerne diverse
artefakter, såsom revner eller pletter i billedet. Specielt i
vores testbillede, der som tidligere nævnt er malet med en masse
prikker, er det en stor hjælp at sløre billedet, så farverne bliver mere
ensartede. Vi vil nu se på tre forskellige måder at opnå dette på.

De to første et såkaldte lav-pas filtre som har svært ved at bibeholde
kanterne i et billede, men arbejder til gengæld direkte på billedet. Den
tredie metode bibeholder til en vis grad kanterne bedre, men kan ikke
arbejde direkte på billedet, hvilket kræver et større pladsforbrug.

\subsubsection*{Simpel sløring}

\subsubsection*{Gaussisk sløring}
Lav-pas filter. Hjælp.

\subsubsection*{Sløring ved statistisk median}
Den sidste metode vi vil nævne er i grunden meget simpel. Grundidéen er
at finde den statistiske median i pixelværdierne rundt om en givet pixel
og tildele medianværdien til denne. Givet et antal pixels, er det
trivielt at sætte dem i en liste og sortere dem efter deres værdi. Hvis
antallet af elementer i listen er ulige, er det midterste element i den
sorterede liste medianen. Er der et lige antal elementer i listen,
defineres medianen som gennemsnittet af de to midterste elementer.
Pixels vælges i et $N \times M$ vindue med den originale pixel i
centrum, som vist i figur \ref{red_box_nxm} og vi vil således altid have
et ulige antal elementer i listen.

\begin{figure}[!h]
    \centering
    \subfloat[]{\label{red_box_nxm}
        \includegraphics[scale=0.42,angle=0]{afsnit/vores_implementation/billeder/sloering/red_pixel_box}
    }\\
    \subfloat[]{\label{3_3_vindue}
        \renewcommand{\arraystretch}{1.8}
        \begin{tabular}{|c|c|c|}
            \hline
            35 & 98  & 23 \\\hline
            48 & \cellcolor[gray]{0.5}42 & 0 \\\hline
            8  & 12   & 29 \\\hline
        \end{tabular}
        }\hspace{1em}
    \subfloat[]{\label{sorteret_median}
        \renewcommand{\arraystretch}{1.5}
        \centering
        \begin{tabular}{|c|c|c|c|c|c|c|c|c|}
            \hline
            0 & 8 & 12 & 23 & \cellcolor[gray]{0.5}29 & 35 & 42 & 48 & 98\\\hline
        \end{tabular}
        }
        \caption[]{
            Bestemmelse af median for pixel med koordinaterne $(2, 2)$.
            \textbf{\ref{red_box_nxm})} Pixels i et $3\times3$ vindue
            omkring $(2, 2)$ er markeret med rødt.
            \textbf{\ref{3_3_vindue})} Værdierne i $3\times3$ vinduet.
            Det ses den originale pixel har værdien $42$.
            \textbf{\ref{sorteret_median})} Den sorterede liste med
            værdierne fra vinduet. Det ses at medianen har værdien
            $29$. Den originale pixel vil da skifte værdi fra $42$ til
            $29$.
        }
\end{figure}

Denne metode kan ikke køres direkte på det originale billede, da dette
vil interferere med fastsættelse af medianen for alle pixels. Man må
derfor oprette en kopi af det originale billede og sætte de fundne
medianværdier i denne. Man finder således altid medianen i forhold til det
originale billede.

\subsubsection*{Eksempler}

% Hold on, this is figure-madness
\begin{figure}[!h]
    \centering
    \subfloat[Original]{\label{simple_original}\includegraphics[angle=0,width=0.3\textwidth]{afsnit/vores_implementation/billeder/sloering/original}}\hspace{1em}
    \subfloat[$3 \times 3$ vindue]{\label{simple_3_3}\includegraphics[angle=0,width=0.3\textwidth]{afsnit/vores_implementation/billeder/sloering/simple_3_3}}\hspace{1em}
    \subfloat[$7 \times 7$ vindue]{\label{simple_7_7}\includegraphics[angle=0,width=0.3\textwidth]{afsnit/vores_implementation/billeder/sloering/simple_7_7}}
    \caption[]{
        \textbf{\ref{simple_original})} Zoom af detajler i det originale billede.
        \textbf{\ref{simple_3_3})}
        \textbf{\ref{simple_7_7})}
    }
    \label{simple_metode}
\end{figure}

\begin{figure}[!h]
    \centering
    \subfloat[Original]{\label{gaussian_original}\includegraphics[angle=0,width=0.3\textwidth]{afsnit/vores_implementation/billeder/sloering/original}}\hspace{1em}
    \subfloat[$3 \times 3$ vindue]{\label{gaussian_3_3}\includegraphics[angle=0,width=0.3\textwidth]{afsnit/vores_implementation/billeder/sloering/gaussian_3_3}}\hspace{1em}
    \subfloat[$7 \times 7$ vindue]{\label{gaussian_7_7}\includegraphics[angle=0,width=0.3\textwidth]{afsnit/vores_implementation/billeder/sloering/gaussian_7_7}}
    \caption[]{
        \textbf{\ref{gaussian_original})} Zoom af detajler i det originale billede.
        \textbf{\ref{gaussian_3_3})}
        \textbf{\ref{gaussian_7_7})}
    }
    \label{gaussian_metode}
\end{figure}

\begin{figure}[!h]
    \centering
    \subfloat[Original]{\label{median_original}\includegraphics[angle=0,width=0.3\textwidth]{afsnit/vores_implementation/billeder/sloering/original}}\hspace{1em}
    \subfloat[$3 \times 3$ vindue]{\label{median_3_3}\includegraphics[angle=0,width=0.3\textwidth]{afsnit/vores_implementation/billeder/sloering/median_3_3}}\hspace{1em}
    \subfloat[$7 \times 7$ vindue]{\label{median_7_7}\includegraphics[angle=0,width=0.3\textwidth]{afsnit/vores_implementation/billeder/sloering/median_7_7}}
    \caption[]{
        \textbf{\ref{median_original})} Zoom af detajler i det originale billede.
        \textbf{\ref{median_3_3})} Median med et vindue på $3\times{}3$.
        Farverne er blevet mere ensartede mens kanterne stadig er
        skarpe.
        \textbf{\ref{median_7_7})} Median med et vindue på $7\times{}7$. Farverne
        er meget ensartede, men det ses at kanterne er blevet mere
        udvisket med det større vindue.
    }
    \label{median_metode}
\end{figure}

}

% vim: set tw=72 spell spelllang=da:
