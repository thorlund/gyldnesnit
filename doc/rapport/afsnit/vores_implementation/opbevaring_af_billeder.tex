{
En central del af den automatiserede analyse af malerier er den database
hvor vi opbevarer metadata og resultater. Vi vil til at starte med kun
koncentrere os om maleriernes metadata og først i afsnit
\ref{section_results} se på hvordan vi i databasen opbevarer resultater.
Dette gøres fordi vi endnu ikke har været inde over hvilke resultater
der kunne være interessante at gemme.

Vi bruger SQLite til selve databasen, hovedsageligt fordi der ikke kræves
nogen videre konfiguration af en sådan database. Den underliggende
database er dog underordnet, da vi bruger Python-pakken \emph{SQLObject}
som giver et abstraktionslag til en bred vifte af databaser. Vi opretter
blot de tabeller vi ønsker at have i databasen som klasser gennem
Python. Ligeledes får vi en sådan klasse tilbage når der laves
forespørgsler til databasen. Da \emph{SQLObject} klarer al kommunikation
med databasen er det derfor muligt at skifte den underliggende database
ud hvis man ønsker det. Vi ser dog ikke nogen grund til at bruge en
anden foreløbig, da SQLite opfylder vores behov. Endvidere har SQLite
den umiddelbare fordel at selve databasen lægges i en fil i filsystemet.
Det er derfor en let sag at tage sikkerhedskopier af databasen uden alt
for meget besvær.

Vores korpus består af billeder hentet fra hjemmesiden \cite{wgahu} som
indeholder europæiske kunstartikler fra år 1001 -- 1900. I
kunstartiklerne, hvor det samlede antal er omkring 23.000, indgår
møbler, skulpturer, mosaikker og malerier, hvor sidstnævnte vil være
vores fokus. Fra \cite{wgahu} tilbydes en kommasepareret fil over hele
deres database som vi har brugt til at populere vores egen database med.
Der gives mange oplysninger om den enkelte artikel samt dennes kunstner.
Vi har konstrueret en parser som trækker disse informationer ud fra
filen og lægger dem ind i tabellerne \ref{artistTable0} og
\ref{paintingTable0} i databasen. Da vi primært vil beskæftige os med
malerier vil vi nu blot omtale kunstartikler som malerier.

\begin{table}[!h]
    \centering
    \begin{tabular}{|l||c|c|c|c|c|c|}
        \hline
        \bf{artist} \hspace{0.5cm} & \underline{artistId} & name & born & died & school & timeline \\\hline
    \end{tabular}
    \caption{Databasetabel for kunstner}
    \label{artistTable0}
\end{table}

\begin{table}[!h]
    \centering
    \begin{tabular}{|l||c|c|c|c|c|c}
        \hline
        \bf{painting} \hspace{0.5cm} & \underline{paintingId} & artistId & title & date & paint & $\cdots$ \\\hline
    \end{tabular}\\ \vspace{0.2cm}\hspace{1.2cm}
    \begin{tabular}{c|c|c|c|c|c|c}
        \hline
        $\cdots$ & material & location & url & form & type & $\cdots$ \\\hline
    \end{tabular}\\ \vspace{0.2cm}\hspace{1.4cm}
    \begin{tabular}{c|c|c|c|c|c|}
        \hline
        $\cdots$ & realHeight & realWidth & height & width & filepath \\\hline
    \end{tabular}
    \caption{Databasetabel for malerier}
    \label{paintingTable0}
\end{table}

Ovenstående databaseskema lægger vægt på at vi let kan forespørge
databasen ved en lang række parametre, såsom et maleris fysiske
størrelse eller kunstneres fødselsår. Vi har, at en der til en kunstner
kan være tilknyttet en række malerier og at der til et givet maleri kun
kan være én kunstner. I den kommaseparerede fil fra \cite{wgahu} er
malerier opstillet efter kunstneren, hvilket gør det let først at
oprette denne i databasen og derefter oprette de efterfølgende malerier
tilknyttet denne kunstner.

Den konstruerede parser til den kommaseparerede fil er dog ret grov, da
folkene bag \cite{wgahu} ikke har lagt meget vægt på at være konsistente
i deres formulering af en kunstners fødsels- og dødsår eller en
genstands dimensioner. En følge deraf er, at nogle kunstnere, hvor
\cite{wgahu} ikke har en klar indikation af dennes levealder, ikke bliver
registreret i databasen. Vi kan dog stadig slå kunstneren op ved at
bruge feltet ``timeline'' i tabel \ref{artistTable0} som angiver hvilken
periode kunstneren tilhører. Vi har i enkelte tilfælde set os nødsaget
til at rette i den kommaseparerede fil i tilfælde hvor der er blevet
indsat tegn der helt umuliggør korrekt parsing, såsom ekstra komma
eller semikolon.

Givet den kommaseparerede fil fra \cite{wgahu} er det en smal sag at
konstruere en crawler som henter alle billederne fra hjemmesiden ned. I
filen gives nemlig en henvisning til hvor man kan finde et billede af
genstanden på deres side. Billederne hentes ned og gemmes på filsystemet
i samme mappestruktur som der bruges på \cite{wgahu}. Her indeles
filerne i mapper navngivet efter deres kunstner. Kunstnerenes mapper
inddeles efter forbogstav. På denne måde undgås problemet at to billeder
kan have det samme filnavn. Filstrukturen er grafisk illustreret i figur
\ref{mappestruktur}. Vi gemmer da blot stien til en fil på filsystemet i
databasen.

% Mappestruktur
\begin{figure}[!h]
    \centering
$
\xymatrix{
 &  &   & \ar @{-} [d] \textrm{/res}  &                                                     \\
 &  &   & \ar @{-} [d] \textrm{/wga.hu}  &                                                  \\
 &  &   & \ar @{-} [dl] \ar @{-} [d] \ar @{--} [dr] \textrm{/art} &                         \\
 &  & \ar @{-} [dl] \ar @{-} [d] \ar @{--} [dr] \textrm{/a} & \textrm{/b} & \cdots          \\
 & \ar @{-} [dl] \ar @{--} [d] \textrm{/aachen} & \ar @{--} [d] \textrm{/abadia} & \cdots   \\
\textrm{allegory.jpg} & \textrm{bacchus.jpg} & \cdots &   &
}
$
    \caption{Mappestruktur til filer fra
        \href{http://www.wga.hu}{http://www.wga.hu}}
    \label{mappestruktur}
\end{figure}

Som tidligere nævnt så vil vi i afsnit \ref{section_results} bygge
videre på databasen når vi skal gemme resultater fra en automatiseret
analyse. Indtil videre vil den beskrevne database blot bruges til at
trække filnavne på malerier ud til videre analyse.

}
% vim: set tw=72 spell spelllang=da:
