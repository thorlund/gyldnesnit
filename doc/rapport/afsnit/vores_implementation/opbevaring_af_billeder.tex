% Denne fil bliver ikke brugt mere
{
En central del af den automatiserede analyse af malerier er den database
hvor vi opbevarer metadata og resultater. Vi vil til at starte med kun
at koncentrere os om maleriernes metadata og de egentlige billedfiler. I
afsnit \ref{section_results} vil vi arbejde videre på databaseskemaet og
se på hvordan vi opbevarer resultater i databasen.  Vi benytter os af
databaseskemaet vist i tabel \ref{artistTable0} og \ref{paintingTable0}.

\begin{table}[!h]
    \centering
    \begin{tabular}{|l||c|c|c|c|c|c|}
        \hline
        \bf{artist} \hspace{0.5cm} & \underline{artistId} & name & born & died & school & timeline \\\hline
    \end{tabular}
    \caption{Databasetabel for kunstner}
    \label{artistTable0}
\end{table}

\begin{table}[!h]
    \centering
    \begin{tabular}{|l||c|c|c|c|c|c}
        \hline
        \bf{painting} \hspace{0.5cm} & \underline{paintingId} & artistId & title & date & paint & $\cdots$ \\\hline
    \end{tabular}\\ \vspace{0.2cm}\hspace{1.2cm}
    \begin{tabular}{c|c|c|c|c|c|c}
        \hline
        $\cdots$ & material & location & url & form & type & $\cdots$ \\\hline
    \end{tabular}\\ \vspace{0.2cm}\hspace{1.4cm}
    \begin{tabular}{c|c|c|c|c|c|}
        \hline
        $\cdots$ & realHeight & realWidth & height & width & filepath \\\hline
    \end{tabular}
    \caption{Databasetabel for malerier}
    \label{paintingTable0}
\end{table}

Databaseskemaet lægger vægt på at vi let kan forespørge
databasen ved en lang række parametre, såsom et maleris fysiske
størrelse og kunstneres fødselsår. Vi har, at en der til en kunstner
kan være tilknyttet en række malerier og at der til et givet maleri kun
kan være én kunstner. Billederne som vi vil analysere hentes fra et
online kunstarkiv og gemmes på filsystemet i samme mappestruktur som i
arkivet. Her indeles filerne i mapper navngivet efter deres kunstner.
Kunstnerenes mapper inddeles efter forbogstav. På denne måde undgås
problemet at to billeder kan have det samme filnavn. Filstrukturen er
grafisk illustreret i figur \ref{mappestruktur}. Vi gemmer da blot stien
til en fil på filsystemet i databasen.

% Mappestruktur
\begin{figure}[!h]
    \centering
$
\xymatrix{
 &  &   & \ar @{-} [d] \textrm{/res}  &                                                     \\
 &  &   & \ar @{-} [d] \textrm{/wga.hu}  &                                                  \\
 &  &   & \ar @{-} [dl] \ar @{-} [d] \ar @{--} [dr] \textrm{/art} &                         \\
 &  & \ar @{-} [dl] \ar @{-} [d] \ar @{--} [dr] \textrm{/a} & \textrm{/b} & \cdots          \\
 & \ar @{-} [dl] \ar @{-} [d] \textrm{/aachen} & \ar @{--} [d] \textrm{/abadia} & \cdots    \\
\textrm{allegory.jpg} & \textrm{bacchus.jpg} & \cdots &   &
}
$
    \caption{Mappestruktur til filer fra
        \href{http://www.wga.hu}{http://www.wga.hu}}
    \label{mappestruktur}
\end{figure}

Som tidligere nævnt så vil vi i afsnit \ref{section_results} bygge
videre på databasen når vi skal gemme resultater fra en automatiseret
analyse. Indtil videre vil den beskrevne database blot bruges til at
trække filnavne på malerier ud til videre analyse.

}
% vim: set tw=72 spell spelllang=da:
