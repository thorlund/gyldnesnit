{
{\sffamily Vi vil i dette kapitel se på hvordan vi vil analysere
billeder for tilstedeværelsen af det gyldne snit. Vi vil derfor starte
med at give en kort introduktion til billedbehandling og problemerne der
er tilknyttet. Vi vil her introducere nogle nøglebegreber i forbindelse
med analysen på malerier. Det øvrige afsnit ser således ud:

\paragraph{Afsnit \ref{section_naiv}} Introducerer en simpel
fremgangsmåde til at afgøre om et billede opfylder det gyldne snit.

\paragraph{Afsnit \ref{section_opdeling}} Omhandler problematikken
ved at gå fra virkelige malerier til en digital gengivelse.

\paragraph{Afsnit \ref{section_udtraek}} Beskriver hvordan vi ved brug
af forskellige metoder fra billedbehandling, vil trække regioner ud af
et billede.

\paragraph{Afsnit \ref{section_database}} Gennemgår opbygningen af vores
bagvedliggende database og forklarer tankerne bag.
}

\section{Kort introduktion til billedbehandling\label{section_kort_intro}}
{
{\sffamily Inden for billedbehandling dækker begrebet \emph{feature
detection} over den opgave at detektere \emph{features} i et givet
billede.  \emph{Feature} oversættes bedst med \emph{træk} eller
\emph{kendetegn}, men nøjagtig \emph{hvad} et træk \emph{er}, er ikke
klart defineret og skal tilpasses den enkelte
opgave\cite{WikiFeatureDetection}\cite{SIOlsen}. Et andet område inden
for detektion af træk kaldes for \emph{blob detection}, hvor en
\emph{blob} kort kan beskrives som en ensfarvet region i billedet. Vi
vil derfor fremover referere til en blob som en region, og i det
følgende vil vi komme ind på præcis hvad vi forstår ved en sådan. Da det
imidlertid kun er de interessante regioner, der ønskes udvalgt i
billederne, præsenteres i afsnit \ref{section_naiv}, en simpel
fremgangsmåde, som fortæller os, hvornår en region kan betegnes som
interessant. Vi vil dog først se på, hvordan digitale billeder
repræsenteres. For en mere dybdegående indledning til billedbehandling
henvises til \cite{SIOlsen}.
}

\subsection{Repræsentation af digitale billeder}
Et digitalt billede er, som nævnt i afsnit
\ref{section_computer_betragter}, sammensat af pixels. En pixel er et punkt i et
koordinatsystem, hvortil der er knyttet en værdi. Et gråtonebillede er
givet ved afbildningen
\begin{equation}
    \mathbb{Z}^{+}\times{} \mathbb{Z}^{+} \rightarrow \mathbb{Z}^{+}
\end{equation}
fra to positive heltal over i et positivt heltal.  Funktionen $G(x, y)
\in \mathbb{Z}^{+}$ angiver mængden af hvid farve i en pixel med
koordinaterne $(x, y)$.

Som et simpelt eksempel siger vi nu, at en pixel i et billede kan antage
to værdier: $0$ og $1$. Vi har altså afbildningen $\mathbb{Z}^{+}\times{}
\mathbb{Z}^{+} \rightarrow \{0, 1\}$. En pixel med værdi $0$ vil ikke
blive farvet, dvs. den forbliver sort, mens en pixel med værdien $1$ vil
blive farvet hvid.  Figur \ref{billede_pixels} viser et sådant simpelt,
binært billede.

% Please, PLEASE, do not try this at home!
% This is the UGLY way to tex things :P
\begin{figure}[!h]
    \renewcommand{\arraystretch}{1.5}
    \centering
    \begin{tabular}{cc|c|c|c|}
        % Start crying
           & \multicolumn{4}{c}{\hspace{1.5em}$y$}\\
           & \multicolumn{4}{c}{\hspace{1.6em}0\hspace{1.2em}1\hspace{1.2em}2} \\\cline{3-5}
           &  0 & 1                                     & \cellcolor{black}\textcolor{white}{0} & 1                                     \\\cline{3-5}
      $x$  &  1 & \cellcolor{black}\textcolor{white}{0} & \cellcolor{black}\textcolor{white}{0} & \cellcolor{black}\textcolor{white}{0} \\\cline{3-5}
           &  2 & 1                                     & \cellcolor{black}\textcolor{white}{0} & 1                                     \\\cline{3-5}
    \end{tabular}
    \caption[]{Et simpelt $3 \times 3$ billede vist som pixels.}
    \label{billede_pixels}
\end{figure}

Bemærk, at koordinatsystemet starter i øverste venstre hjørne, med
stigende $x$- og $y$-værdier mod nedre højre hjørne.

Et billede kan da opstilles som en $N \times M$ matrix som vist i
ligning \ref{billede_matrix}.

\begin{equation}
    \mathbf{I} = \left ( \begin{array}{ccc}
        1 & 0 & 1 \\
        0 & 0 & 0 \\
        1 & 0 & 1
    \end{array} \right )
    \label{billede_matrix}
\end{equation}

At bruge billedet som en matrix har nogle beregningsmæssige fordele, men
det falder uden for denne introduktions formål at gå ind i dette.

\subsection{Regioner i vilkårlige malerier}
Givet et digitalt billede, defineres en \textbf{region} som en sammenhængende
gruppe af pixels med samme farve. Billedet i figur \ref{billede_pixels}
vil da have en sort region formet som et kryds.  Egentlig har vi også
fire små regioner i hjørnerne, hver enkelt på én pixel. I praksis
tillades dog en vis afvigelse i farven. Alt efter hvordan man definerer
afvigelsen, kan man da få regioner ud fra digitale gengivelser af
malerier såsom en himmel, i et maleri af et landskab, eller et ansigt.
En region kan altså være hvad som helst i billedet.

\subsection{Antagelser}
Det er meget svært at opstille nogen regler for, hvad der er interessant
i et arbitrært maleri. Vi må dog antage, at det, som af et menneske
betragtes som interessant, fremstår tydeligt i billedet. At noget
fremstår tydeligt betyder, at den region, som udgør det interessante
område, er klart afgrænset i maleriet. Endvidere antager vi, at den
interessante region har en vis størrelse.

}

% vim: set tw=72 spell spelllang=da:


\section{Naiv algoritme\label{section_naiv}}
{
% Image scale
\def\imgscale{0.34}

\textsf{Vi vil i det følgende forklare vores tanker bag en meget simpelt
fremgangsmåde, som har til formål at afgøre om et billede opfylder det
gyldne snit. For at afgøre dette trækker vi regioner ud af billeder og
vurderer dem nu efter deres placering, størrelse og form. Her antager
vi, at vi allerede har trukket regionerne ud, men afsnit
\ref{section_udtraek} vil forklare hvordan dette foregår. Vi siger at et
billede opfylder det gyldne snit, hvis en eller flere interessante
regioner kan siges at ligge i det gyldne snit.  I det følgende vil vi se
på hvornår en region ligger i det gyldne snit samt hvornår vi har med en
interessant region at gøre.
}

\subsection{Regionens placering}
Når vi har trukket regioner ud af et billede og vil afgøre om de ligger
i det gyldne snit, er det indlysende at deres placering har afgørende
betydning.  Vi vil nu komme frem til en definition hvorved man kan
afgøre om en region i billedet er placeret i det gyldne snit.

Vi starter med at se på det meget simple tilfælde, hvor en region
åbenlyst ligger placeret i det gyldne snit.  Et sådan eksempel ses i
figur \ref{pos_naiv_1}, hvor regionen vi betragter er farvet sort.  Den
røde linje markerer det gyldne snit.  Dette farveskema vil være
gennemgående i det følgende.  Det ses at regionen nærmest tangerer
linjen.

\begin{figure}[h]
    \begin{center}
        \includegraphics[scale=\imgscale,angle=0]{afsnit/vores_implementation/billeder/naiv_algoritme/naiv_positiv_blob_1}
    \end{center}
    \caption[En positiv region]{En region som tangerer det gyldne snit.
    Denne region er positiv.}
    \label{pos_naiv_1}
\end{figure}

I praksis vil vi dog sjældent have at regioner ligger helt præcist på
snittet.
%På grund af den måde, vi trækker regioner ud af billedet på,
%kan vi ikke være sikre på at regionen er fyldt helt ud til dens kanter,
%fordi der i disse områder stadig er stor overgang i farverne.  Dette
%bevirker at en regioner kan blive repræsenteret som mindre end de
%virkelig er.
Vi har også, at det gyldne snit baserer sig på det
irrationelle tal $\varphi$ og vi kan derfor ikke regne os helt nøjagtig
frem til vores det egentlige snit ligger. Dette vil vi komme nærmere ind
på i næste afsnit. Vi indfører derfor en margen hvori vi vil acceptere
regioner.  I praksis betyder det at vi \emph{ikke kun} kigger på den
linje der deler billedet ved det gyldne snit, men faktisk tager vi et
bånd, sammensat af en række snit, og bruger dette bånd som et bredere
gyldent snit.  Derved behøver en region ikke at tangere det gyldne snit
helt nøjagtig for at kunne betragtes som liggende i det gyldne snit. For
at give et eksempel, så vil regionen i figur \ref{pos_naiv_margin_1}
anses som værende placeret i det gyldne snit.  Vær opmærksom på at vores
bånd, rent visuelt i de følgende illustrationer, kan være stærkt
overdrevet.
\begin{figure}[h]
    \begin{center}
        \includegraphics[scale=\imgscale,angle=0]{afsnit/vores_implementation/billeder/naiv_algoritme/naiv_positiv_blob_margin_1}
    \end{center}
    \caption[Positiv region i margen]{En region som
    ligger indenfor en margen af det gyldne snit betragtes som
    positiv. De blå stiplede linjer angiver vores bånd.}
    \label{pos_naiv_margin_1}
\end{figure}

Vi ser nu på det rektangel der der begrænser en region.  En side af
dette rektangel vil vi kalde for en kant. Ved at studere figur
\ref{bbox_section} ses det let, at en region skal have mindst én kant
indenfor båndet om det gyldne snit, før vi kan sige at regionen ligger i
det gyldne snit. Dette betyder, at hvis en region har en kant indenfor
båndet, så ligger denne region i det gyldne snit.

\begin{figure}[h]
    \begin{center}
        \includegraphics[scale=\imgscale,angle=0]{afsnit/vores_implementation/billeder/naiv_algoritme/bbox_section}
    \end{center}
    \caption[Afgrænsende rektangler]{Den øverste region kan ikke
    siges at ligge i det gyldne snit, da den ikke har nogen kanter
    indenfor båndet. Den nederste region derimod, har én kant
    indenfor båndet og ligger derfor i det gyldne snit.}
    \label{bbox_section}
\end{figure}

\begin{figure}[!h]
    \begin{center}
        \includegraphics[scale=0.42,angle=0]{afsnit/vores_implementation/billeder/naiv_algoritme/bathers_mockup_blob}
    \end{center}
    \caption[Interessante regioner i praksis]{I praksis vil de
    fundne regioner være langt mere komplekse.}
    \label{realworld_example}
\end{figure}

Med ovenstående definition i tankerne kigger vi nu på hvordan et rigtigt
billede kan tage sig ud når vi vil finde regioner. I figur
\ref{realworld_example} ser vi, at der i dette tilfælde vil blive
udvalgt mange små regioner, som egentlig ikke kan tillægges nogen
betydning.  Vi vil nu gå videre og opsætte nogle kriterier for hvornår
en region er interessant.

\subsection{Regionens størrelse}
Når vi skal til at afgøre hvorvidt en region er interessant er det
oplagt at se på størrelsen af regionen.  Vi siger derfor at en region
skal have en vis størrelse før den kan betragtes som værende
interessant. I praksis vil regionens areal afspejle dens størrelse, hvor
arealet er det antal pixels regionen optager i billedet.  Grænsen for et
acceptabelt areal skal sættes i forhold til billedets størrelse.

Omvendt er vi heller ikke interesseret i at få for store regioner med i
betragtningerne.  F.eks. vil en himmel i et maleri udgøre en ret stor
sammehængende region.  De fleste af sådanne regioner vil dog ikke blive
taget i betragtning fordi de krydser snittet.  Hvis vi kigger på det
meget simple billede i figur \ref{pos_naiv_1} skal man også huske på at
der faktisk vises to regioner. Den sorte skikkelse er en region ligesom
den hvide baggrund er en region.  Baggrunden kan ikke siges at være en
interessant region, hvorfor det er vigtigt at sortere disse regioner
fra.

Indtil videre har vi kun illustreret ét snit i billedet.  Der er
imidlertid tre andre snit hvor vi også kan dele et billede efter det
gyldne snit.  Hvor vi har kigget på et vertikalt snit, vender vi nu
opmærksomheden mod et horisontalt snit.  Det generelle tilfælde vil være
at en region kun er interessant i forhold til enten et vertikalt eller
et horisontalt snit.  Et eksempel på dette ses i figur
\ref{pos_horiz_naiv_margin_1}.  Der kan dog forekomme specialtilfælde,
hvor en region vil være positiv i flere snit, hvilket vi vil komme ind
på i et senere kapitel.
\begin{figure}[H]
    \begin{center}
        \includegraphics[scale=\imgscale,angle=0]{afsnit/vores_implementation/billeder/naiv_algoritme/naiv_horiz_positiv_blob_1}
    \end{center}
    \caption[Positiv horisontal region]{En positiv region med en
    kant i et horisontalt bånd.}
    \label{pos_horiz_naiv_margin_1}
\end{figure}
Det ses tydeligt at regionen i figur \ref{pos_horiz_naiv_margin_1} ikke
kan tages i betragtning i forhold til et vertikalt snit, da regionen,
uanset snittets placering, vil krydse dette.  Det ses dog at regionen
har en kant indeni et horisontalt bånd og derfor kan klassificeres som
en region der ligger i det gyldne snit, dog i forhold til et horizontalt
bånd.

\subsection{Regionens form}
Regionens form kan også give informationer om hvorvidt vi har med en
interessant region at gøre.  I praksis er det dog meget svært at sige
noget om selve den fysiske form af en region, men vi kan sige noget om
dens masse.  En regions masse skal forstås som forholdet mellem
regionens areal og arealet af det rektangel der afgrænser regionen.
Dette forhold giver information om hvor massiv en region er.  En massiv
region vil være mere interessant end en meget spinkel region.  Figur
\ref{region_mass} illustrerer to forskellige regioner med forskellig
masse.
\begin{figure}[h]
    \begin{center}
        \includegraphics[scale=\imgscale,angle=0]{afsnit/vores_implementation/billeder/naiv_algoritme/bbox_area_ratio}
    \end{center}
    \caption[Regioners masse]{To forskellige regioner med vidt forskellige forhold
    mellem selve regionens areal og arealet af det rektangel der
    afgrænser regionen.}
    \label{region_mass}
\end{figure}
I praksis vil der blive udregnet en tærskelværdi i forhold til et
billedes størrelse for hvor massiv en region skal være for at kunne
karakteriseres som interessant.

\subsection{Sammenfatning af betingelser}
Vi samler nu op på de ovenstående krav for bestemmelse af hvornår en
region er en interessant og hvornår en region ligger i det
gyldne snit.

\noindent For at en region kan betegnes som \textbf{interessant} skal den
\begin{enumerate}
        \renewcommand{\labelenumi}{(\alph{enumi})}
    \item have et areal større end en tærskelværdi der sættes i
        forhold til billedets størrelse
    \item have en masse større end en variabel tærskelværdi.
\end{enumerate}
og før en region kan betragtes som \textbf{liggende i det gyldne
snit} skal den
\begin{enumerate}
        \renewcommand{\labelenumi}{(\alph{enumi})}
        \setcounter{enumi}{2}
    \item have en kant i båndet om det gyldne snit
\end{enumerate}

Bemærk, at kun interessante regioner vil blive taget i betragtning, når vi vil
afgøre om der ligger regioner i det gyldne snit. Dette betyder også at
en region godt kan blive klassificeret som interessant, men ikke
nødvendigvis liggende i det gyldne snit.

Et billede opfylder altså det gyldne snit hvis vi har en eller flere
interessante regioner der ligger i det gyldne snit.  Det er oplagt at
give regioner som er positiv i flere snit større vægt, hvilket vi også
vil komme ind på i et senere kapitel. Bemærk, at selvom ovenstående
fremgangsmåde tager udgangspunkt i det gyldne snit, så kan metoden
anvendes på ethvert snit i billedet.

I kapitel \ref{chap_afproevning} vil vi komme nærmere ind på hvad
tærskelværdierne i praksis skal sættes til.

\subsection{Begrænsninger}
Denne naive tilgang har nogle begrænsninger.  Den mest åbenlyse er, at
interessante regioner med symmetriakse i det gyldne snit, men med kanter udenfor
båndet, ikke vil blive karakterisseret som liggende i det gyldne snit.
Regioner hvor et segment af denne faktisk ligger i båndet, som den
øverste region i figur \ref{bbox_section}, hvor den vertikale del faktisk
ligger i snittet, vil ikke blive udvalgt.  Det kræver en videre
segmentering af de fundne regioner før at sådanne regioner vil blive
udvalgt.

}

% vim: set tw=72 spell spelllang=da:


\section{Opdeling af billeder\label{section_opdeling}}
\subsection*{Trunkerings- og afrundingsproblemer}
Mange af de metoder, vi bruger til at udregne, det gyldne
snits position eller størrelsen af en margin, udregnes med brøker. Hvorimod et
billede opbygges af pixels. Dette gør, at vi bliver nødt til at tage
approksimationer af udregningerne for at få dem tilbage til et helt tal.
Men hvor meget af data er gået tabt, og hvor mange af resultaterne kan
man stole på?

\subsubsection{Acceptabel afvigelse}
\note{Nogle referanse}Som beskrevet i afsnit \ref{mange_tal}, udregnes
det gyldne snit med mange decimaler. En kunstner, hvor god han end er,
har ingen chance for at male så præcist at man kan sige at strøget
ligger nøjagtigt oven på snittet selv om hans ententioner er at ramme
snittet. Vi kommer derfor til at have en vis uprecis hed på de data vi
få fra billedet selv. Vi starter med at se på alle de ting, som kan
skabe en usikkerhed fra malerens side. Man kan gå ud fra at den
procentvise afvigelse ikke er særlig stor, da vi ikke har bestræbelser
på at arbejde på abstrakt malerier, dog har vi sat den procentvise
afvigelse til 0.5 \%. Det vil sige at en maler med et lærred på 100
cm,maksimalt vil male $0.5$ cm forkert.

Når maleren vælger en ramme og et lærred, har vi igen problematikken,
selv om maleren spicifikt gå efter at bygge maleriet op efter det gyldne
snit, kan snittets placering i maleriet have forskubbet sig, ved dårlige
valg at ramme eller lærred. Derfor sætter vi den afvigelse til $1\%$. Da
vi igen mener at dette er den maksimale afvigelse, der kan opstå.

Når maleren maler en region i et maleri, forekommerer der normalt en
lille kant rundt om objektet, et omrids. Dette omrids kan vores
algoritmer ikke tage højde for, og vi må derfor modregne omridset, så vi
er sikre på at vi ser på regionen og ikke dens omrids. Da et omrids ikke
er særligt stort, har vi sat denne procentsats til $0.5\%$. 

Alt i alt giver det en afvigelse på vores aktuelle udtrækning af data
fra malerierne på $2\%$ det vil sige at finder vi en region, som ligger
på pixel 200, i et billedet, der er $500$ cm bredt, befinder den sig
faktisk i intervallet [190,210]. Måden hvorpå vi tager højde for den
forskel, er ved hjælp af marginer, som nævnt i afsnit
\ref{section_naiv}.

\subsection{Inddeling af billede efter snit}

I et billede betegnes højden og bredden som hhv. $H$ og $B$, se figur
\ref{cut}. Der er 4 gyldne snit, 2 vertikale og 2 horisontale, som vist
i figur \ref{lenasnit2}. For at finde ud af, hvor de 4 snit skal ligge i
billedet, multipliceres B og H med $\varPhi$,og man får to tal. Disse
betegner, hvor mange pixels, det gyldne snit befinder sig fra hhv. $H$
og $B$ f.eks vil de 2 horisontale snit, i et billedet, som har $B =
4000$ pixel, ligge hhv. $4000 \cdot \varPhi \approx 2472$ pixels fra
billedets øvre og nedre kant.

\begin{figure}[h]
	\begin{center}
		\includegraphics[scale=0.42,angle=0]{afsnit/vores_implementation/billeder/naiv_algoritme/Lenagolden}
	\end{center}
	\caption[]{Billedet som har indtegnet de fire gyldne snit}
	\label{lenasnit2}
\end{figure}

\begin{figure}[h]
	\begin{center}
		\includegraphics[scale=0.42,angle=0]{afsnit/vores_implementation/billeder/naiv_algoritme/Cut}
	\end{center}
	\caption[]{Billedets højde og bredde betegnes hvv. H og B. De 4 snit er navngivet.}
	\label{cut}
\end{figure}

De 4 snit tildeles hvert deres Id, "snit 0,1,2 og 3" så vi kan kende
forskel på de individuelde snit, Id'erne placering kan ses i figur
\ref{cut}. Vi vil i resten af rapporten kalde snittene efter deres Id.
Hvis vi gerne vil finde snittet som ligger i miden kommer der kun 2
snit, med vær deres Id "snit 0 og 1" som kan ses i figur \ref{Cut2}

\begin{figure}[h]
	\begin{center}
		\includegraphics[scale=0.42,angle=0]{afsnit/vores_implementation/billeder/naiv_algoritme/2Cut}
	\end{center}
	\caption[]{Billedet skæres her kun af 2 snit}
	\label{2Cut}
\end{figure}

\subsubsection{Heltal i det gyldne snit}

I eksemplet med 4000 pixels ovenfor, approksimerer vi antal pixels ved
at afrunde resultatet $2472.13595 \approx 2472$, se udregning
\ref{afrundning}. Det betyder at vi mister 0.13595 pixels i præction,
hvilket svarer til en misvisning af punktet på 0.00339875 $\%$ i forholdt
til $B$ på billedet. Se udregning \ref{afrundning2}.

\begin{equation}
	4000 \cdot \varPhi = 4000(\sqrt{5}-1)/2 = 2472.13595 \approx 2472 \label{afrundning}
\end{equation}

\begin{equation}
	0.13595/4000 \cdot 100 = 0.00339875 \label{afrundning2}
\end{equation}

Det er en meget lille del af selve billedet og skulle ikke give nogle
misvisninger i forhold til udregningen. For at gøre det lidt mere
generelt, sætter vi trunkeringsfejlen til $0.5$, da det er den maksimale
afrundingsfacktor som kan forekomme. Hvis billedet har en størrelse på
500 pixels, hvilket er det mindste billedet vi har, giver dette en fejlmargin
på $0.1 \%$. Dette tal bliver adderet til fejlsatsen ovenfor, og giver
en samlet afvigelse på $2.1\%$.

\subsubsection{Snitratio}
End til vider har vi kun arbejder med det gyldne snit, men andre snit i
billedet kan godt optræde, derfor indføre vi en nu betegnelse
snitratio, som betegner en procents sats for hvor lang inde i billedet
snittet befinder sig. Det vil sige at hvis en snitration er på $0.2$. Et
billedet har $B$ på 4000 vil et snit befinde sig i pixel $4000*0.2 =
800$.

\subsubsection{Heltal ved udregning af Margin}
Når vi har 2 forskellige snitratioer, f.eks. $\varPhi$ og $\frac{2}{3}$,
som ligger meget tæt på hinanden, og vi gerne vil sammenligne hvilken
regioner der ligger i snitratioernens snit, er det vigtigt at margin for
vært af de 2 snitratioers snit ikke krydser hinanden. 

Hvis margin krydser. Vil det indebære, at den samme region bliver fundet
af begge snit. Dette vil give et skævt billedet af forskellen på de to.
Derfor må vi sørge for at marginerne ikke krydser. Hvis $x$ betegner
antal pixels i $B$ eller $H$, og vi vil se på, hvor mange pixels, der er
mellem snitratio $\frac{2}{3}$ og $\varPhi$, multiplicerer vi $x$ med de
to snitratioen for at finde deres placering. Derefter subtraheres vi et
af de snit som befinder sig tetest på hinanden i vær snitratio med
hinanden.

\begin{eqnarray}
	\frac{x2}{3} - \frac{x2}{\sqrt{5}+1} & = & x(\frac{2}{3} - \frac{2}{\sqrt{5} + 1}) \nonumber \\
	& = & x(0.666667-0.618034) \\ \nonumber
	& = & x(0.048633)
\end{eqnarray}

Vi har nu fundet antal pixels mellem de to snit. Vi vil gerne undgå at
de to marginens ikke krydser hinanden, så der dividere vi med 2 og
afrunder værdien.

\begin{equation}
	\left\lfloor \frac{0.048633x}{2}\right\rfloor = \left\lfloor0.024316x \right\rfloor
\end{equation}

Tallet $2.4316$ er altså den minimale procentvise størrelse som vores
margin må have, nå vi sammen liner det gyldne snit og $\frac{2}{3}$.
Det betyder også at vi ikke må sammen line snit som ligger særlig
meget tætter på hinanden, da 0.021 er den minimale procent margin som vi må have.
$\lfloor 0.024316x \rfloor$ giver os et antal pixels som skildre de to snit. For at vise
hvor stort marginen egentlige kan være, bruger jeg denne formel på to
billeder, et som svare til vores mindste billedet, 500 pixels, og et
som svare til vores største billedet, 4000 pixels. Ved 500 pixels
bliver resultatet

\begin{equation}
	 \lfloor 500(0.024316)\rfloor = 12
\end{equation}

Det er en fint margin, da vores fejl på udregningerne ligger på 2.1 \%,
som svare til $\lceil 500*0.021 \rceil = 11$ pixels, som er 1 pixels fra vores
margin

Ved 4000 pixels giver det.

\begin{equation}
	 \left\lfloor 4000(0.024316)\right\rfloor = 97
\end{equation}

Som også er god nok da $4000*0.021 = 84$ pixels.
% vim: set tw=72 spell spelllang=da:


\section{Udtrækning af regioner i et billede\label{section_udtraek}}
{
{\sffamily Dette afsnit har til formål at beskrive hvordan vi trækker
regioner ud af billedet. Kort sagt forsøger vi at segmentere billedet
ved primært ved at bruge en metode kaldet floodfill. Vi præparerer dog
først billedet ved at finde kanter og sløre billedet. Vi vil også komme
ind på hvilke andre metoder vi har forsøgt os med. Først vil vi dog se
på hvilke tanker vi har gjort os om valg af programmeringssprog og
implementationen generelt.
%navnligt den funktion i OpenCV der er buggy, GoodFeaturesToTrack og
%Perona and Malik fra Octave. Det følgende afsnit om Floodfill bør være
%en del af dette afsnit, ligesom vi skal have lignende beskrivelser af
%sløring og kant detektion.
}

\subsection{Kantdetektion}
% Denne fil er inkluderet i udtraekning_af_regioner.tex
{
Beskrivelse af kantdetektion (med billeder).

\subsubsection*{Metode}

\subsubsection*{Eksempler}

\begin{figure}[!h]
    \begin{center}
        \includegraphics[width=0.8\textwidth]{afsnit/vores_implementation/billeder/kantdetektion/canny_20_20}
    \end{center}
    \caption[]{Canny kantdetektion med tærskelværdierne $(20, 20)$.}
    \label{bathers}
\end{figure}
}

% vim: set tw=72 spell spelllang=da:


\subsection{Sløring}
% Denne fil er inkluderet i udtraekning_af_regioner.tex
{
Sløring, som kommer fra det engelske ord ``blur'', er en gruppering af
filtre som bruges til at fjerne støj og uregelmæssigheder i billeder. Vi
så i afsnit \ref{subsec_floodfill}, at metoden floodfill nogle gange kan
have svært ved at fylde hele regionen ud. Specielt i figur
\ref{dot_ff_var_7_7} ses at himlen har små huller. En sløring af
billedet kan hjælpe med at glatte farverne ud, således at vi dækker mere
af regionen. Sløring af billedet kan også hjælpe til at fjerne diverse
artefakter, såsom revner eller pletter i billedet. Specielt i
vores testbillede, der som tidligere nævnt er malet med en masse
prikker, er det en stor hjælp at sløre billedet, så farverne bliver mere
ensartede. Vi vil nu se på tre forskellige måder at opnå dette på.

De to første et såkaldte lav-pas filtre som har svært ved at bibeholde
kanterne i et billede, men arbejder til gengæld direkte på billedet. Den
tredie metode bibeholder til en vis grad kanterne bedre, men kan ikke
arbejde direkte på billedet, hvilket kræver et større pladsforbrug.

\subsubsection*{Simpel sløring}

\subsubsection*{Gaussisk sløring}
Ved Gaussisk sløring ind gør der 3 steps, først at finde en kernel,
multiplicere kernel rigtigt på billedet, kaldet Convolution. Udtrak
værdigerne fra multiplikationen i et nyt billedet, som er det sløret
billedet.

\subsubsection*{Kernel}
En kernel er en lille matrice, som betegner hvor stor del af billedet og
hvor meget af pixels værdiger for farverne rund om et punkt, skal tages
med i udregningen for den nye sløret farve.

\subsubsection*{Convolution}
Convoluting er en simpel matematisk metode, som beskriver hvordan man
kan gange 2 matricer sammen som ikke har sammen mål, men samme
dimission.

\begin{figure}[h]
	\begin{center}
		\includegraphics[scale=1,angle=0]{afsnit/vores_implementation/billeder/sloering/convolution}
	\end{center}
	\caption[]{Et billedet og en kernel}
	\label{Convolution}
\end{figure}

Som man kan se i figur \ref{Convolution} er selve billedet og kernel i
vid forskellige størrelser. Måden Convolution virker på at man
multiplicere kernelen ovenpå de værdiger som ligger rund om den pixel,
som vi gerne vil finde den sløret farve af og tager gennemsnittet af den
værdig, f.eks farven på pixel $(4,4)$, bliver

\begin{equation}
	(4,4) = ((3,3)*4+(4,3)*2+(5,3)*3+(3,4)*3+(4,4)*9+(5,4)*3+(3,5)*2+(4,5)*4+(5,5)*9)*\frac{1}{39} 
\end{equation}

I dette eksempel, er punktet $(5,5)$ og $(4,4)$ vægtet højre i forhold til de andet, så farven vil blive sløret i den retning. Dette skyldes at kernelen er bygget tilfældig op, i en rigtige sløring er kernelen meget specifikt udvalgt for at give det beste resultat.
 
\subsubsection*{Gaussian kernel}
Måde Gaussisk sløring bygger sin kernel op på, at ved brug af Gaussian
fordelings formlen, som bliver brugt utallige steder i den matematiske
verden, som en anderkedt og respektere fordelings metode. XXX(ved ikke
om jeg skal komme ind på selve gaussan fordelige, da dette snilt kan
komme op på den del)


\begin{equation}
	G(x,y) = \frac{1}{2\pi\sigma^2}e^{-\frac{x^2+y^2}{2\sigma^2}}
\end{equation}

Hvis $\sigma = 1$ og vi lader x og y køre fra -2 til 2, med steps size på 1, og multiplicere så den laveste værdig kommer til at være 1 og runder værdien af. Få vi en kernel, se firgur \ref{gauss}.

\begin{figure}[h]
	\begin{center}
		\includegraphics[scale=0.5,angle=0]{afsnit/vores_implementation/billeder/sloering/gauss}
	\end{center}
	\caption[]{Kernel for gauss sløring}
	\label{gauss}
\end{figure}

Denne kernel kan så bruges ved hjælp af concolution til at danne det sløret billet, som vi kan se i figur \ref{gaussian_metode}

\subsubsection*{Sløring ved statistisk median}
Den sidste metode vi vil nævne er i grunden meget simpel. Grundidéen er
at finde den statistiske median i pixelværdierne rundt om en givet pixel
og tildele medianværdien til denne. Givet et antal pixels, er det
trivielt at sætte dem i en liste og sortere dem efter deres værdi. Hvis
antallet af elementer i listen er ulige, er det midterste element i den
sorterede liste medianen. Er der et lige antal elementer i listen,
defineres medianen som gennemsnittet af de to midterste elementer.
Pixels vælges i et $N \times M$ vindue med den originale pixel i
centrum, som vist i figur \ref{red_box_nxm} og vi vil således altid have
et ulige antal elementer i listen.

\begin{figure}[!h]
    \centering
    \subfloat[]{\label{red_box_nxm}
        \includegraphics[scale=0.42,angle=0]{afsnit/vores_implementation/billeder/sloering/red_pixel_box}
    }\\
    \subfloat[]{\label{3_3_vindue}
        \renewcommand{\arraystretch}{1.8}
        \begin{tabular}{|c|c|c|}
            \hline
            35 & 98  & 23 \\\hline
            48 & \cellcolor[gray]{0.5}42 & 0 \\\hline
            8  & 12   & 29 \\\hline
        \end{tabular}
        }\hspace{1em}
    \subfloat[]{\label{sorteret_median}
        \renewcommand{\arraystretch}{1.5}
        \centering
        \begin{tabular}{|c|c|c|c|c|c|c|c|c|}
            \hline
            0 & 8 & 12 & 23 & \cellcolor[gray]{0.5}29 & 35 & 42 & 48 & 98\\\hline
        \end{tabular}
        }
        \caption[]{
            Bestemmelse af median for pixel med koordinaterne $(2, 2)$.
            \textbf{\ref{red_box_nxm})} Pixels i et $3\times3$ vindue
            omkring $(2, 2)$ er markeret med rødt.
            \textbf{\ref{3_3_vindue})} Værdierne i $3\times3$ vinduet.
            Det ses den originale pixel har værdien $42$.
            \textbf{\ref{sorteret_median})} Den sorterede liste med
            værdierne fra vinduet. Det ses at medianen har værdien
            $29$. Den originale pixel vil da skifte værdi fra $42$ til
            $29$.
        }
\end{figure}

Denne metode kan ikke køres direkte på det originale billede, da dette
vil interferere med fastsættelse af medianen for alle pixels. Man må
derfor oprette en kopi af det originale billede og sætte de fundne
medianværdier i denne. Man finder således altid medianen i forhold til det
originale billede.

\subsubsection*{Eksempler}

% Hold on, this is figure-madness
\begin{figure}[!h]
    \centering
    \subfloat[Original]{\label{simple_original}\includegraphics[angle=0,width=0.3\textwidth]{afsnit/vores_implementation/billeder/sloering/original}}\hspace{1em}
    \subfloat[$3 \times 3$ vindue]{\label{simple_3_3}\includegraphics[angle=0,width=0.3\textwidth]{afsnit/vores_implementation/billeder/sloering/simple_3_3}}\hspace{1em}
    \subfloat[$7 \times 7$ vindue]{\label{simple_7_7}\includegraphics[angle=0,width=0.3\textwidth]{afsnit/vores_implementation/billeder/sloering/simple_7_7}}
    \caption[]{
        \textbf{\ref{simple_original})} Zoom af detajler i det originale billede.
        \textbf{\ref{simple_3_3})}
        \textbf{\ref{simple_7_7})}
    }
    \label{simple_metode}
\end{figure}

\begin{figure}[!h]
    \centering
    \subfloat[Original]{\label{gaussian_original}\includegraphics[angle=0,width=0.3\textwidth]{afsnit/vores_implementation/billeder/sloering/original}}\hspace{1em}
    \subfloat[$3 \times 3$ vindue]{\label{gaussian_3_3}\includegraphics[angle=0,width=0.3\textwidth]{afsnit/vores_implementation/billeder/sloering/gaussian_3_3}}\hspace{1em}
    \subfloat[$7 \times 7$ vindue]{\label{gaussian_7_7}\includegraphics[angle=0,width=0.3\textwidth]{afsnit/vores_implementation/billeder/sloering/gaussian_7_7}}
    \caption[]{
        \textbf{\ref{gaussian_original})} Zoom af detajler i det originale billede.
        \textbf{\ref{gaussian_3_3})}
        \textbf{\ref{gaussian_7_7})}
    }
    \label{gaussian_metode}
\end{figure}

\begin{figure}[!h]
    \centering
    \subfloat[Original]{\label{median_original}\includegraphics[angle=0,width=0.3\textwidth]{afsnit/vores_implementation/billeder/sloering/original}}\hspace{1em}
    \subfloat[$3 \times 3$ vindue]{\label{median_3_3}\includegraphics[angle=0,width=0.3\textwidth]{afsnit/vores_implementation/billeder/sloering/median_3_3}}\hspace{1em}
    \subfloat[$7 \times 7$ vindue]{\label{median_7_7}\includegraphics[angle=0,width=0.3\textwidth]{afsnit/vores_implementation/billeder/sloering/median_7_7}}
    \caption[]{
        \textbf{\ref{median_original})} Zoom af detajler i det originale billede.
        \textbf{\ref{median_3_3})} Median med et vindue på $3\times{}3$.
        Farverne er blevet mere ensartede mens kanterne stadig er
        skarpe.
        \textbf{\ref{median_7_7})} Median med et vindue på $7\times{}7$. Farverne
        er meget ensartede, men det ses at kanterne er blevet mere
        udvisket med det større vindue.
    }
    \label{median_metode}
\end{figure}

}

% vim: set tw=72 spell spelllang=da:


\subsection{Floodfill}
\subsection*{Floodfill}

\subsubsection*{Teorien}
Floodfill omhandler udfyldning af områder i et billet som har samme fave eller en fave som ligger inde få en vis afvigelse fra den orginale fave. Det vil sige, der væljes en pixel i billet, denne pixel har faven (r,g,b). Ud fra denne pixel i billdet, ses der på de nabo pixels som ligger dirgunale og vertikale, hvis nogle af de dirgunale eller vertikale pixels har en fave som ligger inde for en vis afvigelse af faven, faves de. Ud fra de nye favet pixels, bliver helle procedyren gendtaget, ind til at der ikke er flere pixels som kan faves.

\subsubsection*{Implementation}
I Implementationen af floodfill, er der visse ting man kan stille på for at få de resultat som er ønsket. Man kan stille på hvor meget fave må variere med. Man kan sætte floodfill til at regne variansen af fave ud fra den pixel som er valt fra start, eller fra fave af den pixel som functionen er komme til. hved at regne varianden ud fra start pixlen, få metode til at indskranke sig en del, og ikke komme inde i alle hjørner af en blob, tilgændgel har metode svære hved at tegne over kanter og komme ind i en anden område, Denne måde at bruge floodfilll på, gørd også at metoden ikke har store udsving på hvor meget der bliver favet, ud fra favens varians. Hved at arbejde ud fra metoden som regner på den nye pixels fave, få man en metode som få meget af bloben med, og som kan overskue at en blob kan skifte fave langsomt i forhold til solens position eller små skift i bloblen. tilgængel er denne framgang måde ret følsom, og kommer der ved til at flyde over bloben som den er i gang med at udfylde.  

\subsubsection*{Overvejelser}
For at få denne metode til at virke på 25000 billeder, hvor en del af billederne ikke har samme fave tone eller er blevet falmet. Må der udregnes, for vært billedet, hvad for en varians i fave der skal bruges.


\subsection{Programmeringssprog og biblioteker}
{
{\sffamily Ved valg af programmeringssprog har vi først og fremmest lagt
vægt på at kunne udarbejde en prototype hurtigt og bruge et sprog, som
er let at gå til. Det valgte sprog skal også gøre det nemt at udvide den
endelige implementation. Vi har også gerne villet undgå at skulle
konstruere komplicerede datastrukturer for relativt simple metoder, både
af hensyn til tidspresset og til implementationens kompleksitet. Af
ovenstående grunde har vi besluttet at udarbejde vores løsning i
programmeringssproget \textbf{Python}, da netop dette sprog er yderst
velegnet at skrive forholdsvis avancerede prototyper i. Python er
ydermere meget fleksibelt med hensyn til datastrukturer og byder
umiddelbart på en lang række, for problemstillingen relevante, pakker.

(Skal man skrive noget om Pythons udbredelse, anerkendelse og brug?)
}

\subsection{OpenCV}
Til udførelse af billedmanipulationer benytter vi os af et bibliotek skrevet
i C og C++, der hedder \emph{OpenCV}. Biblioteket er udviklet af Intel
og tilbyder, udover et solidt udvalg af algoritmer, bindinger til
Python.  Endelig er det meget veldokumenteret og giver referencer til
publikationer om bibliotekets algoritmer. Biblioteket er udviklet med
specielt henblik på real-tids behandling af billeder, f.eks. med et
videokamera som kilde, men det egner sig også til brug på enkelte
billeder.  \emph{OpenCV} tilbyder mange brugbare datastrukturer med
hensyn til arbejdet med billeder i Python.

Der er også andre biblioteker til billedbehandling i Python. Her kan
nævnes \emph{PIL} (Python Image Library) og \emph{PythonMagick}
(ImageMagick bindings), men de er ikke nær så grundige som
\emph{OpenCV}.

\subsubsection{Andre muligheder}
Der er to helt oplagte muligheder, med hensyn til programmeringssprog,
når man taler om billedbehandling, nemlig Matlab og dets Open
Source-alternativ Octave. Disse sprog blev dog valgt fra, da vores
samlede erfaring med udvikling i disse sprog ikke var stor nok.
Endvidere finder vi, at disse sprog, på trods af, at de især egner sig
til den type beregninger, vi skal lave, er besværlige at lave større
programmer med. Matlab og Octave er dog blevet brugt til at sammenligne
resultater og teste alternative metoder med.

Da \emph{OpenCV} er skrevet i C/C++, ville det også være oplagt at bruge
et af disse sprog. Vores erfaring er dog, at man let kommer til at bruge
mere tid på at konstruere de fornødne datastrukturer og hjælpemetoder,
end på at fokusere på opgavens kerne. En senere implementation, med
fokus på køretid, kunne med fordel implementeres i C/C++, da man så
ville have fuld kontrol over, hvilke strukturer der bliver brugt i
programmet.

\subsection{Værktøjer til databasen}
Vi bruger \textbf{SQLite} til selve databasen, hovedsagelig fordi der ikke
kræves nogen videre konfiguration af en sådan database. Den
underliggende database er dog underordnet, da vi bruger Python-pakken
\emph{SQLObject}, som giver et abstraktionslag til en bred vifte af
databaser. Vi opretter blot de tabeller, vi ønsker at have i databasen,
som klasser i Python og får ligeledes en sådan klasse tilbage, når der
laves forespørgsler til databasen. Da \emph{SQLObject} klarer al
kommunikation med databasen, er det derfor muligt at skifte den
underliggende database ud, hvis man ønsker det. SQLite har endvidere den
umiddelbare fordel, at selve databasen eksisterer som en fil i
filsystemet.  Det er derfor en let sag at tage sikkerhedskopier af
databasen uden alt for meget besvær.

\subsection{Andre værktøjer}
Vi gør også brug af statistikprogrammet \textbf{R} til at behandle og
præsentere vores resultater.

}

% vim: set tw=72 spell spelllang=da:


}

% vim: set tw=72 spell spelllang=da:


\section{Opbevaring af data\label{section_database}}
{
{\sffamily I forbindelse med analyse på store datasæt, spiller den
bagvedliggende database en central rolle. Det er fra databasen, vi
henter billeder ind til analyse og samtidig også her, resultaterne
bliver gemt. Vi præsenterer herunder det databaseskema, som databasen er
opbygget efter.  Efterfølgende kaster vi et nærmere blik på de enkelte
tabeller i databasen, hvor vi først vil kigge på, hvordan vi opbevarer
maleriernes metadata, og endelig på hvordan resultater bliver opbevaret.
}

\subsection{Databaseskema}
Tabellerne \ref{artistTable}, \ref{paintingTable}, \ref{runTable},
\ref{resultTable} og \ref{regionTable} herunder, udgør databaseskemaet.
Der er i alle henseender lagt vægt på at eliminere redundans og på
muligheden for senere udvidelse.

\begin{table}[!h]
    \centering
    \begin{tabular}{|l||c|c|c|c|c|c|}
        \hline
        \bf{artist} \hspace{0.5cm} & \underline{artistId} & name & born & died & school & timeline \\\hline
    \end{tabular}
    \caption{Databasetabel for kunstner.}
    \label{artistTable}
\end{table}

\begin{table}[!h]
    \centering
    \begin{tabular}{|l||c|c|c|c|c|c}
        \hline
        \bf{painting} \hspace{0.5cm} & \underline{paintingId} & artistId & title & date & paint & $\cdots$ \\\hline
    \end{tabular}\\ \vspace{0.2cm}\hspace{1.2cm}
    \begin{tabular}{c|c|c|c|c|c|c}
        \hline
        $\cdots$ & material & location & url & form & type & $\cdots$ \\\hline
    \end{tabular}\\ \vspace{0.2cm}\hspace{1.4cm}
    \begin{tabular}{c|c|c|c|c|c|}
        \hline
        $\cdots$ & realHeight & realWidth & height & width & filepath \\\hline
    \end{tabular}
    \caption{Databasetabel for malerier.}
    \label{paintingTable}
\end{table}

\begin{table}[!h]
    \centering
    \begin{tabular}{|l||c|c|c|c|c|c|c|}
        \hline
        \bf{run} \hspace{0.5cm} & \underline{runId} & trsh1 & trsh2 & lo & up & marginPercentage & method \\\hline
    \end{tabular}
    \caption{Databasetabel for en kørsel.}
    \label{runTable}
\end{table}

\begin{table}[!h]
    \centering
    \begin{tabular}{|l||c|c|c|c|c|c|}
        \hline
        \bf{result} \hspace{0.5cm} & \underline{resultId} & runId & paintingId & cutRatio & cutNo & numberOfRegions \\\hline
    \end{tabular}
    \caption{Databasetabel for resultater.}
    \label{resultTable}
\end{table}

\begin{table}[!h]
    \centering
    \begin{tabular}{|l||c|c|c|c|c|c|c|}
        \hline
        \bf{region} \hspace{0.5cm} & \underline{regionId} & resultId & x & y & height & width & area \\\hline
    \end{tabular}
    \caption{Databasetabel for regioner.}
    \label{regionTable}
\end{table}

\subsection{Metadata og billeder\label{section_opbv_billeder}}
Vi starter med at kigge på, hvordan vi opbevarer maleriernes metadata.
Denne information findes i tabellerne \texttt{artist} (tabel
\ref{artistTable}) og \texttt{painting} (tabel \ref{paintingTable}).
Disse to tabeller lægger vægt på, at man let skal kunne forespørge
databasen ved en lang række parametre, såsom et maleris fysiske
størrelse og kunstnerens fødselsår. Vi har, at en kunstner kan være
tilknyttet ét eller flere malerier, og at der, til et givet maleri, kun
kan være én kunstner. Billederne, vi vil analysere, hentes fra et online
kunstarkiv og gemmes på filsystemet i en mappestruktur, der ligner den i
arkivet. Her inddeles filerne i mapper navngivet efter kunstner.
Mapperne inddeles efter forbogstav. På denne måde undgås det, at to
billeder tildeles samme filnavn. Vi gemmer kun stien til en fil på
filsystemet i databasen. Filstrukturen er grafisk illustreret i figur
\ref{mappestruktur}.

% Mappestruktur
\begin{figure}[!h]
    \centering
$
\xymatrix{
 &  &   & \ar @{-} [d] \textrm{/res}  &                                                     \\
 &  &   & \ar @{-} [d] \textrm{/wga.hu}  &                                                  \\
 &  &   & \ar @{-} [dl] \ar @{-} [d] \ar @{--} [dr] \textrm{/art} &                         \\
 &  & \ar @{-} [dl] \ar @{-} [d] \ar @{--} [dr] \textrm{/a} & \textrm{/b} & \cdots          \\
 & \ar @{-} [dl] \ar @{-} [d] \textrm{/aachen} & \ar @{--} [d] \textrm{/abadia} & \cdots    \\
\textrm{allegory.jpg} & \textrm{bacchus.jpg} & \cdots &   &
}
$
    \caption{Mappestruktur til filer fra
        \href{http://www.wga.hu}{http://www.wga.hu}.}
    \label{mappestruktur}
\end{figure}

\subsection{Resultater fra kørsler\label{section_results}}
Når vi har trukket regioner ud af billedet --- jvf. afsnit
\ref{section_udtraek} --- og vurderet dem efter den naive algoritme
givet i afsnit \ref{section_naiv}, står vi tilbage med et egentligt
resultat. Vi ønsker at gemme dette resultat i databasen, så vi på et
senere tidspunkt kan bruge det i en samlet analyse af resultaterne. Det
øvrige databaseskema, der udgøres af tabellerne \texttt{run} (tabel
\ref{runTable}), \texttt{result} (tabel \ref{resultTable}) og
\texttt{region} (tabel \ref{regionTable}), lægger vægt på at kunne gemme
data fra flere forskellige kørsler med forskellige parametre og mulighed
for at genskabe kørte analyser. Vi vil nu kigge på betydningen af de
ovenstående tabeller og se på, hvordan tabellerne giver mulighed for
designmålene.

Hvis vi vil genskabe et fundet resultat, har vi brug for at vide hvilke
parametre, vi har brugt for at komme frem til resultatet. Til dette
formål har vi tabellen \texttt{run} (tabel \ref{runTable}), som
beskriver en kørsel. Denne tabel holder de parametre, som er fælles for
alle billeder i en kørsel, bl. a. de tærskelværdier, som bruges til
kantdetektion (\texttt{trsh1} og \texttt{trsh2}), samt nedre og øvre
grænse for floodfill (\texttt{lo} og \texttt{up}). Her holdes også en
procentsats for, hvor stor et margin vi bruger i udvælgelse af
regioner. Endelig har vi et felt, der angiver, hvilken metode der er
blevet brugt til at finde resultatet. Dette er en tekststreng og vil i
tilfældet af den naive algoritme være sat til \texttt{'naive'}. Felterne
\texttt{trsh1}, \texttt{trsh2} og \texttt{marignPercentage} er
repræsenteret som floats i databasen, mens \texttt{lo} og \texttt{up}
står som heltal. Afslutningsvis har hver kørsel et unikt \emph{id}, så
at vi kan tilknytte indgange i tabellen \texttt{result} (tabel
\ref{resultTable}) til et sæt af parametre.

Vi beskriver et resultat som det antal af regioner, vi får ud fra
analysen af et snit på et givet billede. Som beskrevet i afsnit
\ref{section_opdeling} vil vi, givet en snitratio, typisk søge regioner
i nærheden af fire snit. En undtagelse er, hvis snitratioen deler
billedet i to lige store dele, og vi vil kun i dette tilfælde have to
snit at se på. Tabellen \texttt{result} fortæller os hvilket af de
mulige snit vi har med at gøre, hvilket billede resultatet er
tilknyttet, hvilke parametre der er blevet brugt, samt hvor mange
regioner vi har fundet. Tabellen gør det muligt at gemme resultater fra
kørsler med forskellige parametre, hvorved man kan have data fra
separate kørsler i databasen. Man har da grundlag for at sammenligne
kørsler med forskellige metoder og parametre.

Tabellen \texttt{region} (tabel \ref{regionTable}) holder alle de fundne
regioner fra vores analyse. Hver region henviser til det resultat, som
denne tilhører. Vi kan således skelne de enkelte regioner fra hinanden
og afgøre, i hvilket snit af billedet de ligger. En region bliver
repræsenteret som dens areal og begrænsende rektangel.

\subsection{Vurdering}
Databaseskemaet har været underlagt følgende designmål:

\begin{itemize}
    \item Minimering af redundans
    \item Mulighed for senere udvidelse
    \item Mulighed for rekonstruktion af kørte analyser
    \item Mulighed for at kunne analysere flere snit i én kørsel
\end{itemize}

Vi har dog feltet \texttt{numberOfRgions} i tabellen \texttt{result} som
på sin vis er redundant, da vi blot kan finde antallet af tilknyttede
regioner ved at bruge simple SQL-sætninger. At have antallet stående
direkte i databasen giver dog et umiddelbart bedre overblik over
resultaterne: Når databasen vokser, som følge af ændringer på
parametrene, vil SQL-sætningerne skulle søge meget af databasen igennem
for at returnere simple forespørgselser. Derfor har vi valgt at gemme
antallet at fundne regioner direkte i databasen.

Databaseskemaet er let at udvide, hvis vi, i udviklingen af mere
avancerede metoder, skulle få brug for flere indgange. Det skal også
bemærkes, at tabellerne \ref{runTable}, \ref{resultTable} og
\ref{regionTable} er specifikke for netop den naive algoritme.
Tabellerne er kun tilknyttet maleriernes metadata ved feltet
\texttt{paintingId}. Man kan derved let udvide databasen til at
indeholde data om kørsler, hvor en anden metode til udtrækning af
regioner, eller til bedømmelse af samme, har været brugt.

}

% vim: set tw=72 spell spelllang=da:


\section{Udvidelser til vurdering af regioner\label{section_udvidelser}}
{
{\sffamily Vi præsenterer her nogle udvidelser til den naive algoritme
som afgør, om et billede har interessante regioner i det gyldne snit.
Som beskrevet i \ref{section_naiv} har den naive fremgangsmåde nogle
ulemper, som vi gerne vil forsøge at reducere. Bemærk, at vi ikke vil
forbedre metoden som trækker regioner ud af et billede, eller ændre på
definition af en interessant region. Vi forbedrer udelukkende metoden,
der skal afgøre, om en interessant region er placeret i det gyldne snit,
og i det følgende præsenteres to metoder til dette formål. Den første
metode, bygger videre på den naive algoritme, hvor den ideelle placering
i det gyldne snit redefineres. Den anden metode, vi præsenterer, tager
afstand fra den binære klassifikation og søger i stedet at tildele
interessante regioner en værdi, der angiver, i hvor høj grad de ligger i
det gyldne snit.

Vi antager i det følgende, at vi har perfekt udtrækning af regioner i et
maleri, og at den digitale gengivelse af maleriet således bliver
segmenteret helt som ønsket. Dette betyder, at figurerer der en person i
maleriet, da vil denne blive opfattet som én sammenhængende region.
}

\subsection{Udvidelse til den naive metode}
{
Denne udvidelse sigter mod at forbedre den eksisterende bedømmelse af
interessante regioner i forhold til det gyldne snit. Specielt er der
en type af interessante regioner som bliver fravalgt af den naive
metode, nemlig regioner med massemidtpunkt i det gyldne snit, men med et
afgrænsende rektangel uden for margin. Et eksempel på dette kan ses i
figur \ref{hus}, hvor den sorte region ikke anses som liggende i det
gyldne snit af den naive fremgangsmåde. Vi vil gerne have at sådanne
regioner klassificeres som en positiv interessant region.

\begin{figure}[h]
	\begin{center}
		\includegraphics[scale=0.3,angle=0]{afsnit/vores_implementation/billeder/udvidet_loesning/husworks.png}
	\end{center}
	\caption[]{Et hus som bliver skåret over så midten ligger inde for snittet margin}
	\label{hus}
\end{figure}


\subsubsection{Opdeling af ragion med et grid}
% Original
%\subsubsection{Opdeling af ragion med et grid}
%Måden vi deller ragionen op på, er hved hjælp af et gridt som vi
%tilføjer vær baunding box, se figur \ref{grid}. Da vi ved hvilken farve
%baunding boxens ragion er, tager vi alle de punkter i grittet som har
%denne farve, og få der ved en bedre beskrivels af hvordan ragionen ser
%ud. For at gøre algoritmen lidt hurtiger, er vores gridt punkter lavet
%med 1 pixels mellemrum. 

\begin{figure}[h]
	\centering
	\includegraphics[scale=0.76,angle=0]{afsnit/vores_implementation/billeder/udvidet_loesning/udvidetloesninglayer.png}
	\caption[]{Et grid over en ragion i baunding box}
	\label{grid}
\end{figure}

\subsubsection{Bedømmelse med hensyn til massemidtpunkt}
Givet en region $R$ betegner vi antallet af punkter i regionen med
$|R|$. Vi antager at vi betragter et vertikalt snit $G$ i et billede.
Det gælder for alle punkter $p \in R$ at de kan befinde sig ovenpå, til
højre eller til venstre for $G$. I denne sammenhæng lader vi $R_r$ og
$R_l$ beskrive punkter henholdsvis til højre og venstre for snittet $G$.
Afstanden fra en punkt til kanten af et billede kaldes $D_p$, hvor
kanten er origo i billedet. Med disse informationer kan vi afgøre om
snitte deler regionen to lige store dele, samt om store dele af regionen
befinder sig langt væk fra snittet. Vi vil gerne kigge på en regions
massemidtpunkt og se om dette ligger inden for margin. Vi beregner
regionens massemidtpunkt ved funktionen


\begin{eqnarray}
    m(R) & = & \frac{\sum_{p \in R}{D_p}}{|R|} \label{masssemidpunkt}
    \label{MPunkt}
\end{eqnarray}

hvor m(R) giver os en værdi for hvad for et lige linie der delle ragionen
op i 2 delle, som er mest ens.

\begin{figure}[h]
	\begin{center}
		\includegraphics[scale=0.5,angle=0]{afsnit/vores_implementation/billeder/udvidet_loesning/cOMCutMargin.png}
	\end{center}
	\caption[]{Regioner hvor massemidtpunktet, snit og margin er tegnet ind, som man kan se ligger midtpunktet inde for marginen}
	\label{cOMCutMargin}
\end{figure}

Hvis $m(R)$ ligger inden for margin, se figur \ref{cOMCutMargin}. Bliver
regionen godtaget som liggende i snittet. Det er dog ikke helt nok kun
at bedømme regionerne efter massemidtpunkt, da man kan konstruere en
region som vi ikke vil godtage, men som har massemidtpunkt inde for
marginen, se figur \ref{dontwork}. 

\begin{figure}[h]
	\begin{center}
		\includegraphics[scale=0.5,angle=0]{afsnit/vores_implementation/billeder/udvidet_loesning/dontWork.png}
	\end{center}
	\caption[]{Figur som har et massemidtpunkt indenfor margin, men vi ikke vil have med i godtaget regioner, på grund af den lange stang}
	\label{dontwork}
\end{figure}

Måden vi få løst problemet på er ved at lave en nyt tjek som kan sortere
de regioner fra som vi ikke vil have godkende. Det gør vi hjælp af
funktion

\begin{eqnarray}
    f(R) & = & \frac{|R_{l}| - |R_{r}|}{|R|}
    \label{Fordeling}
\end{eqnarray}

Som sammenligner antal pixels på begge sider af snittet og giver en
procent på hvor stor forskel der er, det vil sige at $f(R) \in [-1,1]$.
Hvis $f(R)$ er positivt, er der $f(R)$ procent flere pixels på $|R_l|$
side og vise verser. hvis den abselutte værdig af $f(R)$ er over $0.75$
betyder det at regionen er få skævt fordelt og den sortere den fra. Få
at give et eksempel på hvornår vores naive algoritme virker, se figur \ref{centerOfMass}
\begin{figure}[h]
	\begin{center}
		\includegraphics[scale=0.35,angle=0]{afsnit/vores_implementation/billeder/udvidet_loesning/centerOfMass.png}
	\end{center}
	\caption[]{Eksempel på hvordan den udvidet algoritme virker på et billedet hvor den naive algoritme ikke vil have fundet det}
	\label{centerOfMass}
\end{figure}


}

% vim: set tw=72 spell spelllang=da:


\subsection{Topografisk kort til omkostninger}
{

Hvad er et topografisk kort? Eksempler på topografiske kort kan ses i
figur \ref{topography_plus} og \ref{topography_times}.

Vi vil nu beskrive en metode hvorved regioner bliver tildelt en
omkostning. Denne omkostning beregnes ud fra regionens placering i
billedet på baggrund af et topografisk kort. Således vil regioner ikke
blive vurderet efter hvorvidt de ligger i det gyldne snit eller ej, men
efter \emph{hvor meget} de ligger i det gyldne snit.

\subsubsection*{Generation af topografisk kort}

Givet en afbildning af et maleri ved matricen $\mathbf{I}$ med dimensioner
$N \times{} M$, kan man opstille et topografisk kort $\mathbf{T}$ med dimensioner
$N \times{} M$.

Det topografiske kort generes ud fra to vektorer $\mathbf{X}$ og
$\mathbf{Y}$ med dimensioner på henholdsvis $N \times 1$ og $1 \times M$.
\begin{equation}
    \mathbf{X} = \left(
    \begin{array}{c}
        x_1     \\
        x_2     \\
        \vdots  \\
        x_n
    \end{array} \right)
    \label{x_vector}
\end{equation}
og
\begin{equation}
    \mathbf{Y}^{t} = \left(
    \begin{array}{c}
        y_1     \\
        y_2     \\
        \vdots  \\
        y_m
    \end{array} \right)
\end{equation}

Vi betragter vektoren $\mathbf{X}$ som liniestykket givet ved $AB$ i
figur \ref{topograph_line}. Længden af liniestykket betegnes ved $|AB|$.
Det følger af ligning \ref{x_vector} at $|AB| = n$. På liniestykket $AB$
er det gyldne snit placeret ved $G$ hvilket betegner indeks $\lfloor
n\varPhi \rfloor$ i $\mathbf{X}$. Et margin er angivet ved punkterne $(G
- \delta) = m$ og $(G + \delta) = m'$, hvor $\delta$ er størrelsen på
margin. Ligeledes betegner $m$ og $m'$ indeks $\lfloor n \varPhi \pm
\delta \rfloor$ i $\mathbf{X}$.  Endvidere har vi at $|Ap| = \lfloor
\frac{1}{2}|AB| \rfloor \leq |pB|$ og $|m'q| = \lfloor \frac{1}{2}|qB|
\rfloor \leq |qB|$. I figur \ref{topograph_line} er kun liniestykket
$pB$ segmenteret, men $Ap$ segmenteres symmetrisk.

Vi placerer nu et nyt punkt $x$ på $pB$. I det et-dimensionelle plan kan
længden $|Gx|$ bruges som mål for hvor tæt $x$ er på det gyldne snit.
Punktet $x$ har dog ingen udstrækning, hvorfor vi ikke blot kan bruge
længden i praksis. Vi betragter nu en region $R \in \mathbb{Z}^{+}$.
Regionen $R$ er en liniestykke i det et-dimensionelle plan. Vi kan da
udregne placeringen af $R$ i forhold til punktet $G$ ved at summere alle
afstandene fra punkterne $x$ i $R$ til $G$.  Dette medfører at lange
liniestykker bliver tildelt højere værdi end små. Vi udregner således
$\frac{\sum_{x \in R}{|Gx|}}{|R|} = |G(\frac{|R|}{2})|$, hvor $|R|$ er
længden, dvs. antallet af punkter i $R$. Dette svarer til at beregne
afstanden fra regionen midtpunkt til $G$. Dette er ikke ønskværdigt, da
vi kan have to regioner med forskellig længde, men med samme midtpunkt.
Hvis vi lader $R_{max}$ betegne den større region og $R_{min}$ være den
mindre, hvor $ \frac{|R_{max}|}{2} = \frac{|R_{min}|}{2}$, da må
$R_{max}$ nødvendigvis have et ekstrema tættere på $G$ end $R_{min}$.
Det er derfor ikke retfærdigt at give begge regioner den samme værdi.

Vi ønsker at belønne punkter der ligger i eller tæt ved det gyldne snit,
men også omvendt give stor omkostning til punkter der ikke ligger i det
gyldne snit. Til dette bruges vektoren $\mathbf{X}$ som vil angive
omkostningen for hvert punkt på $AB$. Da vi ønsker at belønne punkter i
det gyldne snit er der ingen omkostning til punkter der ligger
i det gyldne snit. Vi sætter derfor $\mathbf{X}_{|AG|}$ til $0$.

\begin{figure}[!h]
    \centering
    \begin{picture}(240,30)
        \put(0, 10){$A$}
        \put(3, -5){\line(0, 1){10}}

        \put(116, 10){$p$}
        \put(118, -5){\line(0, 1){10}}

        \put(131, 10){$m$}
        \put(134, -4){\line(0, 1){8}}

        \put(144, 10){$G$}
        \put(147, -4){\line(0, 1){8}}

        \put(157, 10){$m'$}
        \put(160, -4){\line(0, 1){8}}

        \put(195, 10){$q$}
        \put(198, -4){\line(0, 1){8}}

        \put(233, 10){$B$}
        \put(236, -5){\line(0, 1){10}}

        \put(182, 10){$x$}
        \put(185, 0){\circle*{3}}

        \put(3, 0){\line(1, 0){233}}
    \end{picture}
    \caption[]{Liniestykke}
    \label{topograph_line}
\end{figure}

%Vektorerne angiver omkostningen for at have et punkt i den givne
%dimension. Hvis vi betragter det gyldne snit vil vi gerne have at
%værdierne i $\mathbf{X}_{\lfloor n \varPhi \rfloor}$ og
%$\mathbf{X}_{\lfloor n(1 - \varPhi) \rfloor}$ sættes til $0$, da det
%ikke skal have nogen omkostning at have et punkt i det gyldne snit.
%Ligeledes sættes $\mathbf{Y}_{\lfloor m \varPhi \rfloor}$ og
%$\mathbf{Y}_{\lfloor m(1 - \varPhi) \rfloor}$ også til $0$. Vi bruger
%også her et margin $\sigma$ hvor omkostningen for punkter er lav. Indeks
%$\lfloor n \varPhi \pm \sigma \rfloor$ sættes da til at have
%omkostningen $1$. Dette gøres helt analogt for indeks $\lfloor n(1 -
%\varPhi) \rfloor$ og for vektoren $\mathbf{Y}$.

\begin{figure}[h]
    \setlength\fboxsep{0pt}
    \setlength\fboxrule{0.5pt}
    \begin{center}
        \fbox{\includegraphics[width=0.8\textwidth]{afsnit/vores_implementation/billeder/udvidet_loesning/topographic_plus.png}}
    \end{center}
    \caption[]{Topografisk kort over omkostninger for regioner ud fra det
    gyldne snit. Værdien $T_{(x, y)}$ er givet ved funktionen
    $t(i, j) = C^{x}_{i} + C^{y}_{j}$ hvor $C^{x}$ og $C^{y}$ angiver
    omkostningsvektorerne.}
    \label{topography_plus}
\end{figure}

\begin{figure}[h]
    \setlength\fboxsep{0pt}
    \setlength\fboxrule{0.5pt}
    \begin{center}
        \fbox{\includegraphics[width=0.8\textwidth]{afsnit/vores_implementation/billeder/udvidet_loesning/topographic_times.png}}
    \end{center}
    \caption[]{Topografisk kort over omkostninger for regioner ud fra det
    gyldne snit. Her er værdierne udregnet ved at multiplicere værdierne
    fra omkostningsvektorene.}
    \label{topography_times}
\end{figure}

\subsubsection*{Omkostningsfunktionen}

Givet en mængde $R \in \{\mathbb{Z}^{+}\times\mathbb{Z}^{+}\}$, som
angiver punkterne i en given region, kan man finde omkostningen
$C$ ved
\begin{equation}
    C(R) = \sum_{(i, j) \in R}{\frac{T_{ij}}{nm}}
\end{equation}

}

% vim: set tw=72 spell spelllang=da:


\subsection{Implementering af udvidelser}
Vi har valgt at implementere den først nævnte udvidelse til bedømmelse
af interessante regioner. Denne er valgt, da den er en simpel
modifikation af den eksisterende metode, hvilket gør, at vi også kan
afprøve metoden på vores korpus. Strukturen i metoden med topografiske
kort er meget langt væk fra den eksisterende metode, hvorfor vi har
vurderet, at denne ikke skulle implementeres. Topografiske kort indfører
også et mål på regionerne, hvilket vi ikke benytter os af i den
eksisterende metode.

}

% vim: set tw=72 spell spelllang=da:



}

% vim: set tw=72 spell spelllang=da:
