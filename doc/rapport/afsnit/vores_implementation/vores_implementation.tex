{
{\sffamily Vi bør introducere problematikken ved billedbehandling her.
Hvorfor er det at computeren ikke bare kan ``se'' ligesom mennesket.
Hvordan opfatter computeren et billede. Giv en latterlig analogi, f.eks.
en blind mand der bliver fortalt hvilken farve hver kvadratcentimeter
har på et maleri og at han ud fra de oplysninger skal afgøre hvordan
billedet tager sig ud.

Vi bør i denne sammenhæng også komme ind på at billedbehandling, især
ikke det givne problem, er veldefineret. Det er nok en afart af det
ovenstående, men det bør understreges.

Vi skal introducere at vi i det følgende vil komme frem til en simpel
algoritme hvorfra man kan afgøre om et billede opfylder det gyldne snit.
Det er her vi skal have introduceret de forskellige begreber.
}

\section{Naiv algoritme\label{naiv_algoritme}}
{
% Image scale
\def\imgscale{0.34}

\textsf{Vi vil i det følgende forklare vores tanker bag en meget simpelt
fremgangsmåde, som har til formål at afgøre om et billede opfylder det
gyldne snit. For at afgøre dette trækker vi regioner ud af billeder og
vurderer dem nu efter deres placering, størrelse og form. Her antager
vi, at vi allerede har trukket regionerne ud, men afsnit
\ref{section_udtraek} vil forklare hvordan dette foregår. Vi siger at et
billede opfylder det gyldne snit, hvis en eller flere interessante
regioner kan siges at ligge i det gyldne snit.  I det følgende vil vi se
på hvornår en region ligger i det gyldne snit samt hvornår vi har med en
interessant region at gøre.
}

\subsection{Regionens placering}
Når vi har trukket regioner ud af et billede og vil afgøre om de ligger
i det gyldne snit, er det indlysende at deres placering har afgørende
betydning.  Vi vil nu komme frem til en definition hvorved man kan
afgøre om en region i billedet er placeret i det gyldne snit.

Vi starter med at se på det meget simple tilfælde, hvor en region
åbenlyst ligger placeret i det gyldne snit.  Et sådan eksempel ses i
figur \ref{pos_naiv_1}, hvor regionen vi betragter er farvet sort.  Den
røde linje markerer det gyldne snit.  Dette farveskema vil være
gennemgående i det følgende.  Det ses at regionen nærmest tangerer
linjen.

\begin{figure}[h]
    \begin{center}
        \includegraphics[scale=\imgscale,angle=0]{afsnit/vores_implementation/billeder/naiv_algoritme/naiv_positiv_blob_1}
    \end{center}
    \caption[En positiv region]{En region som tangerer det gyldne snit.
    Denne region er positiv.}
    \label{pos_naiv_1}
\end{figure}

I praksis vil vi dog sjældent have at regioner ligger helt præcist på
snittet.
%På grund af den måde, vi trækker regioner ud af billedet på,
%kan vi ikke være sikre på at regionen er fyldt helt ud til dens kanter,
%fordi der i disse områder stadig er stor overgang i farverne.  Dette
%bevirker at en regioner kan blive repræsenteret som mindre end de
%virkelig er.
Vi har også, at det gyldne snit baserer sig på det
irrationelle tal $\varphi$ og vi kan derfor ikke regne os helt nøjagtig
frem til vores det egentlige snit ligger. Dette vil vi komme nærmere ind
på i næste afsnit. Vi indfører derfor en margen hvori vi vil acceptere
regioner.  I praksis betyder det at vi \emph{ikke kun} kigger på den
linje der deler billedet ved det gyldne snit, men faktisk tager vi et
bånd, sammensat af en række snit, og bruger dette bånd som et bredere
gyldent snit.  Derved behøver en region ikke at tangere det gyldne snit
helt nøjagtig for at kunne betragtes som liggende i det gyldne snit. For
at give et eksempel, så vil regionen i figur \ref{pos_naiv_margin_1}
anses som værende placeret i det gyldne snit.  Vær opmærksom på at vores
bånd, rent visuelt i de følgende illustrationer, kan være stærkt
overdrevet.
\begin{figure}[h]
    \begin{center}
        \includegraphics[scale=\imgscale,angle=0]{afsnit/vores_implementation/billeder/naiv_algoritme/naiv_positiv_blob_margin_1}
    \end{center}
    \caption[Positiv region i margen]{En region som
    ligger indenfor en margen af det gyldne snit betragtes som
    positiv. De blå stiplede linjer angiver vores bånd.}
    \label{pos_naiv_margin_1}
\end{figure}

Vi ser nu på det rektangel der der begrænser en region.  En side af
dette rektangel vil vi kalde for en kant. Ved at studere figur
\ref{bbox_section} ses det let, at en region skal have mindst én kant
indenfor båndet om det gyldne snit, før vi kan sige at regionen ligger i
det gyldne snit. Dette betyder, at hvis en region har en kant indenfor
båndet, så ligger denne region i det gyldne snit.

\begin{figure}[h]
    \begin{center}
        \includegraphics[scale=\imgscale,angle=0]{afsnit/vores_implementation/billeder/naiv_algoritme/bbox_section}
    \end{center}
    \caption[Afgrænsende rektangler]{Den øverste region kan ikke
    siges at ligge i det gyldne snit, da den ikke har nogen kanter
    indenfor båndet. Den nederste region derimod, har én kant
    indenfor båndet og ligger derfor i det gyldne snit.}
    \label{bbox_section}
\end{figure}

\begin{figure}[!h]
    \begin{center}
        \includegraphics[scale=0.42,angle=0]{afsnit/vores_implementation/billeder/naiv_algoritme/bathers_mockup_blob}
    \end{center}
    \caption[Interessante regioner i praksis]{I praksis vil de
    fundne regioner være langt mere komplekse.}
    \label{realworld_example}
\end{figure}

Med ovenstående definition i tankerne kigger vi nu på hvordan et rigtigt
billede kan tage sig ud når vi vil finde regioner. I figur
\ref{realworld_example} ser vi, at der i dette tilfælde vil blive
udvalgt mange små regioner, som egentlig ikke kan tillægges nogen
betydning.  Vi vil nu gå videre og opsætte nogle kriterier for hvornår
en region er interessant.

\subsection{Regionens størrelse}
Når vi skal til at afgøre hvorvidt en region er interessant er det
oplagt at se på størrelsen af regionen.  Vi siger derfor at en region
skal have en vis størrelse før den kan betragtes som værende
interessant. I praksis vil regionens areal afspejle dens størrelse, hvor
arealet er det antal pixels regionen optager i billedet.  Grænsen for et
acceptabelt areal skal sættes i forhold til billedets størrelse.

Omvendt er vi heller ikke interesseret i at få for store regioner med i
betragtningerne.  F.eks. vil en himmel i et maleri udgøre en ret stor
sammehængende region.  De fleste af sådanne regioner vil dog ikke blive
taget i betragtning fordi de krydser snittet.  Hvis vi kigger på det
meget simple billede i figur \ref{pos_naiv_1} skal man også huske på at
der faktisk vises to regioner. Den sorte skikkelse er en region ligesom
den hvide baggrund er en region.  Baggrunden kan ikke siges at være en
interessant region, hvorfor det er vigtigt at sortere disse regioner
fra.

Indtil videre har vi kun illustreret ét snit i billedet.  Der er
imidlertid tre andre snit hvor vi også kan dele et billede efter det
gyldne snit.  Hvor vi har kigget på et vertikalt snit, vender vi nu
opmærksomheden mod et horisontalt snit.  Det generelle tilfælde vil være
at en region kun er interessant i forhold til enten et vertikalt eller
et horisontalt snit.  Et eksempel på dette ses i figur
\ref{pos_horiz_naiv_margin_1}.  Der kan dog forekomme specialtilfælde,
hvor en region vil være positiv i flere snit, hvilket vi vil komme ind
på i et senere kapitel.
\begin{figure}[H]
    \begin{center}
        \includegraphics[scale=\imgscale,angle=0]{afsnit/vores_implementation/billeder/naiv_algoritme/naiv_horiz_positiv_blob_1}
    \end{center}
    \caption[Positiv horisontal region]{En positiv region med en
    kant i et horisontalt bånd.}
    \label{pos_horiz_naiv_margin_1}
\end{figure}
Det ses tydeligt at regionen i figur \ref{pos_horiz_naiv_margin_1} ikke
kan tages i betragtning i forhold til et vertikalt snit, da regionen,
uanset snittets placering, vil krydse dette.  Det ses dog at regionen
har en kant indeni et horisontalt bånd og derfor kan klassificeres som
en region der ligger i det gyldne snit, dog i forhold til et horizontalt
bånd.

\subsection{Regionens form}
Regionens form kan også give informationer om hvorvidt vi har med en
interessant region at gøre.  I praksis er det dog meget svært at sige
noget om selve den fysiske form af en region, men vi kan sige noget om
dens masse.  En regions masse skal forstås som forholdet mellem
regionens areal og arealet af det rektangel der afgrænser regionen.
Dette forhold giver information om hvor massiv en region er.  En massiv
region vil være mere interessant end en meget spinkel region.  Figur
\ref{region_mass} illustrerer to forskellige regioner med forskellig
masse.
\begin{figure}[h]
    \begin{center}
        \includegraphics[scale=\imgscale,angle=0]{afsnit/vores_implementation/billeder/naiv_algoritme/bbox_area_ratio}
    \end{center}
    \caption[Regioners masse]{To forskellige regioner med vidt forskellige forhold
    mellem selve regionens areal og arealet af det rektangel der
    afgrænser regionen.}
    \label{region_mass}
\end{figure}
I praksis vil der blive udregnet en tærskelværdi i forhold til et
billedes størrelse for hvor massiv en region skal være for at kunne
karakteriseres som interessant.

\subsection{Sammenfatning af betingelser}
Vi samler nu op på de ovenstående krav for bestemmelse af hvornår en
region er en interessant og hvornår en region ligger i det
gyldne snit.

\noindent For at en region kan betegnes som \textbf{interessant} skal den
\begin{enumerate}
        \renewcommand{\labelenumi}{(\alph{enumi})}
    \item have et areal større end en tærskelværdi der sættes i
        forhold til billedets størrelse
    \item have en masse større end en variabel tærskelværdi.
\end{enumerate}
og før en region kan betragtes som \textbf{liggende i det gyldne
snit} skal den
\begin{enumerate}
        \renewcommand{\labelenumi}{(\alph{enumi})}
        \setcounter{enumi}{2}
    \item have en kant i båndet om det gyldne snit
\end{enumerate}

Bemærk, at kun interessante regioner vil blive taget i betragtning, når vi vil
afgøre om der ligger regioner i det gyldne snit. Dette betyder også at
en region godt kan blive klassificeret som interessant, men ikke
nødvendigvis liggende i det gyldne snit.

Et billede opfylder altså det gyldne snit hvis vi har en eller flere
interessante regioner der ligger i det gyldne snit.  Det er oplagt at
give regioner som er positiv i flere snit større vægt, hvilket vi også
vil komme ind på i et senere kapitel. Bemærk, at selvom ovenstående
fremgangsmåde tager udgangspunkt i det gyldne snit, så kan metoden
anvendes på ethvert snit i billedet.

I kapitel \ref{chap_afproevning} vil vi komme nærmere ind på hvad
tærskelværdierne i praksis skal sættes til.

\subsection{Begrænsninger}
Denne naive tilgang har nogle begrænsninger.  Den mest åbenlyse er, at
interessante regioner med symmetriakse i det gyldne snit, men med kanter udenfor
båndet, ikke vil blive karakterisseret som liggende i det gyldne snit.
Regioner hvor et segment af denne faktisk ligger i båndet, som den
øverste region i figur \ref{bbox_section}, hvor den vertikale del faktisk
ligger i snittet, vil ikke blive udvalgt.  Det kræver en videre
segmentering af de fundne regioner før at sådanne regioner vil blive
udvalgt.

}

% vim: set tw=72 spell spelllang=da:


\section{Opdeling af billeder\label{section_opdeling}}
\subsection*{Trunkerings- og afrundingsproblemer}
Mange af de metoder, vi bruger til at udregne, det gyldne
snits position eller størrelsen af en margin, udregnes med brøker. Hvorimod et
billede opbygges af pixels. Dette gør, at vi bliver nødt til at tage
approksimationer af udregningerne for at få dem tilbage til et helt tal.
Men hvor meget af data er gået tabt, og hvor mange af resultaterne kan
man stole på?

\subsubsection{Acceptabel afvigelse}
\note{Nogle referanse}Som beskrevet i afsnit \ref{mange_tal}, udregnes
det gyldne snit med mange decimaler. En kunstner, hvor god han end er,
har ingen chance for at male så præcist at man kan sige at strøget
ligger nøjagtigt oven på snittet selv om hans ententioner er at ramme
snittet. Vi kommer derfor til at have en vis uprecis hed på de data vi
få fra billedet selv. Vi starter med at se på alle de ting, som kan
skabe en usikkerhed fra malerens side. Man kan gå ud fra at den
procentvise afvigelse ikke er særlig stor, da vi ikke har bestræbelser
på at arbejde på abstrakt malerier, dog har vi sat den procentvise
afvigelse til 0.5 \%. Det vil sige at en maler med et lærred på 100
cm,maksimalt vil male $0.5$ cm forkert.

Når maleren vælger en ramme og et lærred, har vi igen problematikken,
selv om maleren spicifikt gå efter at bygge maleriet op efter det gyldne
snit, kan snittets placering i maleriet have forskubbet sig, ved dårlige
valg at ramme eller lærred. Derfor sætter vi den afvigelse til $1\%$. Da
vi igen mener at dette er den maksimale afvigelse, der kan opstå.

Når maleren maler en region i et maleri, forekommerer der normalt en
lille kant rundt om objektet, et omrids. Dette omrids kan vores
algoritmer ikke tage højde for, og vi må derfor modregne omridset, så vi
er sikre på at vi ser på regionen og ikke dens omrids. Da et omrids ikke
er særligt stort, har vi sat denne procentsats til $0.5\%$. 

Alt i alt giver det en afvigelse på vores aktuelle udtrækning af data
fra malerierne på $2\%$ det vil sige at finder vi en region, som ligger
på pixel 200, i et billedet, der er $500$ cm bredt, befinder den sig
faktisk i intervallet [190,210]. Måden hvorpå vi tager højde for den
forskel, er ved hjælp af marginer, som nævnt i afsnit
\ref{section_naiv}.

\subsection{Inddeling af billede efter snit}

I et billede betegnes højden og bredden som hhv. $H$ og $B$, se figur
\ref{cut}. Der er 4 gyldne snit, 2 vertikale og 2 horisontale, som vist
i figur \ref{lenasnit2}. For at finde ud af, hvor de 4 snit skal ligge i
billedet, multipliceres B og H med $\varPhi$,og man får to tal. Disse
betegner, hvor mange pixels, det gyldne snit befinder sig fra hhv. $H$
og $B$ f.eks vil de 2 horisontale snit, i et billedet, som har $B =
4000$ pixel, ligge hhv. $4000 \cdot \varPhi \approx 2472$ pixels fra
billedets øvre og nedre kant.

\begin{figure}[h]
	\begin{center}
		\includegraphics[scale=0.42,angle=0]{afsnit/vores_implementation/billeder/naiv_algoritme/Lenagolden}
	\end{center}
	\caption[]{Billedet som har indtegnet de fire gyldne snit}
	\label{lenasnit2}
\end{figure}

\begin{figure}[h]
	\begin{center}
		\includegraphics[scale=0.42,angle=0]{afsnit/vores_implementation/billeder/naiv_algoritme/Cut}
	\end{center}
	\caption[]{Billedets højde og bredde betegnes hvv. H og B. De 4 snit er navngivet.}
	\label{cut}
\end{figure}

De 4 snit tildeles hvert deres Id, "snit 0,1,2 og 3" så vi kan kende
forskel på de individuelde snit, Id'erne placering kan ses i figur
\ref{cut}. Vi vil i resten af rapporten kalde snittene efter deres Id.
Hvis vi gerne vil finde snittet som ligger i miden kommer der kun 2
snit, med vær deres Id "snit 0 og 1" som kan ses i figur \ref{Cut2}

\begin{figure}[h]
	\begin{center}
		\includegraphics[scale=0.42,angle=0]{afsnit/vores_implementation/billeder/naiv_algoritme/2Cut}
	\end{center}
	\caption[]{Billedet skæres her kun af 2 snit}
	\label{2Cut}
\end{figure}

\subsubsection{Heltal i det gyldne snit}

I eksemplet med 4000 pixels ovenfor, approksimerer vi antal pixels ved
at afrunde resultatet $2472.13595 \approx 2472$, se udregning
\ref{afrundning}. Det betyder at vi mister 0.13595 pixels i præction,
hvilket svarer til en misvisning af punktet på 0.00339875 $\%$ i forholdt
til $B$ på billedet. Se udregning \ref{afrundning2}.

\begin{equation}
	4000 \cdot \varPhi = 4000(\sqrt{5}-1)/2 = 2472.13595 \approx 2472 \label{afrundning}
\end{equation}

\begin{equation}
	0.13595/4000 \cdot 100 = 0.00339875 \label{afrundning2}
\end{equation}

Det er en meget lille del af selve billedet og skulle ikke give nogle
misvisninger i forhold til udregningen. For at gøre det lidt mere
generelt, sætter vi trunkeringsfejlen til $0.5$, da det er den maksimale
afrundingsfacktor som kan forekomme. Hvis billedet har en størrelse på
500 pixels, hvilket er det mindste billedet vi har, giver dette en fejlmargin
på $0.1 \%$. Dette tal bliver adderet til fejlsatsen ovenfor, og giver
en samlet afvigelse på $2.1\%$.

\subsubsection{Snitratio}
End til vider har vi kun arbejder med det gyldne snit, men andre snit i
billedet kan godt optræde, derfor indføre vi en nu betegnelse
snitratio, som betegner en procents sats for hvor lang inde i billedet
snittet befinder sig. Det vil sige at hvis en snitration er på $0.2$. Et
billedet har $B$ på 4000 vil et snit befinde sig i pixel $4000*0.2 =
800$.

\subsubsection{Heltal ved udregning af Margin}
Når vi har 2 forskellige snitratioer, f.eks. $\varPhi$ og $\frac{2}{3}$,
som ligger meget tæt på hinanden, og vi gerne vil sammenligne hvilken
regioner der ligger i snitratioernens snit, er det vigtigt at margin for
vært af de 2 snitratioers snit ikke krydser hinanden. 

Hvis margin krydser. Vil det indebære, at den samme region bliver fundet
af begge snit. Dette vil give et skævt billedet af forskellen på de to.
Derfor må vi sørge for at marginerne ikke krydser. Hvis $x$ betegner
antal pixels i $B$ eller $H$, og vi vil se på, hvor mange pixels, der er
mellem snitratio $\frac{2}{3}$ og $\varPhi$, multiplicerer vi $x$ med de
to snitratioen for at finde deres placering. Derefter subtraheres vi et
af de snit som befinder sig tetest på hinanden i vær snitratio med
hinanden.

\begin{eqnarray}
	\frac{x2}{3} - \frac{x2}{\sqrt{5}+1} & = & x(\frac{2}{3} - \frac{2}{\sqrt{5} + 1}) \nonumber \\
	& = & x(0.666667-0.618034) \\ \nonumber
	& = & x(0.048633)
\end{eqnarray}

Vi har nu fundet antal pixels mellem de to snit. Vi vil gerne undgå at
de to marginens ikke krydser hinanden, så der dividere vi med 2 og
afrunder værdien.

\begin{equation}
	\left\lfloor \frac{0.048633x}{2}\right\rfloor = \left\lfloor0.024316x \right\rfloor
\end{equation}

Tallet $2.4316$ er altså den minimale procentvise størrelse som vores
margin må have, nå vi sammen liner det gyldne snit og $\frac{2}{3}$.
Det betyder også at vi ikke må sammen line snit som ligger særlig
meget tætter på hinanden, da 0.021 er den minimale procent margin som vi må have.
$\lfloor 0.024316x \rfloor$ giver os et antal pixels som skildre de to snit. For at vise
hvor stort marginen egentlige kan være, bruger jeg denne formel på to
billeder, et som svare til vores mindste billedet, 500 pixels, og et
som svare til vores største billedet, 4000 pixels. Ved 500 pixels
bliver resultatet

\begin{equation}
	 \lfloor 500(0.024316)\rfloor = 12
\end{equation}

Det er en fint margin, da vores fejl på udregningerne ligger på 2.1 \%,
som svare til $\lceil 500*0.021 \rceil = 11$ pixels, som er 1 pixels fra vores
margin

Ved 4000 pixels giver det.

\begin{equation}
	 \left\lfloor 4000(0.024316)\right\rfloor = 97
\end{equation}

Som også er god nok da $4000*0.021 = 84$ pixels.
% vim: set tw=72 spell spelllang=da:


\section{Udtrækning af regioner i et billede\label{section_udtraek}}
{
{\sffamily Dette afsnit har til formål at beskrive hvordan vi trækker
regioner ud af billedet. Kort sagt forsøger vi at segmentere billedet
ved primært ved at bruge en metode kaldet floodfill. Vi præparerer dog
først billedet ved at finde kanter og sløre billedet. Vi vil også komme
ind på hvilke andre metoder vi har forsøgt os med. Først vil vi dog se
på hvilke tanker vi har gjort os om valg af programmeringssprog og
implementationen generelt.
%navnligt den funktion i OpenCV der er buggy, GoodFeaturesToTrack og
%Perona and Malik fra Octave. Det følgende afsnit om Floodfill bør være
%en del af dette afsnit, ligesom vi skal have lignende beskrivelser af
%sløring og kant detektion.
}

\subsection{Kantdetektion}
% Denne fil er inkluderet i udtraekning_af_regioner.tex
{
Beskrivelse af kantdetektion (med billeder).

\subsubsection*{Metode}

\subsubsection*{Eksempler}

\begin{figure}[!h]
    \begin{center}
        \includegraphics[width=0.8\textwidth]{afsnit/vores_implementation/billeder/kantdetektion/canny_20_20}
    \end{center}
    \caption[]{Canny kantdetektion med tærskelværdierne $(20, 20)$.}
    \label{bathers}
\end{figure}
}

% vim: set tw=72 spell spelllang=da:


\subsection{Sløring}
% Denne fil er inkluderet i udtraekning_af_regioner.tex
{
Sløring, som kommer fra det engelske ord ``blur'', er en gruppering af
filtre som bruges til at fjerne støj og uregelmæssigheder i billeder. Vi
så i afsnit \ref{subsec_floodfill}, at metoden floodfill nogle gange kan
have svært ved at fylde hele regionen ud. Specielt i figur
\ref{dot_ff_var_7_7} ses at himlen har små huller. En sløring af
billedet kan hjælpe med at glatte farverne ud, således at vi dækker mere
af regionen. Sløring af billedet kan også hjælpe til at fjerne diverse
artefakter, såsom revner eller pletter i billedet. Specielt i
vores testbillede, der som tidligere nævnt er malet med en masse
prikker, er det en stor hjælp at sløre billedet, så farverne bliver mere
ensartede. Vi vil nu se på tre forskellige måder at opnå dette på.

De to første et såkaldte lav-pas filtre som har svært ved at bibeholde
kanterne i et billede, men arbejder til gengæld direkte på billedet. Den
tredie metode bibeholder til en vis grad kanterne bedre, men kan ikke
arbejde direkte på billedet, hvilket kræver et større pladsforbrug.

\subsubsection*{Simpel sløring}

\subsubsection*{Gaussisk sløring}
Ved Gaussisk sløring ind gør der 3 steps, først at finde en kernel,
multiplicere kernel rigtigt på billedet, kaldet Convolution. Udtrak
værdigerne fra multiplikationen i et nyt billedet, som er det sløret
billedet.

\subsubsection*{Kernel}
En kernel er en lille matrice, som betegner hvor stor del af billedet og
hvor meget af pixels værdiger for farverne rund om et punkt, skal tages
med i udregningen for den nye sløret farve.

\subsubsection*{Convolution}
Convoluting er en simpel matematisk metode, som beskriver hvordan man
kan gange 2 matricer sammen som ikke har sammen mål, men samme
dimission.

\begin{figure}[h]
	\begin{center}
		\includegraphics[scale=1,angle=0]{afsnit/vores_implementation/billeder/sloering/convolution}
	\end{center}
	\caption[]{Et billedet og en kernel}
	\label{Convolution}
\end{figure}

Som man kan se i figur \ref{Convolution} er selve billedet og kernel i
vid forskellige størrelser. Måden Convolution virker på at man
multiplicere kernelen ovenpå de værdiger som ligger rund om den pixel,
som vi gerne vil finde den sløret farve af og tager gennemsnittet af den
værdig, f.eks farven på pixel $(4,4)$, bliver

\begin{equation}
	(4,4) = ((3,3)*4+(4,3)*2+(5,3)*3+(3,4)*3+(4,4)*9+(5,4)*3+(3,5)*2+(4,5)*4+(5,5)*9)*\frac{1}{39} 
\end{equation}

I dette eksempel, er punktet $(5,5)$ og $(4,4)$ vægtet højre i forhold til de andet, så farven vil blive sløret i den retning. Dette skyldes at kernelen er bygget tilfældig op, i en rigtige sløring er kernelen meget specifikt udvalgt for at give det beste resultat.
 
\subsubsection*{Gaussian kernel}
Måde Gaussisk sløring bygger sin kernel op på, at ved brug af Gaussian
fordelings formlen, som bliver brugt utallige steder i den matematiske
verden, som en anderkedt og respektere fordelings metode. XXX(ved ikke
om jeg skal komme ind på selve gaussan fordelige, da dette snilt kan
komme op på den del)


\begin{equation}
	G(x,y) = \frac{1}{2\pi\sigma^2}e^{-\frac{x^2+y^2}{2\sigma^2}}
\end{equation}

Hvis $\sigma = 1$ og vi lader x og y køre fra -2 til 2, med steps size på 1, og multiplicere så den laveste værdig kommer til at være 1 og runder værdien af. Få vi en kernel, se firgur \ref{gauss}.

\begin{figure}[h]
	\begin{center}
		\includegraphics[scale=0.5,angle=0]{afsnit/vores_implementation/billeder/sloering/gauss}
	\end{center}
	\caption[]{Kernel for gauss sløring}
	\label{gauss}
\end{figure}

Denne kernel kan så bruges ved hjælp af concolution til at danne det sløret billet, som vi kan se i figur \ref{gaussian_metode}

\subsubsection*{Sløring ved statistisk median}
Den sidste metode vi vil nævne er i grunden meget simpel. Grundidéen er
at finde den statistiske median i pixelværdierne rundt om en givet pixel
og tildele medianværdien til denne. Givet et antal pixels, er det
trivielt at sætte dem i en liste og sortere dem efter deres værdi. Hvis
antallet af elementer i listen er ulige, er det midterste element i den
sorterede liste medianen. Er der et lige antal elementer i listen,
defineres medianen som gennemsnittet af de to midterste elementer.
Pixels vælges i et $N \times M$ vindue med den originale pixel i
centrum, som vist i figur \ref{red_box_nxm} og vi vil således altid have
et ulige antal elementer i listen.

\begin{figure}[!h]
    \centering
    \subfloat[]{\label{red_box_nxm}
        \includegraphics[scale=0.42,angle=0]{afsnit/vores_implementation/billeder/sloering/red_pixel_box}
    }\\
    \subfloat[]{\label{3_3_vindue}
        \renewcommand{\arraystretch}{1.8}
        \begin{tabular}{|c|c|c|}
            \hline
            35 & 98  & 23 \\\hline
            48 & \cellcolor[gray]{0.5}42 & 0 \\\hline
            8  & 12   & 29 \\\hline
        \end{tabular}
        }\hspace{1em}
    \subfloat[]{\label{sorteret_median}
        \renewcommand{\arraystretch}{1.5}
        \centering
        \begin{tabular}{|c|c|c|c|c|c|c|c|c|}
            \hline
            0 & 8 & 12 & 23 & \cellcolor[gray]{0.5}29 & 35 & 42 & 48 & 98\\\hline
        \end{tabular}
        }
        \caption[]{
            Bestemmelse af median for pixel med koordinaterne $(2, 2)$.
            \textbf{\ref{red_box_nxm})} Pixels i et $3\times3$ vindue
            omkring $(2, 2)$ er markeret med rødt.
            \textbf{\ref{3_3_vindue})} Værdierne i $3\times3$ vinduet.
            Det ses den originale pixel har værdien $42$.
            \textbf{\ref{sorteret_median})} Den sorterede liste med
            værdierne fra vinduet. Det ses at medianen har værdien
            $29$. Den originale pixel vil da skifte værdi fra $42$ til
            $29$.
        }
\end{figure}

Denne metode kan ikke køres direkte på det originale billede, da dette
vil interferere med fastsættelse af medianen for alle pixels. Man må
derfor oprette en kopi af det originale billede og sætte de fundne
medianværdier i denne. Man finder således altid medianen i forhold til det
originale billede.

\subsubsection*{Eksempler}

% Hold on, this is figure-madness
\begin{figure}[!h]
    \centering
    \subfloat[Original]{\label{simple_original}\includegraphics[angle=0,width=0.3\textwidth]{afsnit/vores_implementation/billeder/sloering/original}}\hspace{1em}
    \subfloat[$3 \times 3$ vindue]{\label{simple_3_3}\includegraphics[angle=0,width=0.3\textwidth]{afsnit/vores_implementation/billeder/sloering/simple_3_3}}\hspace{1em}
    \subfloat[$7 \times 7$ vindue]{\label{simple_7_7}\includegraphics[angle=0,width=0.3\textwidth]{afsnit/vores_implementation/billeder/sloering/simple_7_7}}
    \caption[]{
        \textbf{\ref{simple_original})} Zoom af detajler i det originale billede.
        \textbf{\ref{simple_3_3})}
        \textbf{\ref{simple_7_7})}
    }
    \label{simple_metode}
\end{figure}

\begin{figure}[!h]
    \centering
    \subfloat[Original]{\label{gaussian_original}\includegraphics[angle=0,width=0.3\textwidth]{afsnit/vores_implementation/billeder/sloering/original}}\hspace{1em}
    \subfloat[$3 \times 3$ vindue]{\label{gaussian_3_3}\includegraphics[angle=0,width=0.3\textwidth]{afsnit/vores_implementation/billeder/sloering/gaussian_3_3}}\hspace{1em}
    \subfloat[$7 \times 7$ vindue]{\label{gaussian_7_7}\includegraphics[angle=0,width=0.3\textwidth]{afsnit/vores_implementation/billeder/sloering/gaussian_7_7}}
    \caption[]{
        \textbf{\ref{gaussian_original})} Zoom af detajler i det originale billede.
        \textbf{\ref{gaussian_3_3})}
        \textbf{\ref{gaussian_7_7})}
    }
    \label{gaussian_metode}
\end{figure}

\begin{figure}[!h]
    \centering
    \subfloat[Original]{\label{median_original}\includegraphics[angle=0,width=0.3\textwidth]{afsnit/vores_implementation/billeder/sloering/original}}\hspace{1em}
    \subfloat[$3 \times 3$ vindue]{\label{median_3_3}\includegraphics[angle=0,width=0.3\textwidth]{afsnit/vores_implementation/billeder/sloering/median_3_3}}\hspace{1em}
    \subfloat[$7 \times 7$ vindue]{\label{median_7_7}\includegraphics[angle=0,width=0.3\textwidth]{afsnit/vores_implementation/billeder/sloering/median_7_7}}
    \caption[]{
        \textbf{\ref{median_original})} Zoom af detajler i det originale billede.
        \textbf{\ref{median_3_3})} Median med et vindue på $3\times{}3$.
        Farverne er blevet mere ensartede mens kanterne stadig er
        skarpe.
        \textbf{\ref{median_7_7})} Median med et vindue på $7\times{}7$. Farverne
        er meget ensartede, men det ses at kanterne er blevet mere
        udvisket med det større vindue.
    }
    \label{median_metode}
\end{figure}

}

% vim: set tw=72 spell spelllang=da:


\subsection{Floodfill}
\subsection*{Floodfill}

\subsubsection*{Teorien}
Floodfill omhandler udfyldning af områder i et billet som har samme fave eller en fave som ligger inde få en vis afvigelse fra den orginale fave. Det vil sige, der væljes en pixel i billet, denne pixel har faven (r,g,b). Ud fra denne pixel i billdet, ses der på de nabo pixels som ligger dirgunale og vertikale, hvis nogle af de dirgunale eller vertikale pixels har en fave som ligger inde for en vis afvigelse af faven, faves de. Ud fra de nye favet pixels, bliver helle procedyren gendtaget, ind til at der ikke er flere pixels som kan faves.

\subsubsection*{Implementation}
I Implementationen af floodfill, er der visse ting man kan stille på for at få de resultat som er ønsket. Man kan stille på hvor meget fave må variere med. Man kan sætte floodfill til at regne variansen af fave ud fra den pixel som er valt fra start, eller fra fave af den pixel som functionen er komme til. hved at regne varianden ud fra start pixlen, få metode til at indskranke sig en del, og ikke komme inde i alle hjørner af en blob, tilgændgel har metode svære hved at tegne over kanter og komme ind i en anden område, Denne måde at bruge floodfilll på, gørd også at metoden ikke har store udsving på hvor meget der bliver favet, ud fra favens varians. Hved at arbejde ud fra metoden som regner på den nye pixels fave, få man en metode som få meget af bloben med, og som kan overskue at en blob kan skifte fave langsomt i forhold til solens position eller små skift i bloblen. tilgængel er denne framgang måde ret følsom, og kommer der ved til at flyde over bloben som den er i gang med at udfylde.  

\subsubsection*{Overvejelser}
For at få denne metode til at virke på 25000 billeder, hvor en del af billederne ikke har samme fave tone eller er blevet falmet. Må der udregnes, for vært billedet, hvad for en varians i fave der skal bruges.


\subsection{Programmeringssprog og biblioteker}
{
{\sffamily Ved valg af programmeringssprog har vi først og fremmest lagt
vægt på at kunne udarbejde en prototype hurtigt og bruge et sprog, som
er let at gå til. Det valgte sprog skal også gøre det nemt at udvide den
endelige implementation. Vi har også gerne villet undgå at skulle
konstruere komplicerede datastrukturer for relativt simple metoder, både
af hensyn til tidspresset og til implementationens kompleksitet. Af
ovenstående grunde har vi besluttet at udarbejde vores løsning i
programmeringssproget \textbf{Python}, da netop dette sprog er yderst
velegnet at skrive forholdsvis avancerede prototyper i. Python er
ydermere meget fleksibelt med hensyn til datastrukturer og byder
umiddelbart på en lang række, for problemstillingen relevante, pakker.

(Skal man skrive noget om Pythons udbredelse, anerkendelse og brug?)
}

\subsection{OpenCV}
Til udførelse af billedmanipulationer benytter vi os af et bibliotek skrevet
i C og C++, der hedder \emph{OpenCV}. Biblioteket er udviklet af Intel
og tilbyder, udover et solidt udvalg af algoritmer, bindinger til
Python.  Endelig er det meget veldokumenteret og giver referencer til
publikationer om bibliotekets algoritmer. Biblioteket er udviklet med
specielt henblik på real-tids behandling af billeder, f.eks. med et
videokamera som kilde, men det egner sig også til brug på enkelte
billeder.  \emph{OpenCV} tilbyder mange brugbare datastrukturer med
hensyn til arbejdet med billeder i Python.

Der er også andre biblioteker til billedbehandling i Python. Her kan
nævnes \emph{PIL} (Python Image Library) og \emph{PythonMagick}
(ImageMagick bindings), men de er ikke nær så grundige som
\emph{OpenCV}.

\subsubsection{Andre muligheder}
Der er to helt oplagte muligheder, med hensyn til programmeringssprog,
når man taler om billedbehandling, nemlig Matlab og dets Open
Source-alternativ Octave. Disse sprog blev dog valgt fra, da vores
samlede erfaring med udvikling i disse sprog ikke var stor nok.
Endvidere finder vi, at disse sprog, på trods af, at de især egner sig
til den type beregninger, vi skal lave, er besværlige at lave større
programmer med. Matlab og Octave er dog blevet brugt til at sammenligne
resultater og teste alternative metoder med.

Da \emph{OpenCV} er skrevet i C/C++, ville det også være oplagt at bruge
et af disse sprog. Vores erfaring er dog, at man let kommer til at bruge
mere tid på at konstruere de fornødne datastrukturer og hjælpemetoder,
end på at fokusere på opgavens kerne. En senere implementation, med
fokus på køretid, kunne med fordel implementeres i C/C++, da man så
ville have fuld kontrol over, hvilke strukturer der bliver brugt i
programmet.

\subsection{Værktøjer til databasen}
Vi bruger \textbf{SQLite} til selve databasen, hovedsagelig fordi der ikke
kræves nogen videre konfiguration af en sådan database. Den
underliggende database er dog underordnet, da vi bruger Python-pakken
\emph{SQLObject}, som giver et abstraktionslag til en bred vifte af
databaser. Vi opretter blot de tabeller, vi ønsker at have i databasen,
som klasser i Python og får ligeledes en sådan klasse tilbage, når der
laves forespørgsler til databasen. Da \emph{SQLObject} klarer al
kommunikation med databasen, er det derfor muligt at skifte den
underliggende database ud, hvis man ønsker det. SQLite har endvidere den
umiddelbare fordel, at selve databasen eksisterer som en fil i
filsystemet.  Det er derfor en let sag at tage sikkerhedskopier af
databasen uden alt for meget besvær.

\subsection{Andre værktøjer}
Vi gør også brug af statistikprogrammet \textbf{R} til at behandle og
præsentere vores resultater.

}

% vim: set tw=72 spell spelllang=da:


}

% vim: set tw=72 spell spelllang=da:


\section{Opbevaring af data}
{
{\sffamily I forbindelse med analyse på store datasæt, spiller den
bagvedliggende database en central rolle. Det er fra databasen, vi
henter billeder ind til analyse og samtidig også her, resultaterne
bliver gemt. Vi præsenterer herunder det databaseskema, som databasen er
opbygget efter.  Efterfølgende kaster vi et nærmere blik på de enkelte
tabeller i databasen, hvor vi først vil kigge på, hvordan vi opbevarer
maleriernes metadata, og endelig på hvordan resultater bliver opbevaret.
}

\subsection{Databaseskema}
Tabellerne \ref{artistTable}, \ref{paintingTable}, \ref{runTable},
\ref{resultTable} og \ref{regionTable} herunder, udgør databaseskemaet.
Der er i alle henseender lagt vægt på at eliminere redundans og på
muligheden for senere udvidelse.

\begin{table}[!h]
    \centering
    \begin{tabular}{|l||c|c|c|c|c|c|}
        \hline
        \bf{artist} \hspace{0.5cm} & \underline{artistId} & name & born & died & school & timeline \\\hline
    \end{tabular}
    \caption{Databasetabel for kunstner.}
    \label{artistTable}
\end{table}

\begin{table}[!h]
    \centering
    \begin{tabular}{|l||c|c|c|c|c|c}
        \hline
        \bf{painting} \hspace{0.5cm} & \underline{paintingId} & artistId & title & date & paint & $\cdots$ \\\hline
    \end{tabular}\\ \vspace{0.2cm}\hspace{1.2cm}
    \begin{tabular}{c|c|c|c|c|c|c}
        \hline
        $\cdots$ & material & location & url & form & type & $\cdots$ \\\hline
    \end{tabular}\\ \vspace{0.2cm}\hspace{1.4cm}
    \begin{tabular}{c|c|c|c|c|c|}
        \hline
        $\cdots$ & realHeight & realWidth & height & width & filepath \\\hline
    \end{tabular}
    \caption{Databasetabel for malerier.}
    \label{paintingTable}
\end{table}

\begin{table}[!h]
    \centering
    \begin{tabular}{|l||c|c|c|c|c|c|c|}
        \hline
        \bf{run} \hspace{0.5cm} & \underline{runId} & trsh1 & trsh2 & lo & up & marginPercentage & method \\\hline
    \end{tabular}
    \caption{Databasetabel for en kørsel.}
    \label{runTable}
\end{table}

\begin{table}[!h]
    \centering
    \begin{tabular}{|l||c|c|c|c|c|c|}
        \hline
        \bf{result} \hspace{0.5cm} & \underline{resultId} & runId & paintingId & cutRatio & cutNo & numberOfRegions \\\hline
    \end{tabular}
    \caption{Databasetabel for resultater.}
    \label{resultTable}
\end{table}

\begin{table}[!h]
    \centering
    \begin{tabular}{|l||c|c|c|c|c|c|c|}
        \hline
        \bf{region} \hspace{0.5cm} & \underline{regionId} & resultId & x & y & height & width & area \\\hline
    \end{tabular}
    \caption{Databasetabel for regioner.}
    \label{regionTable}
\end{table}

\subsection{Metadata og billeder\label{section_opbv_billeder}}
Vi starter med at kigge på, hvordan vi opbevarer maleriernes metadata.
Denne information findes i tabellerne \texttt{artist} (tabel
\ref{artistTable}) og \texttt{painting} (tabel \ref{paintingTable}).
Disse to tabeller lægger vægt på, at man let skal kunne forespørge
databasen ved en lang række parametre, såsom et maleris fysiske
størrelse og kunstnerens fødselsår. Vi har, at en kunstner kan være
tilknyttet ét eller flere malerier, og at der, til et givet maleri, kun
kan være én kunstner. Billederne, vi vil analysere, hentes fra et online
kunstarkiv og gemmes på filsystemet i en mappestruktur, der ligner den i
arkivet. Her inddeles filerne i mapper navngivet efter kunstner.
Mapperne inddeles efter forbogstav. På denne måde undgås det, at to
billeder tildeles samme filnavn. Vi gemmer kun stien til en fil på
filsystemet i databasen. Filstrukturen er grafisk illustreret i figur
\ref{mappestruktur}.

% Mappestruktur
\begin{figure}[!h]
    \centering
$
\xymatrix{
 &  &   & \ar @{-} [d] \textrm{/res}  &                                                     \\
 &  &   & \ar @{-} [d] \textrm{/wga.hu}  &                                                  \\
 &  &   & \ar @{-} [dl] \ar @{-} [d] \ar @{--} [dr] \textrm{/art} &                         \\
 &  & \ar @{-} [dl] \ar @{-} [d] \ar @{--} [dr] \textrm{/a} & \textrm{/b} & \cdots          \\
 & \ar @{-} [dl] \ar @{-} [d] \textrm{/aachen} & \ar @{--} [d] \textrm{/abadia} & \cdots    \\
\textrm{allegory.jpg} & \textrm{bacchus.jpg} & \cdots &   &
}
$
    \caption{Mappestruktur til filer fra
        \href{http://www.wga.hu}{http://www.wga.hu}.}
    \label{mappestruktur}
\end{figure}

\subsection{Resultater fra kørsler\label{section_results}}
Når vi har trukket regioner ud af billedet --- jvf. afsnit
\ref{section_udtraek} --- og vurderet dem efter den naive algoritme
givet i afsnit \ref{section_naiv}, står vi tilbage med et egentligt
resultat. Vi ønsker at gemme dette resultat i databasen, så vi på et
senere tidspunkt kan bruge det i en samlet analyse af resultaterne. Det
øvrige databaseskema, der udgøres af tabellerne \texttt{run} (tabel
\ref{runTable}), \texttt{result} (tabel \ref{resultTable}) og
\texttt{region} (tabel \ref{regionTable}), lægger vægt på at kunne gemme
data fra flere forskellige kørsler med forskellige parametre og mulighed
for at genskabe kørte analyser. Vi vil nu kigge på betydningen af de
ovenstående tabeller og se på, hvordan tabellerne giver mulighed for
designmålene.

Hvis vi vil genskabe et fundet resultat, har vi brug for at vide hvilke
parametre, vi har brugt for at komme frem til resultatet. Til dette
formål har vi tabellen \texttt{run} (tabel \ref{runTable}), som
beskriver en kørsel. Denne tabel holder de parametre, som er fælles for
alle billeder i en kørsel, bl. a. de tærskelværdier, som bruges til
kantdetektion (\texttt{trsh1} og \texttt{trsh2}), samt nedre og øvre
grænse for floodfill (\texttt{lo} og \texttt{up}). Her holdes også en
procentsats for, hvor stor et margin vi bruger i udvælgelse af
regioner. Endelig har vi et felt, der angiver, hvilken metode der er
blevet brugt til at finde resultatet. Dette er en tekststreng og vil i
tilfældet af den naive algoritme være sat til \texttt{'naive'}. Felterne
\texttt{trsh1}, \texttt{trsh2} og \texttt{marignPercentage} er
repræsenteret som floats i databasen, mens \texttt{lo} og \texttt{up}
står som heltal. Afslutningsvis har hver kørsel et unikt \emph{id}, så
at vi kan tilknytte indgange i tabellen \texttt{result} (tabel
\ref{resultTable}) til et sæt af parametre.

Vi beskriver et resultat som det antal af regioner, vi får ud fra
analysen af et snit på et givet billede. Som beskrevet i afsnit
\ref{section_opdeling} vil vi, givet en snitratio, typisk søge regioner
i nærheden af fire snit. En undtagelse er, hvis snitratioen deler
billedet i to lige store dele, og vi vil kun i dette tilfælde have to
snit at se på. Tabellen \texttt{result} fortæller os hvilket af de
mulige snit vi har med at gøre, hvilket billede resultatet er
tilknyttet, hvilke parametre der er blevet brugt, samt hvor mange
regioner vi har fundet. Tabellen gør det muligt at gemme resultater fra
kørsler med forskellige parametre, hvorved man kan have data fra
separate kørsler i databasen. Man har da grundlag for at sammenligne
kørsler med forskellige metoder og parametre.

Tabellen \texttt{region} (tabel \ref{regionTable}) holder alle de fundne
regioner fra vores analyse. Hver region henviser til det resultat, som
denne tilhører. Vi kan således skelne de enkelte regioner fra hinanden
og afgøre, i hvilket snit af billedet de ligger. En region bliver
repræsenteret som dens areal og begrænsende rektangel.

\subsection{Vurdering}
Databaseskemaet har været underlagt følgende designmål:

\begin{itemize}
    \item Minimering af redundans
    \item Mulighed for senere udvidelse
    \item Mulighed for rekonstruktion af kørte analyser
    \item Mulighed for at kunne analysere flere snit i én kørsel
\end{itemize}

Vi har dog feltet \texttt{numberOfRgions} i tabellen \texttt{result} som
på sin vis er redundant, da vi blot kan finde antallet af tilknyttede
regioner ved at bruge simple SQL-sætninger. At have antallet stående
direkte i databasen giver dog et umiddelbart bedre overblik over
resultaterne: Når databasen vokser, som følge af ændringer på
parametrene, vil SQL-sætningerne skulle søge meget af databasen igennem
for at returnere simple forespørgselser. Derfor har vi valgt at gemme
antallet at fundne regioner direkte i databasen.

Databaseskemaet er let at udvide, hvis vi, i udviklingen af mere
avancerede metoder, skulle få brug for flere indgange. Det skal også
bemærkes, at tabellerne \ref{runTable}, \ref{resultTable} og
\ref{regionTable} er specifikke for netop den naive algoritme.
Tabellerne er kun tilknyttet maleriernes metadata ved feltet
\texttt{paintingId}. Man kan derved let udvide databasen til at
indeholde data om kørsler, hvor en anden metode til udtrækning af
regioner, eller til bedømmelse af samme, har været brugt.

}

% vim: set tw=72 spell spelllang=da:


\section{Implementation}
%% Bemærk:
%%          Programmeringssprog skrives med stort begyndelsesbogstav (første gang fed, (hvis gennemgående))
%%          Pakker skrives med kursiv (\emph{})
{
{\sffamily Dette kapitel har til formål at gennemgå, hvordan
principperne fra kapitel \ref{chap_indledning} og metoderne fra kapitel
\ref{chap_detektion} er blevet implementeret, samt hvilke problemer, der
kan opstå i denne forbindelse. Vi vil også komme ind på, hvordan
resultater --- som dem vist i kapitel \ref{chap_afproevning} ---
repræsenteres og gemmes, samt på hvordan vi kører analysen på malerierne
i vores database.  Vi forklarer programmet, og metoderne deri, ``fra
bunden og op'', dvs. at vi bevæger os fra lille kompleksitet, hvor der
lægges ud med de helt grundlæggende strukturer og principper, til stor
kompleksitet, når de enkelte dele sættes sammen.  I bilag
\ref{appendix_struktur} og \ref{appendix_vejledning} er vedlagt en
oversigt over programmets opdeling og en kort brugervejledning.
Indledningsvis kigger vi på, hvilke programmeringssprog og biblioteker
vi benytter os af.
}

\section{Programmeringssprog og biblioteker\label{section_programmeringssprog}}
{
{\sffamily Ved valg af programmeringssprog har vi først og fremmest lagt
vægt på at kunne udarbejde en prototype hurtigt og bruge et sprog, som
er let at gå til. Det valgte sprog skal også gøre det nemt at udvide den
endelige implementation. Vi har også gerne villet undgå at skulle
konstruere komplicerede datastrukturer for relativt simple metoder, både
af hensyn til tidspresset og til implementationens kompleksitet. Af
ovenstående grunde har vi besluttet at udarbejde vores løsning i
programmeringssproget \textbf{Python}, da netop dette sprog er yderst
velegnet at skrive forholdsvis avancerede prototyper i. Python er
ydermere meget fleksibelt med hensyn til datastrukturer og byder
umiddelbart på en lang række, for problemstillingen relevante, pakker.

(Skal man skrive noget om Pythons udbredelse, anerkendelse og brug?)
}

\subsection{OpenCV}
Til udførelse af billedmanipulationer benytter vi os af et bibliotek skrevet
i C og C++, der hedder \emph{OpenCV}. Biblioteket er udviklet af Intel
og tilbyder, udover et solidt udvalg af algoritmer, bindinger til
Python.  Endelig er det meget veldokumenteret og giver referencer til
publikationer om bibliotekets algoritmer. Biblioteket er udviklet med
specielt henblik på real-tids behandling af billeder, f.eks. med et
videokamera som kilde, men det egner sig også til brug på enkelte
billeder.  \emph{OpenCV} tilbyder mange brugbare datastrukturer med
hensyn til arbejdet med billeder i Python.

Der er også andre biblioteker til billedbehandling i Python. Her kan
nævnes \emph{PIL} (Python Image Library) og \emph{PythonMagick}
(ImageMagick bindings), men de er ikke nær så grundige som
\emph{OpenCV}.

\subsubsection{Andre muligheder}
Der er to helt oplagte muligheder, med hensyn til programmeringssprog,
når man taler om billedbehandling, nemlig Matlab og dets Open
Source-alternativ Octave. Disse sprog blev dog valgt fra, da vores
samlede erfaring med udvikling i disse sprog ikke var stor nok.
Endvidere finder vi, at disse sprog, på trods af, at de især egner sig
til den type beregninger, vi skal lave, er besværlige at lave større
programmer med. Matlab og Octave er dog blevet brugt til at sammenligne
resultater og teste alternative metoder med.

Da \emph{OpenCV} er skrevet i C/C++, ville det også være oplagt at bruge
et af disse sprog. Vores erfaring er dog, at man let kommer til at bruge
mere tid på at konstruere de fornødne datastrukturer og hjælpemetoder,
end på at fokusere på opgavens kerne. En senere implementation, med
fokus på køretid, kunne med fordel implementeres i C/C++, da man så
ville have fuld kontrol over, hvilke strukturer der bliver brugt i
programmet.

\subsection{Værktøjer til databasen}
Vi bruger \textbf{SQLite} til selve databasen, hovedsagelig fordi der ikke
kræves nogen videre konfiguration af en sådan database. Den
underliggende database er dog underordnet, da vi bruger Python-pakken
\emph{SQLObject}, som giver et abstraktionslag til en bred vifte af
databaser. Vi opretter blot de tabeller, vi ønsker at have i databasen,
som klasser i Python og får ligeledes en sådan klasse tilbage, når der
laves forespørgsler til databasen. Da \emph{SQLObject} klarer al
kommunikation med databasen, er det derfor muligt at skifte den
underliggende database ud, hvis man ønsker det. SQLite har endvidere den
umiddelbare fordel, at selve databasen eksisterer som en fil i
filsystemet.  Det er derfor en let sag at tage sikkerhedskopier af
databasen uden alt for meget besvær.

\subsection{Andre værktøjer}
Vi gør også brug af statistikprogrammet \textbf{R} til at behandle og
præsentere vores resultater.

}

% vim: set tw=72 spell spelllang=da:


\section{Billedbehandling med OpenCV\label{section_impBilledbehandling}}
{
{\sffamily I det følgende kaster vi et blik på detaljerne i programmets
billedbehandlingsmetoder. Efter en teknisk introduktion til digitale
billeder vises det, hvilke datastrukturer og metoder \emph{OpenCV}
stiller til rådighed, samt hvordan disse bruges til at udtrække regioner
i digitale billeder. Endelig vil vi undersøge, hvilke svagheder vores
implementering til udtrækning af regioner har.
}

\subsection{Digitale billeder}
I afsnit \ref{section_kort_intro} blev der givet en kort introduktion
til den digitale repræsentation af billeder. Det blev antaget, at en
pixel kunne antage værdier i mængden $\{0, 1\}$, men i praksis kan
pixels godt antage andre værdier. Vi arbejder med billeder, hvor værdien
for hver pixel er repræsenteret ved tre 8 bit størrelser, hver med
værdier i mængden $\{0, 1, 2, \cdots, 254, 255\}$. Sammensætningen af de
tre værdier, som beskrives som kanaler eller farvebånd, kaldes for en
RGB-farve, hvor tallene repræsenteret ved $(R,G,B)$ angiver mængden af
hhv. rød, grøn og blå farve i en pixel. Et sådant billede, kaldes for et
RGB-billede. Vores korpus består af sådanne RGB-billeder med 8 bits, og
vi benytter derfor også denne repræsentation internt. Der er enkelte
undtagelser, hvor der kun bruges én kanal, således at vi arbejder med
gråtonebilleder. For en uddybning af billeders repræsentation henvises
til \cite{SIOlsen}.

Vi har også tidligere, i ligning \ref{billede_matrix},  vist, at et
digitalt billede kan skrives som en matrix. Ved denne matrix kan man få
værdien for en pixel med koordinater $(x,y)$ ved elementet
$\mathbf{I}_{xy}$. Vi har altså, at x-aksen går fra øverste venstre
hjørne i nedadgående retning og y-aksen fra øverste venstre hjørne mod
højre. \emph{OpenCV} bruger dog et andet koordinatsystem, hvor akserne
er vendt om, som vist i figur \ref{opencv_koordinatsystem}. Man skal
derfor huske på, at matricen, som\emph{OpenCV} bruger, er transponeret,
og vi skal derfor bytte om på koordinaterne. For at få værdien på
pixelen med koordinaterne $(x,y)$, skal vi altså bruge værdien
$\mathbf{I}_{yx}$. Bemærk de omvendte koordinater.

\begin{figure}[!b]
    \centering
    \begin{picture}(122,55)
        \put(61, 50){$x$}
        \put(-10, 22){$y$}
        \put(0, 45){\circle*{3}}
        \put(-1, 45){\vector(1, 0){120}}
        \put(0, 45){\vector(0, -1){48}}

    \end{picture}
    \caption[]{Koordinatsystemet, som bliver brugt i \emph{OpenCV}.}
    \label{opencv_koordinatsystem}
\end{figure}

Billedmatricen $\mathbf{I}$ repræsenteres i \emph{OpenCV} som et
dobbelt-array. I figur \ref{opencv_billedematrix} er vist, hvordan denne
struktur ser ud, når vi har indlæst billedet givet i figur
\ref{billede_pixels}. Hvis billedet er indlæst i variablen \texttt{I},
kan vi i Python tilgå pixelen med koordinater $(x,y) = (0,2)$ ved
$\textbf{I}_{2,0} =~$\texttt{I[2][0]}.

\begin{figure}[t]
    \begin{verbatim}
                        I = [ [0, 1, 0],
                              [1, 1, 1],
                              [0, 1, 0] ]
    \end{verbatim}
\vspace{-2em}
\caption{Repræsentation af billedmatricen fra ligning
\ref{billede_pixels} i \emph{OpenCV}.}
\label{opencv_billedematrix}
\end{figure}

\subsection{Resultaters struktur\label{resultat_struktur}}
Indledningsvist vil vi introducere to vigtige datastrukturer, som vi
bruger fra \emph{OpenCV}. Alle datastrukturer og metoder fra
\emph{OpenCV} har præfikset ``cv'', hvilket gør det let at skelne vores
egne metoder fra dem, der er givet i \emph{OpenCV}. Vi præsenterer
desuden en notation for datastrukturer, der er underlagt strukturen vist
i \eqref{types_class}.
\begin{multline}
    \textbf{class~} [\textit{name}] = \{ \\
    \shoveleft{\qquad[\textit{type}] : [\textit{varName}]} \\
    \shoveleft{\}}\shoveright{}
    \label{types_class}
\end{multline}
I \eqref{example_class} ses et eksempel på en struktur kaldet
\textbf{ExampleClass}.
\begin{multline}
    \textbf{class~} \textrm{ExampleClass} = \{ \\
    \shoveleft{\qquad\textbf{int} : \textit{intValue}} \\
    \shoveleft{\qquad\textbf{string} : \textit{stringValue}} \\
    \shoveleft{\qquad\textbf{int[2]} : \textit{arrayValues}} \\
    \shoveleft{\}}\shoveright{}
    \label{example_class}
\end{multline}
Bemærk, at en strukturs navn bliver skrevet med fed skrift i
brødteksten, når der refereres til den. Ligeledes vil en strukturs navn
blive skrevet med fed i pseudokoden, hvis der refereres til den. Vi kan
definere en ny struktur, \textbf{NewClass}, som illustrerer dette, i
\eqref{new_class} herunder.
\begin{multline}
    \textbf{class~} \textrm{NewClass} = \{ \\
    \shoveleft{\qquad\textbf{ExampleClass} : \textit{exampleInstance}} \\
    \shoveleft{\qquad\textbf{string} : \textit{stringValue}} \\
    \shoveleft{\}}\shoveright{}
    \label{new_class}
\end{multline}
Med ovenstående notation kan vi nu beskrive to vigtige strukturer fra
\emph{OpenCV} kaldet \textbf{cvRect} og \textbf{cvConnectedComp}.

\subsubsection{cvRect}
Denne struktur beskriver et rektangel ved at angive dets øverste venstre
hjørne og dets dimensioner. Strukturen vises i \eqref{cvRect_class}.
\begin{multline}
    \textbf{class~} \textrm{cvRect} = \{ \\
    \shoveleft{\qquad\textbf{int} : \textit{x}} \\
    \shoveleft{\qquad\textbf{int} : \textit{y}} \\
    \shoveleft{\qquad\textbf{int} : \textit{height}} \\
    \shoveleft{\qquad\textbf{int} : \textit{width}} \\
    \shoveleft{\}}\shoveright{}
    \label{cvRect_class}
\end{multline}
Implementeringen bruger denne struktur til at angive en regions
begrænsende rektangel. Vi vil i afsnit \ref{section_vurdering_regioner}
se, hvordan strukturen \textbf{cvRect} bruges i vurderingen af regioner.

\subsubsection{cvConnectedComp}
\emph{OpenCV} har implementeret floodfill-metoden, beskrevet i afsnit
\ref{subsec_floodfill}, som gør brug af en datastruktur kaldet
\textbf{cvConnectedComp}. Denne struktur bruges til at beskrive den
region i billedet, som metoden fylder ud.  Strukturen er vist i
\eqref{cvConnectedComp_class}.
\begin{multline}
    \textbf{class~} \textrm{cvConnectedComp} = \{ \\
    \shoveleft{\qquad\textbf{double} : \textit{area}} \\
    \shoveleft{\qquad\textbf{float} : \textit{value}} \\
    \shoveleft{\qquad\textbf{cvRect} : \textit{rect}} \\
    \shoveleft{\}}\shoveright{}
    \label{cvConnectedComp_class}
\end{multline}
Strukturen indeholder en regions begrænsende rektangel og regionens
areal.

\subsubsection{Resultater}
Nu introduceres en ny struktur, som repræsenterer Pythons
\emph{dictionary}, forkortet \emph{dict}.  En \emph{dict} viser vi som i
\ref{def_dict}.
\begin{eqnarray}
    \langle[\textit{dictName}]\rangle = \{ [\textit{key}] : [\textit{data}] \}
    \label{def_dict}
\end{eqnarray}
Som et simpelt eksempel kan vi konstruere en \emph{dict}
$\angles{LuckyNumbers}$, som indeholder personers lykketal:
% TeX-Gods, please forgive me :(
\begin{multline}
    \angles{LuckyNumbers} = \{ \qquad \textrm{Tom Cruise} : 4 , \\
    \textrm{Arthur Dent} : 42 \qquad\qquad\\
    \shoveleft{\}}\shoveright{}
    \label{lucky_dict}
\end{multline}
% In honor of the danish Tom Cruise
Data i en \emph{dict}, kan godt være af mere komplicerede typer, f. eks.
kan man have en anden \emph{dict}. Man kan således lave en hierakisk
struktur, som egner sig til opbevaring af fundne regioner i et billede.
Hierakiet er illustreret ved grammatikken i
\eqref{resultat_hieraki}.

\begin{equation}
    \begin{split}
        \textit{ImageRegions}  & \to  (\textit{RatioRegions~ImageRegions}')\\
        \textit{ImageRegions}' & \to  \textit{RatioRegions~ImageRegions}'\\
        \textit{ImageRegions}' & \to  \\
        \textit{RatioRegions}  & \to  (\textit{CutRegions~CutRegions~RatioRegions}')\\
        \textit{RatioRegions}' & \to  \textit{CutRegions~CutRegions}\\
        \textit{RatioRegions}' & \to  \\
        \textit{CutRegions}    & \to  (\textbf{CV\_RGB}, \textbf{cvConnectedComp})\textit{~CutRegions} \\
        \textit{CutRegions}    & \to
    \end{split}
    \label{resultat_hieraki}
\end{equation}

Vi har da, at et billede kan have et endeligt antal snitratioer, som vi
ønsker at undersøge. Disse snitratioer har enten to eller fire snit,
hvor der i hvert snit er et antal fundne regioner med en tilhørende
farve.

Hver region i et snit bliver tildelt et \emph{ID}, som er en
strengrepræsentation af regionens farve.  Hvis det antages, at regionens
farve er hvid, vil den have RGB-værdien $(255, 255, 255)$, og den får
derfor tildelt $ID = \textrm{'255255255'}$. Vi konstruerer nu
$\angles{CutRegions}$, som indeholder de fundne regioner for et snit.
$\angles{CutRegions}$ er vist herunder i \eqref{CutRegions_dict}.
\begin{multline}
    \angles{CutRegions} = \{ \textit{~RegionId} : \\
    (\textbf{CV\_RGB~}\textit{color}, \textbf{cvConnectedComp~}\textit{region}) \}\quad
    \label{CutRegions_dict}
\end{multline}

\noindent I $\angles{CutRegions}$ bruges regionens ID som nøgle. Vi kan nu
konstruere den \emph{dict} på niveauet over, som vi betegner
$\angles{RatioRegions}$. Den ses i \eqref{RatioRegions_dict}.
\begin{eqnarray}
    \angles{RatioRegions} = \{ \textit{~CutNo} : \angles{CutRegions} \}
    \label{RatioRegions_dict}
\end{eqnarray}

\noindent $\angles{RatioRegions}$ holder alle regioner for en givet
snitratio.  Hvert snit, tilknyttet en snitratio, blev i afsnit
\ref{section_opdeling} tildelt et ID i mængden $\{0,1,2,3\}$. Vi bruger
snittets ID som nøgle.  Data i $\angles{RatioRegions}$ er den instans af
$\angles{CutRegions}$, som tilhører snittet.

Det sidste niveau i hierakiet fra figur \ref{resultat_hieraki} er endnu
en \emph{dict}, givet ved $\angles{ImageRegions}$, vist i
\eqref{ImageRegions_dict}.
\begin{eqnarray}
    \angles{ImageRegions} = \{ \textit{~CutRatio} : \angles{RatioRegions} \}
    \label{ImageRegions_dict}
\end{eqnarray}

\noindent $\angles{ImageRegions}$ har nøglen $CutRatio$, som angiver en
snitratio til et billede. Hver snitratio har et antal snit, som endelig
har et antal fundne regioner. Vi har således fået beskrevet den
struktur, som resultater bliver repræsenteret ved.

Resultatet fra en analyse på et billede, med snitratioerne $0.5$ og
$0.618$, ligner da det nedenstående:
%% !!! Bwadr-tex !!!
\begin{multline}
    \angles{ImageRegions} = \{ 0.5 : \{ 0 : \{ \textrm{'012345678'} : (color, region) \},\\
                                            \{ \textrm{'123456789'} : (color, region) \}
                                            \},\\
                                            \{ 1 : \{ \textrm{'012345678'} : ( \cdots ) \}, 
                                            \{ \cdots \} \}, \\
                              0.618 : \{ 0 : \{ \cdots \} \}, \{ 1 :
                              \cdots \}, \{ 2 : \cdots \}, \{ 3 : \cdots
                              \} \\
    \shoveleft{\}}\shoveright{}
    \label{CutRegions_dict}
\end{multline}

\subsection{Udtrækning af regioner}
I figur \ref{graphic_pipeline} er givet den række af manipulationer, et
billede skal igennem i programmet før regionerne i et givet snit kan
trækkes ud.  Resten af dette afsnit, vil omhandle den egentlige
implementering af fremgangmåden samt de problemer, der knytter sig til
den. Indledningsvist vises, hvordan snit i billedet repræsenteres.
\begin{figure}[!h]
    \includegraphics[width=\textwidth]{afsnit/implementation/billeder/billedbehandling/pipeline.png}
    \caption{Skema over grafiske modifikationer i forbindelse med
    udtrækning af regioner.}
    \label{graphic_pipeline}
\end{figure}

\subsubsection{Repræsentation af snit i billedet}
Metoderne i dette afsnit er ikke specifikke for snitratioen $\varPhi$,
men kan overføres til ethvert andet snit i billedet med en arbitrær
snitratio.

Snit i et billede bliver repræsenteret ved et linjestykke. Vi
præsenterer nu datastrukturen \textbf{Line}, som består af et
linjestykkes to endepunkter.
\begin{multline}
    \textbf{class~} \textrm{Line} = \{ \\
    \shoveleft{\qquad\textbf{cvPoint} : \textit{p1}} \\
    \shoveleft{\qquad\textbf{cvPoint} : \textit{p2}} \\
    \shoveleft{\}}\shoveright{}
    \label{Line_class}
\end{multline}

\subsubsection{Præparation af billedet}
Det første skridt i præparation af billedet er, at detektere kanterne på
dets objekter.  Kanter detekteres i et sort/hvid-billede, så vi starter
med at lave en sort/hvid kopi af det originale billede. Inden den
egentlige kantdetektion, sløres det sort/hvide billede, således at vi
kun betragter kanter, som fremstår tydeligt i billedet.

I andet skridt af fremgangsmåden sløres det originale billede. Både i
sløringingen inden kantdetektion og denne bruges simpel sløring.

Sidste skridt er fremhævelse af de detekterede kanter. Det gøres ved at
oprette et nyt billede med samme størrelse som originalbilledet. Alle
pixels i det nye billede farves sorte. Dernæst kopieres alle pixels fra
det originale billede over i det sorte billede, med undtagelse af de
pixels, hvor der er detekteret en kant. Derved forbliver kanterne sorte.
At lade kanterne være sorte kan have nogle utilsigtede konsekvenser. Vi
henviser til afsnit \ref{subsec_svagheder} for en gennemgang af disse.

Præparation af et billede er fremstillet som pseudokode i kodeboks
\ref{pseudo_prepare}.

\begin{lstlisting}[caption={Pseudokode for metoder til præparation af
    billeder.},captionpos=b,label={pseudo_prepare},numbers=left,
    frame=tb, breaklines=false, float=h]
def GetEdges(original, threshold1, threshold2):
    # Create images
    gray = cvCreateImage(cv.cvGetSize(original), 8, 1)
    out  = cvCreateImage(cv.cvGetSize(original), 8, 1)

    # Convert to B/W
    cvCvtColor(original, gray, CV_BGR2GRAY)

    # Simple blur
    cvSmooth(gray, out, CV_BLUR, 3, 3, 0)

    # Edge detect
    cvCanny(gray, out, threshold1, threshold2)

    return out

def EnhanceEdges(img, edges):
    workImage = cvCreateImage(cv.cvGetSize(original), 8, 3)

    # Superimpose the edges onto the image
    cvNot(edgeImage, edgeImage)
    cvCopy(img, workImage, edges)

    # Get the edges back to white
    cvNot(edges, edges)

    return workImage

def Preprocessing(original, thresholds):
    # Find edges
    edgeImage = GetEdges(original, threshold1, threshold2)

    # Blur
    blurImage = cvCreateImage(cv.cvGetSize(original), 8, 3)
    cvSmooth(original, blurImage, CV_BLUR, 3, 3, 0)

    # Enhance edges
    enhanced = EnhanceEdges(blurImage, edges)

    return (enhanced, edges)

\end{lstlisting}

\subsubsection{Segmentering med floodfill-metoden}
% Plan
% Vi optimerer, maler ikke alle pixel
% Vi bruger det kantdetekterede billede
% Vi laver linjestykker
% Vi bemærker at ikke hele regionen bliver fyldt ud
% Vi maler to gange, men det er en fejl
% Vi retter koden
Vi beskriver nu metoden til at trække regioner ud af et præpareret
billede. Metoden bygger videre på beskrivelsen givet i
\ref{sammensaetning_af_metoder}, hvor floodfill-metoden bruges på hver
pixel langs et snit. I praksis er denne fremgangsmåde meget langsom,
især med store billeder. Vi vil derfor komme frem til en metode, som kun
bruger floodfill-metoden på de nødvendige pixels.

Vi bruger de detekterede kanter til at hjælpe med at segmentere
billedet, da vi vil bruge floodfill med en ny farve, når vi passerer en
kant. Vi trækker derfor de punkter ud, hvor en kant krydser det givne
snit, ved metoden vist i kodeboks \ref{pseudo_GetPoints}.

\begin{lstlisting}[caption={Pseudokode for metode til at finde punkter,
    hvor en kant krydser det givne snit.},captionpos=b,label={pseudo_GetPoints},numbers=left,
    frame=tb, breaklines=false, float=h]
def GetPoints(edges, cut):
    points = []
    for pixel on cut:
        if not (color(pixel) == 0):
            points += pixel
\end{lstlisting}

Vi repræsenterer snittet som et linjestykke, og vi kan således dele dette
i mindre segmenter mellem kanter. Et eksempel på et horisontalt snit er
vist i figur \ref{impUdtraek_kantpunkter}.

\begin{figure}[!h]
    \centering
    \begin{picture}(240,30)
        \put(0, 10){$A$}
        \put(3, -5){\line(0, 1){10}}

        \put(82, 10){$e_1$}
        \put(85, 0){\circle*{3}}

        \put(102, 10){$e_2$}
        \put(105, 0){\circle*{3}}

        \put(134, 10){$e_3$}
        \put(137, 0){\circle*{3}}

        \put(158, 10){$e_4$}
        \put(161, 0){\circle*{3}}

        \put(200, 10){$e_5$}
        \put(203, 0){\circle*{3}}

        \put(221, 10){$e_6$}
        \put(224, 0){\circle*{3}}

        \put(233, 10){$B$}
        \put(236, -5){\line(0, 1){10}}

        \put(3, 0){\line(1, 0){233}}
    \end{picture}
    \caption[]{Punkter, hvor der er en kant, der krydser snittet.}
    \label{impUdtraek_kantpunkter}
\end{figure}

Vi vil nu bruge floodfill således, at hvert linjestykke bliver tildelt
en tilfældig farve, som endnu ikke er blevet givet til et andet
linjestykke. Hvert linjestykke males derefter med denne. Naivt ville man
bruge floodfill på midten af hvert linjestykke, men dette er ikke
holdbart, da vi ikke kan være sikre på, at hele linjestykket farves.
Figur \ref{floodfill_taerskel_problem} viser, at det ikke er lige meget
\emph{hvor} på linjestykket, man bruger floodfill, når vi har med
arbitrære billeder at gøre. Vi vil derfor gå langs snittet og bruge
floodfill på alle pixels, som endnu ikke er blevet farvet af
floodfill-metoden.  Hver gang, vi bruger floodfill, returneres en
instans af strukturen \textbf{cvConnectedComp}.

Vi skal også være sikre på, at floodfill returnerer hele den fundne
region. Den returnerede \textbf{cvConnectedComp} er den \emph{senest
farvede} region. I figur \ref{floodfill_return_entire_region} vises en
situation, hvor vi ikke har farvet hele linjestykket, og når vi udvider
den fundne region, returnerer floodfill kun et undersæt af alle pixels i
regionen. Vi skal derfor bruge floodfill mere end én gang for at sikre
os, at hele regionen returneres. Vi har dertil udviklet pseudokoden i
kodeboks \ref{pseudo_udtraek_org}.

\begin{figure}[h]
    \setlength\fboxsep{0pt}
    \setlength\fboxrule{0.5pt}
    \centering
    \fbox{\includegraphics[width=0.8\textwidth]{afsnit/implementation/billeder/billedbehandling/floodfill_color.png}}
    \caption[]{Problemet med floodfill-metoden i billeder med flere
    farver. Hvid farve er ikke blevet malet af floodfill-metoden endnu.
    Den lyseblå region dækker ikke hele linjestykket ned til kanten.}
    \label{floodfill_taerskel_problem}
\end{figure}

\begin{lstlisting}[caption={Original pseudokode til udtrækning af
    regioner. Denne kan returnere den samme region flere
    gange.},captionpos=b,label={pseudo_udtraek_org},numbers=left,
    frame=tb, breaklines=false, float=h]
for lineSegment in Cut:
    # Get a new color that is not in the component dictionary
    color = getRandomColor()
    region = cvConnectedComp()

    for pixel in lineSegment:

        # Check if the color of the pixel equals current color
        if not (color(pixel) ==  color):

            # Check if the color of the pixel are in the saved regions
            if not (color(pixel) in CutRegions):
                cv.cvFloodFill(img, pixel, color, lowerThres, upperThres, region)

    # Color the last pixel again to make sure that
    # the returned component is the entire region
    cv.cvFloodFill(img, pixel, color, lowerThres, upperThres, region)

    # Put the results in the CutRegions-dictionary
    CutRegions[color.toString()] = (color, region)
\end{lstlisting}

\begin{figure}[p]
    \setlength\fboxsep{0pt}
    \setlength\fboxrule{0.5pt}
    \centering
    \subfloat[En tilføjelse til den lyseblå region. Det begrænsende
    rektangel svarer kun til udvidelsen.]{
        \label{new_reg_small_box}
        \fbox{\includegraphics[angle=0,width=0.8\textwidth]{afsnit/implementation/billeder/billedbehandling/floodfill_color_new_reg_small_box}}
        }\\
    \subfloat[Ved anvedelse af floodfill på regionen igen, markeres hele den
    lyseblå region som ønsket.]{
        \label{new_reg_big_box}
        \fbox{\includegraphics[angle=0,width=0.8\textwidth]{afsnit/implementation/billeder/billedbehandling/floodfill_color_new_reg_big_box}}
        }
    \caption[]{
    Floodfills opførsel ved udvidelse af regioner.
    }
    \label{floodfill_return_entire_region}
\end{figure}

Vi har i kodeboks \ref{pseudo_udtraek_org} forsøgt at tage højde for
problemet, nævnt i figur \ref{new_reg_small_box}, ved et ekstra kald til
\texttt{cvFloodFill} i linje 17. Dette vil farve den sidste pixel en
ekstra gang for at sikre, at hele regionen returneres. Kaldet gør dog
også, at der \emph{altid} bliver returneret en region for hvert segment.
Dette er ikke ønskværdigt, specielt ikke hvis vi er sprunget over alle
pixels på et linjestykke. Dette sker netop, når hele regionen allerede
er blevet fyldt ud. I praksis betyder dette, at der kan blive fundet
flere regioner i et billede, end der reelt er. I bilag \ref{appendix_bug}
er givet en håndkørsel af den fejlende kode, og i afsnit
\ref{program_bug} ses på hvordan dette påvirker resultaterne.

Fejlen i linje 17 har eksisteret selv under vores kørte eksperiment i
afsnit \ref{section_naiv_koersel}. Metoden er senere blevet rettet til
at have den korrekte opførsel, og den reviderede metode er vist i
kodeboks \ref{pseudo_udtraek_rev}.

\begin{lstlisting}[caption={Revideret pseudokode til udtrækning af
    regioner. Returnerer ingen
    duplikater.},captionpos=b,label={pseudo_udtraek_rev},numbers=left,
    frame=tb, breaklines=false, float=h]
for lineSegment in Cut:
    # Get a new color that is not in the component dictionary
    color = getRandomColor()

    # Set region to None, as we have not yet found any
    region = None

    for pixel in lineSegment:

        # Check if the color of the pixel equals current color
        if not (color(pixel) ==  color):

            # Check if the color of the pixel are in the saved regions
            if not (color(pixel) in CutRegions):
                # Now we've got a new region
                region = cvConnectedComp()
                cv.cvFloodFill(img, pixel, color, lowerThres, upperThres, region)

                # Color the pixel again to make sure that
                # the returned component is the entire region
                cv.cvFloodFill(img, pixel, color, lowerThres, upperThres, region)

    # If we have found a region, then put the result in the CutRegions-dictionary
    if not (region is None):
        CutRegions[color.toString()] = (color, region)
\end{lstlisting}

Metoden i kodeboks \ref{pseudo_udtraek_rev} bruger \texttt{cvFloodFill}
to gange, hver gang man møder en ikke-farvet pixel. Dette koster lidt
køretid men sikrer, at man altid får hele regionen returneret i
\textbf{cvConnectedComp}.

%Man kan fristes til at flytte kaldet i linje
%21 ind i \texttt{if}-sætningen i linje 24, men dette åbner op for
%svagheden igen, da vi ikke ved, om den sidste pixel tilhører den
%aktuelle region. Derfor er vi nødt til at ofre lidt køretid, for at
%være sikre på resultatet. Metoden benytter et ``først til
%mølle''-princip, hvor en pixel, når den først et blevet tilknyttet en
%region, altid vil tilhøre denne.

Vi trækker dog kun de regioner ud som rører \emph{snittet}. I kapitel
\ref{chap_detektion} indførtes et margin, således at også regioner, som
ligger tæt på snittet, kan trækkes ud. Vi vil nu udvide pseudokoden i
kodeboks \ref{pseudo_udtraek_rev} til også at udtrække regioner, som
krydser margin. Vi navngiver metoden \texttt{GetCutRegions}, og den
tager et snit som argument. Metoden ses i kodeboks
\ref{pseudo_udtraek_margin} og trækker først regioner ud med hensyn til
det nedre margin, så med hensyn til det øvre margin og til sidst med
hensyn til selve snittet. Alle regioner gemmes i den samme instans af
$\angles{CutRatios}$. For selve udregningen af margin henvises til
afsnit \ref{subsec_margin_udregning}.

\begin{lstlisting}[caption={Pseudokode til udtrækning af regioner med
    margin.},captionpos=b,label={pseudo_udtraek_margin},numbers=left,
    frame=tb, breaklines=false, float=h]
def GetCutRegions(img, edges, cut):
    # Calculate lowerMargin and upperMargin
    (lowerMargin, upperMargin) = calculateMargins(cut)
    Cuts = [lowerMargin, upperMargin, cut]

    # Initialize an empty CutRegions-dict
    CutRegions = {}

    for Cut in Cuts:
        # Use the edge image to find the line segments
        for lineSegment in Cut:
            # Get a new color that is not in the component dictionary
            color = getRandomColor()

            # Set region to None, as we have not yet found any
            region = None

            for pixel in lineSegment:

                # Check if the color of the pixel equals current color
                if not (color(pixel) ==  color):

                    # Check if the color of the pixel are in the saved regions
                    if not (color(pixel) in CutRegions):
                        # Now we've got a new region
                        region = cvConnectedComp()
                        cv.cvFloodFill(img, pixel, color,
                                    lowerThres, upperThres, region)

                        # Color the pixel again to make sure that
                        # the returned component is the entire region
                        cv.cvFloodFill(img, pixel, color,
                                    lowerThres, upperThres, region)

            # If we have found a region,
            # then put the result in the CutRegions-dictionary
            if not (region is None):
                CutRegions[color.toString()] = (color, region)

    return CutRegions
\end{lstlisting}

Vi kan nu kombinere alle metoderne, således at vi kan trække regioner ud
af arbitrære billeder. Til dette bruges pseudokoden vist i kodeboks
\ref{pseudo_udtraek_all}.

\begin{lstlisting}[caption={Fuld udtrækning af regioner i et arbitært
    billede.},captionpos=b,label={pseudo_udtraek_all},numbers=left,
    frame=tb, breaklines=false, float=h]
def NaiveExtraction(original):
    # Preprocess the image
    (enhanced, edges) = Preprocessing(original, thresholds)

    # Get regions
    return GetCutRegions(enhanced, edges, cut)
\end{lstlisting}

\subsection{Svagheder\label{subsec_svagheder}}
Den endelige metode for udtrækning af regioner, med hensyn til et givet
snit i billedet, har nogle svagheder. Den første er allerede nævnt:
Metoden trækker kun regioner ud \emph{med hensyn} til et snit. Dette
betyder, at kun regioner, som rører margin eller selve snittet, bliver
trukket ud. Foregående skal tages helt bogstaveligt, i den forstand, at
vi \emph{skal} have, at en region har en pixel enten på det nedre
margin, det øvre margin eller på selve snittet for at blive trukket ud.
Vi kan altså godt have interessante regioner, som faktisk har mindst én
kant af deres begrænsende rektangel inden for margin, men som ikke
bliver trukket ud. Et eksempel er vist i figur \ref{respect_to_cut},
hvor den sorte region ikke vil blive trukket ud, selvom den har to
kanter inden for margin.

\begin{figure}[h]
    \setlength\fboxsep{0pt}
    \setlength\fboxrule{0.5pt}
    \centering
    \fbox{\includegraphics[width=0.8\textwidth]{afsnit/implementation/billeder/billedbehandling/respect_to_cut.png}}
    \caption[]{Sort region, som ikke bliver trukket ud af billedet.
    Selvom regionens begrænsende rektangel ligger inden for margin, så
    krydser regionen hverken nedre margin, øvre margin eller selve
    snittet. Derfor opdages regionen ikke.}
    \label{respect_to_cut}
\end{figure}

\subsubsection{Valg af tilfældige RGB-værdier}
Man skal endvidere være opmærksom på, at hver gang vi markerer en ny
region, tildeles denne en tilfældig farve.  Vi kan, i vores valg af
farve, være uheldige at vælge en, som bliver brugt i det originale
billede. Når vi kalder \texttt{cvFloodFill} igen, for at være sikre på, at
hele regionen bliver returneret, kan vi smelte to regioner sammen, som
egentlig ikke burde hænge sammen. I figur \ref{floodfill_colors} er
denne situation vist på det præparerede billede fra figur \ref{bathers}.

Ligeledes kan vi være uheldige at støde på en pixel, som har en farve
lig med en allerede farvet region. I dette tilfælde vil denne pixel
blive anset som værende del af en eksisterende region, hvilket
resulterer i at vi ikke bruger \texttt{cvFloodFill} på denne. Vi kan dog
håbe, at denne pixel bliver inkluderet i den efterfølgende iteration.
Vi vælger aldrig, at farve en region med en farve, som allerede er brugt
til en anden region, men vi kan være uheldige og vælge en farve, som
ligger inden for floodfill-metodens tilladte afvigelse, således at to
regioner smeltes sammen.

\begin{figure}[h]
    \setlength\fboxsep{0pt}
    \setlength\fboxrule{0.5pt}
    \centering
    \subfloat[Originalt præpareret billede inden vi vælger en tilfældig
    farve til floodfill.]{
        \label{colors_1}
        \fbox{\includegraphics[width=0.3\textwidth]{afsnit/implementation/billeder/billedbehandling/pre_floodfill_1}}}\hspace{1em}
    \subfloat[Region fyldes med farve lig eksisterende pixels i
    billedet. Pixels i drengens hår har nu samme farve som kroppen.]{
        \label{colors_2}
        \fbox{\includegraphics[width=0.3\textwidth]{afsnit/implementation/billeder/billedbehandling/pre_floodfill_2}}}\\
    \subfloat[Når floodfill bruges en ekstra gang, for at være sikker på
    at hele regionen returneres, smeltes drengens krop og hår sammen.]{
        \label{colors_3}
        \fbox{\includegraphics[width=0.3\textwidth]{afsnit/implementation/billeder/billedbehandling/pre_floodfill_3}}}\hspace{1em}
    \subfloat[Hvis den første region havde fået en anden farve, ville vi
    ikke have smeltet to regioner sammen.]{
        \label{colors_4}
        \fbox{\includegraphics[width=0.3\textwidth]{afsnit/implementation/billeder/billedbehandling/pre_floodfill_4}}}\\
    \caption[]{Uheldige valg af tilfældig farve til regioner.}
    \label{floodfill_colors}
\end{figure}

\subsubsection{Ikke-sammenhængende regioner}
Metoden i kodeboks \ref{pseudo_udtraek_margin}, trækker højest én region
ud per segment. Dette giver mening, set i lyset af hvad
udtrækningsmetoden er blevet udviklet til, nemlig udtrækning af
\emph{sammenhængende regioner} afgrænset ved kantdetektion. I afsnit
\ref{section_computer_betragter} antog vi, at interessante regioner i et
billede, er tydeligt afgrænset. En region er ikke tydeligt afgrænset,
hvis vi faktisk \emph{har} flere regioner indenfor et segment. I figur
\ref{usammenhaengende_region} betragter vi en hypotetisk situation, hvor
en region ikke har været tydeligt afgrænset. Billedet i figur
\ref{usammenhaengende_region}, illustrerer et billede, som allerede er
blevet segmenteret ved metoden i kodeboks \ref{pseudo_udtraek_margin}.
De sorte cirkler markerer dér, hvor vi har fundet kanter, som krydser
snittet. Vi har altså fem segmenter på snittet. På det længste segment,
ser vi, at den mørkeblå region er blevet afbrudt af en tidligere fundet
region. Den mørkeblå region er således ikke sammenhængende. Metoden
returnerer kun den nederste mørkeblå region, da vi ikke kan forbinde
de to mørkeblå regioner. Vi mister derfor informationen om at der
faktisk befinder sig en region i midten af billedet.

Denne opførsel, er en konsekvens af vores antagelse, om at interessante
regioner er tydeligt afgrænset. Vi accepterer derfor, at situationen i
figur \ref{usammenhaengende_region} kan forekomme, og at vi derfor kan
undlade at trække enkelte regioner ud.

\begin{figure}[t]
    \setlength\fboxsep{0pt}
    \setlength\fboxrule{0.5pt}
    \centering
    \fbox{\includegraphics[width=0.8\textwidth]{afsnit/implementation/billeder/billedbehandling/usammenhaengende_region.png}}
    \caption[]{Et segmenteret billede. Sorte cirkler angiver steder,
    hvor vi har detekteret en kant, der krydser snittet. Den mørkeblå
    region er usammenhængende over snittet, da den afbrydes af den røde
    region. Kun den nederste mørkeblå region er blevet returneret.}
    \label{usammenhaengende_region}
\end{figure}

\subsubsection{Usikkerhed ved fremhævelse af kanter}
De fremhævede kanter, tegnes i det slørede billede med sort farve, som
vælges uanset hvilke farver, der er brugt i maleriet i forvejen. Vi kan
derfor støde på problemer med mørke malerier eller lokalt i mørke
områder af et billede. Vi kan også være uheldige, at komme til at
forbinde to regioner, ved at markere kanterne med sort.

På grund af, at kanterne fremhæves ved at male i billedet, kan vi, ved
floodfill-metoden, komme til at male kanter, som har samme retning, som
det snit vi betragter. Dvs, at lodrette kanter kan blive malet ved
vertikale snit og omvendt kan vandrette kanter risikere at blive malet
ved horisontale snit.

\subsection{Andre tilgange og forbedringer}
Dette afsnit vil kort nævne andre fremgangsmåder, vi har prøvet, for at
trække regioner ud af et billede. Vi har tidligt i udviklingen af
programmet, eksperimenteret både med Octave og Matlab. Til at starte
med, har vi brugt disse sprog til at komme problemstillingen nærmere,
og udvikle naive implementationer af enkelte algoritmer. Vi ser nu på
hvad der kan gøres for at forbedre udtrækningen af regioner.

\subsubsection{Udtrækning af alle regioner}
At vi kan have regioner som ikke trækkes ud af billedet, som i figur
\ref{respect_to_cut}, er beklagelig, men kan løses ved pseudokoden i
kodeboks \ref{pseudo_fix}.  Her er tanken, at man i stedet for kun at
trække regioner ud på margin og selve snittet, så gøres det på hver
pixel mellem margin. På denne måde kan regioner, som den i figur
\ref{respect_to_cut}, trækkes ud af billedet. Vores implementation gør
dog ikke dette. Når regioner kun trækkes ud, med hensyn til et snit, kan
vi ved denne metode heller ikke garantere en fuld segmentering af
billedet.

\begin{lstlisting}[caption={Pseudokode til udtrækning af regioner med
    margin.},captionpos=b,label={pseudo_fix},numbers=left,
    frame=tb, breaklines=false, float=h]
def FixGetRegions(img, edges, cut):
    # Calculate lowerMargin and upperMargin
    (lowerMargin, upperMargin) = calculateMargins(cut)

    for pixel in (lowerMargin to upperMargin):
        # Extract regions
        pass
\end{lstlisting}

Vi har også eksperimenteret med, først at skalere billedet ned, inden
man trak regioner ud. Dette gør nemlig, at kanterne forbliver intakte,
mens farverne, til en vis grad, bliver mere ensartede. Vi stødte dog på
problemer når billedet skulle skaleres op igen, eller rettere, få
resultaterne skalleret op, så de passede til originalbilledet. Vi mister
også meget præcision, når vi bruger et nedskaleret billede til at finde
regioner i.


I præparationen af billedet, vil vi gerne have, at farverne i billedet
bliver ensartede, men vi ønsker også at bibeholde kanterne. Vi forsøgte
at bruge en metode i \emph{OpenCV}, der hedder
\texttt{cvPyrMeanShiftFiltering}, som netop segmenterer billedet efter
hvilke farver der ligner hinanden, men en fejl i biblioteket forsagede
altid en segmenteringsfejl i det underliggende C-program. Vi blev siden
gjort opmærksom på en sløringsmetode af Perona og
Malik\cite{perona1990scale}, hvor det grafiske resultat tilnærmer sig
det ønskede. Kun i Octaves \emph{Image}-pakke er denne implementeret. Vi
lavede en hurtig afprøvning, hvor resultatet fra denne metode blev
analyseret, men på basis at det endelige resultat, vurderede vi, at vi
godt kunne nøjes med at fremhæve kanterne i billedet efter en simpel
sløring.

\subsubsection{Bedre fremhævelse af kanter}

\emph{OpenCV} gør det faktisk muligt, at bruge en maske, så man kan
kalde \texttt{cvFloodFill} med et kantdetekteret billede, således at man
ikke maler over de steder i billedet, hvor der er detekteret en kant. I
praksis viste det sig, at være yderst besværligt at bruge denne
funktion, da det kantdetekterede billede skal være to pixel større, i
hver dimension --- det kantdetekterede billede skal have en ramme, på én
pixel. En meget naiv løsning, hvor pixels kopieres én ad gangen, var
tidskrævende på store billeder og vi beholdt derfor de sorte kanter.
Alternativt kunne man beskære det originale billede, men vi besluttede,
at vi ikke ville manipulere med dimensionerne på vores inddata og bevare
det originale billede.

\subsubsection{Optimeringer}
Et hurtigt kig på arbejdsgangen i figur \ref{graphic_pipeline}, viser
allerede mindst ét sted, hvor vi kan optimere programmet. Billedet bliver
nemlig sløret to steder, både i forbindelse med kantdetektion og den
generelle sløring. Det er derfor oplagt, kun at sløre billedet én gang
og passere det slørede billede videre, som argument til de metoder, som
skal bruge det. Dette kan dog kun gøres, hvis vi ønsker at bruge den
samme sløringsmetode, til de to billeder.

Endvidere, kan selve udtrækningen af regioner måske forbedres, ved at bruge
sløring ved statistisk median. Denne metode har vi dog ikke fået testet
ordentlig igennem. Ved brug af en anden sløringsmetode, kunne det godt
tænkes, at fremhævelse af kanter bliver helt overflødig

}

% vim: set tw=72 spell spelllang=da:


\section{Vurdering af regioner\label{section_vurdering_regioner}}
{
{\sffamily Vi vender nu opmærksomheden mod selve implementationen af de
metoder, som afgør, hvorvidt en udtrukket regionen kan siges at være
interessant, samt hvordan vi har implementeret den naive fremgangsmåde,
som bedømmer, om en region ligger i det gyldne snit. Alle metoder
vedrørende vurdering af regioner er implementeret i filen
\texttt{regionSelector.py} i mappen \texttt{lib/}.  Vi har igen, at
metoderne ikke er specifikke for det gyldne snit, men kan anvendes på
ethvert snit i billedet. Sidst i afsnittet viser vi, hvordan vi har
implementeret den videregående vurdering, hvor regioner bedømmes ud fra
deres massemidtpunkt. Vi starter med at se på en fælles datastruktur der
bruges, når regioner bliver vurderet.
}

\subsection{Datastruktur til betingelser}
Når vi skal afgøre, hvorvidt et antal udtrukne regioner er interessante
og ligger i snittet, er der en række betingelser, der skal være opfyldt.
Til disse, er der forbundet nogle udregninger, som vil være de samme for
hver region. Vi bruger derfor en struktur, som indholder resultaterne
fra disse udregninger, således at de ikke skal udføres for hver eneste
region vi kontrollerer. Datastrukturen kaldes \textbf{Constraints} og
ses herunder i \eqref{Constraints_class}.
\begin{multline}
    \textbf{class~} \textrm{Constraints} = \{ \\
    \shoveleft{\qquad\textbf{int} : \textit{coordinate}} \\
    \shoveleft{\qquad\textbf{double} : \textit{minSize}} \\
    \shoveleft{\qquad\textbf{double} : \textit{minMass}} \\
    \shoveleft{\qquad\textbf{int[}2\delta + 1\textbf{]} : \textit{acceptRange}} \\
    \shoveleft{\}}\shoveright{}
    \label{Constraints_class}
\end{multline}
Variablene \texttt{minSize} og \texttt{minMass} relaterer sig kun til
klassificering af interessante regioner, mens \texttt{coordinate} og
\texttt{acceptRange} hører til klassificering af regioner, liggende i
snittet. De enkelte variable vil, i det følgende, blive forklaret
nærmere, når det er relevant.

\subsection{Interessante regioner}
I afsnit \ref{section_naiv}, blev det defineret, at for at en region,
kan betegnes som værende \textbf{interessant}, skal den
\begin{enumerate}
        \renewcommand{\labelenumi}{(\alph{enumi})}
    \item have et areal større end en tærskelværdi, der sættes i
        forhold til billedets størrelse
    \item have en masse større end en tærskelværdi, der ligeledes,
        sættes i forhold til billedets størrelse,
\end{enumerate}
Bemærk, at vurderingen af interessante regioner, ikke har noget at gøre
med hverken snitratio eller margin. Vi undersøger udelukkende de
udtrukne regioners areal og masse. Regioner bliver vurderet, umiddelbart
efter de er blevet trukket ud, så vi har regionerne til rådighed som en
instans af $\angles{CutRegions}$. Vi definerer nu to metoder; én til at
kontrollere en regions størrelse og én til at kontrollere dens masse. Vi
kalder disse \texttt{checkSize} og \texttt{checkMass}. De kan ses i
kodeboks \ref{pseudo_size_mass}.

\begin{lstlisting}[caption={Metoder til at konstollere en regions
    størrelse og masse.},captionpos=b,label={pseudo_size_mass},
    frame=tb, breaklines=false, float=b]
def checkSize(component, constraints):
    "Test if the component have size greater than the minumum size
    defined by the constraints."
    return component.area >= constraints.minSize

def checkMass(component, constraints):
    "Check if the component have mass greater than the minimum mass
    defined by the contraints."
    rect = component.rect
    mass = component.area/(rect.width * rect.height)
    return mass >= constraints.minMass
\end{lstlisting}

Med metoderne i kodeboks \ref{pseudo_size_mass}, returnerer begge en
sandhedsværdi for hvorvidt en region lever op til betingelserne for en
interessant region, og vi da kontrollere hver enkelt. Vi laver nu en ny
metode, som returnerer en \emph{dict} med kun de interessante regioner.
Metoden, kaldet \texttt{GetInterestingRegions}, ses i kodeboks
\ref{pseudo_GetInterestingRegions}. Den tager, som argument, den instans
af $\angles{CutRegions}$, som returneres fra \texttt{ExtractRegions} i
kodeboks \ref{pseudo_udtraek_margin}.

\begin{lstlisting}[caption={Metode som returnerer kun de insteressante
    regioner, givet en instans af $\angles{CutRegions}$}, captionpos=b,
    label={pseudo_GetInterestingRegions}, frame=tb, breaklines=false,
    float=t]
def GetInterestingRegions(CutRegions, constraints):
    interestingRegions = {}
    for id in CutRegions:
        component = CutRegions[id][1]
        passSizeCheck = checkSize(component, constraints)
        passMassCheck = checkMass(component, constraints)
        if (passSizeCheck and passMassCheck):
            interestingRegions[id] = CutRegions[id]
    return interestingRegions
\end{lstlisting}

I kodeboks \ref{pseudo_GetInterestingRegions} bruger vi en instans af
\textbf{Constraints} som argument til metoderne, som kontrollerer
regionens størrelse og masse. Vi skal derfor, inden metoden
\texttt{GetInterestingRegions} kaldes, have initialiseret vores
betingelser, så de passer til billedet. Vi har i kapitel
\ref{chap_afproevning}, fastsat en procentsats for en regions
minimumareal i forhold til billedets størrelse, og denne
minimumstørrelse findes ved udregningen i \eqref{region_min_size}
herunder.
\begin{equation}
    \mathtt{minSize} =
    \lfloor
    \mathrm{minSizePercentage}\cdot\mathrm{height}\cdot\mathrm{width}
    \rfloor
    \label{region_min_size}
\end{equation}
Ligeledes, har vi fastsat en procentsats for en regions minimummasse, men denne
skal vi ikke regne videre på, da metoden \texttt{checkMass} også regner
en procentsats ud for den givne region. Vi gemmer den fastsatte
procentsats, for regioners minimummasse direkte i vores instans af
\textbf{Constraints}. I \texttt{checkMass} sammenlignes minimummassen
med regionen masse direkte.

\subsubsection{Overvejelser}
Udvælgelsen af interessante regioner, kunne godt, være mere
sofistikeret.  Vi undersøger kun regioner for deres størrelse og masse,
hvilket stadig tillader mange regioner, som egentlig er uinteressante,
pga. deres form.  I udvælgelsen af interessante regioner, kunne man
derfor kigge på regionens form eller udstrækning, ved at undersøge
dennes massemidtpunkt.  Hvis massen er koncentreret langt væk fra
snittet, er denne region ikke interessant. Vi skal dog passe på, at vi
ikke tager beslutninger, som egentlig vedrører, om regionen er placeret
i snittet. Vi ønsker, i denne udvælgelse af regioner, udelukkende at
bestemme, hvorvidt regionen skal tages op til videre overvejelse, for om
denne ligger i snittet.

Kun hvis vores søgning for objekter i billedet, bliver mere specifik,
giver det mening, at undersøge regionerne nærmere i udvælgelsen af
interessante regioner. Vi kan forestille os, en situation, hvor man
udelukkende vil finde ansigter, placeret i det gyldne snit. I dette
tilfælde, skal vi selvfølgelig ikke sende en region til videre
vurdering, hvis denne \emph{ikke} er et ansigt. Vi har dog ikke en sådan
specifik søgning, hvorfor vi kun kan frasortere regioner, ud fra
informationen om deres størrelse og masse.

\subsection{Naiv vurdering af regioner}
Vi vil i dette afsnit se på, hvordan den naive fremgangsmåde, givet i
kapitel \ref{chap_detektion}, vurderer hvorvidt en region ligger i det
gyldne snit. Vi vil dog først give en forklaring på hvordan vores margin
bliver repræsenteret.

\subsubsection{Udregning af margin\label{subsec_margin_udregning}}
Inden vi ser på, hvordan det vurderes, hvorvidt en region er placeret i
et givet snit, skal vi se på hvordan vi egentlig regner margin ud. Som
nævnt, både i kapitel \ref{chap_detektion} og \ref{chap_afproevning},
bruges en procentsats til at angive vores margin. Denne procentsats vil,
hvis vi kun undersøger ét snit, blive sat til $2.4\%$ af billedets højde
eller bredde, alt efter hvilken orientering det aktuelle snit har.
Procentsatsen er baseret på forskellen mellem det gyldne snit og snittet
ved to tredjedele. Vi har implementeret fastsættelsen af denne
procentsats således, at \emph{hvis} man ønsker at sammenligne snit, som
ligger tættere på hinanden end det gyldne snit og to tredjedele, så
findes den procentsats der gør, at disse snits margin ikke overlapper. I
praksis gives en liste med snitratioer, der ønskes undersøgt, og fra
denne liste, findes den mindste differens mellem ratioerne,. Den mindste
differens mellem snitratioer, bruges da, som procentsats for alle snit i
analysen. Hvis den mindste differens, mellem snitrationerne, er større
end $2.4\%$, sættes procentsatsen for margin til $2.4\%$. Vi antager i
det følgende, at vores procentsats for margin er blevet sat til $2.4\%$.

Når vi skal vurdere regioner med hensyn til vores margin, får vi brug
for den eksakte pixelstørrelse på margin. I filen
\texttt{marginCalculator.py} i mappen \texttt{lib/} er der implementeret
metoder til dette. Her bruges metoden \texttt{getPixels}, som, givet et
billede, et snit og en procentsats for margin, returnerer afstanden fra
snittet til margin i pixels, som vi også skriver som $\delta$. Det er
også i \texttt{marginCalculator.py} som finder den mindste differens
mellem snitratioer, ved metoden \texttt{getPercentage}.

\begin{figure}[b]
    \centering
    \begin{picture}(122,55)
        \put(61, 50){$x$}
        \put(-10, 22){$y$}
        \put(0, 45){\circle*{3}}
        \put(-1, 45){\vector(1, 0){120}}
        \put(0, 45){\vector(0, -1){48}}

        \color{red}
        \put(88, 50){\line(0, -1){55}}

        \color{blue}
        \put(84, 50){\line(0, -1){55}}
        \put(92, 50){\line(0, -1){55}}

        \color{black}

        \put(66, 30){$-^{x}$}
        \put(78, 30){\vector(1, 0){20}}

        \put(100, 30){$+^{x} $}
        \put(98, 30){\vector(-1, 0){20}}


    \end{picture}
    \caption[]{Koordinatsystem med indtegnet snit og margin. Bemærk, at
    ved det vertikale snit, skal vi kun betragte regioners $x$-værdi,
    når vi skal bedømme om de ligger inden for margin. Ligeledes kan
    margin repræsenteres kun ved de tilladte $x$-værdier.}
    \label{margin_koordinatsystem}
\end{figure}
Når vi, i den naive vurdering af regioner, skal afgøre om en region
ligger inden for vores margin, kan vi udnytte, at vi enten betragter et
vertikalt eller horisontalt snit.  Når vi har et vertikalt snit, behøver
vi kun at betragte $x$-værdier, da $y$-værdien ikke influerer på det
vertikale snit. Omvendt med horisontale snit, behøver vi kun at betragte
$y$-værdier, da $x$-værdierne, i denne situation, ikke har betydning for
snittet.  Tilfældet, for det vertikale snit, er vist i figur
\ref{margin_koordinatsystem}.  Denne egenskab gør, at vi nu kan oprette
et sæt bestående af netop kun de koordinater som ligger inden for
margin. Hvis vi betragter et vertikalt snit, kan vi i Python oprette
sættet af accepterende $x$-værdier ved at bruge strukturen
\texttt{range}. F.eks. vil \texttt{range(1, 4)} returnere listen
$\{1,2,3\}$.  Vores implementation bruger endvidere variablen
\texttt{coordinate} i \textbf{Constraints}, som, lidt misvisende,
indikerer om snittet vi undersøger er vertikalt eller horisontalt. Er
snittet vertikalt, sættes \texttt{coordinate} til $0$, og $1$ for et
horisontalt snit. For ethvert snit, kan vi da finde de accepterende
$x$-værdier, som vist i kodeboks \ref{pseudo_acceptRange}. Variablen
\texttt{margin}, der tages som argument, er del pixelstørrelse der er
returnet fra metoden \texttt{getPixels} fra
\texttt{marginCalculator.py}.

\begin{lstlisting}[caption={Pseudokode},captionpos=b,label={pseudo_acceptRange},
    frame=tb, breaklines=false, float=t]
GetAcceptRange(cut, margin, coordinate):
    if coordinate:
        # Horizontal cut
        lower_bound = cut.p1.y - margin
        upper_bound = cut.p1.y + margin
        acceptRange = range(lower_bound, upper_bound)
    else:
        # Vertical cut
        lower_bound = cut.p1.x - margin
        upper_bound = cut.p1.x + margin
        acceptRange = range(lower_bound, upper_bound)

    return acceptRange
\end{lstlisting}

\subsubsection{Kontrol på en regions afgrænsende rektangel}
\begin{lstlisting}[caption={Metode, som kontrollerer, hvorvidt en region
    har en kant af det afgrænsende rektangel inden for margin.},
    captionpos=b, label={pseudo_position}, frame=tb, breaklines=false,
    float=b]
def checkPosition(component, constraints):
    "Test if the component have a bounding box inside the accepting
    rectangle defined in the constraints."
    d = component.rect.width
    p = component.rect.x
    if constraints.coordinate:
        d = component.rect.height
        p = component.rect.y

    lowerInRange = p in constraints.acceptRange
    upperInRange = (p + d) in constraints.acceptRange

    return lowerInRange or upperInRange
\end{lstlisting}
Når vi har fundet værdierne i \texttt{acceptRange}, og dermed også
fastfast \texttt{coordinate}, dvs. vi ved hvilken orientering snittet
har, kan vi endelig sige om en interessant region ligger placeret i
snittet eller ej. Vi har allerede udnyttet, at vi kun behøver at
betragte én koordinat, når vi kun har vertikale og horisontale snit. Det
er derfor ligetil at kontrollere, om en regions afgrænsende rektangel,
har en kant inden for margin. Regionen er repræsenteret som en instans
af \textbf{cvConnectedComp}, hvori der er gemt en instans af
\textbf{cvRect}. Vi har en metode, kaldet \texttt{checkPosition}, som,
alt efter om vi har et vertikalt eller et horisontalt snit, returnerer
en sandhedsværdi for om den relevante koordinat findes i de accepterende
koordinater. Metoden ses i kodeboks \ref{pseudo_position}.

Vi kan nu sammensætte en metode som returnerer alle interessante
regioner, med en kant inden for margin. Metoden tager en instans af
$\angles{CutRegions}$ og en instans af \textbf{Constraints} som
argumenter. Vi genererer altså vores betingelser inden vi begynder at
frasortere regioner, således at alle informationer om snittet og krav
for regioner ligge i instansen af \textbf{Constraints}. Den endelige
metode, for vurdering efter den naive fremgangsmåde, kaldes
\texttt{GetInterestingRegionsInCut} og er vist i kodeboks
\ref{pseudo_GetInterestingRegionsInCut}.

\begin{lstlisting}[caption={Pseudokode, som returnerer alle interessante
    regioner, der har en kant, af deres afgrænsende rektangel, inden for
    margin.},
    captionpos=b, label={pseudo_GetInterestingRegionsInCut}, frame=tb, breaklines=false,
    float=t]
def GetInterestingRegionsInCut(CutRegions, constraints):

    # First remove uninteresting regions
    interestingRegions = GetInterestingRegions(CutRegions, constraints)

    # Initialize an empty dict
    interestingRegionsInCut = {}

    # Check every interesting region if it's bounding box
    # has an edge inside the margin
    for id in interestingRegions:
        component = interestingRegions[id][1]
        if checkPosition(component, constraints):
            interestingRegionsInCut[id] = interestingRegions[id]

    # The resulting dict contains only interesting regions
    # with an edge inside the margin
    return interestingRegionsInCut
\end{lstlisting}

\subsection{Udvidet vurdering af regioner}
\subsubsection{Gitter (Grid)}
\subsubsection{Massemidtpunkt}

% vim: set tw=72 spell spelllang=da:


\section{Gennemgang af en kørsel\label{section_koersel}}
{
{\sffamily
For at beskrive integrationen og samspillet mellem en kørsel og
databasen vil der følge en beskrivelse af de vigtige kald mellem 
programmet og databasen. 
}


Filen start.py er distributør af arbejde til de andre dele af programkoden.
Diagram \ref{start_workflow} og pseudokode \ref{pseudo_workflow} giver et overblik over hvad der bliver
forklaret i resten af dette kapitel. 
\begin{figure}[h!]
	\begin{center}
		\includegraphics[scale=0.5]{afsnit/implementation/billeder/workflow_start_py.png}
	\end{center}
	\caption{De blå pile er ting, som sker en enkelt gang, mens de blå
	\label{start_workflow}
	bliver gentaget indtil der ikke er flere billeder at arbejde på}
\end{figure}
\begin{lstlisting}[caption={Pseudokode for
start},frame=tb,label={pseudo_workflow}]
cuts = experiment.generateCuts()
experiment.setSettings(settings)
experiment.setGlobalSettings(globalSettings)
db = Database(globalSettings)
db.construct(Database)
run = m.createNewRun(settings)
paintings = m.Painting.select(m.Painting.q.form=="painting")
for painting in paintings:
	paintingContainer = Painting(painting)
	paintingContainer.setResults(paintingAnalyzer.analyze(paintingContainer,settings))
	m.saveResults(run.id,paintingContainer)
\end{lstlisting}
\subsection{Eksperimenter}
Eksperimenter kontrollere hvilke snit og indstillinger en given kørsel
skal afvikles med.  \ref{pseudo_experiment} er et eksempel
på hvordan et eksperiment evt. kunne se ud.
\begin{lstlisting}[caption={Pseudokode for et
experiment, som checker på $\varPhi$ og $\frac{2}{3}$},frame=tb,label={pseudo_experiment}]
def generateCuts():
	cuts = [goldenLibrary.PHI,2/3]
	return cuts
def setSettings(settings):
	settings.setMarginPercentage(0.024)
	return 0
def setGlobalSettings(globalSettings):
	return 0
\end{lstlisting}
%Settings 
\subsection{Globale indstillinger}
De globale indstillinger er placeringen på databasen og den
kommasepareredefil. Dette er indstillinger, som ikke ændre sig fra
kørsel til kørsel, dog er de vitale for hele programmet.

\subsection{Kørselsindstillinger}
Kørselsindstilinger kan afvige i de forskellige kørsler. De består af
tærskelværdier for floodfill og kantdetektion, hvilken analysemetode,
samt størrelsen på margin\ref{terskelverdi} og de snit, som
skal undersøges. De her variabler betyder meget for resultaterne og derfor er vigtige
at kunne ændre fra kørsel til kørsel. 

\subsection{Initialisering af databasen}
Databasen er konstrueret ved at gennemløbe den kommaseparedefil fra
wga.hu.
Ved at bruge \emph{SQLObject} er det ligetil at konstruere tabeller i
databasen. Vi har i afsnit \ref{section_database} givet det database
skema som vi opbygger databasen efter. I kodeboks
\ref{code_tabel_artist} er vist hvordan tabellen \texttt{artist}
konstrueres ved brug af \emph{SQLObject} i Python.

\begin{lstlisting}[caption={Pythonkode for oprettelse af tabeller i
    databasen.}, captionpos=b, label={code_tabel_artist}, frame=tb,
    breaklines=false, float=hb]
import sqlobject as s

class Artist(s.SQLObject):
    "
    _id_, name, born, died, school, timeline
    "
    name = s.StringCol()
    born = s.IntCol()
    died = s.IntCol()
    school = s.StringCol()
    timeline = s.StringCol()
\end{lstlisting}

Når man vil oprette en ny kunstner i databasen gøres det som vist
i kodeboks \ref{code_new_artist}.

\begin{lstlisting}[caption={Oprettelse af en kunstner i databasen.},
    captionpos=b, label={code_new_artist}, frame=tb, breaklines=false,
    float=h]
# Init variables
name = "Homer Simpson"
born = 1968
died = 2000
school = "Springfield"
timeline = "1950-2000"

# Create the artist in the database
Artist(name=name, born=born, died=died, school=school, timeline=timeline)
\end{lstlisting}

\emph{SQLObject} opretter automatisk et id-felt til alle tabeller i
databasen. Vi kan udnytte dette til at lave \emph{foreign keys} i
tabellerne. Vi viser i kodeboks \ref{code_tabel_result} hvordan tabellen
\texttt{result} oprettes i databasen, hvor det er interessant at bemærke
hvorledes de to \emph{foreign keys} oprettes.

\begin{lstlisting}[caption={Pythonkode for oprettelse af \emph{foreign
    keys} i databasen.}, captionpos=b, label={code_tabel_result}, frame=tb,
    breaklines=false, float=h]
class Result(s.SQLObject):
    "
    _id_, ^runId, ^paintingId, cutRatio, cutNo, numberOfRegions
    "

    run = s.ForeignKey('Run')
    painting = s.ForeignKey('Painting')
    cutRatio = s.FloatCol()
    cutNo = s.IntCol()
    numberOfRegions = s.IntCol()
\end{lstlisting}

Dette sker kun i det tilfælde at databasen ikke findes ved starten af en
kørsel. Den kommasepareredefil bliver parset før enhver kørsel og bliver
linje for linje sammenlignet med databasen for at finde evt. mangler.
Det er værd at bemærke at databasen ikke kan håndtere at blive afbrudt i dette stadie første gang den bliver kørt.
Grunden til dette er at det er en meget overfladisk sammenligning, kun
er billedes placering på wga.hu, som bliver sammenlignet, da denne er
unik.
Omvendt giver det mulighed for at den kommaseparerede fil kan opdateres uden nogen
problemer. Billederne automatisk bliver hentet hvis de mangler
giver det også mulighed for at flytte databasen til en anden
maskine uden nogen problemer.
Der er kun to tilfælde databasen kan håndtere udvidelser, hvis den
kommasepareredefil bliver opdateret og beholder sin nuværende form, og
hvis en testdatabases \texttt{count} variable bliver sat til en højere
værdi, dette tilfælde forklares senere.
Følgende pseudokode forklarer hvad databasen gennemgår hver gang der
startes en ny kørsel.
\begin{lstlisting}[caption={Pseudokode for database
initialisering},frame=tb,label={pseudo_init_db}]
csvfile = open(Settings.csvfilelocation)
for line in csvfile:
	line = parser.parse(line)
	if not os.path.isfile(line.path):
		download(line.url)
	if database.Painting.select(database.Painting.url==line.url).count() == 0:
		database.Painting.insert(line)
\end{lstlisting}
\subsubsection{Parsing af kommasepareret fil}
I den kommaseparfilen gives endvidere mange oplysninger, om den enkelte artikel samt
dennes kunstner.  Vi har konstrueret en parser, som trækker disse
informationer ud fra filen og lægger dem ind i databasen. Da vi primært
vil beskæftige os med malerier, vil vi nu blot omtale kunstartikler som
malerier.

Den konstruerede parser, til den kommaseparerede fil, er dog ret grov,
da folkene bag hjemmesiden \cite{wgahu} ikke har lagt meget vægt på, at være
konsistente i deres formulering af en kunstners fødsels- og dødsår eller
en genstands dimensioner. En følge deraf er, at nogle kunstnere, hvor
hjemmesiden\cite{wgahu} ikke har en klar indikation af dennes levealder, ikke
bliver registreret i databasen. Vi kan dog stadig slå kunstneren op ved
at bruge feltet ``timeline'', som angiver hvilken periode kunstneren
tilhører. Vi har i enkelte tilfælde, set os nødsaget til at rette i den
kommaseparerede fil, hvor der er blevet indsat tegn, som helt umuliggør
korrekt parsing af filen, såsom ekstra komma eller semikolon.

\subsubsection{Testdatabase}\label{test_db}
For at alle kan arbejde på det samme billeder undervejs i udviklingen er
det muligt at konstruere en lille testdatabase, dette gøres ved at sætte
\texttt{testdatabase} variablen til True og 
\texttt{count} variablen til det antal af billeder der ønskes. En fuld beskrivelse kan findes i \ref{brugervejl_test_db}


\subsection{Analysen}
Analysen bliver kørt på alle billeder, som har typen "painting" i den
kommaseparerede fil. Som beskrevet i linje 8-10 i \ref{pseudo_workflow}
så bliver billedet sendt igennem Painting klassen, hvor det bliver
konverteret til et \emph{Opencv} billede. Når det er konverteret sendes
det til paintingAnalyser, hvis formål er at sende det videre til
udtrækning af regioner. Efter alle regioner er trukket ud gemmes
resultaterne i databasen.

\subsection{Genskabelse af parametre og resultater}
At kunne genskabe de fundne resultater fra en analyse har meget stor
betydning, dels for at kunne udtage stikprøver i udviklingen af hele
programmet, men også for at kunne fremvise grafiske resultater. Vi har
allerede været inde på, at man for at kunne genskabe et resultat, skal
vide hvilke parametre der oprindeligt har været brugt. Ovenstående
databaseskema gør det let at hente disse parametre ud. Hvis vi får et
resultat med overraskende mange regioner og gerne vil undersøge dette
tilfælde, har vi metoder til rådighed der giver os lige nøjagtig de
informationer vi har brug for at vise dette grafisk. Helt konkret har vi
metoderne vist i listing \ref{rekonst_koersel} til rådighed.

\vspace{0.5cm}
\begin{lstlisting}[caption={Metoder til rekonstruktion af kørsler},captionpos=b,label={rekonst_koersel},numbers=none]
def getSettingsForRunId(runId):
    """Return the settings instance for a given run"""
    pass

def getCutRatiosForRunId(runId):
    """Return the list of cut ratios for a given run"""
    pass

def getSettingsForResultId(resultId):
    """Return the settings instance for a given result"""
    pass

def getSettingsForRegionId(regionId):
    """Return the settings instance for a given region"""
    pass

def getCutRatioForRegionId(regionId):
    """Return the list of cut ratios for a given region"""
    pass

def getCutNoForRegionId(regionId):
    """Return the cut number for a given region"""
    pass

def getRegionsForResultId(resultId):
    """Return the list of regions for a given result"""
    pass
\end{lstlisting}

Selvom metoderne i listing \ref{rekonst_koersel} ikke viser noget
egentlig kode, bør det ud fra sammenhængen være klart hvad disse metoder
gør. Alle metoder der starter med \texttt{getSettings} returnerer
klassen \texttt{Settings} som vist i listing \ref{settings_klassen} med
indstillinger tilpasset den enkelte forespørgelse.
\vspace{0.5cm}
\begin{lstlisting}[caption={Settings-klassen med standardindstillinger},captionpos=b,label={settings_klassen},numbers=none]
class Settings:
    """These are the default settings for the analysis"""
    edgeThreshold1 = 78
    edgeThreshold2 = 2.5 * edgeThreshold1
    lo = 4
    up = 4
    cutRatios = None
    marginPercentage = 0.009
    method = 'naive'
    ...
\end{lstlisting}

Det ses at vi har mulighed for at trække de fundne regioner ved et
snit ud og vi behøver derfor ikke at køre nogen analyse på billedet hvis
vi blot ønsker at få de fundne regioners begrænsende areal vist. I dette
tilfælde kan vi nøjes med at forespørge databasen om de regioner der er
tilknyttet et snit vi gerne vil undersøge og traversere gennem den liste
af regioner vi får tilbage. Hver region er repræsenteret som en klasse
hvor vi kan trække rektanglet ud og vi bruger da \emph{OpenCV} til at
tegne rektanglet på det tilknyttede billede.
}
% vim: set tw=72 spell spelllang=da:


\section{Database\label{section_imp_database}}
Databasen er lavet i sqlite.  Formålet med databasen er todelt,
opbevaring af metadata og opsamling af statestik fra kørsler.
Problemet med dette er at når databasen skal konstrueres, er det ikke
muligt at sige noget definitivt omkring statestik delen af databasen.
At splitte databasen op i to, er en mulighed.  En del af de udvidede
løsninger opridset i synopsisen benytter en del af metadataerne omkring
billederne, derfor er denne løsning af tvivlsom anvendlighed.  En
løsning, der tilbyder udvidelse, må derfor være at tilstræbe.
Basisløsningen består udelukkende af opbevaringen af metadata, grunden
til dette er udelukkende planlægning af projektet.  Database skemaet
til basisløsningen bygger endnu videre på den struktur, der er
udleveret af \cite{wgahu}.

\begin{center}
\begin{tabular}{|l||c|c|c|c|c|c|}
    \hline
    \bf{artist} \hspace{0.5cm} & \underline{artistId} & name & born & died & school & timeline \\\hline
\end{tabular}

\begin{tabular}{|l||c|c|c|c|c|c}
    \hline
    \bf{painting} \hspace{0.5cm} & \underline{paintingId} & artistId & title & date & paint & $\cdots$ \\\hline
\end{tabular}\\ \vspace{0.2cm}\hspace{1.2cm}
\begin{tabular}{c|c|c|c|c|c|c}
    \hline
    $\cdots$ & material & location & url & form & type & $\cdots$ \\\hline
\end{tabular}\\ \vspace{0.2cm}\hspace{1.4cm}
\begin{tabular}{c|c|c|c|c|c|}
    \hline
    $\cdots$ & realHeight & realWidth & height & width & filepath \\\hline
\end{tabular}

\begin{tabular}{|l||c|c|c|c|c|c|c|}
    \hline
    \bf{run} \hspace{0.5cm} & \underline{runId} & trsh1 & trsh2 & lo & up & marginPercentage & method \\\hline
\end{tabular}

\begin{tabular}{|l||c|c|c|c|c|c|}
    \hline
    \bf{result} \hspace{0.5cm} & \underline{resultId} & runId & paintingId & cutRatio & cutNo & numberOfRegions \\\hline
\end{tabular}

\begin{tabular}{|l||c|c|c|c|c|c|c|}
    \hline
    \bf{region} \hspace{0.5cm} & \underline{regionId} & resultId & x & y & height & width & area \\\hline
\end{tabular}
\end{center}

\newpage
\includegraphics[scale=0.9]{afsnit/vores_implementation/billeder/ER}

% vim: set tw=72 spell spelllang=da:


}

% vim: set tw=72 spell spelllang=da:


}

% vim: set tw=72 spell spelllang=da:
