{
{\sffamily Dette afsnit har til formål at beskrive hvordan vi trækker
regioner ud af billedet. Kort sagt forsøger vi at segmentere billedet
ved primært ved at bruge en metode kaldet floodfill. Billedet bliver
først præpareret ved at finde kanter, en metode betegnet som
kantdetektion, og viske farverne sammen ved en metode kaldet sløring. Vi
vil i det følgende forklare, hvordan vi kombinerer disse metoder til at
finde regioner i en digital gengivelse af et maleri. Vi vil også komme
ind på hvilke andre metoder vi har forsøgt os med og til sidst se på
hvilke tanker vi har gjort os om valg af programmeringssprog og
implementationen generelt.
%navnligt den funktion i OpenCV der er buggy, GoodFeaturesToTrack og
%Perona and Malik fra Octave. Det følgende afsnit om Floodfill bør være
%en del af dette afsnit, ligesom vi skal have lignende beskrivelser af
%sløring og kant detektion.
}

\subsection{Floodfill}                                  % Denne metode er udgangs punktet
\subsection*{Floodfill}

\subsubsection*{Teorien}
Floodfill omhandler udfyldning af områder i et billet som har samme fave eller en fave som ligger inde få en vis afvigelse fra den orginale fave. Det vil sige, der væljes en pixel i billet, denne pixel har faven (r,g,b). Ud fra denne pixel i billdet, ses der på de nabo pixels som ligger dirgunale og vertikale, hvis nogle af de dirgunale eller vertikale pixels har en fave som ligger inde for en vis afvigelse af faven, faves de. Ud fra de nye favet pixels, bliver helle procedyren gendtaget, ind til at der ikke er flere pixels som kan faves.

\subsubsection*{Implementation}
I Implementationen af floodfill, er der visse ting man kan stille på for at få de resultat som er ønsket. Man kan stille på hvor meget fave må variere med. Man kan sætte floodfill til at regne variansen af fave ud fra den pixel som er valt fra start, eller fra fave af den pixel som functionen er komme til. hved at regne varianden ud fra start pixlen, få metode til at indskranke sig en del, og ikke komme inde i alle hjørner af en blob, tilgændgel har metode svære hved at tegne over kanter og komme ind i en anden område, Denne måde at bruge floodfilll på, gørd også at metoden ikke har store udsving på hvor meget der bliver favet, ud fra favens varians. Hved at arbejde ud fra metoden som regner på den nye pixels fave, få man en metode som få meget af bloben med, og som kan overskue at en blob kan skifte fave langsomt i forhold til solens position eller små skift i bloblen. tilgængel er denne framgang måde ret følsom, og kommer der ved til at flyde over bloben som den er i gang med at udfylde.  

\subsubsection*{Overvejelser}
For at få denne metode til at virke på 25000 billeder, hvor en del af billederne ikke har samme fave tone eller er blevet falmet. Må der udregnes, for vært billedet, hvad for en varians i fave der skal bruges.


\subsection{Sløring}                                    % Her maler vi større regioner
% Denne fil er inkluderet i udtraekning_af_regioner.tex
{
Sløring, som kommer fra det engelske ord ``blur'', er en gruppering af
filtre som bruges til at fjerne støj og uregelmæssigheder i billeder. Vi
så i afsnit \ref{subsec_floodfill}, at metoden floodfill nogle gange kan
have svært ved at fylde hele regionen ud. Specielt i figur
\ref{dot_ff_var_7_7} ses at himlen har små huller. En sløring af
billedet kan hjælpe med at glatte farverne ud, således at vi dækker mere
af regionen. Sløring af billedet kan også hjælpe til at fjerne diverse
artefakter, såsom revner eller pletter i billedet. Specielt i
vores testbillede, der som tidligere nævnt er malet med en masse
prikker, er det en stor hjælp at sløre billedet, så farverne bliver mere
ensartede. Vi vil nu se på tre forskellige måder at opnå dette på.

De to første et såkaldte lav-pas filtre som har svært ved at bibeholde
kanterne i et billede, men arbejder til gengæld direkte på billedet. Den
tredie metode bibeholder til en vis grad kanterne bedre, men kan ikke
arbejde direkte på billedet, hvilket kræver et større pladsforbrug.

\subsubsection*{Simpel sløring}

\subsubsection*{Gaussisk sløring}
Ved Gaussisk sløring ind gør der 3 steps, først at finde en kernel,
multiplicere kernel rigtigt på billedet, kaldet Convolution. Udtrak
værdigerne fra multiplikationen i et nyt billedet, som er det sløret
billedet.

\subsubsection*{Kernel}
En kernel er en lille matrice, som betegner hvor stor del af billedet og
hvor meget af pixels værdiger for farverne rund om et punkt, skal tages
med i udregningen for den nye sløret farve.

\subsubsection*{Convolution}
Convoluting er en simpel matematisk metode, som beskriver hvordan man
kan gange 2 matricer sammen som ikke har sammen mål, men samme
dimission.

\begin{figure}[h]
	\begin{center}
		\includegraphics[scale=1,angle=0]{afsnit/vores_implementation/billeder/sloering/convolution}
	\end{center}
	\caption[]{Et billedet og en kernel}
	\label{Convolution}
\end{figure}

Som man kan se i figur \ref{Convolution} er selve billedet og kernel i
vid forskellige størrelser. Måden Convolution virker på at man
multiplicere kernelen ovenpå de værdiger som ligger rund om den pixel,
som vi gerne vil finde den sløret farve af og tager gennemsnittet af den
værdig, f.eks farven på pixel $(4,4)$, bliver

\begin{equation}
	(4,4) = ((3,3)*4+(4,3)*2+(5,3)*3+(3,4)*3+(4,4)*9+(5,4)*3+(3,5)*2+(4,5)*4+(5,5)*9)*\frac{1}{39} 
\end{equation}

I dette eksempel, er punktet $(5,5)$ og $(4,4)$ vægtet højre i forhold til de andet, så farven vil blive sløret i den retning. Dette skyldes at kernelen er bygget tilfældig op, i en rigtige sløring er kernelen meget specifikt udvalgt for at give det beste resultat.
 
\subsubsection*{Gaussian kernel}
Måde Gaussisk sløring bygger sin kernel op på, at ved brug af Gaussian
fordelings formlen, som bliver brugt utallige steder i den matematiske
verden, som en anderkedt og respektere fordelings metode. XXX(ved ikke
om jeg skal komme ind på selve gaussan fordelige, da dette snilt kan
komme op på den del)


\begin{equation}
	G(x,y) = \frac{1}{2\pi\sigma^2}e^{-\frac{x^2+y^2}{2\sigma^2}}
\end{equation}

Hvis $\sigma = 1$ og vi lader x og y køre fra -2 til 2, med steps size på 1, og multiplicere så den laveste værdig kommer til at være 1 og runder værdien af. Få vi en kernel, se firgur \ref{gauss}.

\begin{figure}[h]
	\begin{center}
		\includegraphics[scale=0.5,angle=0]{afsnit/vores_implementation/billeder/sloering/gauss}
	\end{center}
	\caption[]{Kernel for gauss sløring}
	\label{gauss}
\end{figure}

Denne kernel kan så bruges ved hjælp af concolution til at danne det sløret billet, som vi kan se i figur \ref{gaussian_metode}

\subsubsection*{Sløring ved statistisk median}
Den sidste metode vi vil nævne er i grunden meget simpel. Grundidéen er
at finde den statistiske median i pixelværdierne rundt om en givet pixel
og tildele medianværdien til denne. Givet et antal pixels, er det
trivielt at sætte dem i en liste og sortere dem efter deres værdi. Hvis
antallet af elementer i listen er ulige, er det midterste element i den
sorterede liste medianen. Er der et lige antal elementer i listen,
defineres medianen som gennemsnittet af de to midterste elementer.
Pixels vælges i et $N \times M$ vindue med den originale pixel i
centrum, som vist i figur \ref{red_box_nxm} og vi vil således altid have
et ulige antal elementer i listen.

\begin{figure}[!h]
    \centering
    \subfloat[]{\label{red_box_nxm}
        \includegraphics[scale=0.42,angle=0]{afsnit/vores_implementation/billeder/sloering/red_pixel_box}
    }\\
    \subfloat[]{\label{3_3_vindue}
        \renewcommand{\arraystretch}{1.8}
        \begin{tabular}{|c|c|c|}
            \hline
            35 & 98  & 23 \\\hline
            48 & \cellcolor[gray]{0.5}42 & 0 \\\hline
            8  & 12   & 29 \\\hline
        \end{tabular}
        }\hspace{1em}
    \subfloat[]{\label{sorteret_median}
        \renewcommand{\arraystretch}{1.5}
        \centering
        \begin{tabular}{|c|c|c|c|c|c|c|c|c|}
            \hline
            0 & 8 & 12 & 23 & \cellcolor[gray]{0.5}29 & 35 & 42 & 48 & 98\\\hline
        \end{tabular}
        }
        \caption[]{
            Bestemmelse af median for pixel med koordinaterne $(2, 2)$.
            \textbf{\ref{red_box_nxm})} Pixels i et $3\times3$ vindue
            omkring $(2, 2)$ er markeret med rødt.
            \textbf{\ref{3_3_vindue})} Værdierne i $3\times3$ vinduet.
            Det ses den originale pixel har værdien $42$.
            \textbf{\ref{sorteret_median})} Den sorterede liste med
            værdierne fra vinduet. Det ses at medianen har værdien
            $29$. Den originale pixel vil da skifte værdi fra $42$ til
            $29$.
        }
\end{figure}

Denne metode kan ikke køres direkte på det originale billede, da dette
vil interferere med fastsættelse af medianen for alle pixels. Man må
derfor oprette en kopi af det originale billede og sætte de fundne
medianværdier i denne. Man finder således altid medianen i forhold til det
originale billede.

\subsubsection*{Eksempler}

% Hold on, this is figure-madness
\begin{figure}[!h]
    \centering
    \subfloat[Original]{\label{simple_original}\includegraphics[angle=0,width=0.3\textwidth]{afsnit/vores_implementation/billeder/sloering/original}}\hspace{1em}
    \subfloat[$3 \times 3$ vindue]{\label{simple_3_3}\includegraphics[angle=0,width=0.3\textwidth]{afsnit/vores_implementation/billeder/sloering/simple_3_3}}\hspace{1em}
    \subfloat[$7 \times 7$ vindue]{\label{simple_7_7}\includegraphics[angle=0,width=0.3\textwidth]{afsnit/vores_implementation/billeder/sloering/simple_7_7}}
    \caption[]{
        \textbf{\ref{simple_original})} Zoom af detajler i det originale billede.
        \textbf{\ref{simple_3_3})}
        \textbf{\ref{simple_7_7})}
    }
    \label{simple_metode}
\end{figure}

\begin{figure}[!h]
    \centering
    \subfloat[Original]{\label{gaussian_original}\includegraphics[angle=0,width=0.3\textwidth]{afsnit/vores_implementation/billeder/sloering/original}}\hspace{1em}
    \subfloat[$3 \times 3$ vindue]{\label{gaussian_3_3}\includegraphics[angle=0,width=0.3\textwidth]{afsnit/vores_implementation/billeder/sloering/gaussian_3_3}}\hspace{1em}
    \subfloat[$7 \times 7$ vindue]{\label{gaussian_7_7}\includegraphics[angle=0,width=0.3\textwidth]{afsnit/vores_implementation/billeder/sloering/gaussian_7_7}}
    \caption[]{
        \textbf{\ref{gaussian_original})} Zoom af detajler i det originale billede.
        \textbf{\ref{gaussian_3_3})}
        \textbf{\ref{gaussian_7_7})}
    }
    \label{gaussian_metode}
\end{figure}

\begin{figure}[!h]
    \centering
    \subfloat[Original]{\label{median_original}\includegraphics[angle=0,width=0.3\textwidth]{afsnit/vores_implementation/billeder/sloering/original}}\hspace{1em}
    \subfloat[$3 \times 3$ vindue]{\label{median_3_3}\includegraphics[angle=0,width=0.3\textwidth]{afsnit/vores_implementation/billeder/sloering/median_3_3}}\hspace{1em}
    \subfloat[$7 \times 7$ vindue]{\label{median_7_7}\includegraphics[angle=0,width=0.3\textwidth]{afsnit/vores_implementation/billeder/sloering/median_7_7}}
    \caption[]{
        \textbf{\ref{median_original})} Zoom af detajler i det originale billede.
        \textbf{\ref{median_3_3})} Median med et vindue på $3\times{}3$.
        Farverne er blevet mere ensartede mens kanterne stadig er
        skarpe.
        \textbf{\ref{median_7_7})} Median med et vindue på $7\times{}7$. Farverne
        er meget ensartede, men det ses at kanterne er blevet mere
        udvisket med det større vindue.
    }
    \label{median_metode}
\end{figure}

}

% vim: set tw=72 spell spelllang=da:


\subsection{Kantdetektion}                              % Vi vil gerne afgrænse floodfill
% Denne fil er inkluderet i udtraekning_af_regioner.tex
{
Beskrivelse af kantdetektion (med billeder).

\subsubsection*{Metode}

\subsubsection*{Eksempler}

\begin{figure}[!h]
    \begin{center}
        \includegraphics[width=0.8\textwidth]{afsnit/vores_implementation/billeder/kantdetektion/canny_20_20}
    \end{center}
    \caption[]{Canny kantdetektion med tærskelværdierne $(20, 20)$.}
    \label{bathers}
\end{figure}
}

% vim: set tw=72 spell spelllang=da:


\subsection{Sammensætning af metoder}
Forklar hvordan vi kombinerer metoderne

\subsection{Programmeringssprog og biblioteker}
{
{\sffamily Ved valg af programmeringssprog har vi først og fremmest lagt
vægt på at kunne udarbejde en prototype hurtigt og bruge et sprog, som
er let at gå til. Det valgte sprog skal også gøre det nemt at udvide den
endelige implementation. Vi har også gerne villet undgå at skulle
konstruere komplicerede datastrukturer for relativt simple metoder, både
af hensyn til tidspresset og til implementationens kompleksitet. Af
ovenstående grunde har vi besluttet at udarbejde vores løsning i
programmeringssproget \textbf{Python}, da netop dette sprog er yderst
velegnet at skrive forholdsvis avancerede prototyper i. Python er
ydermere meget fleksibelt med hensyn til datastrukturer og byder
umiddelbart på en lang række, for problemstillingen relevante, pakker.

(Skal man skrive noget om Pythons udbredelse, anerkendelse og brug?)
}

\subsection{OpenCV}
Til udførelse af billedmanipulationer benytter vi os af et bibliotek skrevet
i C og C++, der hedder \emph{OpenCV}. Biblioteket er udviklet af Intel
og tilbyder, udover et solidt udvalg af algoritmer, bindinger til
Python.  Endelig er det meget veldokumenteret og giver referencer til
publikationer om bibliotekets algoritmer. Biblioteket er udviklet med
specielt henblik på real-tids behandling af billeder, f.eks. med et
videokamera som kilde, men det egner sig også til brug på enkelte
billeder.  \emph{OpenCV} tilbyder mange brugbare datastrukturer med
hensyn til arbejdet med billeder i Python.

Der er også andre biblioteker til billedbehandling i Python. Her kan
nævnes \emph{PIL} (Python Image Library) og \emph{PythonMagick}
(ImageMagick bindings), men de er ikke nær så grundige som
\emph{OpenCV}.

\subsubsection{Andre muligheder}
Der er to helt oplagte muligheder, med hensyn til programmeringssprog,
når man taler om billedbehandling, nemlig Matlab og dets Open
Source-alternativ Octave. Disse sprog blev dog valgt fra, da vores
samlede erfaring med udvikling i disse sprog ikke var stor nok.
Endvidere finder vi, at disse sprog, på trods af, at de især egner sig
til den type beregninger, vi skal lave, er besværlige at lave større
programmer med. Matlab og Octave er dog blevet brugt til at sammenligne
resultater og teste alternative metoder med.

Da \emph{OpenCV} er skrevet i C/C++, ville det også være oplagt at bruge
et af disse sprog. Vores erfaring er dog, at man let kommer til at bruge
mere tid på at konstruere de fornødne datastrukturer og hjælpemetoder,
end på at fokusere på opgavens kerne. En senere implementation, med
fokus på køretid, kunne med fordel implementeres i C/C++, da man så
ville have fuld kontrol over, hvilke strukturer der bliver brugt i
programmet.

\subsection{Værktøjer til databasen}
Vi bruger \textbf{SQLite} til selve databasen, hovedsagelig fordi der ikke
kræves nogen videre konfiguration af en sådan database. Den
underliggende database er dog underordnet, da vi bruger Python-pakken
\emph{SQLObject}, som giver et abstraktionslag til en bred vifte af
databaser. Vi opretter blot de tabeller, vi ønsker at have i databasen,
som klasser i Python og får ligeledes en sådan klasse tilbage, når der
laves forespørgsler til databasen. Da \emph{SQLObject} klarer al
kommunikation med databasen, er det derfor muligt at skifte den
underliggende database ud, hvis man ønsker det. SQLite har endvidere den
umiddelbare fordel, at selve databasen eksisterer som en fil i
filsystemet.  Det er derfor en let sag at tage sikkerhedskopier af
databasen uden alt for meget besvær.

\subsection{Andre værktøjer}
Vi gør også brug af statistikprogrammet \textbf{R} til at behandle og
præsentere vores resultater.

}

% vim: set tw=72 spell spelllang=da:


}

% vim: set tw=72 spell spelllang=da:
