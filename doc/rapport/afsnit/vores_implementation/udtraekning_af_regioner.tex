{
{\sffamily Dette afsnit har til formål at beskrive hvordan vi trækker
regioner ud af billedet. Kort sagt forsøger vi at segmentere billedet
ved primært ved at bruge en metode kaldet floodfill. Vi præparerer dog
først billedet ved at finde kanter og sløre billedet. Vi vil også komme
ind på hvilke andre metoder vi har forsøgt os med. Først vil vi dog se
på hvilke tanker vi har gjort os om valg af programmeringssprog og
implementationen generelt.
%navnligt den funktion i OpenCV der er buggy, GoodFeaturesToTrack og
%Perona and Malik fra Octave. Det følgende afsnit om Floodfill bør være
%en del af dette afsnit, ligesom vi skal have lignende beskrivelser af
%sløring og kant detektion.
}

\subsection{Programmeringssprog og biblioteker}
Ved valg af programmeringssprog har vi først og fremmest lagt vægt på at
kunne udarbejde en prototype hurtigt og have et sprog som er let at gå
til. Det valgte sprog skal også gøre det nemt at udvide den endelige
implementation. Vi har også gerne ville undgå at skulle konstruere
komplicerede datastrukturer for relativt simple metoder, både af hensyn
til tidspres, men også til implementationens kompleksitet. Af
ovenstående grunde har vi besluttet at udarbejde vores løsning i
programmeringssproget Python, da netop dette sprog er yderst velegnet
til at skrive forholdsvis avancerede prototyper i. Python er
ydermere meget fleksibelt med hensyn til datastrukturer og byder
umiddelbart på en lang række pakker relevant for problemstillingen.

(Skal man skrive noget om Pythons udbredelse, anerkendelse og brug?)

\subsubsection{Andre muligheder}
Der er to helt oplagte mugligheder med hensyn til programmeringssprog
når man taler om billedbehandling, nemlig Matlab og dets Open
Source-alternativ Octave. Matlab/Octave blev valgt fra, da vores samlede
erfaring med udvikling i disse sprog ikke var stor nok. Endvidere, selv
om de især egner sig til de beregninger vi skal lave, så er disse sprog
besværlige at lave større programmer med.

Octave, Matlab, C, C++, Ruby. Fordele/ulemper.

\subsubsection*{OpenCV}

\subsubsection*{SQLite}


Hvad bruger vi, hvorfor og hvordan? Indsæt eventuelt et diagram over
programmet.

\subsection{Kantdetektion}
Indsæt beskrivelse af kantdetektion.

\subsection{Sløring}
Indsæt beskrivelse af sløring (blur).

\subsection{Floodfill}
\subsection*{Floodfill}

\subsubsection*{Teorien}
Floodfill omhandler udfyldning af områder i et billet som har samme fave eller en fave som ligger inde få en vis afvigelse fra den orginale fave. Det vil sige, der væljes en pixel i billet, denne pixel har faven (r,g,b). Ud fra denne pixel i billdet, ses der på de nabo pixels som ligger dirgunale og vertikale, hvis nogle af de dirgunale eller vertikale pixels har en fave som ligger inde for en vis afvigelse af faven, faves de. Ud fra de nye favet pixels, bliver helle procedyren gendtaget, ind til at der ikke er flere pixels som kan faves.

\subsubsection*{Implementation}
I Implementationen af floodfill, er der visse ting man kan stille på for at få de resultat som er ønsket. Man kan stille på hvor meget fave må variere med. Man kan sætte floodfill til at regne variansen af fave ud fra den pixel som er valt fra start, eller fra fave af den pixel som functionen er komme til. hved at regne varianden ud fra start pixlen, få metode til at indskranke sig en del, og ikke komme inde i alle hjørner af en blob, tilgændgel har metode svære hved at tegne over kanter og komme ind i en anden område, Denne måde at bruge floodfilll på, gørd også at metoden ikke har store udsving på hvor meget der bliver favet, ud fra favens varians. Hved at arbejde ud fra metoden som regner på den nye pixels fave, få man en metode som få meget af bloben med, og som kan overskue at en blob kan skifte fave langsomt i forhold til solens position eller små skift i bloblen. tilgængel er denne framgang måde ret følsom, og kommer der ved til at flyde over bloben som den er i gang med at udfylde.  

\subsubsection*{Overvejelser}
For at få denne metode til at virke på 25000 billeder, hvor en del af billederne ikke har samme fave tone eller er blevet falmet. Må der udregnes, for vært billedet, hvad for en varians i fave der skal bruges.


}

% vim: set tw=72 spell spelllang=da:
