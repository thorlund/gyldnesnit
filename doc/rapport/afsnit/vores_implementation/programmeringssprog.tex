% Denne fil er inkluderet i udtraekning_af_regioner.tex
{
Ved valg af programmeringssprog har vi først og fremmest lagt vægt på at
kunne udarbejde en prototype hurtigt og have et sprog som er let at gå
til. Det valgte sprog skal også gøre det nemt at udvide den endelige
implementation. Vi har også gerne ville undgå at skulle konstruere
komplicerede datastrukturer for relativt simple metoder, både af hensyn
til tidspres, men også til implementationens kompleksitet. Af
ovenstående grunde har vi besluttet at udarbejde vores løsning i
programmeringssproget Python, da netop dette sprog er yderst velegnet
til at skrive forholdsvis avancerede prototyper i. Python er
ydermere meget fleksibelt med hensyn til datastrukturer og byder
umiddelbart på en lang række pakker relevant for problemstillingen.

(Skal man skrive noget om Pythons udbredelse, anerkendelse og brug?)

\subsubsection*{OpenCV}
Til at udføre billedmanipulation benytter vi os af et bibliotek skrevet
i C og C++ der hedder \emph{OpenCV}. Biblioteket er udviklet af Intel og
tilbyder, udover et solidt udvalg af algoritmer, bindinger til Python.
Endelig er det meget veldokumenteret og giver referencer til
publikationer om bibliotekets algoritmer. Biblioteket er udviklet med
specielt henblik på real-tids behandling af billeder, f.eks. med et
videokamera som kilde, men egner sig også til brug på enkelte billeder.
\emph{OpenCV} tilbyder mange brugbare datastrukturer med hensyn til
arbejdet med billeder i Python.

Der tilbydes også andre biblioteker til billedbehandling i Python. Her
kan nævnes \emph{PIL} (Python Image Library) og \emph{PythonMagick}
(ImageMagick bindings), men de er ikke nær så grundige som
\emph{OpenCv}. Der findes dog metoder i \emph{OpenCV} til at samarbejde
med \emph{PIL}, men vi har ikke set det nødvendigt at udnytte dette.

%\subsubsection*{SQLite}
% Der snakkes om dette i Opbevaring af billeder, mon ikke det er nok?

\subsubsection{Andre muligheder}
Der er to helt oplagte mugligheder med hensyn til programmeringssprog
når man taler om billedbehandling, nemlig Matlab og dets Open
Source-alternativ Octave. De blev dog valgt fra, da vores samlede
erfaring med udvikling i disse sprog ikke var stor nok. Endvidere, selv
om de især egner sig til de beregninger vi skal lave, så finder vi at
disse sprog besværlige at lave større programmer med. Matlab og Octave
er dog blevet brugt til at sammenligne resultater og teste alternative
metoder med. Dette vil vi vende tilbage til.

Da \emph{OpenCV} er skrevet i C/C++ ville det også være oplagt at bruge et af
disse sprog. Vores erfaring er dog, at man let kommer til at bruge mere
tid på at konstruere de fornødne datastrukturer og hjælpemetoder hvilket
let flytter fokus fra opgavens kerne. En senere implementation, med
fokus på køretid, kunne med fordel implementeres i C/C++, da man vil
have fuld kontrol over hvilke strukturer der bliver brugt i programmet.
}

% vim: set tw=72 spell spelllang=da:
