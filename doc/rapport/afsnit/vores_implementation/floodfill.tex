\subsection*{Floodfill}

\subsubsection*{Teorien}
Floodfill omhandler udfyldning af områder i et billet som har samme fave eller en fave som ligger inde få en vis afvigelse fra den orginale fave. Det vil sige, der væljes en pixel i billet, denne pixel har faven (r,g,b). Ud fra denne pixel i billdet, ses der på de nabo pixels som ligger dirgunale og vertikale, hvis nogle af de dirgunale eller vertikale pixels har en fave som ligger inde for en vis afvigelse af faven, faves de. Ud fra de nye favet pixels, bliver helle procedyren gendtaget, ind til at der ikke er flere pixels som kan faves.

\subsubsection*{Implementation}
I Implementationen af floodfill, er der visse ting man kan stille på for at få de resultat som er ønsket. Man kan stille på hvor meget fave må variere med. Man kan sætte floodfill til at regne variansen af fave ud fra den pixel som er valt fra start, eller fra fave af den pixel som functionen er komme til. hved at regne varianden ud fra start pixlen, få metode til at indskranke sig en del, og ikke komme inde i alle hjørner af en blob, tilgændgel har metode svære hved at tegne over kanter og komme ind i en anden område, Denne måde at bruge floodfilll på, gørd også at metoden ikke har store udsving på hvor meget der bliver favet, ud fra favens varians. Hved at arbejde ud fra metoden som regner på den nye pixels fave, få man en metode som få meget af bloben med, og som kan overskue at en blob kan skifte fave langsomt i forhold til solens position eller små skift i bloblen. tilgængel er denne framgang måde ret følsom, og kommer der ved til at flyde over bloben som den er i gang med at udfylde.  

\subsubsection*{Overvejelser}
For at få denne metode til at virke på 25000 billeder, hvor en del af billederne ikke har samme fave tone eller er blevet falmet. Må der udregnes, for vært billedet, hvad for en varians i fave der skal bruges.
