% Denne fil bliver ikke brugt mere
{
Når vi har trukket regioner ud af billedet og vurderet dem efter den
naive algoritme givet i afsnit \ref{naiv_algoritme} kan vi stå tilbage
med et egentligt resultat. Vi ønsker at gemme dette resultat i en
database så vi på et senere tidspunkt kan bruge det i en samlet analyse
af resultaterne.  Vi så i afsnit \ref{section_opbv_billeder} på et
databaseskema til opbevaring af maleriers metadata. Vi bygger nu videre
på dette skema så vi kan tilknytte resultater fra en analyse til de
enkelte malerier. Det fulde databaseskema ses nedenfor og bliver
efterfølgende forklaret.

\begin{table}[!h]
    \centering
    \begin{tabular}{|l||c|c|c|c|c|c|}
        \hline
        \bf{artist} \hspace{0.5cm} & \underline{artistId} & name & born & died & school & timeline \\\hline
    \end{tabular}
    \caption{Databasetabel for kunstner}
    \label{artistTable}
\end{table}

\begin{table}[!h]
    \centering
    \begin{tabular}{|l||c|c|c|c|c|c}
        \hline
        \bf{painting} \hspace{0.5cm} & \underline{paintingId} & artistId & title & date & paint & $\cdots$ \\\hline
    \end{tabular}\\ \vspace{0.2cm}\hspace{1.2cm}
    \begin{tabular}{c|c|c|c|c|c|c}
        \hline
        $\cdots$ & material & location & url & form & type & $\cdots$ \\\hline
    \end{tabular}\\ \vspace{0.2cm}\hspace{1.4cm}
    \begin{tabular}{c|c|c|c|c|c|}
        \hline
        $\cdots$ & realHeight & realWidth & height & width & filepath \\\hline
    \end{tabular}
    \caption{Databasetabel for malerier}
    \label{paintingTable}
\end{table}

\begin{table}[!h]
    \centering
    \begin{tabular}{|l||c|c|c|c|c|c|c|}
        \hline
        \bf{run} \hspace{0.5cm} & \underline{runId} & trsh1 & trsh2 & lo & up & marginPercentage & method \\\hline
    \end{tabular}
    \caption{Databasetabel for en kørsel}
    \label{runTable}
\end{table}

\begin{table}[!h]
    \centering
    \begin{tabular}{|l||c|c|c|c|c|c|}
        \hline
        \bf{result} \hspace{0.5cm} & \underline{resultId} & runId & paintingId & cutRatio & cutNo & numberOfRegions \\\hline
    \end{tabular}
    \caption{Databasetabel for resultater}
    \label{resultTable}
\end{table}

\begin{table}[!h]
    \centering
    \begin{tabular}{|l||c|c|c|c|c|c|c|}
        \hline
        \bf{region} \hspace{0.5cm} & \underline{regionId} & resultId & x & y & height & width & area \\\hline
    \end{tabular}
    \caption{Databasetabel for regioner}
    \label{regionTable}
\end{table}

Tabellerne \ref{artistTable} og \ref{paintingTable} er de samme som vist
i afsnit \ref{section_opbv_billeder}, men er taget med her blot for at
vise databaseskemaet i sin helhed. Det øvrige databaseskema lægger vægt
på at minimere redundans, at kunne gemme data fra flere forskellige
kørsler med forskellige parametre og mulighed for at genskabe kørte
analyser. Vi vil nu kigge på betydningen af de ovenstående tabeller i
databasen og deres relation til analysen på et billede.

Hvis vi vil genskabe et fundet resultat, har vi brug for at vide hvilke
parametre vi har brugt for at komme frem til resultatet. Til dette
formål har vi tabellen \texttt{run} (tabel \ref{runTable}) som beskriver
en kørsel. Denne tabel holder de parametre som er fælles for alle
billeder i en kørsel. Her har vi adgang til de tærskelparametre som
bruges til kantdetektion (\texttt{trsh1} og \texttt{trsh2}) samt nedre
og øvre grænse for floodfill (\texttt{lo} og \texttt{up}). Her holdes
også en procentsats for hvor stort et margin vi bruger i floodfill og
udvælgelse af regioner. Endelig har vi et felt der angiver hvilken
metode der er blevet brugt til at finde resultatet. Dette er en
tekststreng og vil i tilfældet af den naive algoritme være sat til
\texttt{'naive'}. Felterne \texttt{trsh1}, \texttt{trsh2} og
\texttt{marignPercentage} er repræsenteret som floats i databasen,
mens \texttt{lo} og \texttt{up} står som heltal. Afslutningsvis har
hver kørsel en unik id. Således vi kan tilknytte indgange i tabellen
\texttt{result} (tabel \ref{resultTable}) til et sæt af parametre.

Vi beskriver et resultat, som det antal af regioner vi får ud fra
analysen af et snit på et givet billede. Givet en snitratio vil en
analyse typisk have fire snit hvor vi vil finde regioner i nærheden af.
En undtagelse er hvis snitratioen deler billedet i to lige store dele.
Vi vil i dette tilfælde kun have to snit vi kigger på. Tabellen
\texttt{result} fortæller os hvilket af de mulige snit vi har med at
gøre, hvilket billede resultatet er tilknyttet, hvilke parametre der er
blevet brugt samt hvor mange regioner vi har fundet.

Tabellen \texttt{region} (tabel \ref{regionTable}) holder alle de fundne
regioner fra vores analyse. Hver region henviser til det resultat som
denne tilhører. Vi kan således skelne de enkelte regioner fra hinanden
og afgøre i hvilket snit af billedetde ligger.  På grund af
begrænsninger i \emph{OpenCV} kan vi ikke gemme regionens præcise form,
men kun dennes begrænsende rektangel og regionens areal.

Tabellen \texttt{result} gør det muligt at gemme resultater fra kørsler
med forskellige parametre, hvorved man kan have data fra separate
kørsler i databasen. Man har da grundlag for at sammenligne kørsler med
forskellige metoder og parametre.

Vi nævnte tidligere at databaseskemaet var udviklet med blandt andet det
formål at mindske redundans. Vi har dog feltet \texttt{numberOfRgions} i
tabellen \texttt{result} som på sin vis er redundant, da vi blot kan
finde antallet af tilknyttede regioner ved at bruge simple
SQL-sætninger. At have antallet stående direkte i databasen giver dog et
umiddelbart bedre overblik over resultaterne end ved at bruge
SQL-sætninger. Når databasen vokser, ved at ændre på parametrene, vil
SQL-sætningerne skulle søge meget af databasen igennem for at returnere
en meget simpel forespørgsel. Derfor har vi valgt at gemme antallet at
fundne regioner direkte i databasen.

Databaseskemaet er let at udvide, hvis vi i udviklingen af mere
avancerede metoder, skulle få brug for flere indgange. Specielt kunne
tabellen \texttt{region} udvides med en vægt som et mål for ``hvor
meget'' denne region ligger i snittet.

}
% vim: set tw=72 spell spelllang=da:
