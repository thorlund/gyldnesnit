{
{\sffamily I forbindelse med analyse på store datasæt, spiller den
bagvedliggende database en central rolle. Det er fra databasen, vi
henter billeder ind til analyse og samtidig også her, resultaterne
bliver gemt. Vi præsenterer herunder det databaseskema, som databasen er
opbygget efter.  Efterfølgende kaster vi et nærmere blik på de enkelte
tabeller i databasen, hvor vi først vil kigge på, hvordan vi opbevarer
maleriernes metadata, og endelig på hvordan resultater bliver opbevaret.
}

\subsection{Databaseskema}
Tabellerne \ref{artistTable}, \ref{paintingTable}, \ref{runTable},
\ref{resultTable} og \ref{regionTable} herunder, udgør databaseskemaet.
Der er i alle henseender lagt vægt på at eliminere redundans og på
muligheden for senere udvidelse.

\begin{table}[!h]
    \centering
    \begin{tabular}{|l||c|c|c|c|c|c|}
        \hline
        \bf{artist} \hspace{0.5cm} & \underline{artistId} & name & born & died & school & timeline \\\hline
    \end{tabular}
    \caption{Databasetabel for kunstner}
    \label{artistTable}
\end{table}

\begin{table}[!h]
    \centering
    \begin{tabular}{|l||c|c|c|c|c|c}
        \hline
        \bf{painting} \hspace{0.5cm} & \underline{paintingId} & artistId & title & date & paint & $\cdots$ \\\hline
    \end{tabular}\\ \vspace{0.2cm}\hspace{1.2cm}
    \begin{tabular}{c|c|c|c|c|c|c}
        \hline
        $\cdots$ & material & location & url & form & type & $\cdots$ \\\hline
    \end{tabular}\\ \vspace{0.2cm}\hspace{1.4cm}
    \begin{tabular}{c|c|c|c|c|c|}
        \hline
        $\cdots$ & realHeight & realWidth & height & width & filepath \\\hline
    \end{tabular}
    \caption{Databasetabel for malerier}
    \label{paintingTable}
\end{table}

\begin{table}[!h]
    \centering
    \begin{tabular}{|l||c|c|c|c|c|c|c|}
        \hline
        \bf{run} \hspace{0.5cm} & \underline{runId} & trsh1 & trsh2 & lo & up & marginPercentage & method \\\hline
    \end{tabular}
    \caption{Databasetabel for en kørsel}
    \label{runTable}
\end{table}

\begin{table}[!h]
    \centering
    \begin{tabular}{|l||c|c|c|c|c|c|}
        \hline
        \bf{result} \hspace{0.5cm} & \underline{resultId} & runId & paintingId & cutRatio & cutNo & numberOfRegions \\\hline
    \end{tabular}
    \caption{Databasetabel for resultater}
    \label{resultTable}
\end{table}

\begin{table}[!h]
    \centering
    \begin{tabular}{|l||c|c|c|c|c|c|c|}
        \hline
        \bf{region} \hspace{0.5cm} & \underline{regionId} & resultId & x & y & height & width & area \\\hline
    \end{tabular}
    \caption{Databasetabel for regioner}
    \label{regionTable}
\end{table}

\subsection{Metadata og billeder\label{section_opbv_billeder}}
Vi starter med at kigge på, hvordan vi opbevarer maleriernes metadata.
Denne information findes i tabellerne \texttt{artist} (tabel
\ref{artistTable}) og \texttt{painting} (tabel \ref{paintingTable}).
Disse to tabeller lægger vægt på, at man let skal kunne forespørge
databasen ved en lang række parametre, såsom et maleris fysiske
størrelse og kunstnerens fødselsår. Vi har, at en kunstner kan være
tilknyttet ét eller flere malerier, og at der, til et givet maleri, kun
kan være én kunstner. Billederne, vi vil analysere, hentes fra et online
kunstarkiv og gemmes på filsystemet i en mappestruktur, der ligner den i
arkivet. Her inddeles filerne i mapper navngivet efter kunstner.
Mapperne inddeles efter forbogstav. På denne måde undgås det, at to
billeder tildeles samme filnavn. Vi gemmer kun stien til en fil på
filsystemet i databasen. Filstrukturen er grafisk illustreret i figur
\ref{mappestruktur}.

% Mappestruktur
\begin{figure}[!h]
    \centering
$
\xymatrix{
 &  &   & \ar @{-} [d] \textrm{/res}  &                                                     \\
 &  &   & \ar @{-} [d] \textrm{/wga.hu}  &                                                  \\
 &  &   & \ar @{-} [dl] \ar @{-} [d] \ar @{--} [dr] \textrm{/art} &                         \\
 &  & \ar @{-} [dl] \ar @{-} [d] \ar @{--} [dr] \textrm{/a} & \textrm{/b} & \cdots          \\
 & \ar @{-} [dl] \ar @{-} [d] \textrm{/aachen} & \ar @{--} [d] \textrm{/abadia} & \cdots    \\
\textrm{allegory.jpg} & \textrm{bacchus.jpg} & \cdots &   &
}
$
    \caption{Mappestruktur til filer fra
        \href{http://www.wga.hu}{http://www.wga.hu}}
    \label{mappestruktur}
\end{figure}

\subsection{Resultater fra kørsler\label{section_results}}
Når vi har trukket regioner ud af billedet --- jvf. afsnit
\ref{section_udtraek} --- og vurderet dem efter den naive algoritme
givet i afsnit \ref{section_naiv}, står vi tilbage med et egentligt
resultat. Vi ønsker at gemme dette resultat i databasen, så vi på et
senere tidspunkt kan bruge det i en samlet analyse af resultaterne. Det
øvrige databaseskema, der udgøres af tabellerne \texttt{run} (tabel
\ref{runTable}), \texttt{result} (tabel \ref{resultTable}) og
\texttt{region} (tabel \ref{regionTable}), lægger vægt på at kunne gemme
data fra flere forskellige kørsler med forskellige parametre og mulighed
for at genskabe kørte analyser. Vi vil nu kigge på betydningen af de
ovenstående tabeller og se på, hvordan tabellerne giver mulighed for
designmålene.

Hvis vi vil genskabe et fundet resultat, har vi brug for at vide hvilke
parametre, vi har brugt for at komme frem til resultatet. Til dette
formål har vi tabellen \texttt{run} (tabel \ref{runTable}), som
beskriver en kørsel. Denne tabel holder de parametre, som er fælles for
alle billeder i en kørsel, bl. a. de tærskelværdier, som bruges til
kantdetektion (\texttt{trsh1} og \texttt{trsh2}), samt nedre og øvre
grænse for floodfill (\texttt{lo} og \texttt{up}). Her holdes også en
procentsats for, hvor stor et margin vi bruger i udvælgelse af
regioner. Endelig har vi et felt, der angiver, hvilken metode der er
blevet brugt til at finde resultatet. Dette er en tekststreng og vil i
tilfældet af den naive algoritme være sat til \texttt{'naive'}. Felterne
\texttt{trsh1}, \texttt{trsh2} og \texttt{marignPercentage} er
repræsenteret som floats i databasen, mens \texttt{lo} og \texttt{up}
står som heltal. Afslutningsvis har hver kørsel et unikt \emph{id}, så
at vi kan tilknytte indgange i tabellen \texttt{result} (tabel
\ref{resultTable}) til et sæt af parametre.

Vi beskriver et resultat som det antal af regioner, vi får ud fra
analysen af et snit på et givet billede. Som beskrevet i afsnit
\ref{section_opdeling} vil vi, givet en snitratio, typisk søge regioner
i nærheden af fire snit. En undtagelse er, hvis snitratioen deler
billedet i to lige store dele, og vi vil kun i dette tilfælde have to
snit at se på. Tabellen \texttt{result} fortæller os hvilket af de
mulige snit vi har med at gøre, hvilket billede resultatet er
tilknyttet, hvilke parametre der er blevet brugt, samt hvor mange
regioner vi har fundet. Tabellen gør det muligt at gemme resultater fra
kørsler med forskellige parametre, hvorved man kan have data fra
separate kørsler i databasen. Man har da grundlag for at sammenligne
kørsler med forskellige metoder og parametre.

Tabellen \texttt{region} (tabel \ref{regionTable}) holder alle de fundne
regioner fra vores analyse. Hver region henviser til det resultat, som
denne tilhører. Vi kan således skelne de enkelte regioner fra hinanden
og afgøre, i hvilket snit af billedet de ligger. En region bliver
repræsenteret som dens areal og begrænsende rektangel.

\subsection{Vurdering}
Databaseskemaet har været underlagt følgende designmål:

\begin{itemize}
    \item Minimering af redundans
    \item Mulighed for senere udvidelse
    \item Mulighed for rekonstruktion af kørte analyser
    \item Mulighed for at kunne analysere flere snit i én kørsel
\end{itemize}

Vi har dog feltet \texttt{numberOfRgions} i tabellen \texttt{result} som
på sin vis er redundant, da vi blot kan finde antallet af tilknyttede
regioner ved at bruge simple SQL-sætninger. At have antallet stående
direkte i databasen giver dog et umiddelbart bedre overblik over
resultaterne: Når databasen vokser, som følge af ændringer på
parametrene, vil SQL-sætningerne skulle søge meget af databasen igennem
for at returnere simple forespørgselser. Derfor har vi valgt at gemme
antallet at fundne regioner direkte i databasen.

Databaseskemaet er let at udvide, hvis vi, i udviklingen af mere
avancerede metoder, skulle få brug for flere indgange. Det skal også
bemærkes, at tabellerne \ref{runTable}, \ref{resultTable} og
\ref{regionTable} er specifikke for netop den naive algoritme.
Tabellerne er kun tilknyttet maleriernes metadata ved feltet
\texttt{paintingId}. Man kan derved let udvide databasen til at
indeholde data om kørsler, hvor en anden metode til udtrækning af
regioner, eller til bedømmelse af samme, har været brugt.

}

% vim: set tw=72 spell spelllang=da:
