{
{\sffamily I forbindelse med analyse på store datasæt, spiller den
bagvedliggende database en central rolle. Det er fra databasen vi henter
billeder ind til analyse og samtidig også her hvor resultaterne bliver gemt. Vi
præsenterer herunder det databaseskema som databasen er opbygget efter.
Efterfølgende kaster vi et nærmere blik på de enkelte tabeller i
databasen, hvor vi først vil kigge på hvordan vi opbevarer
maleriernes metadata og endeligt hvordan resultater bliver opbevaret.
}

\subsection{Databaseskema}
Tabellerne \ref{artistTable}, \ref{paintingTable}, \ref{runTable},
\ref{resultTable} og \ref{regionTable} herunder, udgør databaseskemaet.
Der er i alle henseender lagt vægt på at eliminere redundans og mulighed
for senere udvidelse.

\begin{table}[!h]
    \centering
    \begin{tabular}{|l||c|c|c|c|c|c|}
        \hline
        \bf{artist} \hspace{0.5cm} & \underline{artistId} & name & born & died & school & timeline \\\hline
    \end{tabular}
    \caption{Databasetabel for kunstner}
    \label{artistTable}
\end{table}

\begin{table}[!h]
    \centering
    \begin{tabular}{|l||c|c|c|c|c|c}
        \hline
        \bf{painting} \hspace{0.5cm} & \underline{paintingId} & artistId & title & date & paint & $\cdots$ \\\hline
    \end{tabular}\\ \vspace{0.2cm}\hspace{1.2cm}
    \begin{tabular}{c|c|c|c|c|c|c}
        \hline
        $\cdots$ & material & location & url & form & type & $\cdots$ \\\hline
    \end{tabular}\\ \vspace{0.2cm}\hspace{1.4cm}
    \begin{tabular}{c|c|c|c|c|c|}
        \hline
        $\cdots$ & realHeight & realWidth & height & width & filepath \\\hline
    \end{tabular}
    \caption{Databasetabel for malerier}
    \label{paintingTable}
\end{table}

\begin{table}[!h]
    \centering
    \begin{tabular}{|l||c|c|c|c|c|c|c|}
        \hline
        \bf{run} \hspace{0.5cm} & \underline{runId} & trsh1 & trsh2 & lo & up & marginPercentage & method \\\hline
    \end{tabular}
    \caption{Databasetabel for en kørsel}
    \label{runTable}
\end{table}

\begin{table}[!h]
    \centering
    \begin{tabular}{|l||c|c|c|c|c|c|}
        \hline
        \bf{result} \hspace{0.5cm} & \underline{resultId} & runId & paintingId & cutRatio & cutNo & numberOfRegions \\\hline
    \end{tabular}
    \caption{Databasetabel for resultater}
    \label{resultTable}
\end{table}

\begin{table}[!h]
    \centering
    \begin{tabular}{|l||c|c|c|c|c|c|c|}
        \hline
        \bf{region} \hspace{0.5cm} & \underline{regionId} & resultId & x & y & height & width & area \\\hline
    \end{tabular}
    \caption{Databasetabel for regioner}
    \label{regionTable}
\end{table}

\subsection{Metadata og billeder\label{section_opbv_billeder}}
Vi starter med at kigge på hvordan vi opbevarer maleriernes metadata. Vi
holder denne information i tabellerne \texttt{artist} (tabel
\ref{artistTable}) og \texttt{painting} (tabel \ref{paintingTable}).
Disse to tabeller lægger vægtpå at man let kan forespørge databasen ved
en lang række parametre, såsom et maleris fysiske størrelse og
kunstneres fødselsår. Vi har, at der til en kunstner kan være tilknyttet
ét eller flere malerier og at der til et givet maleri, kun kan være én
kunstner. Billederne som vi vil analysere, hentes fra et online
kunstarkiv og gemmes på filsystemet i en mappestruktur lignende den i
arkivet. Her indeles filerne i mapper navngivet efter deres kunstner.
Kunstnerenes mapper inddeles efter forbogstav. På denne måde undgås
problemet at to billeder kan have det samme filnavn. Vi gemmer da blot
stien til en fil på filsystemet i databasen. Filstrukturen er grafisk
illustreret i figur \ref{mappestruktur}.

% Mappestruktur
\begin{figure}[!h]
    \centering
$
\xymatrix{
 &  &   & \ar @{-} [d] \textrm{/res}  &                                                     \\
 &  &   & \ar @{-} [d] \textrm{/wga.hu}  &                                                  \\
 &  &   & \ar @{-} [dl] \ar @{-} [d] \ar @{--} [dr] \textrm{/art} &                         \\
 &  & \ar @{-} [dl] \ar @{-} [d] \ar @{--} [dr] \textrm{/a} & \textrm{/b} & \cdots          \\
 & \ar @{-} [dl] \ar @{-} [d] \textrm{/aachen} & \ar @{--} [d] \textrm{/abadia} & \cdots    \\
\textrm{allegory.jpg} & \textrm{bacchus.jpg} & \cdots &   &
}
$
    \caption{Mappestruktur til filer fra
        \href{http://www.wga.hu}{http://www.wga.hu}}
    \label{mappestruktur}
\end{figure}

\subsection{Resultater fra kørsler\label{section_results}}
Når vi har trukket regioner ud af billedet, jvf. afsnit
\ref{section_udtraek}, og vurderet dem efter den naive algoritme givet i
afsnit \ref{section_naiv}, kan vi stå tilbage med et egentligt
resultat. Vi ønsker at gemme dette resultat i databasen, så vi på et
senere tidspunkt kan bruge det i en samlet analyse af resultaterne. Det
øvrige databaseskema, der udgøres af tabellerne \texttt{run} (tabel
\ref{runTable}), \texttt{result} (tabel \ref{resultTable}) og
\texttt{region} (tabel \ref{regionTable}), lægger vægt på at kunne gemme
data fra flere forskellige kørsler med forskellige parametre og mulighed
for at genskabe kørte analyser. Vi vil nu kigge på betydningen af de
ovenstående tabeller og se på hvordan tabellerne giver mulighed for
designmålene.

Hvis vi vil genskabe et fundet resultat, har vi brug for at vide hvilke
parametre vi har brugt for at komme frem til resultatet. Til dette
formål har vi tabellen \texttt{run} (tabel \ref{runTable}) som beskriver
en kørsel. Denne tabel holder de parametre som er fælles for alle
billeder i en kørsel. Her har vi adgang til de tærskelparametre som
bruges til kantdetektion (\texttt{trsh1} og \texttt{trsh2}) samt nedre
og øvre grænse for floodfill (\texttt{lo} og \texttt{up}). Her holdes
også en procentsats for hvor stort et margin vi bruger i udvælgelse af
regioner. Endelig har vi et felt der angiver hvilken metode der er
blevet brugt til at finde resultatet. Dette er en tekststreng og vil i
tilfældet af den naive algoritme være sat til \texttt{'naive'}. Felterne
\texttt{trsh1}, \texttt{trsh2} og \texttt{marignPercentage} er
repræsenteret som floats i databasen, mens \texttt{lo} og \texttt{up}
står som heltal. Afslutningsvis har hver kørsel en unik id. Således vi
kan tilknytte indgange i tabellen \texttt{result} (tabel
\ref{resultTable}) til et sæt af parametre.

Vi beskriver et resultat, som det antal af regioner vi får ud fra
analysen af et snit på et givet billede. Som beskrevet i afsnit
\ref{section_opdeling} vil vi, givet en snitratio, typisk have fire snit
hvor vi vil finde regioner i nærheden af. En undtagelse er hvis
snitratioen deler billedet i to lige store dele. Vi vil i dette tilfælde
kun have to snit vi kigger på. Tabellen \texttt{result} fortæller os
hvilket af de mulige snit vi har med at gøre, hvilket billede resultatet
er tilknyttet, hvilke parametre der er blevet brugt samt hvor mange
regioner vi har fundet. Tabellen gør det muligt at gemme resultater fra
kørsler med forskellige parametre, hvorved man kan have data fra
separate kørsler i databasen. Man har da grundlag for at sammenligne
kørsler med forskellige metoder og parametre.

Tabellen \texttt{region} (tabel \ref{regionTable}) holder alle de fundne
regioner fra vores analyse. Hver region henviser til det resultat som
denne tilhører. Vi kan således skelne de enkelte regioner fra hinanden
og afgøre i hvilket snit af billedet de ligger. En region bliver
repræsenteret som dens areal og afgrænsende rektangel.

\subsection{Vurdering}
Databaseskemaet har været underlagt følgende designmål:

\begin{itemize}
    \item Minimere redunsdans
    \item Mulighed for senere udvidelse
    \item Mange forespørgelsesmuligheder (andet ord)
    \item Mulighed for rekonstruktion af kørte analyser
    \item Mulighed for at kunne analysere flere snit i én kørsel
\end{itemize}

Vi har dog feltet \texttt{numberOfRgions} i tabellen \texttt{result} som
på sin vis er redundant, da vi blot kan finde antallet af tilknyttede
regioner ved at bruge simple SQL-sætninger. At have antallet stående
direkte i databasen giver dog et umiddelbart bedre overblik over
resultaterne end ved at bruge SQL-sætninger. Når databasen vokser, ved
at ændre på parametrene, vil SQL-sætningerne skulle søge meget af
databasen igennem for at returnere en meget simpel forespørgsel. Derfor
har vi valgt at gemme antallet at fundne regioner direkte i databasen.

Databaseskemaet er let at udvide, hvis vi i udviklingen af mere
avancerede metoder, skulle få brug for flere indgange. Specielt kunne
tabellen \texttt{region} udvides med en vægt som et mål for ``hvor
meget'' denne region ligger i snittet.
}

% vim: set tw=72 spell spelllang=da:
