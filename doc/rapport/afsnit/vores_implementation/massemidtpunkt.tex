{
Denne udvidelse tager udgangspunkt i den eksisterende naive
fremgangsmåde, men søger at redefinere den idéelle position, for at en
region ligger i det gyldne snit. Hvor den naive fremgangsmåde tilgodeser
regioner som ligger op ad snittet, leder vi nu efter regioner som i
højere grad er centreret på snittet. Vi leder nu efter regioner med
massemidtpunkt i det gyldne snit, og kigger ikke længere på denne
regions afgrænsende rektangel. Et eksempel på dette kan ses i
figur \ref{hus}, hvor den sorte region ikke anses som liggende i det
gyldne snit af den naive fremgangsmåde. Vi vil gerne have, at sådanne
regioner, klassificeres som en positiv, interessant region.

\begin{figure}[h]
	\begin{center}
		\includegraphics[scale=0.3,angle=0]{afsnit/vores_implementation/billeder/udvidet_loesning/husworks.png}
	\end{center}
	\caption[]{Et hus som bliver skåret over, så midten ligger inde for
    snittets margin.}
	\label{hus}
\end{figure}

\subsubsection{Opdeling af region med et gitter}
Når vi ikke længere bruger regionens afgrænsende rektangel, har vi brug
for en anden måde at repræsentere denne på. Vi laver derfor en
approksimation, af regionens form og udstrækning, ved at bruge et gitter.
Alt efter hvor finmasket dette gitter er, kan man justere hvor præcis
approksimationen af regionen skal være. Et eksempel på et gitter, ses i
figur \ref{grid}, hvor regionen beskrives ved de punkter, hvor to linjer
krydser, og befinder sig inden for regionen.

I praksis bruges det segmenterede billede fra floodfill-metoden, til at
konstruere regionens approksimation. Vi kontrollerer hver pixel, som er
en del af gitteret, i det afgrænsende rektangel for en given region, om
denne har samme farve, som regionen er blevet tildelt. Vi finder da et
undersæt af punkter til at beskrive regionen.

\begin{figure}[h]
    \centering
    \includegraphics[scale=0.76,angle=0]{afsnit/vores_implementation/billeder/udvidet_loesning/udvidetloesninglayer.png}
    \caption[]{Et gitter over en regions afgrænsende rektangel.}
    \label{grid}
\end{figure}

\subsubsection{Bedømmelse med hensyn til massemidtpunkt}
Givet en region $R$ betegner vi antallet af punkter i regionen med
$|R|$. Vi antager, at vi betragter et vertikalt snit $G$ i et billede.
Det gælder, for alle punkter $p \in R$, at de kan befinde sig ovenpå,
til højre eller til venstre for $G$. I denne sammenhæng lader vi $R_r$
og $R_l$ beskrive punkter, henholdsvis til højre og venstre for snittet
$G$.  Afstanden fra et punkt til kanten af et billede kaldes $D_p$, hvor
kanten er origo i billedet. Med disse informationer, kan vi afgøre, om
snittet deler regionen i to lige store dele, samt om store dele af
regionen, befinder sig langt væk fra snittet. Vi vil gerne kigge på en
regions massemidtpunkt, og se om dette ligger inden for margin. Vi
beregner regionens massemidtpunkt ved funktionen vist i \eqref{MPunkt.}

\begin{eqnarray}
    m(R) & = & \frac{\sum_{p \in R}{D_p}}{|R|} \label{masssemidpunkt}
    \label{MPunkt}
\end{eqnarray}

hvor $m(R)$ giver os en værdi for, hvad for en lige linje, der deler
regionen op i to dele, som er mest ens.

\begin{figure}[h]
    \begin{center}
        \includegraphics[scale=0.5,angle=0]{afsnit/vores_implementation/billeder/udvidet_loesning/cOMCutMargin.png}
    \end{center}
    \caption[]{Region, hvor massemidtpunkt, snit og margin er tegnet
    ind. Det ses, at massemidtpunktet, tegnet med en grøn linje, ligger
    inde for margin.}
    \label{cOMCutMargin}
\end{figure}

Hvis $m(R)$ ligger inden for margin, som tilfældet i figur
\ref{cOMCutMargin}, siger vi at regionen ligger i snittet. Det er dog
ikke helt nok, kun
at bedømme regionerne efter massemidtpunkt, da man kan konstruere
regioner, som vi egentlig ikke vil godtage, men som har massemidtpunkt inde for
marginen. Et eksempel på en sådan region er vist i figur \ref{dontwork},
hvor selve fordelingen af regionen er skæv.

\begin{figure}[h]
    \begin{center}
        \includegraphics[scale=0.5,angle=0]{afsnit/vores_implementation/billeder/udvidet_loesning/dontWork.png}
    \end{center}
    \caption[]{Region, som har massemidtpunkt inden for margin, men som
    ikke kvalificerer sig til at ligge i snittet pga. arealets
    fordeling.}
    \label{dontwork}
\end{figure}

Vi forsøger at løse dette problem, ved at kontrollere regionens
fordeling over snittet. Vi definerer, at en region er jævnt fordelt over
snittet, hvis forholdet mellem punkterne, på den højre og venstre side,
er lavere en $0.75$. Vi finder regionens fordeling, som vist i
\eqref{Fordeling}.

\begin{eqnarray}
    f(R) & = & \frac{|R_{l}| - |R_{r}|}{|R|}
    \label{Fordeling}
\end{eqnarray}

I \eqref{Fordeling} sammenlignes antal punkter, på begge sider af
snittet, og giver en procentsats for, hvor stor forskel der er mellem
siderne. Vi har at $f(R) \in [-1,1]$.  Hvis $f(R)$ er positivt, er der
$f(R)$ procent flere punkter på venstre side og vice versa. Her vises
netop, hvorfor denne fremgangsmåde adskiller sig markant fra den naive.
Den naive fremgangsmåde søger nemlig at udvælge de regioner $R$, hvor
størstedelen af punkter ligger på den ene side af snittet, dvs. $|f(R)|
\geq 0.75$. Den udvidede løsning gør nøjagtig det modsatte og finder
regioner som er jævnt fordelt over snittet.

% Fjernet, da dette vil være første gang vi viser et egentligt resultat.
% Man ved ikke hvad man ser. Sådanne billeder bør vente til afprøvning.
%
%Få
%at give et eksempel på hvornår vores naive algoritme virker, se figur \ref{centerOfMass}
%\begin{figure}[h]
%	\begin{center}
%		\includegraphics[scale=0.35,angle=0]{afsnit/vores_implementation/billeder/udvidet_loesning/centerOfMass.png}
%	\end{center}
%	\caption[]{Eksempel på hvordan den udvidet algoritme virker på et billedet hvor den naive algoritme ikke vil have fundet det}
%	\label{centerOfMass}
%\end{figure}

}

% vim: set tw=72 spell spelllang=da:
