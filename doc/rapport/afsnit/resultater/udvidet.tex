{
{\sffamily I dette afsnit præsenterer vi vores resultater fra
eksperimentet, hvor vi har brugt den udvidede metode til at vurdere
regionerne. Da eksperimentet blev kørt, satte vi opløsningen, på
regionernes approksimation med et gitter, for højt, hvilket resulterede
i, at kørslen ikke fik analyseret hele vores datasæt på grund af
tidsmangel.
}

\subsection{Reduceret datasæt}
Kørslen har nået at analysere $608$ malerier, men som vist herunder i
tabel \ref{ud_tabel_fjern_detaljer}, har vi kun $524$ brugbare
resultater, når vi har fjernet de billeder, som ikke er hele malerier.
Dette svarer til en nedgang på $13.82\%$.

\begin{table}[H]
    \centering
    \begin{tabular}{r@{\ \ }p{12em}r|r@{.}l}
            & Analyserede malerier & $608$ & $100$ & $00\%$   \\
        $-$ & Udsnit af malerier   &  $84$ &  $13$ & $82\%$   \\\hline
            & Resultater           & $524$ &  $86$ & $18\%$
    \end{tabular}
    \caption[]{Udregning af brugbare resultater fra udvidet kørsel.}
    \label{ud_tabel_fjern_detaljer}
\end{table}

Dette svarer til $3.65\%$ af de brugbare resultater fra den
naive kørsel. Vores datasæt bliver gennemgået alfabetisk efter
kunstnerens efternavn, så vi har kun resultater fra kunstnere med
efternavn startende med 'A' og 'B'.

At vi kun har et undersæt af malerierne gør, at vi bliver nødt til at se
på hvilke tidsperioder og skoler, vi har repræsenteret. Graferne
i figur \ref{ud_year_nation} viser, at italienske kunstnere er kraftigt
overrepræsenteret, og at malerierne primært er fra perioden 1401 --
1650. Dette er ikke overraskende, siden datasættet i forvejen
favoriserer denne type malerier.

\begin{figure}[!h]
    \centering
    \subfloat[Årstal]{
    \includegraphics[angle=-90,width=0.42\textwidth]{afsnit/resultater/billeder/year}
        \label{ud_year}}\hspace{1em}
    \subfloat[Skole]{
        \includegraphics[angle=-90,width=0.42\textwidth]{afsnit/resultater/billeder/nation}
        \label{ud_nation}}
    \caption[]{Antal malerier ved årstal og skole, fra resultater
    ved udvidet kørsel.}
    \label{ud_year_nation}
\end{figure}

\subsection{Håndtering af systematisk fejl}
Selvom denne analyse også har brugt den fejlbehæftede metode til
udtrækning af regioner, så sorteres duplikater fra, når vi approksimerer
en region. Vi har gemt den farve, som en region er blevet tildelt af
floodfill, og hvis denne ikke er at finde i regionens begrænsende
rektangel, betyder det, at regionen er blevet malet over senere. Vi
smider derfor denne region væk.

Den udvidede metode indeholder derfor ingen duplikater, men resultaterne
er ikke direkte sammenlignelige med dem fra den naive kørsel.

\subsection{Resultater}
Udregningen i tabel \ref{ud_tabel_fordeling} redegør for, hvor mange af
de brugbare resultater, der har mindst én interessant region liggende i
det gyldne snit. Vi ser, at der i $87.02\%$ af malerierne er blevet
fundet regioner i det gyldne snit, og vi kan derfor ikke afvise hypotese
\ref{hypo_binaer}.

\begin{table}[H]
    \centering
    \begin{tabular}{r@{\ \ }p{12em}r|r@{.}l}
            & Positive resultater   & $456$ &  $87$ & $02\%$ \\
        $+$ & Negative resultater   &  $68$ &  $12$ & $98\%$ \\\hline
            & Resultater i alt      & $524$ & $100$ & $00\%$
    \end{tabular}
    \caption[]{Et positivt resultat beskriver et maleri, hvori der er
    fundet mindst én interessant region i det gyldne snit ved brug af
    den udvidede vurdering af regioner. Over halvdelen af malerierne har
    således mindst én region i det gyldne snit.}
    \label{ud_tabel_fordeling}
\end{table}

Fordelingen af interessante regioner, over de fire gyldne snit i
malerierne, ses i tabel \ref{ud_tabel_fire_snit}. Ved at undersøge
afvigelsen mellem ekstremerne i snit 0 og 2, ser vi, at disse afviger med
$(499 - 405)/405 = 0.2320988 = 23.2 \%$ fra hinanden, og vi kan således
afvise hypotese \ref{hypo_fire_g_snit}.

\begin{table}[H]
    \centering
    \begin{tabular}{r@{\ \ }p{12em}r}
            & Regioner i $GS$ 0         &  $405$ \\
        $+$ & Regioner i $GS$ 1         &  $421$ \\
        $+$ & Regioner i $GS$ 2         &  $499$ \\
        $+$ & Regioner i $GS$ 3         &  $492$ \\\hline
            & Regioner i de fire $GS$   & $1817$
    \end{tabular}
    \caption[]{Forholdet mellem de interessante regioner fundet i de
    fire gyldne snit ved udvidet vurdering. Der er afvigelse mellem
    ekstremerne på $23.2 \%$. Ligesom i den naive kørsel, er det
    nederste horisontale snit det mest brugte.}
    \label{ud_tabel_fire_snit}
\end{table}

I figur \ref{antal_regioner_vertikale_cut_udvidet} er det samlede antal
fundne interessante regioner i de 20 vertikale snit, afbilledet for hele
datasættet.  Det ses, at der er fundet markant flere regioner i de to
snit, som ligger tættest på midten, og færrest ud mod kanterne.

\begin{figure}[h!]
	\begin{center}
		\includegraphics[width=0.9\textwidth]{afsnit/resultater/billeder/cut0cut1eatsperratioU.png}
	\end{center}
	\caption{Antal interessante regioner i hvert af de 20 vertikale snit
	ved udvidet kørsel.}
	\label{antal_regioner_vertikale_cut_udvidet}
\end{figure}

I figur \ref{antal_regioner_horisontale_cut_udvidet} er samme afbildning
lavet for det horisontale plan, hvor venstre side af grafen svarer til
toppen af malerierne. I denne graf er det snitratioerne $0.72$ og $0.82$
som har flest fundne regioner. Fra snitratio $0.52$ og ned falder antal
regioner gradvist. Fra snitratio $0.52$ og op svinger antallet af
regioner. Kanterne er igen klart de laveste i grafen, mht. antallet af
regioner.

\begin{figure}[h!]
	\begin{center}
		\includegraphics[width=0.9\textwidth]{afsnit/resultater/billeder/cut2cut3eatsperratioU.png}
	\end{center}
    \caption{Antal interessante regioner i hvert af de 20 horisontale
    snit ved udvidet kørsel, hvor venstre side af grafen repræsenterer
    øverste del af malerierne.}
    \label{antal_regioner_horisontale_cut_udvidet}
\end{figure}

Vi udregner den procentvise forskel på midten og det gyldne snit i
udregning \ref{Umid}, som giver at der er 27 \% flere interessante
regioner i midten en i det gyldne snit i den udvidet løsning.

\begin{eqnarray}
\sum{G} &=& 405+421+492+499 \label{Umid}\\ \nonumber
			&=& 1817 \\\nonumber
\sum{miten} &=& 517+518+647+623 \\\nonumber
			&=& 2305 \\\nonumber
\frac{\sum{miten}}{\sum{G}} &=& 1.27  \\ \nonumber
\end{eqnarray}

Ud fra disse observationer kan vi konkludere, at der ikke er flere
interessante regioner i det gyldne snit end i midten af malerierne, og
vi kan derfor forkaste hypotese \ref{hypo_alle_andre_snit} og
\ref{hypo_midten}.

I figur \ref{G_vs_to_trejedele_udvidet} er de fire gyldne snit samt
$\frac{2}{3}$ repræsenteret. Det ses, at der ikke er nogen entydighed
omkring hvilken ratio, som er dominerende, og vi kan derfor forkaste hypotese
\ref{hypo_to_tredjedele}.

Figur \ref{G_vs_to_trejedele_udvidet} viser også, at antallet af fundne
regioner i det gyldne snit og snitratioen $\frac{2}{3}$ ligger meget
tæt på hinanden, og den største forskel ligger ikke på mere end $15\%$.
Vi kan derfor ikke afvise hypotese \ref{hypo_15p}.

\begin{figure}[h!]
	\begin{center}
		\includegraphics[width=0.6\textwidth]{afsnit/resultater/billeder/G_vs_to_tredjedeleU.png}
	\end{center}
    \caption{Antal interessante regioner i de fire gyldne snit og deres
    tilhørende $\frac{2}{3}$-snit ved udvidet kørsel.}
	\label{G_vs_to_trejedele_udvidet}
\end{figure}

I graferne i figur \ref{udvidet_year} kan man se, hvor mange
interessante regioner der er fundet i gennemsnit per maleri i alle
tidsperioder.  Tidsperioder, hvor ingen malerier er analyseret, er ikke
taget med. Det ses, at der i perioden 1401 -- 1450 findes mange flere
end i de andre perioder og i visse tilfælde dobbelt så mange. Da
dobbelt så mange regioner er skarpt større end $10\%$ afvigelse, kan vi
forkaste hypotese \ref{hypo_tid}.

\begin{figure}[!h]
	\begin{center}
		\includegraphics[angle=0,width=0.90\textwidth]{afsnit/resultater/billeder/yearcutU.png}
	\end{center}
    \caption{Graf over det gennemsnitlige antal interessante regioner
    fundet i hver tidsperiode. Hver graf repræsenterer hvert deres snit
    i den udvidet kørsel.}
	\label{udvidet_year}
\end{figure}

I graferne i figur \ref{udvidet_nation} ses, at det er meget svingende,
hvor malerier, med mange interessante regioner i det gyldne snit, kommer
fra. Den nederlandske skole ligger lidt foran de andre. Den danske skole har slet ikke
nogen malerier, hvor der er fundet regioner i det gyldne snit. Alle
skoler, hvor analysen ikke har bearbejdet nogle malerier, er sorteret
fra. Da malerier fra den nederlandske skole gennemsnitligt har over
$10\%$ flere fundne regioner end den danske skole og den franske
skole, holder hypotese
\ref{hypo_nation} ikke, og kan derfor forkastes.

\begin{figure}[!h]
	\begin{center}
		\includegraphics[angle=0,width=0.90\textwidth]{afsnit/resultater/billeder/nationcutU.png}
	\end{center}
    \caption{Graf over gennemsnitligt antal interessante regioner fundet
    i det gyldne snit fra hver skole ved udvidet kørsel.}
	\label{udvidet_nation}
\end{figure}

\subsubsection{Antallet af fundne regioner i alle snit}
Analysen har fundet $17,705$ regioner i alle snit i malerierne, med
middelværdi $\mu = 33.79$ og standardafvigelse $\sigma = 18.58$.
Antallet af fundne regioner i malerierne er illustreret i figur
\ref{ud_graf_total_regions}. I figur \ref{ud_qq_total_regions} er vist
et QQ-plot, som viser, hvorvidt vores data er normalfordelt. Dette er
tilfældet, hvis punkterne følger diagonalen i grafen. Vi ser, at de
observerede data følger linjen nogenlunde, men at de har et udsving nederst
til venstre, hvilket antyder, at fordelingen er lidt skæv. Vi har i figur
\ref{ud_hist_total_regions} sammenlignet et histogram over de
observerede værdier med tæthedsfunktionen for normalfordelingen $X \sim
N(\mu = 33.79, \sigma^2 = 345.22)$. De observerede værdier følger dog
ikke de teoretiske værdier særlig pænt, og kun få steder falder de
teoretiske værdier inden for det observerede interval, hvorfor vi ikke
kan komme frem til et troværdigt konfidensinterval.

Med et større antal fundne regioner kunne vi meget vel komme tættere på
en normalfordeling, og vi kunne da udtale os mere sikkert om det
forventede antal fundne regioner i et arbitrært maleri.

\begin{figure}[!h]
    \centering
    \subfloat[]{
        \includegraphics[width=0.49\textwidth]{afsnit/resultater/billeder/exp_totalregions}
        \label{ud_graf_total_regions}
    }
    \subfloat[]{
        \includegraphics[width=0.49\textwidth]{afsnit/resultater/billeder/qq_exp_totalregions}
        \label{ud_qq_total_regions}
    }\\
    \subfloat[]{
        \includegraphics[width=0.62\textwidth]{afsnit/resultater/billeder/hist_exp_totalregions}
        \label{ud_hist_total_regions}
    }
    \caption[]{Fordelingen af fundne regioner på malerier ved udvidet
    vurdering.
    \textbf{\ref{ud_graf_total_regions}:} Fordelingen af de fundne
    regioner.
    \textbf{\ref{ud_qq_total_regions}:} QQ-plot som viser, at vi er tæt
    på at have en normalfordeling med den teoretiske fordeling $X \sim N(\mu,
    \sigma^2)$.
    \textbf{\ref{ud_hist_total_regions}:} Histogram med
    tæthedsfunktionen for normalfordeling, hvor $\mu = 33.79$ og
    $\sigma^2 = 345.22$.
    }
    \label{ud_total_regions_plots}
\end{figure}

\subsection{Opsamling}
Vi vil nu samle op på de ovenstående resultater for vores eksperiment
med udvidet vurdering af interessante regioner. Tabel
\ref{hypoteser_udvidet} viser, hvordan vores resultater forholder sig
til vores opstillede hypoteser.  Vi giver her vores vurdering af, hvad
resultaterne siger om brugen af det gyldne snit, men vær opmærksom på,
at vi ikke er fagfolk i kunstforståelse, og at vores fortolkninger kun
bygger på de observerede resultater.

\begin{table}[!h]
    \centering
    \begin{tabular}{|c|l|c|c|}
        \hline
        \textbf{Hypotese nr.} & \textbf{Beskrivelse} & \textbf{Afvist} &
        \textbf{Ikke afvist}  \\\hline\hline
        1 & Mindst én region i $GS$                     &            & \checkmark   \\\hline
        2 & Alle fire $GS$ lige meget brugt             & \checkmark &              \\\hline
        3 & $1/3$ har lærred med forholdet $1:\varphi $ & \checkmark$^{\textrm{*}}$ &              \\\hline
        4 & Flest regioner i $GS$                       & \checkmark &              \\\hline
        5 & Flere regioner i $GS$ end $\frac{2}{3}$     & \checkmark &              \\\hline
        6 & Flere regioner i $GS$ end i midten          & \checkmark &              \\\hline
        7 & $GS$ brugt lige meget, uanset tidsperiode   & \checkmark &              \\\hline
        8 & $GS$ brugt lige meget, uanset skole         & \checkmark &              \\\hline
        9 & $\frac{2}{3}$ brugt som approksimation til $GS$   &      & \checkmark	\\\hline
    \end{tabular}
    \caption[]{Hypoteser i forhold til den udvidede kørsel. $GS$ bruges som
    forkortelse for det gyldne snit.  $^{\textrm{*}}$Jvf. udregning.
    \ref{tabel_real_dimensions} }
    \label{hypoteser_udvidet}
\end{table}

Hypotese 1 kan ikke afvises, hvilket betyder, at over halvdelen af de
analyserede malerier har mindst én interessant region i det gyldne
snit. Da vi mener, at malerier konstrueret efter det gyldne snit vil
have mindst én interessant region i det gyldne snit, kan vi ikke afvise
at det gyldne snit bliver brugt af kunstnere i malerier.

Vores resultater afviser hypotese 2, hvilket betyder, at antallet af
interessante regioner fundet i de enkelte gyldne snit afviger meget fra
hinanden.  Dette tyder på, at kunstnere favoriserer nogle gyldne snit
frem for andre. Når denne hypotese afvises, mener vi ikke, at det gyldne
snit kan tillægges nogen speciel æstetisk værdi: Tror man på, at det
gyldne snit er specielt æstetisk tiltalende, må \emph{alle} gyldne snit
være lige æstetisk tiltalende. Derfor må man jo netop forvente at finde
nogenlunde samme antal regioner i hvert snit.

Da hypotese 3 ikke bestemmes af, hvilken metode interessante regioner
bedømmes efter, har vi allerede afvist denne. Nærmere forklaring er
givet i afsnit \ref{naiv_opsamling} ved eksperimentet med naiv vurdering
af regioner.

Vi afviser hypotese 4, idet der ikke er blevet fundet flere interessante
regioner i det gyldne snit end i alle andre snit.  I malerier komponeret
efter det gyldne snit, må man forvente, at de interessante regioner
koncentrerer sig om snittet. Vi mener derfor ikke, at det er retfærdigt
at sige, at kunstneren arbejder ud fra det gyldne snit, når et maleri
komponeres.

Hypotese 5 afvises, fordi vi ikke kan finde flere interessante regioner
i det gyldne snit, end i snittet ved $\frac{2}{3}$, i alle fire snit. Vi
mener derfor, at dette er en indikation på, at kunstneren ikke er helt
bevidst om, hvor det gyldne snit ligger, hvis han da forsøger at male
efter dette.

Vi afviser hypotese 6, fordi vi finder flere interessante regioner i
midten af malerier, end i det gyldne snit.  Vi mener derfor, at der er
mere belæg for at sige, at kunstnere opbygger deres malerier ud fra
midten end for at tale om speciel anvendelse af det gyldne snit.

Hypotese 7 afvises af grafen i figur \ref{udvidet_year}, fordi det
gennemsnitlige antal af interessante regioner mellem tidsperioder
afviger med mere end $10 \%$. Det er interessant at se på antallet af
regioner fundet i det gyldne snit omkring de historiske milestene. Det
lader til, at brugen af det gyldne snit falder efter 1550. I tidsperioden
1851 -- 1900 har vi kun to malerier, hvilket kan være forklaringen på
det drastiske fald i antal fundne regioner.

Hypotese 8 bliver afvist, da det gennemsnitlige antal af fundne
interessante regioner i det gyldne snit afviger for meget mellem
skoler. Vi mener, at dette indikerer, at nogle skoler er mere
fascineret af det gyldne snit end andre.

Vi kan ikke afvise hypotese 9. Det betyder, at antallet af interessante
regioner fundet i det gyldne snit, ikke afviger med mere end $15 \%$ fra
antallet fundet i snittet ved $\frac{2}{3}$. Dette kan tyde på, at
snittet ved $\frac{2}{3}$ bruges som en approksimation til det gyldne
snit, hvilket også passer meget fint sammen med at hypotese 5 er blevet
afvist.

} % Eh eh eh. Nallerne væk!

% vim: set tw=72 spell spelllang=da:
