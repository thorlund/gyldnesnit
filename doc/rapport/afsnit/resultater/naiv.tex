{
\subsection{Overvejelser}
Før der konkluderes noget over resultaterne er det vigtigt at se på
grundlaget, det meste af dette er givet tidligere i rapporten.
Fordi det er den første kørsel bliver tærskelværdierne sat ud fra hvad
der tidligere i rapporten er blevet bedømt som optimalt.
Som det kan læses TODO: fix ref når dette afsnit er
lavet og omdefiner\ref{chap:chap_afproevning}.
Problemet ved dette er at de samme tærskelværdier kan give forskellige
resultater på forskellige billeder.
På grund af tidsbegrænsningen af denne opgave er det efterladt til
fremtidigt arbejde, og det efterlades som en usikkerhed.

Der er blevet snakket en del om usikkerheden både i
\ref{chap:chap_afproevning} og om tærskelværdierne.
Det er dog meget vigtigt at kigge på størrelsen af datasættet, før alle
resultater afvises på grund af usikkerhederne.
Følgende SQL sætning tæller antallet af billeder, som benyttes i et
eksperiment.
\begin{verbatim}
sqlite> select Count(*) from painting where form = "painting";
17364
\end{verbatim}
Algoritmen vil altså blive udført på 17364 billeder, for at opsummere så
ligger det gyldne snit, altså fire steder på et billede. Grunden til det
pointeres i kombination med usikkerhed er for at holde fokus på præcis
hvor algoritmen lægger sit arbejde. Hvis algoritmen har en generel
svaghed er det unægteligt problematisk, men speciale tilfælde vil
forsvinde på grund af størrelsen af datasættet.

Databasen er sat op til at kunne indeholde alle de data, der skal
indsamles under eksperimentet. På grund af det behøves der ikke at tages
speciale hensyn til f.eks. at en statistik om hvilken periode der indeholder
flest gyldnesnit. Dette eksempel er en del af den udvidede løsning og
brugt netop for at vise at der ikke skal tages hensyn pga. databasens
opbygning.

Nu er det den første kørsel på en så stor billedebase, derfor er det
vigtigt at så meget data som muligt bliver opsamlet. Det er selvfølgelig
også derfor at der er blevet snakket så meget om alle de forskellige
indstillinger en sidste vigtig ting er hvilke snit skal der arbejdes på.
Det ville selvfølgelig være oplagt at vælge $\Phi$ og $2/3$. Det kunne
dog godt betyde at der gik informationer tabt. Hvordan bygges en kørsel
så op hvis der skal findes mest mulige data, svaret på dette ligger i de
tekniske begrænsninger og usikkerheden om hvor tæt to snit må ligge på
hinanden.
Med de tekniske begrænsninger menes mest af alt at algoritmen er bygget
på fire snit, derfor skal der reelt kun findes snit der dækker halvdelen
af billedet fordi der ellers vil komme meget redundant information.
Derudover er tiden også en begrænsning. Ud fra udokumenterede tests
virker det til at hvert snit tager op mod en halv dag.
Tiden sammen med usikkerheden giver en minimumsgrænse for hvor mange
snit det er muligt at have. Margin skal ifølge \ref{margin} minimum være
$2.4\%$, potentielt betyder dette at der kunne være $50/2.4=20.8$ snit.
Da op til 10 dages ventetid er lidt meget specielt i og med det er den
første kørsel over alle billederne, vælges det halve altså $10$ snit
eller et snit hver $5\%$.
\subsection{Resultaterne}
\begin{verbatim}
number of features in the golden ratio in different periodes

{'1301-1350\r\n': 151894, '1551-1600\r\n': 184246, '1201-1250\r\n': 419, '1851-1900\r\n': 15092, '1101-1150\r\n': 2817, '1651-1700\r\n': 171119, '1351-1400\r\n': 35464, '1251-1300\r\n': 11864, '1451-1500\r\n': 428338, '1701-1750\r\n': 115703, '1151-1200\r\n': 14688, '1751-1800\r\n': 67703, '1801-1850\r\n': 79182, '1601-1650\r\n': 273832, '1401-1450\r\n': 199989, '1501-1550\r\n': 394100}
Which golden ration is the most popular, ranging from 0 to 3
[56092, 57044, 59181, 54152]
features in the different ratios
{0.66803398874999997: 222018, 0.86803398875000004: 206899, 0.56803398875: 229650, 0.96803398875000002: 183833, 0.76803398874999995: 213570, 0.91803398874999997: 208340, 0.81803398875: 208081, 0.71803398875000002: 217432, 0.51803398874999995: 230144, 0.61803398875000004: 226462}
Top 10 cuts, where the most features was found
[239, 250, 254, 257, 274, 288, 298, 326, 430, 436]
Top 10 images
[546, 552, 554, 569, 570, 578, 592, 616, 634, 675]
Top 10 images, with only the features in the golden feature
[73, 75, 76, 77, 78, 86, 87, 87, 99, 147]
Top 10 images, with only features in 2/3 that counts
[144, 146, 171, 204, 209, 211, 221, 228, 251, 300]
\end{verbatim}
TODO:tilføj en ud af hvor mange billeder der var i den periode!
	og hvilke billeder der er i top 10!
	og en fordelen af hvor mange features der er i billeder generelt.

Det mest overraskende er nok hvordan intensiteten af interessante
regioner stiger alt efter hvor tæt på midten snittet ligger.
Der er nogle billeder som indeholder et helt utroligt antal af
interessante regioner.
TODO: Indsæt nogle af de billeder.
}
% vim: set tw=72 spell spelllang=da:
