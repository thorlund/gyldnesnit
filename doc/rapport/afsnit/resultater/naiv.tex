{
Hypotesen for dette eksperiment er at det gyldnesnit ikke bliver brugt
mere end andre snit.
\subsection{Eksperimentsopstilling}
I \ref{chap:chap_afproevning} er de optimale tærskelværdier fundet.
Idet denne hypotese blot kigger på frekvensen for brug af det gyldnesnit
mod andre snit, hvis eneste restriktion er at de skal have et fælles
forhold til det gyldnesnit.
Det er også fordelagtigt at maksimere antallet af andre snit, da det
giver et bedre grundlag for eksperimentet.
Afstanden mellem to snit er begrænset af margin defineret til at være
$2.4\%$\ref{margin}. 
Denne margin skal være tilstede på begge sider af et snit, så derfor vil
hvert snit fylde $(2.4*2)\%$.
Det maksimale antal af snit på et billede må altså være
$100/4.8=20.833$. Hvilket ses på denne figur \ref{snitogmargin}
\begin{figure}[h!]
	\begin{center}
		\includegraphics[scale=0.3]{afsnit/resultater/billeder/20_cuts_med_margin}
	\end{center}
	\caption{Sort:snittene, grøn: margins og rød er midten}
	\label{snitogmargin}
\end{figure}

Eksperimentet bliver kørt på alle 17364 billeder.
\newpage
\subsection{Forventninger}
Det er tydeligt ud fra hypotesen at eksperimentet fejler hvis der bliver
fundet flere interessante regioner i det gyldne snit end i alle andre
snit.

\subsection{Resultater}
%Indsæt graf over de forskellige ratios her.
\begin{figure}
	\begin{center}
		blah blah TODO insæt graf over antallet af regioner fundet i de
		forskellige snit
	\end{center}
	\caption{Antal af detektere interessante regioner på de forskellige snit.}
	\label{diffratios}
\end{figure}
Der er kun to snit, hvor der bliver detekteret et mindre antal
interessante regioner end i gyldne snit nemlig $0.968$ og
$0.868$. Ganske tydeligt er det dog at se at der er et klart forhold til
hvor tæt på midten grafen kommer.
Yderpunkterne $0.958$ og $0.518$ er problematiske, $0.958$'s margin
løber udover billedet, dvs. at den ren principielt går glip af at
detektere en masse interessante regioner.
$0.518$ lider af det modsatte problem, den kan potentielt fange
interessante regioner, på begge sider af midten.
For at være helt præcis så er det i $0.518$ tilfælde:\\
$1-0.518 = 0.482$
$0.518-0.482=0.036$\\
Hvilket er afstanden mellem de to snit.
Der er altså en stimmel på $0.05-0.036 = 0.014 = 1.4\%$ af billedet,
hvor interessante regioner bliver talt to gange.


\begin{verbatim}
number of features in the golden ratio in different periodes

{'1301-1350\r\n': 151894, '1551-1600\r\n': 184246, '1201-1250\r\n': 419, '1851-1900\r\n': 15092, '1101-1150\r\n': 2817, '1651-1700\r\n': 171119, '1351-1400\r\n': 35464, '1251-1300\r\n': 11864, '1451-1500\r\n': 428338, '1701-1750\r\n': 115703, '1151-1200\r\n': 14688, '1751-1800\r\n': 67703, '1801-1850\r\n': 79182, '1601-1650\r\n': 273832, '1401-1450\r\n': 199989, '1501-1550\r\n': 394100}
Which golden ration is the most popular, ranging from 0 to 3
[56092, 57044, 59181, 54152]
features in the different ratios
{0.66803398874999997: 222018, 0.86803398875000004: 206899, 0.56803398875: 229650, 0.96803398875000002: 183833, 0.76803398874999995: 213570, 0.91803398874999997: 208340, 0.81803398875: 208081, 0.71803398875000002: 217432, 0.51803398874999995: 230144, 0.61803398875000004: 226462}
Top 10 cuts, where the most features was found
[239, 250, 254, 257, 274, 288, 298, 326, 430, 436]
Top 10 images
[546, 552, 554, 569, 570, 578, 592, 616, 634, 675]
Top 10 images, with only the features in the golden feature
[73, 75, 76, 77, 78, 86, 87, 87, 99, 147]
Top 10 images, with only features in 2/3 that counts
[144, 146, 171, 204, 209, 211, 221, 228, 251, 300]
\end{verbatim}
TODO:tilføj en ud af hvor mange billeder der var i den periode!
	og hvilke billeder der er i top 10!
	og en fordelen af hvor mange features der er i billeder generelt.

Det mest overraskende er nok hvordan intensiteten af interessante
regioner stiger alt efter hvor tæt på midten snittet ligger.
Der er nogle billeder som indeholder et helt utroligt antal af
interessante regioner.
TODO: Indsæt nogle af de billeder.
}
% vim: set tw=72 spell spelllang=da:
