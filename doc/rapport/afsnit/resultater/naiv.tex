{
{\sffamily I det følgende vil vi samle op på resultaterne fra det kørte
eksperiment med den naive vurdering af interessante regioner. Vores
metode til udtrækning af regioner, har dog indeholdt en fejl, som vi
først vil beskrive.
}

\subsection{Systematisk fejl\label{program_bug}}
I den naive løsning findes en fejl i udtrækningen af regioner, som gør,
at vi finder nogle af regionerne flere gange. Denne fejl er nævnt i
afsnit \ref{section_impBilledbehandling} og gennemgået i bilag
\ref{appendix_bug}. Det betyder, at de resultater, vi fremviser for
kørslen af den naive løsning, ikke stemmer overens med det egentlige
antal. For at give en indblik i hvor mange regioner, der bliver fundet
flere gange, har vi afprøvet programmet med fejlen på 11 malerier og
sammenlignet dem med en kørsel, hvor fejlen er rettet.

Resultaterne for de 11 malerier er opstillet i tabel \ref{bug_tabel},
hvor kolonne 2 og 3 er antal regioner fundet over alle snit i billedet,
uden og med fejl.  Fjerde kolonne er en procentsats for, hvor mange
flere regioner, der bliver fundet, når den fejlbehæftede kode køres. Den
procentvise spredning er $[60 \%,~377 \%]$, så de resultater, vi får,
kan være op til $377\%$ for høje.  Gennemsnitligt er der $172 \%$ flere
regioner. Vi kan godt antage, at antallet af regioner, opgivet i
resultaterne for den naive løsning, er mindst dobbelt så store.

\begin{table}[!h]
    \centering
    \begin{tabular}{|l|c|c|c|}
        \hline
  Maleri  & Bug løst 		& Bug ikke løst		& Procentvis forskel\\\hline
        1   & 70 			& 112 				& 60 \% \\
        2   & 27 			& 45 				& 67 \% \\
        3	& 52 			& 98 				& 88 \% \\
        4   & 84 			& 145 				& 73 \% \\
        5	& 32 			& 57 				& 78 \% \\
        6   & 88 			& 170 				& 93 \% \\
        7   & 85 			& 201 				& 136 \% \\
        8   & 78 			& 197 				& 153 \% \\
        9   & 164 			& 554 				& 238 \% \\
        10	& 16 			& 75 				& 368 \% \\
        11	& 115 			& 548 				& 377 \% \\\hline
	Total	& 881			& 2202				& 150 \% \\\hline
	  \end{tabular}
    \caption[]{Tabel for antal fundne regioner i alle snit i versionen
    med og uden fejlen som duplikerer regioner. Den sidste kolonne
    viser, hvor mange flere procent regioner der bliver fundet.}
    \label{bug_tabel}
\end{table}

De 11 malerier, der er brugt i tabel \ref{bug_tabel}, er valgt ud fra
tilfældige udtræk fra billeddatabasen, dog med de kriterier, at alle
forskellige typer af malerier skal være inkluderet, såsom landskabsmalerier,
portrætter etc.

\subsection{Resultater}
Vi har kørt eksperimentet med naiv vurdering på $17,364$ malerier, men i
vores resultater sorterer vi $2,989$ af disse fra, da de kun er udsnit
af et større maleri.  Som vist i tabel \ref{tabel_fjern_detaljer}
herunder, er dette en nedgang på $17.21 \%$ og vi har $14,375$
brugbare resultater tilbage.

\begin{table}[H]
    \centering
    \begin{tabular}{r@{\ \ }p{12em}r|r@{.}l}
            & Analyserede malerier & $17,364$ & $100$ & $00\%$   \\
        $-$ & Udsnit af malerier   &  $2,989$ &  $17$ & $21\%$   \\\hline
            & Resultater           & $14,375$ &  $82$ & $79\%$
    \end{tabular}
    \caption[]{Udregning af brugbare resultater for naiv kørsel.}
    \label{tabel_fjern_detaljer}
\end{table}

Af de brugbare resultater ser vi i tabel \ref{tabel_fordeling}, at der
i $91.43 \%$ af malerierne er fundet mindst én region, som ligger i
det gyldne snit. Vi kan derfor ikke afvise hypotese \ref{hypo_binaer}.

\begin{table}[H]
    \centering
    \begin{tabular}{r@{\ \ }p{12em}r|r@{.}l}
            & Positive resultater   & $13,143$ &  $91$ & $43\%$ \\
        $+$ & Negative resultater   &  $1,232$ &   $8$ & $57\%$ \\\hline
            & Resultater i alt      & $14,375$ & $100$ & $00\%$
    \end{tabular}
    \caption[]{Et positivt resultat beskriver et maleri, hvori der er
    fundet mindst én region, som ligger i det gyldne snit i den naive
    kørsel. Et negativt resultat er et maleri, hvori der ikke findes
    nogen regioner i det gyldne snit.}
    \label{tabel_fordeling}
\end{table}

Vi vil gerne se på, hvordan fordelingen af fundne interessante regioner
i de fire gyldne snit ser ud. Fordelingen er vist i tabel
\ref{tabel_fire_snit}. Ekstremerne findes i snit 2 og 3 og afviger med
$(46432-41547)/41547 = 0.1175777 = 11.8 \%$. Da $11.8 \% > 10 \%$,
kan vi derfor afvise hypotese \ref{hypo_fire_g_snit}.

\begin{table}[H]
    \centering
    \begin{tabular}{r@{\ \ }p{12em}r}
            & Regioner i $GS$ 0             &  $42,884$ \\
        $+$ & Regioner i $GS$ 1             &  $44,042$ \\
        $+$ & Regioner i $GS$ 2             &  $46,432$ \\
        $+$ & Regioner i $GS$ 3             &  $41,547$ \\\hline
            & Regioner i de fire $GS$       & $168,650$
    \end{tabular}
    \caption[]{Forholdet mellem de interessante regioner fundet i de
    fire gyldne snit ved den naive kørsel. Ekstremerne $41,547$ og
    $46,432$ afviger med $11.8 \%$. Det nederste horisontale snit er det
    mest brugte.}
    \label{tabel_fire_snit}
\end{table}

Vi undersøger nu, hvor mange af de brugbare resultater, der er forsynet
med dimensioner i databasen, således at vi kan undersøge, hvorvidt
lærredet er konstrueret som et gyldent rektangel. Udregningen i tabel
\ref{tabel_med_dimensioner} viser, at ud af de brugbare resultater,
mangler $2,410$ malerier dimensionerne, og vi har da $11,965$ malerier
tilbage, som kan undersøges for den gyldne ratio i lærredets
dimensioner.

\begin{table}[H]
    \centering
    \begin{tabular}{r@{\ \ }p{14em}r|r@{.}l}
            & Resultater                     & $14,375$ & $100$ & $00\%$ \\
        $-$ & Resultater uden dimensioner    &  $2,410$ &  $16$ & $77\%$ \\\hline
            & Resultater med dimensioner     & $11,965$ &  $83$ & $23\%$
    \end{tabular}
    \caption[]{Brugbare resultater med dimensioner i databasen.}
    \label{tabel_med_dimensioner}
\end{table}

Vi undersøger nu, for hvor mange af de $11,965$ malerier, det gælder, at
dets lange side, $L$, divideret med dets korte side, $K$, ligger i
intervallet $G = [1.57920117302, 1.65686680448] = \varphi \pm 2.4\%$.
Tabel \ref{tabel_real_dimensions} viser, at kun $3.99\%$ falder inden
for dette interval, og vi kan således afvise hypotese
\ref{hypo_golden_ractangle}, da denne kræver, at $33.334 \%$ af
malerierne har $L/K \in G$.

\begin{table}[H]
    \centering
    \begin{tabular}{r@{\ \ }p{14em}r|r@{.}l}
            & $L/K \in G$                  &    $478$ &   $3$ & $99\%$ \\
        $+$ & $L/K \notin G$               & $11,487$ &  $96$ & $01\%$ \\\hline
            & Resultater med dimensioner   & $11,965$ & $100$ & $00\%$
    \end{tabular}
    \caption[]{Resultater med dimensioner, hvor disse er et gyldent
    rektangel med en afvigelse på højst $2.4\%$. Under en tredjedel af
    malerierne har et lærred konstrueret efter det gyldne rektangel.}
    \label{tabel_real_dimensions}
\end{table}

Grafen i figur \ref{antal_regioner_vertikale_cut} viser antallet af
interessante regioner fundet i de vertikale snit i vores datasæt. Det
ses, at regionerne koncentrerer sig omkring midten af malerierne.

Figur \ref{antal_regioner_horisontale_cut} viser grafen for antallet af
interessante regioner fundet i det horisontale plan. Vi ser, at der
findes flest regioner i snittet med snitratio $0.92$. Fra midten af
malerierne og op bliver der fundet færre og færre regioner i snittene.
Dette kan skyldes at malerier med horisontlinje normalt har få regioner over
horisonten og mange under horisonten.

Vi udregner den procentvise forskel på midten og det gyldne snit i
udregning \ref{mid}, som giver at der er 2 \% flere interessante
regioner i midten end i det gyldne snit i den naive løsning.

\begin{eqnarray}
\sum{G} &=& 42884+44042+41547+46432 \label{mid}\\ \nonumber
			&=& 174905 \\\nonumber
\sum{midten} &=& 45176+44347+44660+44108 \\\nonumber
			&=& 178291 \\\nonumber
\frac{\sum{midten}}{\sum{G}} &=& 1.02  \\ \nonumber	
\end{eqnarray}

Hvis man samler informationerne for de to grafer (figur
\ref{antal_regioner_vertikale_cut} og
\ref{antal_regioner_horisontale_cut}), kan man se, at regionerne samler
sig fra den horisontale midte og ned i maleriet, mens vi i det vertikale
plan har maksimum i midten af maleriet. Dette strider mod hypotese
\ref{hypo_alle_andre_snit} og \ref{hypo_midten}, og vi kan derfor
forkaste dem.

\begin{figure}[h!]
	\centering
	\includegraphics[width=0.9\textwidth]{afsnit/resultater/billeder/cut2cut3eatsperratio.png}
    \caption{Antal interessante regioner ved naiv kørsel i hvert af de
    20 horisontale snit, hvor venstre side af grafen repræsenterer
    øverst del af malerierne. $G$ står for det gyldne snit. Der findes
    færre interessante regioner i toppen af malerierne.}
	\label{antal_regioner_horisontale_cut}
\end{figure}

\begin{figure}[h!]
	\centering
	\includegraphics[width=0.9\textwidth]{afsnit/resultater/billeder/cut0cut1eatsperratio.png}
    \caption{Antal interessante regioner ved naiv kørsel i hvert af de
    20 vertikale snit. $G$ står for det gyldne snit. Vi finder flest
    regioner i midten af malerierne.}
	\label{antal_regioner_vertikale_cut}
\end{figure}

Grafen i figur \ref{G_vs_to_trejedele} viser forskellen på det gyldne
snit og snittet ved $\frac{2}{3}$ i de fire snit. Som man kan se, er der
ikke meget forskel på det gyldne snit og $\frac{2}{3}$, og den største
procentvise forskel ligger på $2.34\%$. Da $2.34 \% < 15\%$, kan vi ikke
afvise hypotese \ref{hypo_15p}.

Figur \ref{G_vs_to_trejedele} viser også, at selv om snittene ligger tæt
på hinanden, så har det gyldne snit ca. $2 \%$ flere regioner i alle
fire snit. Vi kan derfor ikke afvise hypotese \ref{hypo_to_tredjedele}.

\begin{figure}[h!]
	\centering
	\includegraphics[width=0.6\textwidth]{afsnit/resultater/billeder/G_vs_to_tredjedele.png}
    \caption{Procentvis antal interessante regioner i de fire gyldne
    snit og deres tilhørende $\frac{2}{3}$ snit ved naiv kørsel.}
	\label{G_vs_to_trejedele}
\end{figure}

Grafen i figur \ref{naiv_year} viser, hvor mange interessante regioner
der er fundet i gennemsnit per maleri over alle tidsperioder. Det ses,
at i tidsperioden år 1301--1350 bliver der fundet ca. 17 regioner per
maleri, hvilket er det klart største antal. I årene 1200--1250 bliver
der fundet færrest regioner, ca. 7 i gennemsnit. Faktisk bliver der
fundet mere en dobbelt så mange regioner i årene 1301--1350, end i årene
1200--1250, hvilket betyder, at hypotese \ref{hypo_tid} kan forkastes.

\begin{figure}[!h]
	\centering
	\includegraphics[angle=0,width=0.90\textwidth]{afsnit/resultater/billeder/yearcut.png}
    \caption{Graf over gennemsnitligt antal interessante regioner fundet
    for hver tidsperiode ved naiv kørsel. Hver bjælke repræsenterer en
    tidsperiode.}
	\label{naiv_year}
\end{figure}

\begin{figure}[!h]
	\centering
	\includegraphics[angle=0,width=0.90\textwidth]{afsnit/resultater/billeder/yearNrImage.png}
	\caption{Antal malerier per tidperiode ved naiv kørsel.}
	\label{naiv_yearNrImage}
\end{figure}

I graferne i figur \ref{naiv_nation}, vises samme repræsentation som i
figur \ref{naiv_year}, blot er det her maleriernes skole i stedet for
år. Den græske, nederlandske og catalanske skole har klart flest
interessante regioner per maleri, hvorimod den norske og skotske ligger
i bunden.  Skolerne har en forskel på ca. 12 regioner, hvilket svarer
til ca.  $240\%$. Da $240 \% > 10 \%$, holder hypotese \ref{hypo_nation}
ikke.

\begin{figure}[!h]
	\centering
	\includegraphics[angle=0,width=0.90\textwidth]{afsnit/resultater/billeder/nationcut.png}
    \caption{Graf over gennemsnitligt antal fundne interessante regioner
    fundet i hver sin skole ved naiv kørsel. Hver bjælke
    repræsenterer en skole.}
	\label{naiv_nation}
\end{figure}

\begin{figure}[!h]
	\centering
	\includegraphics[angle=0,width=0.90\textwidth]{afsnit/resultater/billeder/nationNrImage.png}
	\caption{Antal malerier per skole ved naiv kørsel.}
	\label{naiv_nationNrImage}
\end{figure}

\subsubsection{Antallet af fundne regioner over alle snit}
I alle $14,375$ malerier er der i alt fundet $1,650,082$ interessante
regioner. Vi har en middelværdi på $114.79$ med standardafvigelse på
$86.91$. I figur \ref{total_regions_plots} er vist grafer over, hvordan
antallet af fundne regioner fordeler sig i malerierne. Plottet i figur
\ref{graf_total_regions_zoom} ser bort fra $298$ malerier, hvori der
ikke er fundet nogen regioner. Regionerne i malerierne har en fordeling,
der kunne ligne en eksponentialfordeling, som vist i histogrammet i
figur \ref{hist_total_regions}. De observerede værdier afviger dog noget
fra de teoretiske værdier, og vi kan derfor ikke sige, at vi har en
eksponentialfordeling.

\begin{figure}[!h]
    \centering
    \subfloat[]{
        \includegraphics[width=0.49\textwidth]{afsnit/resultater/billeder/totalregions_var}
        \label{graf_total_regions_var}
    }
    \subfloat[]{
        \includegraphics[width=0.49\textwidth]{afsnit/resultater/billeder/totalregions}
        \label{graf_total_regions_zoom}
    }\\
    \subfloat[]{
        \includegraphics[width=0.62\textwidth]{afsnit/resultater/billeder/hist_totalregions}
        \label{hist_total_regions}
    }
    \caption[]{Fordelingen af fundne regioner på malerier.
    \textbf{\ref{graf_total_regions_var}:} Plot som viser hvor mange
    malerier, hvori der er fundet et vist antal interessante regioner
    over alle snit i maleriet.
    \textbf{\ref{graf_total_regions_zoom}:} Samme plot som i figur
    \ref{graf_total_regions_var}, men hvor antal malerier uden regioner
    ikke er vist.
    \textbf{\ref{hist_total_regions}:} Histogram over antal fundne
    regioner, hvor en eksponentialfordeling med $\lambda = 1/114.79$ er
    indtegnet.}
    \label{total_regions_plots}
\end{figure}

Middelværdien fortæller os at i et arbitrært maleri, kan vi forvente at
finde $115$ interessante regioner over alle snit.  Bemærk, at en region
\emph{altid} vil ligge i fire snit og derfor meget vel kan indgå fire
gange. Endvidere skal vi huske på, at hver af disse regioner kan være
udtrukket flere gange, på grund af den fejlbehæftede metode til
udtrækning. Vi ser, at der er et lille antal malerier, hvori der bliver
fundet rigtig mange regioner. Det tynder dog ud omkring $380$ regioner,
hvor antallet af malerier, med flere regioner, forekommer mindre
hyppigt.

\subsection{Opsamling\label{naiv_opsamling}}
Vi har i tabel \ref{hypoteser_naiv} givet en oversigt over, hvordan de
observerede værdier forholder sig til vores hypoteser. Vi vil give vores
vurdering af, hvad vores data kan fortælle os om brugen af det gyldne
snit i malerier. Bemærk, at vi på ingen måde er fagfolk i
kunstforståelse, og at vores fortolkning kun er begrundet ud fra de
resultater, vi har fra vores eksperiment med naiv vurdering af
interessante regioner.

\begin{table}[!h]
    \centering
    \begin{tabular}{|c|l|c|c|}
		\hline
        \textbf{Hypotese nr.} & \textbf{Beskrivelse} & \textbf{Afvist} &
        \textbf{Ikke afvist}  \\\hline\hline
        1 & Mindst én region i $GS$                     &            & \checkmark   \\\hline
        2 & Alle fire $GS$ lige meget brugt             & \checkmark &              \\\hline
        3 & $1/3$ har lærred med forholdet $1:\varphi $ & \checkmark &              \\\hline
        4 & Flest regioner i $GS$                       & \checkmark &              \\\hline
        5 & Flere regioner i $GS$ end $2/3$             &            & \checkmark   \\\hline
        6 & Flere regioner i $GS$ end i midten          & \checkmark &              \\\hline
        7 & $GS$ brugt lige meget, uanset tidsperiode   & \checkmark &              \\\hline
        8 & $GS$ brugt lige meget, uanset skole         & \checkmark &              \\\hline
        9 & $2/3$ brugt som approksimation til $GS$     &            & \checkmark	\\\hline
    \end{tabular}
    \caption[]{Hypoteser i forhold til den naive kørsel. $GS$ bruges som
    forkortelse for det gyldne snit.}
    \label{hypoteser_naiv}
\end{table}

Vi kan ikke afvise hypotese 1. Vi mener, at et maleri, som er komponeret
efter det gyldne snit, vil have mindst én interessant region i det
gyldne snit, og vi kan derfor ikke afvise, at det gyldne snit bliver
brugt i malerkunsten.

Når vi afviser hypotese 2, betyder det, at der ikke er fundet det samme
antal interessante regioner i de fire gyldne snit. Vi mener, at hvis man
antager, at det gyldne snit er specielt æstetisk tiltalende, så må de
fire gyldne snit være lige tiltalende. Alligevel antyder resultaterne,
at kunstnere foretrækker nogle gyldne snit fremfor andre.

Vi ser at kunstnere ikke konstruerer lærredet efter det gyldne
rektangel, hvorfor hypotese 3 afvises. Vi mener, at dette sår tvivl om,
hvorvidt det gyldne rektangel er et specielt æstetisk tiltalende format.
Hvis dette var tilfældet, ville vi forvente, at flere malerier havde
dette format.

Det er vores opfattelse, at hvis kunstneren arbejder ud fra det gyldne
snit, når han komponerer et maleri, må vi forvente at finde flere
interessante regioner i det gyldne snit, end i noget andet snit i
maleriet. Vores resultater har afvist, at dette kan være tilfældet,
hvorfor hypotese 4 ikke holder.

Vi har ikke afvist hypotese 5 og 9. Det betyder, at antallet af
interessante regioner fundet i det gyldne snit dominerer antallet af
regioner fundet i alle fire snit ved $\frac{2}{3}$, men at der ikke er
mere end $15 \%$ afvigelse mellem disse. Dette tyder på, at når det
gyldne snit bliver brugt i et maleri, så er dette et bevidst valg fra
kunstnerens side. Noget tyder dog på, at snittet ved to tredjedele
bruges som approksimation til det gyldne snit. Dette er en modstrid, og
vi kan derfor ikke sige noget fornuftigt om, hvorvidt det gyldne snit
bruges bevidst eller om der bruges en approksimation.

Hypotese 6 afvises, hvilket betyder, at der findes flere interessante
regioner i midten af billedet end i det gyldne snit. Dette tyder på, at
kunstneren ikke arbejder ud fra det gyldne snit, men i stedet ud fra
midten, når han komponerer et maleri.

Fordi vi afviser hypotese 7, hvor vi ser på det gennemsnitlige antal
interessante regioner i det gyldne snit per maleri, kan vi sige, at der
er forskel på, hvor mange regioner der findes i det gyldne snit
tidsperioder imellem.  Dette tyder på, at det gyldne snit gennem
tiden har haft svingende popularitet. Det er spændende at se, som vist i
figur \ref{naiv_year}, hvordan regionerne fordeler sig mht. til
de historiske milestene i det gyldne snits historie. Det tyder faktisk
på en nedgang i brugen af det gyldne snit, da Luca Pacioli udgiver sin
bog \emph{De divina propertione} i 1509\cite{Markowsky1992}. Heller ikke
i 1835, hvor udtrykket \emph{det gyldne snit} bruges for første
gang\cite{Markowsky1992}, ses nogen speciel ændring i brugen af det
gyldne snit.

Endelig afvises også hypotese 8. Helt konkret har vi stor afvigelse i
antallet af interessante regioner fundet i det gyldne snit mellem
skoler. Vi mener, at dette viser, at det gyldne snit har større
udbredelse og popularitet i nogle skoler. Grafen i figur
\ref{naiv_nation} antyder, at der i malerier af den græske skole findes
mange regioner i det gyldne snit. Vi kan dog i figur
\ref{naiv_nationNrImage} se at vores grundlag for at udtale os om netop
denne skole er vagt, på grund af et meget lille antal af disse malerier
i datasættet. På trods af ovenstående tyder data dog på, at det gyldne
snit er mere brugt inden for visse skoler.

}

% vim: set tw=72 spell spelllang=da:
