{
{\sffamily Vi vil i dette kapitel se på, hvilke resultater vi, med vores
metoder, er kommet frem til. Vi præsenterer en række hypoteser og
undersøger, hvorvidt vores data kan be- eller afkræfte disse hypoteser.
Vi vil også gerne se, om vores data afdækker nogle interessante
observationer vedrørende kunstneres oprindelse eller fødselsår.
}

\section{Hypoteser}
{
{\sffamily Vi vil i det følgende præsentere de hypoteser, som vi, på
baggrund af resultaterne fra en analyse på vores datasæt, vil forsøge at
besvare. De fleste af hypoteserne antager den generelle opfattelse, at
det gyldne snit er specielt æstetisk tiltalende og af denne grund, er
meget brugt i malerkunsten. Vi deler hypoteserne op i nogle kategorier,
alt efter hvilket aspekt hypotesen belyser. Inden vi præsenterer de
omtalte hypoteser, er det dog nødvendigt at kaste et kritisk blik på
vores datasæt.
}

\subsection{Datasæt}
Det korpus, vi kører vores analyse på, består af billeder hentet fra
hjemmesiden wga.hu\cite{wgahu}, som er en online billededatabase, med
europæiske kunstartikler fra år 1001 -- 1900. I kunstartiklerne, hvor
det samlede antal er omkring 23.000, indgår møbler, kalkmalerier,
skulpturer, mosaikker og malerier, hvor sidstnævnte, vil være vores
fokus. Over halvdelen af disse kunstartikler står udstillet på museum.
Databasen blev oprettet i 1996, med det formål at præsentere kunst fra
renæssancen (ca.  14. -- 17.  århundrede), men blev senere udvidet, til
også at inkludere kunst fra andre perioder. Dette betyder, at
størstedelen af malerierne vi undersøger, er fra tidsperioden 1450 --
1650 og er malet af italienske kunstnere. Endvidere er langt de fleste
malerier, klassificeret som religiøse.  Disse informationer er givet fra
wga.hu, men er også suppleret i bilag \ref{appendix_grafer} som
grafer.

Vi må af ovenstående grunde forvente, at resultater, fra en analyse på
vores datasæt, vil være farvet af samlingen af malerier, og det derfor
kan være svært at drage nogen konklusioner for malerkunsten generelt, da
resultaterne vil være begrænset, til kun at gælde for et udsnit af
vestlig kultur. Endvidere findes der ingen nyere malerier i datasættet,
hvilket gør at vi ikke kan udtale os om nyere malerkunst.

Billederne, som suppleres fra databasen, er af høj kvalitet, men der er
visse problemer, som vi nævner nedenfor.

\begin{itemize}
    \item \textbf{Beskæring af billeder}\\
        Vi kan ikke vide os sikre på, om billederne i datasættet er
        ordentligt beskåret. Dvs. vi \emph{kan} have, at noget af
        billedrammen er med i billedet. Dette kan muligvis volde lidt
        problemer med udtrækning af regioner, men hvad værre er, så gør
        det vores mål, for hvor det gyldne snit ligger, upræcist. Dette
        har vi dog taget højde for, i kraft af vores margin.  Endeligt
        er der inkluderet billeder af malerier detaljer i databasen, som
        er udsnit af maleriet, således at målene på billedet ikke
        passer.
    \item \textbf{Forvrængning og perspektiv}\\
        Billederne af malerier er taget med et kamera, hvor linsen muligvis kan
        forvrænge billedet. Vi kan derfor have skæve linjer og tage
        forkerte beslutninger, for regioner, pga. dette. Endvidere kan
        billedet være taget skævt, således at billedet hælder til den
        ene side. Vi kan selvfølgelig også have at perspektivet i
        billedet er forkert, fordi billedet er taget fra en skæv vinkel.
    \item \textbf{Opdelte malerier}\\
        Nogle store malerier kan være blevet opdelt, da databasen har
        det formål at vise malerierne på en computerskærm, hvor meget
        store billeder kan være svære at betragte. Dette betyder, at
        nogle billeder ikke viser hele maleriet, men blot er et udsnit,
        hvilket påvirker vores muligheder for at sige noget fornuftigt
        om det gyldne snit i maleriet.
\end{itemize}

Nogle stilarter, såsom kalkmalerier og tegninger, har gennem
udokumenterede afprøvninger, vist sig at være besværlige at analysere,
pga. meget svingende farvegengivelse. Endvidere, kan disse være billeder
af en hvælving i en kirke, som ikke egner sig til analyse for det gyldne
snit. Vi har derfor valgt kun at analysere malerier, kendetegnet ved at
de er beskrevet som ``painting'' fra wga.hu.

Alt det ovenstående vil påvirke resultaterne, ved analyse på vores
datasæt.

\subsection{Generelle hypoteser}
Følgende hypoteser er ret generelle. Altså sådan nogenlunde.

\begin{hypotese}
    Hvis det gyldne snit er meget brugt i malerkunsten, må vi have, at
    over halvdelen, af de analyserede malerier, har én eller flere
    regioner liggende i det gyldne snit.
\end{hypotese}

\begin{hypotese}
    Hvis et gyldent snit er specielt æstetisk tiltalende, må vi have, at
    antallet af regioner liggende hvert af de fire snit, som kan
    betragtes som gyldne, ikke afviger mere end $\pm10\%$ fra hinanden.
\end{hypotese}

\begin{hypotese}
    Hvis det gyldne rektangel er et specielt æstetisk tiltalende format,
    må vi have, at mere end en tredjedel malerierne har et lærred, hvis
    dimensioner er lig $\varphi$. \textbf{Problemer med billeder som er
    skåret op.}
\end{hypotese}

\subsection{Antallet af regioner i det gyldne snit}
Følgende hypoteser omhandler antallet af detekterede regioner i det
gyldne snit. Altså sådan nogenlunde.

\begin{hypotese}
    Hvis det gyldne snit er meget brug i malerkunsten, må vi have, at
    antallet af detekterede regioner i det gyldne snit, er skarpt
    større, end antallet i alle andre snit.
\end{hypotese}

\begin{hypotese}
    Hvis det gyldne snit er bevidst brugt af kunstneren, må vi have, at
    antallet regioner liggende i det gyldne snit, er skarpt større, end
    antallet af regioner liggende i snittet ved to tredjedele.
\end{hypotese}

\begin{hypotese}
    Hvis det gyldne snit er bevidst brugt af kunstneren, må vi have, at
    antallet regioner liggende i det gyldne snit, er skarpt større, end
    antallet af regioner liggende det midterste snit.
\end{hypotese}

\subsection{Tidsperiode, kunstner og land}
Følgende hypoteser har noget at gøre med maleriernes ophav og arbejder
ud fra den antagelse, at hvis det gyldne snit \emph{altid} har været
interessant, så skal det altid have været til stede, uanset
nationalitet, årstal og kunstner. [Bør omformuleres].

\begin{hypotese}
    Hvis det gyldne snit altid har været specielt æstetisk tiltalende,
    må vi have, at antallet af regioner liggende i det gyldne snit,
    tidsperioder imellem, højest kan afvige med $\pm10\%$.
\end{hypotese}

\begin{hypotese}
    Hvis det gyldne snit altid er ligeså  æstetisk tiltalende for alle,
    må vi have, at antallet af regioner liggende i det gyldne snit,
    nationaliteter imellem, højest kan afvige med $\pm10\%$.
\end{hypotese}

\begin{hypotese}
    Hvis det gyldne snit er lige æstetisk tiltalende for alle personer,
    må vi have, at antallet af regioner liggende i det gyldne snit, mod
    antallet af regioner liggende i alle andre snit, højst må afgive med
    $\pm10\%$ mellem de enkelte kunstnere. Vi skal altså have, at
    procentdelen af regioner liggende i det gyldne snit, er nogenlunde
    den samme.  \textbf{Meget kontroversiel og nok ikke værd
    at undersøge}.
\end{hypotese}

\subsection{Det gyldne snit mod \emph{the rule of thirds}}
Vi undersøger her, om der virkelig gøres brug af en approksimation til
det gyldne snit, hvor man arbejder ud fra snittet ved to tredjedele.

\begin{hypotese}
    Hvis det gyldne snit er specielt æstetisk tiltalende, og vi har at
    \emph{the rule of thirds} bliver brugt i malerkunsten, som en
    approksimation til det gyldne snit, må vi have, at antallet af
    regioner liggende i det gyldne snit ikke angiver fra antallet af
    regioner i to tredjedele med mere end $15\%$.
\end{hypotese}

}
% vim: set tw=72 spell spelllang=da:


\section{Kørsel af naiv implementering\label{section_naiv_koersel}}
{
\subsection{Eksperimentsopstilling}
I \ref{chap_afproevning} er de optimale tærskelværdier fundet.
Idet denne hypotese blot kigger på frekvensen for brug af det gyldnesnit
mod andre snit, hvis eneste restriktion er at de skal have et fælles
forhold til det gyldnesnit.
Det er også fordelagtigt at maksimere antallet af andre snit, da det
giver et bedre grundlag for eksperimentet.
Afstanden mellem to snit er begrænset af margin defineret til at være
$2.4\%$\ref{margin}. 
Denne margin skal være tilstede på begge sider af et snit, så derfor vil
hvert snit fylde $(2.4*2)\%$.
Det maksimale antal af snit på et billede må altså være
$100/4.8=20.833$. Hvilket ses på denne figur \ref{snitogmargin}
\begin{figure}[ht]
	\begin{center}
		\includegraphics[scale=0.3]{afsnit/resultater/billeder/20_cuts_med_margin}
	\end{center}
	\caption{Sort:snittene, grøn: margins og rød er midten}
	\label{snitogmargin}
\end{figure}
Yderpunkterne $0.958$ og $0.518$ er problematiske, $0.958$'s margin
løber udover billedet, dvs. at den ren principielt går glip af at
detektere en masse interessante regioner.
$0.518$ lider af det modsatte problem, den kan potentielt fange
interessante regioner, på begge sider af midten.
For at være helt præcis så er det i $0.518$ tilfælde:\\
$1-0.518 = 0.482$
$0.518-0.482=0.036$\\
Hvilket er afstanden mellem de to snit.
Der er altså en stimmel på $0.05-0.036 = 0.014 = 1.4\%$ af billedet,
hvor interessante regioner bliver talt to gange.

Eksperimentet bliver kørt på 17364 billeder.

\newpage
\subsection{Resultater}
Resultaterne strider ikke mod hypotesen, dog tegner \ref{diffratios}
et meget et interessante billede. Det er kun de to snit der ligger
tættere på midten, der indeholder flere interessante regioner.
Antallet af interessante regioner stiger forholdvis konstant mellem
$0.77$ og op til $0.57$.

Med den nuværende algoritme og billedebase burde det gyldnesnit
altså ligge mellem $0.56803398875 +- 2.4\%$.
Og tyder meget på at midten langt mere er stedet kunstrene arbejder
ud fra.
\begin{figure}[ht]
	\begin{minipage}[b]{0.5\linewidth}
		\begin{center}
		\includegraphics[scale=0.4]{afsnit/resultater/billeder/cut0featsperratio.png}
		\caption{Antal af detekterede interessante regioner i det højre
		vertikale snit}
		\label{cut0feats}
		\end{center}
	\end{minipage}
	\hspace{0.5cm}
	\begin{minipage}[b]{0.5\linewidth}
		\begin{center}
		\includegraphics[scale=0.4]{afsnit/resultater/billeder/cut1featsperratio.png}
		\caption{Antal af detekterede interessante regioner i det
		venstre vertikale snit}
		\label{cut1feats}
		\end{center}
	\end{minipage}
\end{figure}
\begin{figure}[ht]
	\begin{minipage}[b]{0.5\linewidth}
		\begin{center}
		\includegraphics[scale=0.4]{afsnit/resultater/billeder/cut2featsperratio.png}
		\caption{Antal af detekterede interessante regioner i det højre
		vertikale snit}
		\label{cut0feats}
		\end{center}
	\end{minipage}
	\hspace{0.5cm}
	\begin{minipage}[b]{0.5\linewidth}
		\begin{center}
		\includegraphics[scale=0.4]{afsnit/resultater/billeder/cut3featsperratio.png}
		\caption{Antal af detekterede interessante regioner i det
		venstre vertikale snit}
		\label{cut1feats}
		\end{center}
	\end{minipage}
\end{figure}

\begin{figure}[h!]
	\begin{center}
		\includegraphics[scale=0.5]{afsnit/resultater/billeder/featsperratio.png}
	\end{center}
	\caption{Antal af detektere interessante regioner på de forskellige snit.}
	\label{diffratios}
\end{figure}




%\begin{verbatim}
%number of features in the golden ratio in different periodes
%
%{'1301-1350\r\n': 151894, '1551-1600\r\n': 184246, '1201-1250\r\n': 419, '1851-1900\r\n': 15092, '1101-1150\r\n': 2817, '1651-1700\r\n': 171119, '1351-1400\r\n': 35464, '1251-1300\r\n': 11864, '1451-1500\r\n': 428338, '1701-1750\r\n': 115703, '1151-1200\r\n': 14688, '1751-1800\r\n': 67703, '1801-1850\r\n': 79182, '1601-1650\r\n': 273832, '1401-1450\r\n': 199989, '1501-1550\r\n': 394100}
%Which golden ration is the most popular, ranging from 0 to 3
%[56092, 57044, 59181, 54152]
%features in the different ratios
%{0.66803398874999997: 222018, 0.86803398875000004: 206899, 0.56803398875: 229650, 0.96803398875000002: 183833, 0.76803398874999995: 213570, 0.91803398874999997: 208340, 0.81803398875: 208081, 0.71803398875000002: 217432, 0.51803398874999995: 230144, 0.61803398875000004: 226462}
%Top 10 cuts, where the most features was found
%[239, 250, 254, 257, 274, 288, 298, 326, 430, 436]
%Top 10 images
%[546, 552, 554, 569, 570, 578, 592, 616, 634, 675]
%Top 10 images, with only the features in the golden feature
%[73, 75, 76, 77, 78, 86, 87, 87, 99, 147]
%Top 10 images, with only features in 2/3 that counts
%[144, 146, 171, 204, 209, 211, 221, 228, 251, 300]
%\end{verbatim}
%TODO:tilføj en ud af hvor mange billeder der var i den periode!
%	og hvilke billeder der er i top 10!
%	og en fordelen af hvor mange features der er i billeder generelt.
%
}
% vim: set tw=72 spell spelllang=da:

\clearpage

\section{Kørsel af udvidet implementering\label{section_udvidet_koersel}}
{
{\sffamily Dette afsnit vil præsentere vores resultater fra
eksperimentet, hvor vi har brugt den udvidede metode, til at vurdere
regionerne. Da eksperimentet blev kørt, har vi sat opløsningen på
regionernes approksimation med et gitter for højt, hvilket er resulteret
i, at kørslen ikke er kommet igennem hele vores datasæt.
}

\subsection{Reduceret datasæt}
Kørslen har nået at analysere $608$ malerier, men som vist herunder i
tabel \ref{ud_tabel_fjern_detaljer}, har vi kun $524$ brugbare
resultater, når vi har fjernet de billeder, som ikke er hele malerier.
Dette svarer til en nedgang på $13.82\%$.

\begin{table}[H]
    \centering
    \begin{tabular}{r@{\ \ }p{12em}r|r@{.}l}
            & Analyserede malerier & $608$ & $100$ & $00\%$   \\
        $-$ & Udsnit af malerier   &  $84$ &  $13$ & $82\%$   \\\hline
            & Resultater           & $524$ &  $86$ & $18\%$
    \end{tabular}
    \caption[]{Udregning af brugbare resultater fra udvidet kørsel.}
    \label{ud_tabel_fjern_detaljer}
\end{table}

Dette svarer til $\mathsf{3.65\%}$ af de brugbare resultater fra den
naive kørsel. Vores datasæt bliver gennemgået alfabetisk efter
kunstnerens efternavn, så vi har kun resultater fra kunstnere med
efternavn startende med 'A' og 'B'.

\begin{figure}[!h]
    \centering
    \subfloat[Årstal]{
    \includegraphics[angle=-90,width=0.42\textwidth]{afsnit/resultater/billeder/year}
        \label{ud_year}}\hspace{1em}
    \subfloat[Nationalitet]{
        \includegraphics[angle=-90,width=0.42\textwidth]{afsnit/resultater/billeder/nation}
        \label{ud_nation}}
    \caption[]{Maleriernes årstal og nationalitet.}
    \label{ud_year_nation}
\end{figure}

\subsection{Håndtering af systematisk fejl}
Selvom denne analyse også har brugt den fejlbehæftede metode, til
udtrækning af regioner, så sorteres duplikater fra, når vi approksimerer
en region. Vi har gemt farven, som en region er blevet tildelt af
floodfill, og hvis denne farve ikke er at finde i regionens begrænsende
rektangel, betyder det, at regionen er blevet malet over senere. Vi
smider derfor denne region væk.

Den udvidede metode indeholder derfor ingen duplikater, men resultaterne
er ikke direkte sammenlignelige med dem fra den naive kørsel.

\subsection{Resultater}
Udregningen i tabel \ref{ud_tabel_fordeling} redegør for, hvor mange af
de brugbare resultater, der har mindst en region liggende i det gyldne
snit. Vi ser at der i $87.02\%$ af malerierne, er blevet fundet regioner
i det gyldne snit, og vi kan derfor ikke afvise hypotese
\ref{hypo_binaer}.

\begin{table}[H]
    \centering
    \begin{tabular}{r@{\ \ }p{12em}r|r@{.}l}
            & Positive resultater   & $456$ &  $87$ & $02\%$ \\
        $+$ & Negative resultater   &  $68$ &  $12$ & $98\%$ \\\hline
            & Resultater i alt      & $524$ & $100$ & $00\%$
    \end{tabular}
    \caption[]{Et positivt resultat beskriver et maleri hvori der er
    fundet mindst en region i det gyldne snit, ved brug af den udvidede
    vurdering af regioner.}
    \label{ud_tabel_fordeling}
\end{table}

Fordelingen af regioner, over de fire gyldne snit i malerierne, ses i
tabel \ref{ud_tabel_fire_snit}. Intet af de fire snit afviger med
mere end $10\%$ fra et andet, og vi kan således ikke afvise hypotese
\ref{hypo_fire_g_snit}.

\begin{table}[H]
    \centering
    \begin{tabular}{r@{\ \ }p{12em}r|r@{.}l}
            & Regioner i snit 0   &  $405$ &  $22$ & $29\%$ \\
        $+$ & Regioner i snit 1   &  $421$ &  $23$ & $17\%$ \\
        $+$ & Regioner i snit 2   &  $499$ &  $27$ & $46\%$ \\
        $+$ & Regioner i snit 3   &  $492$ &  $27$ & $08\%$ \\\hline
            & Regioner i alt      & $1817$ & $100$ & $00\%$
    \end{tabular}
    \caption[]{Forholdet mellem de interessante regioner fundet i de
    fire gyldne snit ved udvidet vurdering.}
    \label{ud_tabel_fire_snit}
\end{table}

\subsubsection{Antallet af fundne regioner over alle snit}
Analysen har fundet $17,705$ regioner over alle snit i malerierne, med
middelværdi $\mu = 33.79$ og standardafgivelse $\sigma = 18.58$.
Antallet af fundne regioner i malerierne er illustreret i figur
\ref{ud_graf_total_regions}. I figur \ref{ud_qq_total_regions} er vist
et QQ-plot, som viser hvorvidt vores data er normalfordelt. Dette er
tilfældet, hvis punkterne følger diagonalen i grafen. Vi ser, at de
observerede data følger linjen nogenlunde, men har et udsving nederst
til venstre, hvilket antyder at fordelingen er lidt skæv. Vi har i figur
\ref{ud_hist_total_regions} sammenlignet et histogram over de
observerede værdier, med tæthedsfunktionen for normalfordelingen $X \sim
N(\mu = 33.79, \sigma^2 = 345.22)$. De observerede værdier, følger dog
ikke de teoretiske værdier særlig pænt, og kun få steder falder de
teoretiske værdier inden for det observerede interval, hvorfor vi ikke
kan komme frem til et troværdigt konfidensinterval.

Med et større antal fundne regioner, kunne vi meget vel komme tættere på
en normalfordeling, og vi kunne da udtale os mere sikkert, om det
forventede antal fundne regioner i et arbitrært billede.

\begin{figure}[!h]
    \centering
    \subfloat[]{
        \includegraphics[width=0.49\textwidth]{afsnit/resultater/billeder/exp_totalregions}
        \label{ud_graf_total_regions}
    }
    \subfloat[]{
        \includegraphics[width=0.49\textwidth]{afsnit/resultater/billeder/qq_exp_totalregions}
        \label{ud_qq_total_regions}
    }\\
    \subfloat[]{
        \includegraphics[width=0.62\textwidth]{afsnit/resultater/billeder/hist_exp_totalregions}
        \label{ud_hist_total_regions}
    }
    \caption[]{Fordelingen af fundne regioner på malerier ved udvidet
    vurdering.
    \textbf{\ref{ud_graf_total_regions}:} Fordelingen af de fundne
    regioner.
    \textbf{\ref{ud_qq_total_regions}:} QQ-plot, som viser at vi er tæt
    på at have en normalfordeling med den teoretiske fordeling $X \sim N(\mu,
    \sigma^2)$.
    \textbf{\ref{ud_hist_total_regions}:} Histrogram med
    tæthedsfunktionen for normalfordeling, hvor $\mu = 33.79$ og
    $\sigma^2 = 345.22$.
    }
    \label{ud_total_regions_plots}
\end{figure}

\subsubsection{Spacer}

% Spacer! Ikke over denne linje HAHAHA

I graf \ref{antal_regioner_vertikale_cut_udvidet} er der afbilledet,
antale sammelet regioner i de 20 vertikale snit for helle datasættet.
Som man kan se er der fundet makant flere regioner i de to snit som
ligger tættest på miden, og færest ud i kanterne. 

\begin{figure}[h!]
	\begin{center}
		\includegraphics[width=0.9\textwidth]{afsnit/resultater/billeder/cut0cut1eatsperratioU.png}
	\end{center}
	\caption{Antal regioner i hvert af de 20 vertikale snit}
	\label{antal_regioner_vertikale_cut_udvidet}
\end{figure}

I graf \ref{antal_regioner_horisontale_cut_udvidet} er der samme
afbildning lavet, bare i det hoisontale plan, med venster side af grafen
svare til toppen af billedet. I denne graf er det 72,82 som peaker. Fra
52 og ned af, falder antal regioner gradvis. hvor i mod fra 52 og op gå
grafen lidt op og lidt ned. Kanterne er igen klart de lavest i grafen.

\begin{figure}[h!]
	\begin{center}
		\includegraphics[width=0.9\textwidth]{afsnit/resultater/billeder/cut2cut3eatsperratioU.png}
	\end{center}
	\caption{Antal regioner i hvert af de 20 horisontale snit, hvor venstre side af grafen repræsentere øverst del af malerierne}
	\label{antal_regioner_horisontale_cut_udvidet}
\end{figure}

Ud fra disse observationer kan vi konkludere at der ikke er flere
regioner i det gyldne snit, en midden af billedet, og kan derfor
forkaste hypotese \ref{hypo_alle_andre_snit} og \ref{hypo_midten}.

I graf \ref{G_vs_to_trejedele_udvidet}, er de fire gyldne snit, samt
$\frac{2}{3}$ representeret. Som man kan se der ikke nogen entydighed på
hvad for et ration, der er dominerne, så vi kan forkaste hyptese
\ref{hypo_to_tredjedele}.

\begin{figure}[h!]
	\begin{center}
		\includegraphics[width=0.6\textwidth]{afsnit/resultater/billeder/G_vs_to_tredjedeleU.png}
	\end{center}
	\caption{Procent vis antal regioner i de fire gyldne snit og deres tilhørene $\frac{2}{3}$ snit}
	\label{G_vs_to_trejedele_udvidet}
\end{figure}

I graferne \ref{udvidet_year}, kan man se de 4 gyldnes snit, og hvor
mange regioner der er fundet i gennemsnit per billedet i alle
tidsperioder. Tidsperioder hvor ingen malerier er analyseret, er ikke
taget med. Som man kan se i tidsperioden 1401-1450 for de fire snit,
bliver der fundet ca 0.2 flere regioner i gennemsnit. Da den
gennemsnit maksimale fundene regioner ligger og svinger mellem 1.1 og
1.5. Må der være mere en ti procent flere regioner i nogle tidsperioder
en andre. Så hypotese \ref{hypo_tid} kan forkastes. 

\begin{figure}[!h]
    \centering
    	\subfloat[Det gyldne snit 0.]{
	       	\includegraphics[angle=-90,width=0.42\textwidth]{afsnit/resultater/billeder/yearcut0U.png}
	       	\label{udvidet_year_cut0}}\hspace{1em}
		\subfloat[Det gyldne snit 1.]{
	       	\includegraphics[angle=-90,width=0.42\textwidth]{afsnit/resultater/billeder/yearcut1U.png}
	       	\label{udvidet_year_cut1}}\hspace{1em}\\
		\subfloat[Det gyldne snit 2.]{
	       	\includegraphics[angle=-90,width=0.42\textwidth]{afsnit/resultater/billeder/yearcut2U.png}
	       	\label{udvidet_year_cut2}}\hspace{1em}
		\subfloat[Det gyldne snit 3.]{
	       	\includegraphics[angle=-90,width=0.42\textwidth]{afsnit/resultater/billeder/yearcut3U.png}
	       	\label{udvidet_year_cut3}}\hspace{1em}
       \caption[]{Fire grafer over antal regioner fundet i vær sig
       tidsperiode. Vær graf repræsentere vær deres snit. Y aksen er det
       gemmesnitlige antal fundene regioner i snittet.}
     \label{udvidet_year}
\end{figure}
I grafene \ref{udvidet_nation}, kan 
\begin{figure}[!h]
    \centering
    	\subfloat[Det gyldne snit 0.]{
	       	\includegraphics[angle=-90,width=0.42\textwidth]{afsnit/resultater/billeder/nationcut0U.png}
	       	\label{udvidet_nation_cut0}}\hspace{1em}
		\subfloat[Det gyldne snit 1.]{
	       	\includegraphics[angle=-90,width=0.42\textwidth]{afsnit/resultater/billeder/nationcut1U.png}
	       	\label{udvidet_nation_cut1}}\hspace{1em}\\
		\subfloat[Det gyldne snit 2.]{
	       	\includegraphics[angle=-90,width=0.42\textwidth]{afsnit/resultater/billeder/nationcut2U.png}
	       	\label{udvidet_nation_cut2}}\hspace{1em}
		\subfloat[Det gyldne snit 3.]{
	       	\includegraphics[angle=-90,width=0.42\textwidth]{afsnit/resultater/billeder/nationcut3U.png}
	       	\label{udvidet_nation_cut3}}\hspace{1em}
       \caption[]{Fire grafer over antal regioner fundet i vær sig
       nationalitet. Vær graf repræsentere vær deres snit. Y aksen er det
       gemmesnitlige antal fundene regioner i snittet.}
     \label{udvidet_nation}
\end{figure}



\begin{table}[!h]
    \centering
    \begin{tabular}{|l|c|c|}
        \hline
            & Afvist & Ikke afvist  \\\hline
        1   &            & \checkmark   \\\hline
        2   &            & \checkmark   \\\hline
        3   & \checkmark$^{\textrm{*}}$ &              \\\hline
        4   & \checkmark &              \\\hline
        5   & \checkmark &    	\\\hline
        6   & \checkmark &              \\\hline
        7   & \checkmark &              \\\hline
        8   &            &              \\\hline
        9   &            & 	\\\hline
    \end{tabular}
    \caption[]{Hypoteser i forhold til den udvidet kørsel.
    $^{\textrm{*}}$Jvf. udregning \ref{tabel_real_dimensions}.
    }
    \label{hypoteser_udvidet}
\end{table}

} % Eh eh eh. Nallerne væk!

% vim: set tw=72 spell spelllang=da:


\section{Sammenligning\label{section_samlede_resultater}}
{
Vi vil i dette afsnit sammenligne resultaterne fra den naive og den
udvidede kørsel.

Hvis vi sammenligner resultaterne for interessante regioner i de
horisontale snit, kan vi ud fra graferne for naiv og udvidet kørsel
(figur \ref{antal_regioner_horisontale_cut} og
\ref{antal_regioner_horisontale_cut_udvidet}) se, at vi i begge kørsler har
et faldende antal regioner fra midten og ud mod venstre. Højre side er
nogenlunde stabil.

Hvis vi sammenligner de vertikale snit, viser graferne for naiv og
udvidet kørsel (figur \ref{antal_regioner_vertikale_cut} og
\ref{antal_regioner_vertikale_cut_udvidet}), at de begge har maksimum i
midten og falder ud mod kanterne.

En sammenligning af interessante regioner fundet i de forskellige
tidsperioder (figur \ref{naiv_year} og \ref{udvidet_year}) viser, at
begge metoder finder mange regioner per maleri i årene 1400 -- 1450, og
at der efter denne tidsperiode bliver fundet relativt få regioner per
maleri.

Hvis vi sammenligner det gennemsnitlige antal regioner fundet per
maleri, skoler imellem (figur \ref{naiv_nation} og
\ref{udvidet_nation}), ses det, at den græske og nederlandske skole er i
top tre ved begge metoder, samt at den spanske skole ligger lavest.

Ovenstående sammenligninger viser, at selvom vi bruger to forskellige
metoder, som begge giver et bud på hvilke regioner, der ligger i det
gyldne snit, så kommer de frem til nogenlunde samme resultater. De to
metoder be- og afkræfter stort set de samme hypoteser, og denne lighed
giver os i højere grad mulighed for at udtale os generelt om det gyldne
snit i digitaliserede malerier.

Bemærk, at antallet af regioner fundet over alle snit i malerierne
er meget forskellige de to metoder imellem (figur
\ref{graf_total_regions_var} og \ref{ud_graf_total_regions}). Ved den
naive metode findes et maksimum på 634 interessante regioner, mens vi
ved den udvidede maksimalt finder 90.  Dette understreger yderligere det
interessante i, at metoderne be- og afkræfter de samme hypoteser.

Det skal dog pointeres, at der er stor forskel i størrelsen på de to
datasæt brugt i kørslerne. Vi kan således ikke ukritisk sammenligne
resultaterne, men kun konstatere at graferne ligner hinanden.

}
% vim: set tw=72 spell spelllang=da:


\section*{Opsummering}
Dette kapitel har præsenteret resultaterne fra to analyser på vores
datasæt fra WGA og holdt dem op imod en samling af hypoteser.
Resultaterne fra en kørslen med den naive vurdering af regioner, har dog
brugt en fejlbehæftet metode til udtrækning af regioner.  Den udvidede
vurdering af regioner retter selv op på denne fejl, men disse resultater
dækker ikke hele datasættet.

Resultaterne giver os, at kun $3,99 \%$ af malerierne har det
gyldne rektangel i lærredet. Vi kan derfor ikke sige, at kunstnere
foretrækker at konstruere lærredet efter det gyldne rektangel.

Fra resultaterne kan vi også se, at der ikke findes flere interessante
regioner i det gyldne snit end i de andre snit i maleriet. Faktisk
findes der konsekvent flest interessante regioner i midten af
malerierne. Vi mener derfor ikke, at det er passende at kalde det gyldne
snit for \emph{specielt} æstetisk tiltalende, da det lader til at
interessante regioner i højere grad placeres i midten af maleriet.

Vi har observeret, at antallet af interessante regioner i det gyldne
snit afviger mellem skoler og tidsperiode. Heraf har vi udledt, at det
gyldne snit ikke er lige tiltalende for alle, men snarere har at gøre
med bl. a. kultur og stilperiode.

Resultaterne viser, at der findes flere interessante regioner i nogle
gyldne snit end i andre. Dette ser vi som en indikation på, at nogle
gyldne snit åbenbart opfattes som mere æstetisk tiltalende end andre. Vi
mener derimod ikke, at man kan udtage delmængder af de fire snit, da de
alle i princippet burde være lige æstetisk tiltalende.

Ud fra resultaterne har vi dog intet belæg for at kunne afvise brugen af
det gyldne snit, da der i langt de fleste malerier kan findes mindst én
interessant region placeret i et gyldent snit.

I det sidste kapitel vil vi kaste et blik tilbage på vores metoder, og
implementering af disse, for at vurdere, på hvilke områder der skal
sættes ind i fremtidigt arbejde.

}
% vim: set tw=72 spell spelllang=da:
