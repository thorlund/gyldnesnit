{
Vi vil i dette afsnit sammenligne resultaterne fra den naive og den
udvidede kørsel.

Hvis vi sammenligner resultaterne for interessante regioner i de
horisontale snit, kan vi ud fra graferne for naiv og udvidet kørsel
(figur \ref{antal_regioner_horisontale_cut} og
\ref{antal_regioner_horisontale_cut_udvidet}) se, at vi i begge kørsler har
et faldende antal regioner fra midten og ud mod venstre. Højre side er
nogenlunde stabil.

Hvis vi sammenligner de vertikale snit, viser graferne for naiv og
udvidet kørsel (figur \ref{antal_regioner_vertikale_cut} og
\ref{antal_regioner_vertikale_cut_udvidet}), at de begge har maksimum i
midten og falder ud mod kanterne.

En sammenligning af interessante regioner fundet i de forskellige
tidsperioder (figur \ref{naiv_year} og \ref{udvidet_year}) viser, at
begge metoder finder mange regioner per maleri i årene 1400 -- 1450, og
at der efter denne tidsperiode bliver fundet relativt få regioner per
maleri.

Hvis vi sammenligner det gennemsnitlige antal regioner fundet per
maleri, skoler imellem (figur \ref{naiv_nation} og
\ref{udvidet_nation}), ses det, at den græske og nederlandske skole er i
top tre ved begge metoder, samt at den spanske skole ligger lavest.

Ovenstående sammenligninger viser, at selvom vi bruger to forskellige
metoder, som begge giver et bud på hvilke regioner, der ligger i det
gyldne snit, så kommer de frem til nogenlunde samme resultater. De to
metoder be- og afkræfter stort set de samme hypoteser, og denne lighed
giver os i højere grad mulighed for at udtale os generelt om det gyldne
snit i digitaliserede malerier.

Bemærk, at antallet af regioner fundet over alle snit i malerierne
er meget forskellige de to metoder imellem (figur
\ref{graf_total_regions_var} og \ref{ud_graf_total_regions}). Ved den
naive metode findes et maksimum på 634 interessante regioner, mens vi
ved den udvidede maksimalt finder 90.  Dette understreger yderligere det
interessante i, at metoderne be- og afkræfter de samme hypoteser.

Det skal dog pointeres, at der er stor forskel i størrelsen på de to
datasæt brugt i kørslerne. Vi kan således ikke ukritisk sammenligne
resultaterne, men kun konstatere at graferne ligner hinanden.

}
% vim: set tw=72 spell spelllang=da:
