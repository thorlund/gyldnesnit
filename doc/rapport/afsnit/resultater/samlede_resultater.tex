{
{\sffamily Vi vil sammenligne resultaterne for den naive metode og den udvidet metode.
}


Hvis vi sammenligner resultaterne for fundene regioner i de horisontale
snit. Kan vi se, ud fra tabellerne \ref{antal_regioner_horisontale_cut} i
de naive resultater og \ref{antal_regioner_horisontale_cut_udvidet} i de udvidet
resultater, at fra midten og ud mod venstre i begge grafer, begynder
antal fundene regioner at falde gradvis. Hvor i mod højre side er nogle
lunde stabil.

Hvis vi sammenligner de vertikale snit. Kan vi se, ud fra tabellerne
\ref{antal_regioner_vertikale_cut} i de naive resultater og
\ref{antal_regioner_vertikale_cut_udvidet} i de udvidet resultater. At de begge
har maksimum i midten og falder ud mod kanterne af graferne.

En Sammenligningen af resultaterne, af regioner fundet i de forskellige
tidsperioder, visser at begge metoder finder mange regioner pr. maleri i
tidsperioden (1400 - 1450) og at efter denne tidsperiode bliver der
fundet relativ få regioner pr. billedet, graferne for tidsperioden kan
ses i graf \ref{naiv_year} og graf \ref{udvidet_year}.

Hvis vi sammenligner de to grafer for den gennemsnitlige antal regioner
fundet pr. maleri nationer i mellem, se graf \ref{naiv_nation} for den
naive løsning og graf \ref{udvidet_nation} for den udvidet. Kan vi se at
Grækenland og Holland er i top tre i begge grafer og at Spanien ligger
lavet i begge grafer. 

Ud for de resultats sammenligninger vi har gjort, kan vi konkludere at.
selv om det er to forskellige metoder, som begge giver et bud på hvilke
regioner der ligger i det gyldne snit. Kommer de frem til nogle lunde de
samme resultater. Og giver grafer som ligner hinanden på mange punkter.
Ensheden af resultatsæt kan kun bekræfte de konklutioner vi har draget
om det gyldne snit.

Det skal dog pointeres at datasættet for de to resultater har ret stor
størrelses forskel, og vi kan derfor ikke direkte sammenligne
resultaterne, kun se om graferne ligner hinanden. 
}
% vim: set tw=72 spell spelllang=da:
