{
{\sffamily Vi vil i det følgende præsentere de hypoteser, som vi, på
baggrund af resultaterne af den efterfølgende analyse på vores datasæt,
vil forsøge at verificere. De fleste af hypoteserne antager den
generelle opfattelse, at det gyldne snit er specielt æstetisk tiltalende
og af denne grund, er meget brugt i malerkunsten. Sidst i afsnittet vil
vi kaste et kritisk blik på vores datasæt, for at se på hvad vi kan
forvente af vores analyse.
}

\begin{hypotese}
    Hvis det gyldne snit er meget brugt i malerkunsten, må vi have, at
    over halvdelen, af de analyserede malerier, har én eller flere
    regioner liggende i det gyldne snit.
\end{hypotese}

\emph{eller}

\begin{hypotese}
    I en analyse på et stort antal malerier, vil over halvdelen af disse
    have mindst én region i det gyldne snit, da det gyldne snit bliver
    brugt meget i malerkunsten.
\end{hypotese}

\emph{eller}

\begin{hypotese}
    Det gyldne snit er meget brugt i malerkunsten, da over halvdelen, af
    de analyserede malerier, har én eller flere regioner liggende i det
    gyldne snit.
    \label{hypo_binaer}
\end{hypotese}

\hrule

\begin{hypotese}
    Hvis et gyldent snit er specielt æstetisk tiltalende, må vi have, at
    antallet af regioner liggende hvert af de fire snit, som kan
    betragtes som gyldne, ikke afviger mere end $\pm10\%$ fra hinanden.
\end{hypotese}

\emph{eller}

\begin{hypotese}
    Et gyldent snit er specielt æstetisk tiltalende, da antallet af
    regioner, liggende i hvert af de fire gyldne snit, ikke afviger med
    mere end $\pm10\%$ fra hinanden.
\end{hypotese}

\hrule

\begin{hypotese}
    Hvis det gyldne rektangel er et specielt æstetisk tiltalende format,
    må vi have, at mere end en tredjedel malerierne har et lærred, hvis
    dimensioner er lig $\varphi\pm2.4\%$.
\end{hypotese}

\emph{eller}

\begin{hypotese}
    Det gyldne rektangel er et specielt æstetisk tiltalende format,
    fordi mere end en tredjedel af alle malerier, har et lærred, hvor
    forholdet mellem dimensionerne  er lig $\varphi\pm2.4\%$.
\end{hypotese}

\hrule

\begin{hypotese}
    Hvis det gyldne snit er bevidst brugt af kunstneren, må vi have, at
    antallet regioner liggende i det gyldne snit, er skarpt større, end
    antallet af regioner liggende i snittet ved to tredjedele.
    \label{hypo_golden_ractangle}
\end{hypotese}

\emph{eller}

\begin{hypotese}
    Det gyldne snit bruges bevidst af kunstnere, da der findes flere
    regioner i det gyldne snit, end i snittet ved to tredjedele.
\end{hypotese}

\hrule

\begin{hypotese}
    Hvis det gyldne snit er meget brug i malerkunsten, må vi have, at
    antallet af detekterede regioner i det gyldne snit, er skarpt
    større, end antallet i alle andre snit.
\end{hypotese}

\emph{eller}

\begin{hypotese}
    Det gyldne snit er meget brugt i malerkunsten, fordi der findes
    flere regioner i det gyldne snit end antallet af regioner liggende i
    ethvert andet snit.
\end{hypotese}

\hrule

\begin{hypotese}
    Hvis det gyldne snit er bevidst brugt af kunstneren, må vi have, at
    antallet regioner liggende i det gyldne snit, er skarpt større, end
    antallet af regioner liggende det midterste snit.
\end{hypotese}

\emph{eller}

\begin{hypotese}
    Det gyldne snit bruges bevidst af kunstnere, da der findes flere
    regioner liggende i det gyldne snit end antallet af regioner
    liggende i det midterste snit.
\end{hypotese}

\hrule

\begin{hypotese}
    Hvis det gyldne snit altid har været specielt æstetisk tiltalende,
    må vi have, at antallet af regioner liggende i det gyldne snit,
    tidsperioder imellem, højest kan afvige med $\pm10\%$.
\end{hypotese}

\emph{eller}

\begin{hypotese}
    Det gyldne snit har altid været lige meget brugt i malerkunsten, fordi
    antallet af regioner i det gyldne snit, tidsperioder imellem, kun
    afviger med $\pm10\%$.
\end{hypotese}

\hrule

\begin{hypotese}
    Hvis det gyldne snit er ligeså  æstetisk tiltalende for alle,
    må vi have, at antallet af regioner liggende i det gyldne snit,
    nationaliteter imellem, højest kan afvige med $\pm10\%$.
\end{hypotese}

\emph{eller}

\begin{hypotese}
    Det gyldne snit er lige æstetisk tiltalende for alle, fordi antallet
    af regioner liggende i det gyldne snit, nationaliteter imellem, kun
    afviger med $\pm10\%$.
\end{hypotese}

\hrule

\begin{hypotese}
    Hvis det gyldne snit er specielt æstetisk tiltalende, og vi har at
    \emph{the rule of thirds} bliver brugt i malerkunsten, som en
    approksimation til det gyldne snit, må vi have, at antallet af
    regioner liggende i det gyldne snit ikke angiver fra antallet af
    regioner i to tredjedele med mere end $15\%$.
\end{hypotese}

\emph{eller}

\begin{hypotese}
    \emph{The Rule of Thirds} bliver brugt i malerkunsten som en
    approksimation til det gyldne snit, fordi antallet af regioner, som
    findes i det gyldne snit, ikke afviger med mere end $\pm15\%$ fra
    antallet af regioner, som findes i snittet ved to tredjedele.
\end{hypotese}

\subsection{Datasæt}
Det korpus, vi kører vores analyse på, består af billeder hentet fra
``The Web Gallery of Art''\cite{wgahu}, som er en online billededatabase, med
europæiske kunstartikler fra år 1001 -- 1900. I kunstartiklerne, hvor
det samlede antal er omkring 23.000, indgår møbler, kalkmalerier,
skulpturer, mosaikker og malerier, hvor sidstnævnte, vil være vores
fokus. Over halvdelen af disse kunstartikler står udstillet på museum.
Databasen blev oprettet i 1996, med det formål at præsentere kunst fra
renæssancen (ca.  14. -- 17.  århundrede), men blev senere udvidet, til
også at inkludere kunst fra andre perioder. Dette betyder, at
størstedelen af malerierne vi undersøger, er fra tidsperioden 1450 --
1650 og er malet af italienske kunstnere. Endvidere er langt de fleste
malerier, klassificeret som religiøse.  Disse informationer er givet fra
WGA, men er også suppleret i bilag \ref{appendix_grafer} som
grafer.

Vi må af ovenstående grunde forvente, at resultater, fra en analyse på
vores datasæt, vil være farvet af samlingen af malerier, og det derfor
kan være svært at drage nogen konklusioner for malerkunsten generelt, da
resultaterne vil være begrænset, til kun at gælde for et udsnit af
vestlig kultur. Endvidere findes der ingen nyere malerier i datasættet,
hvilket gør at vi ikke kan udtale os om nyere malerkunst.

Billederne, som suppleres fra databasen, er af høj kvalitet, men der er
visse problemer, som vi nævner nedenfor.

\begin{itemize}
    \item \textbf{Beskæring af billeder}\\
        Vi kan ikke vide os sikre på, om billederne i datasættet er
        ordentligt beskåret, hvilket betyder at vi \emph{kan} have, at
        noget af billedrammen er med i billedet. Dette kan muligvis
        volde lidt problemer med udtrækning af regioner, men hvad værre
        er, så gør det vores mål, for hvor det gyldne snit ligger,
        upræcist. Dette har vi dog taget højde for, i kraft af vores
        margin.  Endeligt er der inkluderet billeder af malerier
        detaljer i databasen, som er udsnit af maleriet, således at
        målene på billedet ikke passer.
    \item \textbf{Forvrængning og perspektiv}\\
        Billederne af malerier er taget med et kamera, hvor linsen muligvis kan
        forvrænge billedet. Vi kan derfor have skæve linjer og tage
        forkerte beslutninger, for regioner, pga. dette. Endvidere kan
        billedet være taget skævt, således at billedet hælder til den
        ene side. Vi kan selvfølgelig også have at perspektivet i
        billedet er forkert, fordi billedet er taget fra en skæv vinkel.
    \item \textbf{Opdelte malerier}\\
        Nogle store malerier kan være blevet opdelt, da databasen har
        det formål at vise malerierne på en computerskærm, hvor meget
        store billeder kan være svære at betragte. Dette betyder, at
        nogle billeder ikke viser hele maleriet, men blot er et udsnit,
        hvilket påvirker vores muligheder for at sige noget fornuftigt
        om det gyldne snit i maleriet.
\end{itemize}

Nogle stilarter, såsom kalkmalerier og tegninger, har gennem
udokumenterede afprøvninger, vist sig at være besværlige at analysere,
pga. meget svingende farvegengivelse. Endvidere, kan disse være billeder
af en hvælving i en kirke, som ikke egner sig til analyse for det gyldne
snit. Vi har derfor valgt kun at analysere malerier, kendetegnet ved at
de er beskrevet som ``painting'' fra WGA.

Alt det ovenstående vil påvirke resultaterne, ved analyse på vores
datasæt.

}
% vim: set tw=72 spell spelllang=da:
