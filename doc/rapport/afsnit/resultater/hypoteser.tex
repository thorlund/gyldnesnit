{
{\sffamily Vi vil i det følgende præsentere de hypoteser, som vi, på
baggrund af resultaterne fra en analyse, vil forsøge at besvare. De
fleste af hypoteserne antager den generelle opfattelse, at det gyldne
snit er specielt æstetisk tiltalende og af denne grund, er meget brugt i
malerkunsten. Vi deler hypoteserne op i nogle kategorier, alt efter
hvilket aspekt af det gyldne snit vi undersøger. Inden vi præsenterer de
omtalte hypoteser, er det dog nødvendigt at kaste et kritisk blik på
vores korpus.
}

\subsection{Korpus}
Hvilken data har vi rådighed? Hvor har vi den fra? Tidsperiode, lande,
kultur. Er malerierne ``anerkendt'' på nogen måde? Hvem har valgt dem
ud? Er der nogen kriterier for udvælgelse? Er der nogle kunstværker vi
mangler? Billedernes grafiske kvalitet. \textbf{Vi skal ind på at
billeder kan være delt op.}

\paragraph{Flyttet fra noget med database}
Vores korpus består af billeder hentet fra hjemmesiden \cite{wgahu} som
indeholder europæiske kunstartikler fra år 1001 -- 1900. I
kunstartiklerne, hvor det samlede antal er omkring 23.000, indgår
møbler, skulpturer, mosaikker og malerier, hvor sidstnævnte vil være
vores fokus.

\subsection{Generelle hypoteser}
Følgende hypoteser er ret generelle. Altså sådan nogenlunde.

\begin{hypotese}
    Hvis det gyldne snit er meget brugt i malerkunsten, må vi have, at
    over halvdelen, af de analyserede malerier, har én eller flere
    regioner liggende i det gyldne snit.
\end{hypotese}

\begin{hypotese}
    Hvis et gyldent snit er specielt æstetisk tiltalende, må vi have, at
    antallet af regioner liggende hvert af de fire snit, som kan
    betragtes som gyldne, ikke afviger mere end $\pm10\%$ fra hinanden.
\end{hypotese}

\begin{hypotese}
    Hvis det gyldne rektangel er et specielt æstetisk tiltalende format,
    må vi have, at mere end en tredjedel malerierne har et lærred, hvis
    dimensioner er lig $\varphi$. \textbf{Problemer med billeder som er
    skåret op.}
\end{hypotese}

\subsection{Antallet af regioner i det gyldne snit}
Følgende hypoteser omhandler antallet af detekterede regioner i det
gyldne snit. Altså sådan nogenlunde.

\begin{hypotese}
    Hvis det gyldne snit er meget brug i malerkunsten, må vi have, at
    antallet af detekterede regioner i det gyldne snit, er skarpt
    større, end antallet i alle andre snit.
\end{hypotese}

\begin{hypotese}
    Hvis det gyldne snit er bevidst brugt af kunstneren, må vi have, at
    antallet regioner liggende i det gyldne snit, er skarpt større, end
    antallet af regioner liggende i snittet ved to tredjedele.
\end{hypotese}

\begin{hypotese}
    Hvis det gyldne snit er bevidst brugt af kunstneren, må vi have, at
    antallet regioner liggende i det gyldne snit, er skarpt større, end
    antallet af regioner liggende det midterste snit.
\end{hypotese}

\subsection{Tidsperiode, kunstner og land}
Følgende hypoteser har noget at gøre med maleriernes ophav og arbejder
ud fra den antagelse, at hvis det gyldne snit \emph{altid} har været
interessant, så skal det altid have været til stede, uanset
nationalitet, årstal og kunstner. [Bør omformuleres].

\begin{hypotese}
    Hvis det gyldne snit altid har været specielt æstetisk tiltalende,
    må vi have, at antallet af regioner liggende i det gyldne snit,
    tidsperioder imellem, højest kan afvige med $\pm10\%$.
\end{hypotese}

\begin{hypotese}
    Hvis det gyldne snit altid har været specielt æstetisk tiltalende,
    må vi have, at antallet af regioner liggende i det gyldne snit,
    nationaliteter imellem, højest kan afvige med $\pm10\%$.
\end{hypotese}

\begin{hypotese}
    Hvis det gyldne snit er lige æstetisk tiltalende for alle personer,
    må vi have, at antallet af regioner liggende i det gyldne snit, mod
    antallet af regioner liggende i alle andre snit, højst må afgive med
    $\pm10\%$ mellem de enkelte kunstnere. Vi skal altså have, at
    procentdelen af regioner liggende i det gyldne snit er nogenlunde
    den samme.  \textbf{Meget kontroversiel og nok ikke værd
    at undersøge}.
\end{hypotese}

\subsection{Det gyldne snit mod \emph{the rule of thirds}}
Vi undersøger her, om der virkelig gøres brug af en approksimation til
det gyldne snit, hvor man arbejder ud fra snittet ved to tredjedele.

\begin{hypotese}
    Hvis det gyldne snit er specielt æstetisk tiltalende, og vi har at
    \emph{the rule of thirds} bliver brugt i malerkunsten, som en
    approksimation til det gyldne snit, må vi have, at antallet af
    regioner liggende i det gyldne snit ikke angiver fra antallet af
    regioner i to tredjedele med mere end $15\%$.
\end{hypotese}

}
% vim: set tw=72 spell spelllang=da:
