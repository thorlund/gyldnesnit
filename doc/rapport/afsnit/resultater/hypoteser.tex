{
{\sffamily Det er meget svært at kvantificere, hvorvidt det gyldne snit
bliver brugt i malerkunsten, men vi kan opstille nogle hypoteser for,
hvilke resultater vi vil forvente, hvis det gyldne snit bliver brugt.
De følgende hypoteser antager derfor den generelle opfattelse, at det
gyldne snit er specielt æstetisk tiltalende.
}

\begin{hypotese}
    Mere end $50\%$ af de analyserede malerier, har én eller flere
    interessante regioner i det gyldne snit.
    \label{hypo_binaer}
\end{hypotese}

\begin{hypotese}
    Antallet af interessante regioner, fundet i hvert af de fire snit
    tilknyttet snitratioen $\varPhi$, afviger ikke mere end $10\%$ fra
    hinanden.
    \label{hypo_fire_g_snit}
\end{hypotese}

\begin{hypotese}
    Mere end en tredjedel malerierne har et lærred, hvis
    dimensioner er lig $\varphi\pm2.4\%$.
    \label{hypo_golden_ractangle}
\end{hypotese}

\begin{hypotese}
    Antallet af interessante regioner i det gyldne snit er skarpt større end
    antallet af interessante regioner i alle andre snit.
    \label{hypo_alle_andre_snit}
\end{hypotese}

\begin{hypotese}
    Antallet af interessante regioner i snitratioen $\varPhi$ er skarpt
    større end antallet af interessante regioner i snitratioen $\frac{2}{3}$,
    per snit i snitratioen.
    \label{hypo_to_tredjedele}
\end{hypotese}

\begin{hypotese}
    Antallet af interessante regioner i det gyldne snit er skarpt større end
    antallet af interessante regioner i det midterste snit.
    \label{hypo_midten}
\end{hypotese}

\begin{hypotese}
    Det gennemsnitlige antal interessante regioner i det gyldne snit per
    maleri, tidsperioder imellem, afviger med højst $10\%$.
    \label{hypo_tid}
\end{hypotese}

\begin{hypotese}
    Det gennemsnitlige antal interessante regioner i det gyldne snit per
    maleri, nationaliteter imellem, afviger med højst $10\%$.
    \label{hypo_nation}
\end{hypotese}

\begin{hypotese}
    Antallet af interessante regioner i snitratioen $\varPhi$ afviger
    ikke fra antallet af interessante regioner i snitratioen
    $\frac{2}{3}$, per snit, med mere end $15 \%$.
    \label{hypo_15p}
\end{hypotese}

Ovenstående hypoteser er udarbejdet efter hvilke egenskaber, vi
forventer, et maleri konstrueret efter det gyldne snit, har. Endvidere
antager vi, at det gyldne snit \emph{altid} har været at foretrække på
tværs af landegrænser og kultur.

Hvis vi kun har malerier, som i alle henseender er konstrueret efter det
gyldne snit, vil disse opfylde hypoteserne 1 -- 8. Dette vil betyde, at
regionerne i de gyldne snit dominerer, samt at brugen af det gyldne snit
ikke binder sig til nationalitet eller tidsperiode.

Kan en samling af malerier kun bekræfte nogle af hypoteserne fra 1 -- 8,
kan vi dog stadig sige noget om brugen af det gyldne snit. Hvis hypotese
3, som den eneste, ikke bliver opfyldt, kan vi stadig sige, at kunstnere
placerer de interessante regioner i maleriet efter det gyldne snit, men
ikke bruger det gyldne snit i valg af lærred.

Hypotese 5 og 9 er modsætninger. Nr. 5, hvis bekræftet, antyder, at
kunstneren bevidst placerer interessante regioner i det gyldne snit. Nr.
9, derimod, antyder, at snitratioen $\frac{2}{3}$ bliver brugt som
approksimation til det gyldne snit, og at de to snit bruges på lige fod.
Det vil sige, at \emph{hvis} det gyldne snit er specielt æstetisk
tiltalende, og \emph{hvis} snitratioen $\frac{2}{3}$ bruges som
approksimation til det gyldne snit, så vil interessante regioner
koncentrere sig om det gyldne snit og snittet ved to tredjedele, og
denne hypotese vil blive bekræftet.

De tilladte afvigelser er vores bedste bud på, hvornår det gyldne snit
ikke længere kan siges at være til stede i malerierne. Procentsatsen i
hypotese 3 er sat efter den fejlmargin, vi fandt frem til i kapitel
\ref{chap_detektion}. De resterende procentsatser er valgt efter, hvad
der virkede passende, da ingen tidligere har forsøgt at kvantificere
problemstillingen.

\subsection{Datasæt}
Det korpus, vi kører vores analyse på, består af billeder hentet fra
``The Web Gallery of Art''\cite{wgahu}, som er en online billeddatabase med
europæiske kunstartikler fra år 1001 -- 1900. I kunstartiklerne, hvor
det samlede antal er omkring 23.000, indgår møbler, kalkmalerier,
skulpturer, mosaikker og malerier, hvor sidstnævnte, vil være vores
fokus. Over halvdelen af disse kunstartikler står udstillet på museum.
Databasen blev oprettet i 1996 med det formål at præsentere kunst fra
renæssancen (ca.  14. -- 17.  århundrede), men blev senere udvidet til
også at inkludere kunst fra andre perioder. Dette betyder, at
størstedelen af malerierne, vi undersøger, er fra tidsperioden 1450 --
1650 og er malet af italienske kunstnere. Endvidere er langt de fleste
malerier klassificeret som religiøse.  Disse informationer er givet fra
WGA men er også vedlagt i bilag \ref{appendix_grafer} som
grafer.

Vi må af ovenstående grunde forvente, at resultater, fra en analyse på
vores datasæt, vil være farvet af samlingen af malerier. Derfor kan være
svært at drage nogen konklusioner for malerkunsten generelt, da
resultaterne vil være begrænset, til kun at gælde for et udsnit af
vestlig kultur. Endvidere findes der ingen nyere malerier i datasættet,
hvilket gør, at vi ikke kan udtale os om nyere malerkunst.

Billederne, som suppleres fra databasen, er af høj kvalitet, men der er
visse problemer, som vi nævner nedenfor.

\begin{itemize}
    \item \textbf{Beskæring af billeder}\\
        Vi kan være sikre på, at billederne i datasættet er ordentligt
        beskåret, hvilket f. eks. kan betyde, at noget af billedrammen
        kan være med i billedet. Dette kan muligvis volde lidt problemer
        med udtrækning af regioner, men hvad værre er, så gør det vores
        mål, for hvor det gyldne snit ligger, upræcist.  Dette har vi
        dog taget højde for i kraft af vores margin.  Endelig er der
        inkluderet detaljebilleder af malerier i databasen, som er et
        udsnit af et givet maleri, således at målene for det gyldne snit
        ikke passer på billedet.
    \item \textbf{Forvrængning og perspektiv}\\
        Billederne af malerier er taget med et kamera, hvor linsen
        muligvis kan have forvrænget billedet. Vi kan derfor have skæve
        linjer og tage forkerte beslutninger for regioner pga. dette.
        Endvidere kan billedet være taget skævt, således at billedet
        hælder til den ene side.
    \item \textbf{Opdelte malerier}\\
        Nogle store malerier kan være blevet opdelt i flere billeder, da
        databasen har det formål at vise malerierne på en computerskærm,
        hvor meget store billeder kan være svære at betragte. Dette
        betyder, at nogle billeder ikke viser hele maleriet, men blot
        et udsnit, hvilket påvirker vores muligheder for at sige noget
        fornuftigt om det gyldne snit i maleriet.
\end{itemize}

Nogle stilarter, såsom kalkmalerier og tegninger, har gennem
udokumenterede afprøvninger vist sig at være besværlige at analysere
pga. meget svingende farvegengivelse. Endvidere kan disse være billeder
af en hvælving i en kirke, som ikke egner sig til analyse for det gyldne
snit. Vi har derfor valgt kun at analysere malerier beskrevet som
``painting'' fra WGA.

Alt det ovenstående vil påvirke resultaterne ved analyse på vores
datasæt.

\subsection{Eksperimentsopstilling}
I kapitel \ref{chap_afproevning} er de optimale tærskelværdier fundet, og
de er sat derefter.  Det er fordelagtigt at maksimere antallet af andre
snit, da det giver et bedre grundlag for eksperimentet.  Afstanden
mellem to snit er begrænset af en margin defineret til at være $2.4\%$,
som beskrevet i afsnit \ref{margin}.  Denne margin skal være tilstede
på begge sider af et snit, så derfor vil hvert snit fylde $(2.4*2)\%$.
Det maksimale antal af snit på et billede må altså være
$100/4.8=20.833$.  Med en margin på $2.4\%$ vil der være en strimmel på
$0.1\%$ af billedet, som ikke bliver brugt, og mellem to marginer vil
der derfor ligge en strimmel på $0.2\%$.

Den eneste undtagelse findes ved midten. I udregning
\eqref{resul_midt_udregning1} og \eqref{resul_midt_udregning2} ses først
hvordan snittet til venstre for midten findes, og derefter hvor lang
afstanden er mellem de to snit.

\begin{eqnarray}
    1-0.518 = 0.482\label{resul_midt_udregning1}\\
    0.518-0.482 = 0.036\label{resul_midt_udregning2}
\end{eqnarray}

Der er altså en strimmel på $0.048-0.036 = 0.012 = 1.2\%$ af
billedet, hvor interessante regioner potentielt kunne blive talt to gange.
Dette er illustreret i figur \ref{resultat_fejl_midt}.

\begin{figure}[!h]
	\centering
	\fbox{\includegraphics[scale=0.5]{afsnit/resultater/billeder/midt_strimmel}}
	\caption{Kollision mellem de to midterste snit, vist ved de to
	blå linjer. Midten er indikeret ved en sort linje, der ligger cirka
	midt i kollisionsområdet.}
	\label{resultat_fejl_midt}
\end{figure}

}
% vim: set tw=72 spell spelllang=da:
