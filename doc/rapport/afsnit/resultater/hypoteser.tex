{
{\sffamily Alle vore hypoteser antager, at det gyldne snit er specielt
æstetisk tiltalende og derfor er meget brugt i malerkunsten.
}

\subsection{Korpus}
Hvilken data har vi rådighed? Tidsperiode, lande, kultur. Er malerierne
``anerkendt'' på nogen måde? Hvem har valgt dem ud? Er der nogen
kriterier for udvælgelse? Er der nogle kunstværker vi mangler?

\subsection{Generelle hypoteser}
Følgende hypoteser er ret generelle. Altså sådan nogenlunde.

\begin{hypotese}
    Hvis det gyldne snit er meget brugt i malerkunsten, må vi have, at
    over halvdelen, af de analyserede malerier, har én eller flere
    regioner liggende i det gyldne snit.
\end{hypotese}

\begin{hypotese}
    Hvis et gyldent snit er specielt æstetisk tiltalende, må vi have, at
    antallet af regioner liggende hvert af de fire snit, som kan
    betragtes som gyldne, ikke afviger mere end $\pm10\%$ fra hinanden.
\end{hypotese}

\subsection{Antallet af regioner i det gyldne snit}
Følgende hypoteser omhandler antallet af detekterede regioner i det
gyldne snit. Altså sådan nogenlunde.

\begin{hypotese}
    Hvis det gyldne snit er meget brug i malerkunsten, må vi have, at
    antallet af detekterede regioner i det gyldne snit, er skarpt
    større, end antallet i alle andre snit.
\end{hypotese}

\begin{hypotese}
    Hvis det gyldne snit er bevidst brugt af kunstneren, må vi have, at
    antallet regioner liggende i det gyldne snit, er skarpt større, end
    antallet af regioner liggende i snittet ved to tredjedele.
\end{hypotese}

\begin{hypotese}
    Hvis det gyldne snit er bevidst brugt af kunstneren, må vi have, at
    antallet regioner liggende i det gyldne snit, er skarpt større, end
    antallet af regioner liggende det midterste snit.
\end{hypotese}

\subsection{Tidsperiode}
Følgende hypoteser har noget at gøre med maleriernes tidsperiode.

\begin{hypotese}
    Hvis det gyldne snit altid har været specielt æstetisk tiltalende,
    må vi have, at antallet af regioner liggende i det gyldne snit,
    tidsperioder imellem, højest kan afvige med $\pm10\%$.
\end{hypotese}

}
% vim: set tw=72 spell spelllang=da:
