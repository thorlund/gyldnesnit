{
{\sffamily Det følgende afsnit giver en kort introduktion til hvordan
databasen er blevet implementeret. Vi ser på, hvordan tabeller oprettes
i databasen, hvordan databasen populeres samt hvordan vi gemmer
resultater fra kørsler. Endeligt vil vi komme ind på, hvordan vi kan
genskabe enkelte grafiske resultater fra databasen.
}

\subsection{Databaseskema}
Ved at bruge \emph{SQLObject} er det ligetil at konstruere tabeller i
databasen. Vi har i afsnit \ref{section_database} givet det database
skema som vi opbygger databasen efter. I kodeboks
\ref{code_tabel_artist} er vist hvordan tabellen \texttt{artist}
konstrueres ved brug af \emph{SQLObject} i Python.

\begin{lstlisting}[caption={Pythonkode for oprettelse af tabeller i
    databasen.}, captionpos=b, label={code_tabel_artist}, frame=tb,
    breaklines=false, float=hb]
import sqlobject as s

class Artist(s.SQLObject):
    "
    _id_, name, born, died, school, timeline
    "
    name = s.StringCol()
    born = s.IntCol()
    died = s.IntCol()
    school = s.StringCol()
    timeline = s.StringCol()
\end{lstlisting}

Når man vil oprette en ny kunstner i databasen gøres det som vist
i kodeboks \ref{code_new_artist}.

\begin{lstlisting}[caption={Oprettelse af en kunstner i databasen.},
    captionpos=b, label={code_new_artist}, frame=tb, breaklines=false,
    float=h]
# Init variables
name = "Homer Simpson"
born = 1968
died = 2000
school = "Springfield"
timeline = "1950-2000"

# Create the artist in the database
Artist(name=name, born=born, died=died, school=school, timeline=timeline)
\end{lstlisting}

\emph{SQLObject} opretter automatisk et id-felt til alle tabeller i
databasen. Vi kan udnytte dette til at lave \emph{foreign keys} i
tabellerne. Vi viser i kodeboks \ref{code_tabel_result} hvordan tabellen
\texttt{result} oprettes i databasen, hvor det er interessant at bemærke
hvorledes de to \emph{foreign keys} oprettes.

\begin{lstlisting}[caption={Pythonkode for oprettelse af \emph{foreign
    keys} i databasen.}, captionpos=b, label={code_tabel_result}, frame=tb,
    breaklines=false, float=h]
class Result(s.SQLObject):
    "
    _id_, ^runId, ^paintingId, cutRatio, cutNo, numberOfRegions
    "

    run = s.ForeignKey('Run')
    painting = s.ForeignKey('Painting')
    cutRatio = s.FloatCol()
    cutNo = s.IntCol()
    numberOfRegions = s.IntCol()
\end{lstlisting}

\subsection{Populering af databasen}
Vores korpus består af billeder fra en online kunstdatabase kaldet
``The Web Gallery of Art''\cite{wgahu}. Herfra er suppleret en
kommasepareret fil, som indeholder maleriernes metadata, heriblandt også
en webadresse til billedet af maleriet. Det har således været
problemfrit at konstruere en crawler til at hente alle billeder ned til
egen disk, givet denne fil.

\subsubsection{Parsing af kommasepareret fil}
I filen gives endvidere mange oplysninger, om den enkelte artikel samt
dennes kunstner.  Vi har konstrueret en parser, som trækker disse
informationer ud fra filen og lægger dem ind i databasen. Da vi primært
vil beskæftige os med malerier, vil vi nu blot omtale kunstartikler som
malerier.

Den konstruerede parser, til den kommaseparerede fil, er dog ret grov,
da folkene bag \cite{wgahu} ikke har lagt meget vægt på, at være
konsistente i deres formulering af en kunstners fødsels- og dødsår eller
en genstands dimensioner. En følge deraf er, at nogle kunstnere, hvor
\cite{wgahu} ikke har en klar indikation af dennes levealder, ikke
bliver registreret i databasen. Vi kan dog stadig slå kunstneren op ved
at bruge feltet ``timeline'', som angiver hvilken periode kunstneren
tilhører. Vi har i enkelte tilfælde, set os nødsaget til at rette i den
kommaseparerede fil, hvor der er blevet indsat tegn, som helt umuliggør
korrekt parsing af filen, såsom ekstra komma eller semikolon.

\subsubsection{Automatiseret populering}
I mappen \texttt{database/} findes redskaber til en automatiseret
populering af databasen. Scriptet i filen \texttt{\_\_init\_\_.py}
brugere parseren, og sørger for at indsætte værdierne rigtigt i
databasen.

\subsection{Oprettelse af resultater}
\paragraph{Noter}
Denne sektion afhænger til dels af en færdig implementation af den
automatiserede analyse. Hvordan lægger vi resultater fra en kørsel ind i
databasen? Forklar de (smarte) metoder vi har til rådighed. Vi kan blot
kaste klasser ind i databasen. Vores settings-klasse bruges til at
oprette runs, dictionary til at smække resultatet fra analysen ind.

På grund af begrænsninger i \emph{OpenCV} kan vi ikke gemme regionens
præcise form, men kun dennes begrænsende rektangel og regionens areal.

\subsection{Genskabelse af parametre og resultater}
At kunne genskabe de fundne resultater fra en analyse har meget stor
betydning, dels for at kunne udtage stikprøver i udviklingen af hele
programmet, men også for at kunne fremvise grafiske resultater. Vi har
allerede været inde på, at man for at kunne genskabe et resultat, skal
vide hvilke parametre der oprindeligt har været brugt. Ovenstående
databaseskema gør det let at hente disse parametre ud. Hvis vi får et
resultat med overraskende mange regioner og gerne vil undersøge dette
tilfælde, har vi metoder til rådighed der giver os lige nøjagtig de
informationer vi har brug for at vise dette grafisk. Helt konkret har vi
metoderne vist i listing \ref{rekonst_koersel} til rådighed.

\vspace{0.5cm}
\begin{lstlisting}[caption={Metoder til rekonstruktion af kørsler},captionpos=b,label={rekonst_koersel},numbers=none]
def getSettingsForRunId(runId):
    """Return the settings instance for a given run"""
    pass

def getCutRatiosForRunId(runId):
    """Return the list of cut ratios for a given run"""
    pass

def getSettingsForResultId(resultId):
    """Return the settings instance for a given result"""
    pass

def getSettingsForRegionId(regionId):
    """Return the settings instance for a given region"""
    pass

def getCutRatioForRegionId(regionId):
    """Return the list of cut ratios for a given region"""
    pass

def getCutNoForRegionId(regionId):
    """Return the cut number for a given region"""
    pass

def getRegionsForResultId(resultId):
    """Return the list of regions for a given result"""
    pass
\end{lstlisting}

Selvom metoderne i listing \ref{rekonst_koersel} ikke viser noget
egentlig kode, bør det ud fra sammenhængen være klart hvad disse metoder
gør. Alle metoder der starter med \texttt{getSettings} returnerer
klassen \texttt{Settings} som vist i listing \ref{settings_klassen} med
indstillinger tilpasset den enkelte forespørgelse.
\vspace{0.5cm}
\begin{lstlisting}[caption={Settings-klassen med standardindstillinger},captionpos=b,label={settings_klassen},numbers=none]
class Settings:
    """These are the default settings for the analysis"""
    edgeThreshold1 = 78
    edgeThreshold2 = 2.5 * edgeThreshold1
    lo = 4
    up = 4
    cutRatios = None
    marginPercentage = 0.009
    method = 'naive'
    ...
\end{lstlisting}

Det ses at vi har mulighed for at trække de fundne regioner ved et
snit ud og vi behøver derfor ikke at køre nogen analyse på billedet hvis
vi blot ønsker at få de fundne regioners begrænsende areal vist. I dette
tilfælde kan vi nøjes med at forespørge databasen om de regioner der er
tilknyttet et snit vi gerne vil undersøge og traversere gennem den liste
af regioner vi får tilbage. Hver region er repræsenteret som en klasse
hvor vi kan trække rektanglet ud og vi bruger da \emph{OpenCV} til at
tegne rektanglet på det tilknyttede billede.

}

% vim: set tw=72 spell spelllang=da:

