{
{\sffamily Vi giver en kort præsentation af programmets overordnede
struktur. Mappestrukturen er vist i figur \ref{program_struktur}.
Indholdet af mappen \texttt{tests} er blot afprøvninger fra forskellige
stadier i udviklingen af programmet og vi vil derfor ikke komme nærmere
ind på dette.

\begin{figure}[!h]
    \dirtree{%
    .1 src/ .
    .2 database/ .
    .2 experiments/ .
    .2 lib/ .
    .2 model/ .
    .2 painting/ .
    .2 settings/ .
    .2 tests/ .
    .2 \_\_init\_\_.py .
    .2 pictureresource.db .
    .2 start.py .
    }
    \caption[]{Programmets struktur}
    \label{program_struktur}
\end{figure}

I programmets rod ligger billededatabasen i filen
\texttt{pictureresource.db}. Filen \texttt{start.py} starter en analyse
på denne database. Den følgende korte introduktion til de enkelte dele
af programmet går mapperne igennem alfabetisk.

}

\subsection{database/}
\dirtree{%
.1 database/ .
.2 \_\_init\_\_.py .
.2 catalog.csv .
.2 parser.py .
.2 udtraek.py .
}
%(wga.hu csv-fil, parsing af csv-fil, udtraek)
\vspace{1.2em}
Mappen \texttt{database/} vedrører population af databasen ud fra en
csv-fil fra vores korpus. Der er udviklet en parser specifikt til denne
csv-fil. I denne mappe ligger også koden tilknyttet udtræk fra databasen
i forbindelse med de videnskabelige resultater i kapitel
\ref{chap_resultater}.

\subsection{experiments/}
\dirtree{%
.1 experiments/ .
.2 \dots .
}
%(dikumentation af eksperimenter)
\vspace{1.2em}

I mappen \texttt{experiments/} gemmes kørsler tilknyttet de
videnskabelige eksperimenter, således at de let kan hentes ind
\texttt{start.py} i programmets rod og køres på databasen. Det er
nyttigt at have eksperimenterne liggende separat på denne måde, da man
kan oprette forskellige eksperimenter som kører analysen med specielle
parametre eller udtrækningsmetoder.

\subsection{lib/}
\dirtree{%
.1 lib/ .
.2 \_\_init\_\_.py .
.2 edgeDetector.py .
.2 expandedMethod.py .
.2 expandedMethon.py .
.2 featureDetector.py .
.2 goldenLibrary.py .
.2 graphicHelper.py .
.2 grid.py .
.2 lineScanner.py .
.2 marginCalculator.py .
.2 naiveMethod.py .
.2 regionSelector.py .
.2 paintingAnalyzer.py .
.2 statestik.py .
.2 transformations.py .
}
%(goldenLibrary - hjælpefunktioner, paintingAnalyser - sammensætter de
%øvrige metoder, naiveMethod, expandedMethod, regionSelector)
\vspace{1.2em}

\texttt{lib/} indeholder alle metoder vedrørende billedbehandling, dvs.
metoder til udtrækning og bedømmelse af regioner. Filen
\texttt{paintingAnalyser.py} er den centrale komponent i denne
sammenhæng, da den binder hele udtrækningen og bedømmelsen af regioner
sammen. Den distribuerer arbejdet, alt efter hvilken udtræknings- og
bedømmelsesmetode der bruges.  I \texttt{goldenLibrary.py} er der
defineret en række hjælpemetoder som bruges i billedbehandlingen.

\subsection{model/}
\dirtree{%
.1 model/ .
.2 \_\_init\_\_.py .
}
%(database in/ud)
\vspace{1.2em}

I \texttt{model/} er selve databaseskemaet defineret. Her er også
defineret hjælpefunktioner til at gemme specielle datastrukturer og
gøre ofte brugte forespørgsler til databasen lettere.

\subsection{painting/}
\dirtree{%
.1 painting/ .
.2 \_\_init\_\_.py .
}
%(container)
\vspace{1.2em}

Dette er blot en datastruktur til malerier i programmet således at
databasen og billedbehandlingen har et fælles format at arbejde efter.

\subsection{settings/}
\dirtree{%
.1 settings/ .
.2 \_\_init\_\_.py .
}
%(globale indstillinger, indstillinger til kørsel, kan gemmes i database)
\vspace{1.2em}

I \texttt{setting} stilles en klasse til rådighed som indeholder de
forskellige parametre til billedbehandlingen. Dette drejer sig om
tærskelværdier, marginstørrelse og hvilke snitratioer der skal
undersøges. Her er også defineret globale indstillinger til programmet
med specielt henblik på hvor databasen og maleriers metadata er placeret
i filsystemet.

}

% vim: set tw=72 spell spelllang=da:
