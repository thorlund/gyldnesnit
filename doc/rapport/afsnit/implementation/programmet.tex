{
{\sffamily Vi giver en kort præsentation af programmets overordnede
struktur. Mappestrukturen er vist i figur \ref{program_struktur}.

\begin{figure}[!h]
    \dirtree{%
    .1 src/ .
    .2 database/ .
    .2 experiments/ .
    .2 lib/ .
    .2 model/ .
    .2 painting/ .
    .2 settings/ .
    .2 tests/ .
    .2 \_\_init\_\_.py .
    .2 pictureresource.db .
    .2 start.py .
    }
    \caption[]{Programmets struktur}
    \label{program_struktur}
\end{figure}

Mappen \texttt{database/} indeholder kode til initialisering af
databasen og udtræk i sammenhæng med resultater. I mappen
\texttt{experiments/} gemmes kørsler tilknyttet de videnskabelige
eksperimenter. \texttt{lib/} indeholder alle metoder vedrørende
billedbehandling, dvs. metoder til udtrækning og bedømmelse af regioner.
I \texttt{model/} er databaseskemaet og hjælpefunktioner til at gemme
specielle datastrukturer defineret. Mapperne \texttt{painting/} og
\texttt{settings/} tilbyder datastrukturer til henholdsvis malerier og
indstillinger. Vi vil i det følgende kigge lidt nærmere på de enkelte
dele af programmet.
}

\subsection{database/}
\dirtree{%
.1 database/ .
.2 \_\_init\_\_.py .
.2 catalog.csv .
.2 parser.py .
.2 udtraek.py .
}
(wga.hu csv-fil, parsing af csv-fil, udtraek)

\subsection{experiments/}
\dirtree{%
.1 experiments/ .
.2 \dots .
}
(dikumentation af eksperimenter)

\subsection{lib/}
\dirtree{%
.1 lib/ .
.2 \_\_init\_\_.py .
.2 edgeDetector.py .
.2 expandedMethod.py .
.2 expandedMethon.py .
.2 featureDetector.py .
.2 goldenLibrary.py .
.2 graphicHelper.py .
.2 grid.py .
.2 lineScanner.py .
.2 marginCalculator.py .
.2 naiveMethod.py .
.2 regionSelector.py .
.2 paintingAnalyzer.py .
.2 statestik.py .
.2 transformations.py .
}
(goldenLibrary - hjælpefunktioner, paintingAnalyser - sammensætter de
øvrige metoder, naiveMethod, expandedMethod, regionSelector)

\subsection{model/}
\dirtree{%
.1 model/ .
.2 \_\_init\_\_.py .
}
(database in/ud)

\subsection{painting/}
\dirtree{%
.1 painting/ .
.2 \_\_init\_\_.py .
}
(container)

\subsection{settings/}
\dirtree{%
.1 settings/ .
.2 \_\_init\_\_.py .
}
(globale indstillinger, indstillinger til kørsel, kan gemmes i database)

}

% vim: set tw=72 spell spelllang=da:
