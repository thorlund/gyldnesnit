{
{\sffamily Vi vender nu opmærksomheden mod selve implementationen af de
metoder, som afgør, hvorvidt en udtrukket regionen er interessant, samt
hvordan vi har implementeret den naive fremgangsmåde, som bedømmer, om
en region ligger i det gyldne snit. Alle metoder vedrørende vurdering af
regioner er implementeret i filen \texttt{regionSelector.py} i mappen
\texttt{lib/}.  Vi har igen, at metoderne ikke er specifikke for det
gyldne snit, men kan anvendes på ethvert snit i billedet. Sidst i
afsnittet viser vi, hvordan vi har implementeret den udvidede vurdering.
Vi starter med at se på en fælles datastruktur, som bruges, når regioner
bliver vurderet.
}

\subsection{Datastruktur til betingelser}
Når vi skal afgøre, hvorvidt et antal udtrukne regioner er interessante
og ligger i snittet, er der en række betingelser, der skal være opfyldt.
Til disse, er der forbundet nogle udregninger, som vil være de samme for
hver region. Vi bruger derfor en struktur, som indeholder resultaterne
fra disse udregninger, således at de ikke skal udføres for hver eneste
region vi kontrollerer. Datastrukturen kaldes \textbf{Constraints} og
ses herunder i \eqref{Constraints_class}.
\begin{multline}
    \textbf{class~} \textrm{Constraints} = \{ \\
    \shoveleft{\qquad\textbf{int} : \textit{coordinate}} \\
    \shoveleft{\qquad\textbf{double} : \textit{minSize}} \\
    \shoveleft{\qquad\textbf{double} : \textit{minMass}} \\
    \shoveleft{\qquad\textbf{int[}2\delta + 1\textbf{]} : \textit{acceptRange}} \\
    \shoveleft{\}}\shoveright{}
    \label{Constraints_class}
\end{multline}
Variablene \texttt{minSize} og \texttt{minMass} relaterer sig kun til
klassificering af \emph{interessante regioner}, mens \texttt{coordinate}
og \texttt{acceptRange} hører til klassificering af regioner,
\emph{liggende i snittet}. De enkelte variable vil, i det følgende,
blive forklaret nærmere, når det er relevant.

\subsection{Interessante regioner}
Til at afgøre om udtrukket region er interessant, bruges definition
\ref{def_interessant}. Bemærk, at vurderingen af interessante regioner,
ikke har noget at gøre med hverken snitratio eller margin. Vi undersøger
udelukkende de udtrukne regioners areal og masse. Regioner bliver
vurderet, umiddelbart efter de er blevet trukket ud, så vi har
regionerne til rådighed, som en instans af $\angles{CutRegions}$. Vi
definerer nu to metoder; én til at kontrollere en regions størrelse og
én til at kontrollere dens masse. Vi kalder disse \texttt{checkSize} og
\texttt{checkMass}. De kan ses i kodeboks \ref{pseudo_size_mass}.

\begin{lstlisting}[caption={Metoder til at konstollere en regions
    størrelse og masse.},captionpos=b,label={pseudo_size_mass},
    frame=tb, breaklines=false, float]
def checkSize(component, constraints):
    "Test if the component have size greater than the minumum size
    defined by the constraints."
    return component.area >= constraints.minSize

def checkMass(component, constraints):
    "Check if the component have mass greater than the minimum mass
    defined by the contraints."
    rect = component.rect
    mass = component.area/(rect.width * rect.height)
    return mass >= constraints.minMass
\end{lstlisting}

Metoderne i kodeboks \ref{pseudo_size_mass}, returnerer begge en
sandhedsværdi, for hvorvidt en region lever op til betingelserne, for en
interessant region. Vi vil imidlertid gerne kontrollere hver enkelt
region, så vi konstruerer en ny metode, som returnerer en \emph{dict}
kun med de interessante regioner.  Metoden, kaldet
\texttt{GetInterestingRegions}, ses i kodeboks
\ref{pseudo_GetInterestingRegions}. Den tager, som argument, den instans
af $\angles{CutRegions}$, som returneres fra \texttt{ExtractRegions}, se
kodeboks \ref{pseudo_udtraek_margin}.

\begin{lstlisting}[caption={Metode som returnerer kun de insteressante
    regioner, givet en instans af $\angles{CutRegions}$}, captionpos=b,
    label={pseudo_GetInterestingRegions}, frame=tb, breaklines=false,
    float]
def GetInterestingRegions(CutRegions, constraints):
    interestingRegions = {}
    for id in CutRegions:
        component = CutRegions[id][1]
        passSizeCheck = checkSize(component, constraints)
        passMassCheck = checkMass(component, constraints)
        if (passSizeCheck and passMassCheck):
            interestingRegions[id] = CutRegions[id]
    return interestingRegions
\end{lstlisting}

I kodeboks \ref{pseudo_GetInterestingRegions} bruger vi en instans af
\textbf{Constraints} som argument til metoderne, som kontrollerer
regionens størrelse og masse. Vi skal derfor, inden metoden
\texttt{GetInterestingRegions} kaldes, have initialiseret vores
betingelser, så de passer til billedet. I kapitel
\ref{chap_afproevning}, blev fastsat en procentsats for en regions
minimumareal i forhold til billedets størrelse, og denne
minimumstørrelse findes ved udregningen i \eqref{region_min_size}
herunder.
\begin{equation}
    \mathtt{minSize} =
    \lfloor
    \mathrm{minSizePercentage}\cdot\mathrm{height}\cdot\mathrm{width}
    \rfloor
    \label{region_min_size}
\end{equation}
Ligeledes, har vi fastsat en procentsats for en regions minimummasse, men denne
skal vi ikke regne videre på, da metoden \texttt{checkMass} også regner
en procentsats ud for den givne region. Vi gemmer den fastsatte
procentsats, for regioners minimummasse direkte i vores instans af
\textbf{Constraints}. I \texttt{checkMass} sammenlignes minimummassen
med regionen masse direkte.

\clearpage
\subsection{Naiv vurdering af regioner}
Vi vil nu se på, hvordan den naive fremgangsmåde vurderer hvorvidt en
region, ligger i det gyldne snit. Definition \ref{def_naiv} siger, at en
region mindst skal have én kant inden for margin, så vi vil først give
en forklaring på, hvordan dette margin bliver repræsenteret.

\subsubsection{Udregning af margin\label{subsec_margin_udregning}}
Størrelsen på margin er givet ved en procentsats $\Delta$.  Hvis vi kun
undersøger ét snit, bliver $\Delta$ sat til $2.4\%$ af billedets højde
eller bredde, alt efter hvilken orientering det aktuelle snit har.  Vi
har implementeret fastsættelsen af $\Delta$ således, at \emph{hvis} man
ønsker at sammenligne to snit, som ligger tættere på hinanden, end det
gyldne snit og to tredjedele, så findes den værdi af $\Delta$, der gør,
at disse snits margin ikke overlapper. I praksis gives en liste med
snitratioer, der ønskes undersøgt, og fra denne liste, findes den
mindste differens mellem ratioerne. Den mindste differens mellem
snitratioer, bruges da, som $\Delta$ for alle snit i analysen. Hvis den
mindste differens, mellem snitrationerne, er større end $2.4\%$, sættes
$\Delta$ til $2.4\%$.

Når vi skal vurdere regioner med hensyn til vores margin, får vi brug
for den eksakte pixelstørrelse på margin. I filen
\texttt{marginCalculator.py} i mappen \texttt{lib/} er der implementeret
metoder til dette. Her bruges metoden \texttt{getPixels}, som, givet et
billede, et snit og $\Delta$, returnerer afstanden fra
snittet til margin i pixels, som vi også skriver som $\delta$. I
\texttt{marginCalculator.py} finder vi også den mindste differens mellem
snitratioer, ved metoden \texttt{getPercentage}.

\begin{figure}[t]
    \centering
    \begin{picture}(122,55)
        \put(61, 50){$x$}
        \put(-10, 22){$y$}
        \put(0, 45){\circle*{3}}
        \put(-1, 45){\vector(1, 0){120}}
        \put(0, 45){\vector(0, -1){48}}

        \color{red}
        \put(88, 50){\line(0, -1){55}}

        \color{blue}
        \put(84, 50){\line(0, -1){55}}
        \put(92, 50){\line(0, -1){55}}

        \color{black}

        \put(66, 30){$-^{x}$}
        \put(78, 30){\vector(1, 0){20}}

        \put(100, 30){$+^{x} $}
        \put(98, 30){\vector(-1, 0){20}}


    \end{picture}
    \caption[]{Koordinatsystem med indtegnet snit og margin. Bemærk, at
    ved det vertikale snit, skal vi kun betragte regioners $x$-værdi,
    når vi skal bedømme om de ligger inden for margin. Således kan
    margin repræsenteres kun ved de tilladte $x$-værdier.}
    \label{margin_koordinatsystem}
\end{figure}
Vi udnytter, at vi enten betragter et vertikalt eller horisontalt snit.
Når vi har et vertikalt snit, behøver vi kun at betragte $x$-værdier, da
$y$-værdien ikke influerer på det vertikale snit. Omvendt med
horisontale snit, behøver vi kun at betragte $y$-værdier, da
$x$-værdierne, i denne situation, ikke har betydning for snittet.
Tilfældet, for det vertikale snit, er vist i figur
\ref{margin_koordinatsystem}.

Ovenstående gør, at vi nu kan oprette et sæt, bestående af de
koordinater, som ligger inden for margin. Hvis vi betragter et vertikalt
snit, kan vi i Python oprette sættet af accepterende $x$-værdier, ved at
bruge \texttt{range}. Vores implementation bruger endvidere variablen
\texttt{coordinate} i \textbf{Constraints}, som, lidt misvisende,
indikerer om snittet vi undersøger, er vertikalt eller horisontalt. Er
snittet vertikalt, sættes \texttt{coordinate} til \texttt{0}, og
\texttt{1} for et horisontalt snit. For ethvert snit, kan vi da finde de
accepterende $x$-værdier, som vist i kodeboks \ref{pseudo_acceptRange}.
Variablen \texttt{margin}, der tages som argument, er den
pixelstørrelse, der er returnet fra metoden \texttt{getPixels} fra
\texttt{marginCalculator.py}.

\begin{lstlisting}[caption={Metode som genererer sættet af accepterende
    koordinater.},captionpos=b,label={pseudo_acceptRange},
    frame=tb, breaklines=false, float]
GetAcceptRange(cut, margin, coordinate):
    if coordinate:
        # Horizontal cut
        lower_bound = cut.p1.y - margin
        upper_bound = cut.p1.y + margin
        acceptRange = range(lower_bound, upper_bound)
    else:
        # Vertical cut
        lower_bound = cut.p1.x - margin
        upper_bound = cut.p1.x + margin
        acceptRange = range(lower_bound, upper_bound)

    return acceptRange
\end{lstlisting}

\subsubsection{Kontrol på en regions begrænsende rektangel}
\begin{lstlisting}[caption={Metode, som kontrollerer, hvorvidt en region
    har en kant af det begrænsende rektangel inden for margin.},
    captionpos=b, label={pseudo_position}, frame=tb, breaklines=false,
    float]
def checkPosition(component, constraints):
    "Test if the component have a bounding box inside the accepting
    rectangle defined in the constraints."
    d = component.rect.width
    p = component.rect.x
    if constraints.coordinate:
        d = component.rect.height
        p = component.rect.y

    lowerInRange = p in constraints.acceptRange
    upperInRange = (p + d) in constraints.acceptRange

    return lowerInRange or upperInRange
\end{lstlisting}
Når vi har fastsat \texttt{coordinate} og fundet værdierne i
\texttt{acceptRange} kan vi kan afgøre, om en interessant region, ligger
placeret i snittet. Vi har allerede udnyttet, at vi kun behøver at
betragte én koordinat, og det er derfor ligetil, at kontrollere, om en
regions begrænsende rektangel, har en kant inden for margin. Regionen er
repræsenteret som en instans af \textbf{cvConnectedComp}, hvori der er
en instans af \textbf{cvRect}. Vi har en metode, kaldet
\texttt{checkPosition}, som, alt efter om vi har et vertikalt eller et
horisontalt snit, returnerer en sandhedsværdi for, om den relevante
koordinat findes i de accepterende koordinater. Metoden ses i kodeboks
\ref{pseudo_position}.

Vi kan nu sammensætte en metode, som returnerer alle interessante
regioner, med en kant inden for margin. Metoden tager en instans af
$\angles{CutRegions}$ og en instans af \textbf{Constraints} som
argumenter. Vi genererer altså vores betingelser inden vi begynder at
frasortere regioner, således at alle informationer om snittet og krav
for regioner ligger i instansen af \textbf{Constraints}. Den endelige
metode, for vurdering efter den naive fremgangsmåde, kaldes
\texttt{GetInterestingRegionsInCut} og er vist i kodeboks
\ref{pseudo_GetInterestingRegionsInCut}.

\begin{lstlisting}[caption={Pseudokode, som returnerer alle interessante
    regioner, der har en kant, af deres begrænsende rektangel, inden for
    margin.},
    captionpos=b, label={pseudo_GetInterestingRegionsInCut}, frame=tb, breaklines=false,
    float]
def GetInterestingRegionsInCut(CutRegions, constraints):

    # First remove uninteresting regions
    interestingRegions = GetInterestingRegions(CutRegions, constraints)

    # Initialize an empty dict
    interestingRegionsInCut = {}

    # Check every interesting region if it's bounding box
    # has an edge inside the margin
    for id in interestingRegions:
        component = interestingRegions[id][1]
        if checkPosition(component, constraints):
            interestingRegionsInCut[id] = interestingRegions[id]

    # The resulting dict contains only interesting regions
    # with an edge inside the margin
    return interestingRegionsInCut
\end{lstlisting}

\subsection{Udvidet vurdering af regioner}
Vi har også implementeret en udvidet vurdering af interessante
regioner. Grundprincipperne bag denne er gennemgået i afsnit
\ref{subsec_udvidet_massemidtpunkt}. Da vi ikke længere betragter en
regions begrænsende rektangel, men i stedet regionens form og
udstrækning, har brug vi for en anden måde, at repræsentere regionen på.
Vi tager derfor billedet, som \texttt{cvFloodFill} har arbejdet på, og
bruger dette til at approksimere regionens form.

\subsubsection{Approksimation af regioners form og udstrækning}
Fremgangsmåden for at approksimere en regions form og udstrækning er
simpel. Givet et segmenteret billede, fra \texttt{cvFloodFill}, og den
resulterende $\angles{CutRegions}$, fra \texttt{ExtractRegions}, kan vi,
for hver pixel i det begrænsende rektangel, kontrollere om denne har
samme farve, som regionen er blevet tildelt i segmentering af billedet.
Hvis farverne er ens, så gemmes dette punkt i en liste. Hvis regionen er
stor, kan dette godt være en tidskrævende procedure, så i praksis ønsker
vi ikke at kontrollere hver eneste pixel i det begrænsende rektangel. Vi
springer derfor et antal pixels over, for hver gang vi har kontrolleret
en pixel.  På denne måde får vi lavet et gitter, som vist i figur
\ref{grid}.  Pseudokode, for at approksimere en regions form og
udstrækning, er vist i kodeboks \ref{pseudo_GetRegionGrid}, som
definerer metoden \texttt{GetRegionGrid}, der returnerer en liste med
punkter for en region.

\begin{lstlisting}[caption={Metode til at approksimere en regions
    form. Bemærk linje 19, hvor der ses et eksempel på det omvendte
    koordinatsystem i \emph{OpenCV}.}, captionpos=b,
    label={pseudo_GetRegionGrid}, frame=tb,
    breaklines=false, float, numbers=left]
def GetRegionGrid(image, color, component, step):
    rect = component.rect

    # Initialize the area we want to traverse
    lower_x = rect.x
    lower_y = rect.y
    upper_x = lower_x + rect.width
    upper_y = lower_y + rect.height

    # Initialize an empty array
    coordinates = []

    # Traverse the bounding box with the defined step size
    for x in range(lower_x, upper_x, step):
        for y in range(lower_y, upper_y, step):

            # If we find a color in the image that equals the region
            # color, then we save that point
            if isSameColor(color, image[y][x]):
                coordinates.append(cvPoint(x, y))

    # Finally, return the resulting set of points
    return coordinates
\end{lstlisting}
% Sorry, I'm really sorry :(
Til senere brug, vil vi definere en metode, som finder en approksimation
til alle regioner i en instans af $\angles{CutRegions}$. Denne metode
returnerer en modificeret instans af $\angles{CutRegions}$, som vi
kalder for $\angles{CutGridRegions}$, således at den også indeholder
approksimationen til regionen. Strukturen er vist i
\eqref{CutGridRegions_dict}.
\begin{multline}
    \angles{CutRegions} = \{ \textit{~RegionId} : \\
    (\textbf{CV\_RGB~}\textit{color}, \textbf{cvConnectedComp~}\textit{region}, \textbf{cvPoint[]~}\textit{grid}) \}\quad
    \label{CutGridRegions_dict}
\end{multline}
Metoden \texttt{GridIt}, er vist i kodeboks \ref{pseudo_GridIt}, og går
alle regionerne igennem, mens den bygger den nye \emph{dict} op.

\begin{lstlisting}[caption={Metode, som finder approksimationen til alle
    regioner i en instans af $\angles{CutRegions}$.}, captionpos=b,
    label={pseudo_GridIt}, frame=tb,
    breaklines=false, float]
def GridIt(image, CutRegions, step):
    # Initialize an empty dict
    CutGridRegions = {}

    for id in CutRegions:
        # Get the color and component
        color = CutRegions[id][0]
        component = CutRegions[id][1]

        # Get the grid for the region
        grid = GetRegionGrid(image, color, component, step)

        # Save the findings
        CutGridRegions[id] = (color, component, grid)

    # Well, return
    return CutGridRegions
\end{lstlisting}

\subsubsection{Vurdering ved massemidtpunkt og udstrækning}
Definition \ref{def_expanded} fortæller os, at hvis en region har et
massemidtpunkt inden for margin og dens masse i øvrigt er jævnt fordelt
over snittet, så ligger denne i snittet.  En regions fordeling af masse
er givet i definition \ref{def_fordeling_ligning} og en regions
massemidtpunkt er givet i definition \ref{def_massemidtpunkt}. Vi skal
bruge approkimationen af regionens form og udstrækning til at afgøre om
regionen ligger i snittet.

Vi starter med at kigge på fordelingen af en regions masse, hvor vi skal
finde forholdet mellem punkter på højre og venstre side af snittet.
Metoden i kodeboks \ref{pseudo_distribution}, gør netop dette, og
returnerer en sandhedsværdi for, om regionens masse er jævnt fordelt
over snittet. Definition \ref{def_fordeling_procent} siger, at regionens
masse er jævnt fordelt over et snit, hvis forholdet mellem de to sider
er under $0.75$.  Metoden gør brug af snittets orientering, hentet i
$\textbf{Constraints}.\textit{coordinate}$, og det faktum, at vi kan
finde det midterste element i vores accepterende værdier, ved at tage
det $\left(\left\lfloor \frac{|\textit{acceptRange}|}{2}\right\rfloor +
1\right)$'te element i $\textbf{Constraints}.\textit{acceptRange}$.
Dette er selve snittet.

\begin{lstlisting}[caption={Metode som, på baggrund af regionens
    fordeling af masse, afgør, om denne region er jævnt fordelt},
    captionpos=b, label={pseudo_distribution}, frame=tb,
    breaklines=false, float]
def checkDistribution(grid, constraints):
    # If we've got no approximation, there's no distribution
    if len(grid) == 0:
        return False

    # Init variables for right/left-distribution as floats
    rDist = 0.0
    lDist = 0.0

    # Get the middle value of the accepting range, i.e. the cut
    middleIndex = floor(len(constraints.acceptRange)/2) + 1
    cutVal = constraints.acceptRange[middleIndex]

    # Check the position of every pixel in the grid
    for point in grid:
        if constraints.coordinate:
            # Horisontal
            if cutVal > point.y:
                left += 1
            elif cutVal < point.y:
                right += 1
        else:
            # Vertical
            if cutVal > point.x:
                left += 1
            elif cutVal < point.x:
                right += 1

    distribution = abs((lDist - rDist)/len(grid))
    return distribution < 0.75
\end{lstlisting}

Vi skal også bruge en metode, til at finde en regions massemidtpunkt.
Denne metode er vist i kodeboks \ref{pseudo_checkCenterOfMass} og
returnerer en sandhedsværdi for, hvorvidt en regions massemidtpunkt
befinder sig inden for margin.

\begin{lstlisting}[caption={Metode, som kontrollerer, om en regions
    massemidtpunkt er inden for margin.}, captionpos=b,
    label={pseudo_checkCenterOfMass}, frame=tb, breaklines=false,
    float]
def checkCenterOfMass(grid, constraints):
    if len(grid) == 0:
        return False

    sum = 0
    for point in grid:
        if constraints.coordinate:
            # Horisontal
            sum += point.y
        else:
            # Vertical
            sum += point.x

    centerOfMass = sum/len(grid)

    return centerOfMass in constraints.acceptRange
\end{lstlisting}

Endeligt samler vi alle metoderne, så vi kan sortere i alle resultaterne
returneret fra segmentering af billedet. Vi kalder denne metode for
\texttt{GetExpandedRegions} og viser denne i kodeboks
\ref{pseudo_GetExpandedRegions}.

\begin{lstlisting}[caption={Pseudokode, som returnerer alle interessante
    regioner, der er nævnt fordelt over snittet og har et massemidtpunkt
    inden for margin.},
    captionpos=b, label={pseudo_GetExpandedRegions}, frame=tb, breaklines=false,
    float]
def GetExpandedRegions(image, CutRegions, constraints):

    # First remove uninteresting regions
    interestingRegions = GetInterestingRegions(CutRegions, constraints)

    # Initialize the grid
    interestingRegions = GridIt(image, interestingRegions, constraints)

    # Initialize an empty dict
    interestingRegionsInCut = {}

    # Check every interesting region for center of mass
    # and distribution
    for id in interestingRegions:
        grid = interestingRegions[id][2]
        if checkDistribution(grid, constraints)
            and checkCenterOfMass(grid, constraints):
            interestingRegionsInCut[id] = interestingRegions[id]

    # The resulting dict contains only interesting regions
    # that qualify
    return interestingRegionsInCut
\end{lstlisting}

\clearpage

}

% vim: set tw=72 spell spelllang=da:
