{
{\sffamily Vi vender nu opmærksomheden mod selve implementationen af de
metoder, som afgør, hvorvidt en udtrukket regionen kan siges at være
interessant, samt hvordan vi har implementeret den naive fremgangsmåde,
som bedømmer, om en region ligger i det gyldne snit. Alle metoder
vedrørende vurdering af regioner er implementeret i filen
\texttt{regionSelector.py} i mappen \texttt{lib/}.  Vi har igen, at
metoderne ikke er specifikke for det gyldne snit, men kan anvendes på
ethvert snit i billedet. Sidst i afsnittet viser vi, hvordan vi har
implementeret den videregående vurdering, hvor regioner bedømmes ud fra
deres massemidtpunkt. Vi starter med at se på en fælles datastruktur der
bruges, når regioner bliver vurderet.
}

\subsection{Datastruktur til betingelser}
Når vi skal afgøre, hvorvidt et antal udtrukne regioner er interessante
og ligger i snittet, er der en række betingelser, der skal være opfyldt.
Til disse, er der forbundet nogle udregninger, som vil være de samme for
hver region. Vi bruger derfor en struktur, som indeholder resultaterne
fra disse udregninger, således at de ikke skal udføres for hver eneste
region vi kontrollerer. Datastrukturen kaldes \textbf{Constraints} og
ses herunder i \eqref{Constraints_class}.
\begin{multline}
    \textbf{class~} \textrm{Constraints} = \{ \\
    \shoveleft{\qquad\textbf{int} : \textit{coordinate}} \\
    \shoveleft{\qquad\textbf{double} : \textit{minSize}} \\
    \shoveleft{\qquad\textbf{double} : \textit{minMass}} \\
    \shoveleft{\qquad\textbf{int[}2\delta + 1\textbf{]} : \textit{acceptRange}} \\
    \shoveleft{\}}\shoveright{}
    \label{Constraints_class}
\end{multline}
Variablene \texttt{minSize} og \texttt{minMass} relaterer sig kun til
klassificering af interessante regioner, mens \texttt{coordinate} og
\texttt{acceptRange} hører til klassificering af regioner, liggende i
snittet. De enkelte variable vil, i det følgende, blive forklaret
nærmere, når det er relevant.

\subsection{Interessante regioner}
I afsnit \ref{section_naiv}, blev det defineret, at for at en region,
kan betegnes som værende \textbf{interessant}, skal den
\begin{enumerate}
        \renewcommand{\labelenumi}{(\alph{enumi})}
    \item have et areal større end en tærskelværdi, der sættes i
        forhold til billedets størrelse
    \item have en masse større end en tærskelværdi, der ligeledes,
        sættes i forhold til billedets størrelse.
\end{enumerate}
Bemærk, at vurderingen af interessante regioner, ikke har noget at gøre
med hverken snitratio eller margin. Vi undersøger udelukkende de
udtrukne regioners areal og masse. Regioner bliver vurderet, umiddelbart
efter de er blevet trukket ud, så vi har regionerne til rådighed som en
instans af $\angles{CutRegions}$. Vi definerer nu to metoder; én til at
kontrollere en regions størrelse og én til at kontrollere dens masse. Vi
kalder disse \texttt{checkSize} og \texttt{checkMass}. De kan ses i
kodeboks \ref{pseudo_size_mass}.

\begin{lstlisting}[caption={Metoder til at konstollere en regions
    størrelse og masse.},captionpos=b,label={pseudo_size_mass},
    frame=tb, breaklines=false, float=b]
def checkSize(component, constraints):
    "Test if the component have size greater than the minumum size
    defined by the constraints."
    return component.area >= constraints.minSize

def checkMass(component, constraints):
    "Check if the component have mass greater than the minimum mass
    defined by the contraints."
    rect = component.rect
    mass = component.area/(rect.width * rect.height)
    return mass >= constraints.minMass
\end{lstlisting}

Med metoderne i kodeboks \ref{pseudo_size_mass}, returnerer begge en
sandhedsværdi for, hvorvidt en region lever op til betingelserne, for en
interessant region. Vi vil gerne kontrollere hver enkelt region. Vi
laver nu en ny metode, som returnerer en \emph{dict} med kun de
interessante regioner.  Metoden, kaldet \texttt{GetInterestingRegions},
ses i kodeboks \ref{pseudo_GetInterestingRegions}. Den tager, som
argument, den instans af $\angles{CutRegions}$, som returneres fra
\texttt{ExtractRegions} i kodeboks \ref{pseudo_udtraek_margin}.

\begin{lstlisting}[caption={Metode som returnerer kun de insteressante
    regioner, givet en instans af $\angles{CutRegions}$}, captionpos=b,
    label={pseudo_GetInterestingRegions}, frame=tb, breaklines=false,
    float=t]
def GetInterestingRegions(CutRegions, constraints):
    interestingRegions = {}
    for id in CutRegions:
        component = CutRegions[id][1]
        passSizeCheck = checkSize(component, constraints)
        passMassCheck = checkMass(component, constraints)
        if (passSizeCheck and passMassCheck):
            interestingRegions[id] = CutRegions[id]
    return interestingRegions
\end{lstlisting}

I kodeboks \ref{pseudo_GetInterestingRegions} bruger vi en instans af
\textbf{Constraints} som argument til metoderne, som kontrollerer
regionens størrelse og masse. Vi skal derfor, inden metoden
\texttt{GetInterestingRegions} kaldes, have initialiseret vores
betingelser, så de passer til billedet. Vi har i kapitel
\ref{chap_afproevning}, fastsat en procentsats for en regions
minimumareal i forhold til billedets størrelse, og denne
minimumstørrelse findes ved udregningen i \eqref{region_min_size}
herunder.
\begin{equation}
    \mathtt{minSize} =
    \lfloor
    \mathrm{minSizePercentage}\cdot\mathrm{height}\cdot\mathrm{width}
    \rfloor
    \label{region_min_size}
\end{equation}
Ligeledes, har vi fastsat en procentsats for en regions minimummasse, men denne
skal vi ikke regne videre på, da metoden \texttt{checkMass} også regner
en procentsats ud for den givne region. Vi gemmer den fastsatte
procentsats, for regioners minimummasse direkte i vores instans af
\textbf{Constraints}. I \texttt{checkMass} sammenlignes minimummassen
med regionen masse direkte.

\subsubsection{Overvejelser}
Udvælgelsen af interessante regioner, kunne godt, være mere
sofistikeret.  Vi undersøger kun regioner for deres størrelse og masse,
hvilket stadig tillader mange regioner, som egentlig er uinteressante,
pga. deres form.  I udvælgelsen af interessante regioner, kunne man
derfor kigge på regionens form eller udstrækning, ved at undersøge
dennes massemidtpunkt.  Hvis massen er koncentreret langt væk fra
snittet, er denne region ikke interessant. Vi skal dog passe på, at vi
ikke tager beslutninger, som egentlig vedrører, om regionen er placeret
i snittet. Vi ønsker, i denne udvælgelse af regioner, udelukkende at
bestemme, hvorvidt regionen skal tages op til videre overvejelse, for om
denne ligger i snittet.

Kun hvis vores søgning for objekter i billedet, bliver mere specifik,
giver det mening, at undersøge regionerne nærmere i udvælgelsen af
interessante regioner. Vi kan forestille os, en situation, hvor man
udelukkende vil finde ansigter, placeret i det gyldne snit. I dette
tilfælde, skal vi selvfølgelig ikke sende en region til videre
vurdering, hvis denne \emph{ikke} er et ansigt. Vi har dog ikke en sådan
specifik søgning, hvorfor vi kun kan frasortere regioner, ud fra
informationen om deres størrelse og masse.

\subsection{Naiv vurdering af regioner}
Vi vil nu se på, hvordan den naive fremgangsmåde, givet i kapitel
\ref{chap_detektion}, vurderer hvorvidt en region, ligger i det gyldne
snit. Da det eneste krav er, at en region skal have en kant, af dens
afgrænsende rektangel, inden for margin, vil vi først give en forklaring
på, hvordan dette margin bliver repræsenteret.

\subsubsection{Udregning af margin\label{subsec_margin_udregning}}
Som nævnt, både i kapitel \ref{chap_detektion} og
\ref{chap_afproevning}, bruges en procentsats, til at angive vores
margin. Denne procentsats vil, hvis vi kun undersøger ét snit, blive sat
til $2.4\%$ af billedets højde eller bredde, alt efter hvilken
orientering det aktuelle snit har.  Vi har
implementeret fastsættelsen af denne procentsats således, at \emph{hvis}
man ønsker at sammenligne to snit, som ligger tættere på hinanden end det
gyldne snit og to tredjedele, så findes den procentsats der gør, at
disse snits margin ikke overlapper. I praksis gives en liste med
snitratioer, der ønskes undersøgt, og fra denne liste, findes den
mindste differens mellem ratioerne. Den mindste differens mellem
snitratioer, bruges da, som procentsats for alle snit i analysen. Hvis
den mindste differens, mellem snitrationerne, er større end $2.4\%$,
sættes procentsatsen for margin til $2.4\%$.

Når vi skal vurdere regioner med hensyn til vores margin, får vi brug
for den eksakte pixelstørrelse på margin. I filen
\texttt{marginCalculator.py} i mappen \texttt{lib/} er der implementeret
metoder til dette. Her bruges metoden \texttt{getPixels}, som, givet et
billede, et snit og en procentsats for margin, returnerer afstanden fra
snittet til margin i pixels, som vi også skriver som $\delta$. I
\texttt{marginCalculator.py} finder vi også den mindste differens mellem
snitratioer, ved metoden \texttt{getPercentage}.

\begin{figure}[t]
    \centering
    \begin{picture}(122,55)
        \put(61, 50){$x$}
        \put(-10, 22){$y$}
        \put(0, 45){\circle*{3}}
        \put(-1, 45){\vector(1, 0){120}}
        \put(0, 45){\vector(0, -1){48}}

        \color{red}
        \put(88, 50){\line(0, -1){55}}

        \color{blue}
        \put(84, 50){\line(0, -1){55}}
        \put(92, 50){\line(0, -1){55}}

        \color{black}

        \put(66, 30){$-^{x}$}
        \put(78, 30){\vector(1, 0){20}}

        \put(100, 30){$+^{x} $}
        \put(98, 30){\vector(-1, 0){20}}


    \end{picture}
    \caption[]{Koordinatsystem med indtegnet snit og margin. Bemærk, at
    ved det vertikale snit, skal vi kun betragte regioners $x$-værdi,
    når vi skal bedømme om de ligger inden for margin. Således kan
    margin repræsenteres kun ved de tilladte $x$-værdier.}
    \label{margin_koordinatsystem}
\end{figure}
Når vi, i den naive vurdering af regioner, skal afgøre, om en region
ligger inden for vores margin, kan vi udnytte, at vi enten betragter et
vertikalt eller horisontalt snit.  Når vi har et vertikalt snit, behøver
vi kun at betragte $x$-værdier, da $y$-værdien ikke influerer på det
vertikale snit. Omvendt med horisontale snit, behøver vi kun at betragte
$y$-værdier, da $x$-værdierne, i denne situation, ikke har betydning for
snittet.  Tilfældet, for det vertikale snit, er vist i figur
\ref{margin_koordinatsystem}.  Denne egenskab gør, at vi nu kan oprette
et sæt, bestående af netop kun de koordinater, som ligger inden for
margin. Hvis vi betragter et vertikalt snit, kan vi i Python oprette
sættet af accepterende $x$-værdier, ved at bruge strukturen
\texttt{range}. F.eks. vil \texttt{range(1, 4)} returnere listen
$\{1,2,3\}$.  Vores implementation bruger endvidere variablen
\texttt{coordinate} i \textbf{Constraints}, som, lidt misvisende,
indikerer om snittet vi undersøger, er vertikalt eller horisontalt. Er
snittet vertikalt, sættes \texttt{coordinate} til $0$, og $1$ for et
horisontalt snit. For ethvert snit, kan vi da finde de accepterende
$x$-værdier, som vist i kodeboks \ref{pseudo_acceptRange}. Variablen
\texttt{margin}, der tages som argument, er den pixelstørrelse, der er
returnet fra metoden \texttt{getPixels} fra
\texttt{marginCalculator.py}.

\begin{lstlisting}[caption={Metode som genererer sættet af accepterende
    koordinater.},captionpos=b,label={pseudo_acceptRange},
    frame=tb, breaklines=false, float=b]
GetAcceptRange(cut, margin, coordinate):
    if coordinate:
        # Horizontal cut
        lower_bound = cut.p1.y - margin
        upper_bound = cut.p1.y + margin
        acceptRange = range(lower_bound, upper_bound)
    else:
        # Vertical cut
        lower_bound = cut.p1.x - margin
        upper_bound = cut.p1.x + margin
        acceptRange = range(lower_bound, upper_bound)

    return acceptRange
\end{lstlisting}

\subsubsection{Kontrol på en regions afgrænsende rektangel}
\begin{lstlisting}[caption={Metode, som kontrollerer, hvorvidt en region
    har en kant af det afgrænsende rektangel inden for margin.},
    captionpos=b, label={pseudo_position}, frame=tb, breaklines=false,
    float=t]
def checkPosition(component, constraints):
    "Test if the component have a bounding box inside the accepting
    rectangle defined in the constraints."
    d = component.rect.width
    p = component.rect.x
    if constraints.coordinate:
        d = component.rect.height
        p = component.rect.y

    lowerInRange = p in constraints.acceptRange
    upperInRange = (p + d) in constraints.acceptRange

    return lowerInRange or upperInRange
\end{lstlisting}
Når vi har fundet værdierne i \texttt{acceptRange}, og dermed også
fastfast \texttt{coordinate}, dvs. vi ved hvilken orientering snittet
har, kan vi afgøre om en interessant region ligger placeret i snittet
eller ej. Vi har allerede udnyttet, at vi kun behøver at betragte én
koordinat, når vi kun har vertikale og horisontale snit. Det er derfor
ligetil, at kontrollere, om en regions afgrænsende rektangel, har en
kant inden for margin. Regionen er repræsenteret som en instans af
\textbf{cvConnectedComp}, hvori der er gemt en instans af
\textbf{cvRect}. Vi har en metode, kaldet \texttt{checkPosition}, som,
alt efter om vi har et vertikalt eller et horisontalt snit, returnerer
en sandhedsværdi for, om den relevante koordinat findes i de
accepterende koordinater. Metoden ses i kodeboks \ref{pseudo_position}.

Vi kan nu sammensætte en metode som returnerer alle interessante
regioner, med en kant inden for margin. Metoden tager en instans af
$\angles{CutRegions}$ og en instans af \textbf{Constraints} som
argumenter. Vi genererer altså vores betingelser inden vi begynder at
frasortere regioner, således at alle informationer om snittet og krav
for regioner ligger i instansen af \textbf{Constraints}. Den endelige
metode, for vurdering efter den naive fremgangsmåde, kaldes
\texttt{GetInterestingRegionsInCut} og er vist i kodeboks
\ref{pseudo_GetInterestingRegionsInCut}.

\begin{lstlisting}[caption={Pseudokode, som returnerer alle interessante
    regioner, der har en kant, af deres afgrænsende rektangel, inden for
    margin.},
    captionpos=b, label={pseudo_GetInterestingRegionsInCut}, frame=tb, breaklines=false,
    float=hb]
def GetInterestingRegionsInCut(CutRegions, constraints):

    # First remove uninteresting regions
    interestingRegions = GetInterestingRegions(CutRegions, constraints)

    # Initialize an empty dict
    interestingRegionsInCut = {}

    # Check every interesting region if it's bounding box
    # has an edge inside the margin
    for id in interestingRegions:
        component = interestingRegions[id][1]
        if checkPosition(component, constraints):
            interestingRegionsInCut[id] = interestingRegions[id]

    # The resulting dict contains only interesting regions
    # with an edge inside the margin
    return interestingRegionsInCut
\end{lstlisting}

\subsection{Udvidet vurdering af regioner}
Der er også blevet implementeret en anden tilgang til vurdering af
interessante regioner. Grundprincipperne bag denne er gennemgået i
afsnit \ref{subsec_udvidet_massemidtpunkt}. Vi kigger nu ikke længere på
en regions afgrænsende rektangel, men betragter i stedet regionens form
og udstrækning, for at vurdere, om den ligger i snittet, så vi har brug
for en anden måde, at repræsentere regionen på. Vi tager derfor
billedet, som \texttt{cvFloodFill} har arbejdet på, og bruger dette til
at approksimere regionens form.

\subsubsection{Approksimation af regioners form og udstrækning}
Fremgangsmåden for at approksimere en regions form og udstrækning er
simpel. Givet et segmenteret billede, fra \texttt{cvFloodFill}, og den
resulterende $\angles{CutRegions}$, fra \texttt{ExtractRegions}, kan vi,
for hver pixel i det afgrænsende rektangel, kontrollere om denne har
samme farve, som regionen er blevet tildelt. Hvis farverne er ens, så
gemmes dette punkt i en liste. Hvis regionen er stor, kan dette godt
være en tidskrævende procedure, så i praksis ønsker vi ikke at
kontrollere hver eneste pixel i det afgrænsende rektangel. Vi springer
derfor et antal pixels over, for hver gang vi har kontrolleret en pixel.
På denne måde får vi lavet et gitter, som vist i figur \ref{grid}.
Pseudokode, for at approksimere en regions form og udstrækning, er vist
i kodeboks \ref{pseudo_GetRegionGrid}, som definerer metoden
\texttt{GetRegionGrid}, der returnerer en liste med punkter for en
region.

\begin{lstlisting}[caption={Metode til at approksimere en regions
    form. Bemærk linje 19, hvor der ses et eksempel på det omvendte
    koordinatsystem i \emph{OpenCV}.}, captionpos=b,
    label={pseudo_GetRegionGrid}, frame=tb,
    breaklines=false, float=bh, numbers=left]
def GetRegionGrid(image, color, component, step):
    rect = component.rect

    # Initialize the area we want to traverse
    lower_x = rect.x
    lower_y = rect.y
    upper_x = lower_x + rect.width
    upper_y = lower_y + rect.height

    # Initialize an empty array
    coordinates = []

    # Traverse the bounding box with the defined step size
    for x in range(lower_x, upper_x, step):
        for y in range(lower_y, upper_y, step):

            # If we find a color in the image that equals the region
            # color, then we save that point
            if isSameColor(color, image[y][x]):
                coordinates.append(cvPoint(x, y))

    # Finally, return the resulting set of points
    return coordinates
\end{lstlisting}
% Sorry, I'm really sorry :(
Til senere brug, vil vi definere en metode, som finder en approksimation
til alle regioner i en instans af $\angles{CutRegions}$. Denne metode
returnerer en modificeret instans af $\angles{CutRegions}$, som vi
kalder for $\angles{CutGridRegions}$, således at den også indeholder
approksimationen til regionen. Strukturen er vist i
\eqref{CutGridRegions_dict}.
\begin{multline}
    \angles{CutRegions} = \{ \textit{~RegionId} : \\
    (\textbf{CV\_RGB~}\textit{color}, \textbf{cvConnectedComp~}\textit{region}, \textbf{cvPoint[]~}\textit{grid}) \}\quad
    \label{CutGridRegions_dict}
\end{multline}
Metoden \texttt{GridIt}, er vist i kodeboks \ref{pseudo_GridIt}, og går
alle regionerne igennem, mens den bygger den nye \emph{dict} op.

\begin{lstlisting}[caption={Metode, som finder approksimationen til alle
    regioner i en instans af $\angles{CutRegions}$.}, captionpos=b,
    label={pseudo_GridIt}, frame=tb,
    breaklines=false, float=t]
def GridIt(image, CutRegions, step):
    # Initialize an empty dict
    CutGridRegions = {}

    for id in CutRegions:
        # Get the color and component
        color = CutRegions[id][0]
        component = CutRegions[id][1]

        # Get the grid for the region
        grid = GetRegionGrid(image, color, component, step)

        # Save the findings
        CutGridRegions[id] = (color, component, grid)

    # Well, return
    return CutGridRegions
\end{lstlisting}

\subsubsection{Vurdering ved massemidtpunkt og udstrækning}
Vi siger nu, at en region ligger i snittet, hvis denne har et
massemidtpunkt inden for margin og dens masse i øvrigt er jævnt fordelt
over snittet. Til at afgøre, hvorvidt regionens masse er jævnt fordelt
over snittet, bruger vi ligning \eqref{Fordeling}, og til at finde
regionens massemidtpunkt, bruges ligning \eqref{MPunkt}, som begge kan
findes i afsnit \ref{subsec_udvidet_massemidtpunkt}. Vi skal bruge
approkimationen af regionens form og udstrækning til at afgøre
ovenstående egenskaber.

Vi starter med at kigge på fordelingen af en regions masse, hvor vi skal
finde forholdet mellem punkter på højre og venstre side af snittet.
Metoden i kodeboks \ref{pseudo_distribution}, gør netop dette, og
returnerer en sandhedsværdi for, om regionens masse er jævnt fordelt
over snittet. Vi har tidligere fastsat, at regionens masse er jævnt
fordelt over et snit, hvis forholdet mellem de to sider er under $0.75$.
Metoden gør brug af snittets orientering, hentet i
$\textbf{Constraints}.\textit{coordinate}$, og det faktum, at vi kan
finde det midterste element i vores accepterende værdier, ved at tage
det $\left(\left\lfloor \frac{|\textit{acceptRange}|}{2}\right\rfloor +
1\right)$'te element i $\textbf{Constraints}.\textit{acceptRange}$.
Dette er selve snittet.

\begin{lstlisting}[caption={Metode som, på baggrund af regionens
    fordeling af masse, afgør, om denne region er jævnt fordelt},
    captionpos=b, label={pseudo_distribution}, frame=tb,
    breaklines=false, float=t]
def checkDistribution(grid, constraints):
    # If we've got no approximation, there's no distribution
    if len(grid) == 0:
        return False

    # Init variables for right/left-distribution as floats
    rDist = 0.0
    lDist = 0.0

    # Get the middle value of the accepting range, i.e. the cut
    middleIndex = floor(len(constraints.acceptRange)) + 1
    cutVal = constraints.acceptRange[middleIndex]

    # Check the position of every pixel in the grid
    for point in grid:
        if constraints.coordinate:
            # Horisontal
            if cutVal > point.y:
                left += 1
            elif cutVal < point.y:
                right += 1
        else:
            # Vertical
            if cutVal > point.x:
                left += 1
            elif cutVal < point.x:
                right += 1

    distribution = abs((lDist - rDist)/len(grid))
    return distribution < 0.75
\end{lstlisting}

Vi skal også bruge en metode, til at finde en regions massemidtpunkt.
Denne metode er vist i kodeboks \ref{pseudo_checkCenterOfMass} og
returnerer en sandhedsværdi for, hvorvidt en regions massemidtpunkt
befinder sig inden for margin.

\begin{lstlisting}[caption={Metode, som kontrollerer, om en regions
    massemidtpunkt er inden for margin.}, captionpos=b,
    label={pseudo_checkCenterOfMass}, frame=tb, breaklines=false,
    float=t]
def checkCenterOfMass(grid, constraints):
    if len(grid) == 0:
        return False

    sum = 0
    for point in grid:
        if constraints.coordinate:
            # Horisontal
            sum += point.y
        else:
            # Vertical
            sum += point.x

    centerOfMass = sum/len(grid)

    return centerOfMass in constraints.acceptRange
\end{lstlisting}

Endeligt samler vi alle metoderne, så vi kan sortere i alle resultaterne
returneret fra segmentering af billedet. Vi kalder denne metode for
\texttt{GetExpandedRegions} og viser denne i kodeboks
\ref{pseudo_GetExpandedRegions}.

\begin{lstlisting}[caption={Pseudokode, som returnerer alle interessante
    regioner, der er nævnt fordelt over snittet og har et massemidtpunkt
    inden for margin.},
    captionpos=b, label={pseudo_GetExpandedRegions}, frame=tb, breaklines=false,
    float=hb]
def GetInterestingRegionsInCut(image, CutRegions, constraints):

    # First remove uninteresting regions
    interestingRegions = GetInterestingRegions(CutRegions, constraints)

    # Initialize the grid
    interestingRegions = GridIt(image, interestingRegions, constraints)

    # Initialize an empty dict
    interestingRegionsInCut = {}

    # Check every interesting region for center of mass
    # and distribution
    for id in interestingRegions:
        grid = interestingRegions[id][2]
        if checkDistribution(grid, constraints)
            and checkCenterOfMass(grid, constraints):
            interestingRegionsInCut[id] = interestingRegions[id]

    # The resulting dict contains only interesting regions
    # that qualify
    return interestingRegionsInCut
\end{lstlisting}

\subsection{Optimeringer}
Den naive fremgangsmåde er svær at optimere, da den er så enkel i sin
struktur. I den udvidede metode, kan der dog gøres nogle overvejelser
mht. regionens approksimation. Da vi i højere grad kigger kigger på
regionens udstrækning, er det værd at overveje, om approksimationen ved
brug af et gitter, faktisk er for grundig. Ved store regioner, kan man
med fordel bruge et større gitter, men man udelukker da mindre regioner,
da disse kan falde gennem nettet.

Man kan derfor overveje at implementere en variabel gitterstørelse,
afhængig af regionens areal, således at regioner blev approksimeret med
et passende gitter.

Man kunne også bruge en anden approksimation end gitteret. F.eks. virker
det tiltalende at finde det konvekse hylster for en region. Dette
gør, at der skal færre punkter til at beskrive regionen, men vi mister
til gengæld en del information om regionens form. Køretiden på gitteret
kan som sagt, blive meget krævende, ved store regioner, og det kan
tænkes, at det konvekse hylster kan give en mere tilfredsstillende
køretid, mod lidt mindre præcision ved fordeling og massemidtpunkt.
\clearpage

}

% vim: set tw=72 spell spelllang=da:
