{
{\sffamily Vi kaster i det følgende et detaljeret blik på
billedbehandlingsmetoderne i programmet. Efter en kort teknisk
introduktion til digitale billeder, vises hvilke datastrukturer og
metoder \emph{OpenCV} stiller til rådighed og hvordan disse bruges til
at udtrække og bedømme regioner i digitale billeder. Endeligt vil vi
undersøge hvilke problemer vores implementation kan have.
}

\subsection{Digitale billeder}
I afsnit \ref{section_kort_intro} blev der givet en kort introduktion
til den digitale repræsentation af billeder. Vi antog at en pixel kunne
antage værdier i mængden $\{0, 1\}$. I praksis kan pixels godt tage
andre værdier. Vi arbejder med billeder, hvor værdien for hver pixel, er
repræsenteret ved tre 8 bit størrelser, hver især med værdier i mængden
$\{0, 1, 2, \cdots, 254, 255\}$. Sammensætningen af de tre værdier, som
beskrives som kanaler eller farvebånd, kaldes for en RGB-farve, hvor
tallene repræsenteret ved $(R,G,B)$ henholdsvis angiver mængden af rød,
grøn og blå farve i en pixel. Et sådan billede kaldes for et
RGB-billede. For uddybende information omkring billeders repræsentation,
se da \cite{SIOlsen}. Vi benytter os af 8 bit størrelsen, da dette er
industristandarden\footnote{Reference}.

\subsection{Vigtige datastrukturer i \emph{OpenCV}}
\subsubsection{cvScalar}
Bruges til RGB-farver.

\subsubsection{cvPoint}
Beskriver punkter i det to-dimensionelle plan.

\subsubsection{cvRectangle}
Struktur som har et par $x,y$ og en bredde og højde. Koordinaterne
angiver rektanglets øverste venstre hjørne.

\subsection{Centrale metoder}
\subsubsection{cvCanny}
\subsubsection{cvSmooth}
\subsubsection{cvFloodfill}
\subsubsection{RibbonFloodfill}

}

% vim: set tw=72 spell spelllang=da:
