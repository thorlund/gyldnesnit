{
{\sffamily Ved valg af programmeringssprog har vi først og fremmest lagt
vægt på at kunne udarbejde en prototype hurtigt og bruge et sprog, som
er let at gå til. Det valgte sprog skal også gøre det nemt at udvide den
endelige implementation. Vi har også gerne villet undgå at skulle
konstruere komplicerede datastrukturer for relativt simple metoder, både
af hensyn til tidspresset og til implementationens kompleksitet. Af
ovenstående grunde har vi besluttet at udarbejde vores løsning i
programmeringssproget \textbf{Python}\cite{PythonLanguage}. Vores
erfaring er, at dette sprog er yderst velegnet at skrive forholdsvis
avancerede prototyper i.
}

\subsection{OpenCV}
Til udførelse af billedmanipulationer benytter vi os af et bibliotek
skrevet i C og C++, der hedder \emph{OpenCV}\cite{OpenCV}. Biblioteket
er udviklet af Intel og tilbyder, udover et solidt udvalg af algoritmer,
bindinger til Python\cite{OpenCVPython}. Endelig er det meget
veldokumenteret og giver referencer til publikationer om bibliotekets
algoritmer. Biblioteket er udviklet med specielt henblik på real-tids
behandling af billeder, f.eks. med et videokamera som kilde, men det
egner sig også til brug på enkelte billeder.  \emph{OpenCV} tilbyder
mange brugbare datastrukturer med hensyn til arbejdet med billeder i
Python.

Der er også andre biblioteker til billedbehandling i Python. Her kan
nævnes \emph{PIL} (Python Imaging Library)\cite{PIL} og
\emph{PythonMagick} (ImageMagick bindings)\cite{PMck}, men de er ikke
nær så grundige som \emph{OpenCV}.

\subsubsection{Andre muligheder}
Der er to helt oplagte muligheder, med hensyn til programmeringssprog,
når man taler om billedbehandling, nemlig Matlab\cite{MatlabLang} og
dets Open Source-alternativ Octave\cite{Octave}. Disse sprog blev dog
valgt fra, da vores samlede erfaring med udvikling i disse sprog ikke
var stor nok.  Endvidere finder vi, at disse sprog, på trods af, at de
især egner sig til den type beregninger, vi skal lave, er besværlige at
lave større programmer med. Matlab og Octave er dog blevet brugt til at
sammenligne resultater og teste alternative metoder med.

Da \emph{OpenCV} er skrevet i C/C++, ville det også være oplagt at bruge
et af disse sprog. Vores erfaring er dog, at man let kommer til at bruge
mere tid på at konstruere de fornødne datastrukturer og hjælpemetoder,
end på at fokusere på opgavens kerne. En senere implementation, med
fokus på køretid, kunne med fordel implementeres i C/C++, da man så
ville have fuld kontrol over, hvilke strukturer der bliver brugt i
programmet.

\subsection{Værktøjer til databasen}
Vi bruger \textbf{SQLite}\cite{Sqlite} til selve databasen, hovedsagelig
fordi der ikke kræves nogen videre konfiguration af en sådan database.
Den underliggende database er dog underordnet, da vi bruger
Python-pakken \emph{SQLObject}\cite{Sqlobject}, som giver et
abstraktionslag til en bred vifte af databaser. Vi opretter blot de
tabeller, vi ønsker at have i databasen, som klasser i Python og får
ligeledes en sådan klasse tilbage, når der laves forespørgsler til
databasen. Da \emph{SQLObject} klarer al kommunikation med databasen, er
det derfor muligt at skifte den underliggende database ud, hvis man
ønsker det. SQLite har endvidere den umiddelbare fordel, at selve
databasen eksisterer som en fil i filsystemet.  Det er derfor en let sag
at tage sikkerhedskopier af databasen uden alt for meget besvær.

\subsection{Andre værktøjer}
Vi har gjort brug af statistikprogrammet \textbf{R}\cite{Rlang} til at
behandle og præsentere vores resultater i kapitel \ref{chap_resultater}.
Da programmet primært er blevet brugt som hjælpeværktøj, til at
producere grafer, vil vi ikke komme nærmere ind på, hvordan disse
hjælpeprogrammer er blevet udviklet.

}

% vim: set tw=72 spell spelllang=da:
