{
{\sffamily Hvordan ser resultater ud.
}

Vi definerer en ny struktur inspireret fra Pythons \emph{dictionary},
som vi vil forkorte \emph{dict}. I praksis er det en liste som man slår
op i vedbrug af en nøgle. Nøglen kan være hvilken som helst type, men er
gerne en streng eller et heltal. Til hver indgang i listen er der
tilknyttet noget data som også kan være af en vilkårlig type. Man kan
derved have \emph{dicts} inde i en \emph{dict}. En \emph{dict} defineres
som vist i \ref{def_dict}.
\begin{eqnarray}
    \langle[\textit{dictName}]\rangle = \{ [\textit{key}] : [\textit{data}] \}
    \label{def_dict}
\end{eqnarray}
Som et simpelt eksempel kan vi konstruere en \emph{dict}
$\angles{LuckyNumbers}$ som indeholder personers lykketal:
% TeX-Gods, please forgive me :(
\begin{multline}
    \angles{LuckyNumbers} = \{ \qquad \textrm{Tom Cruise} : 4 , \\
    \textrm{Arthur Dent} : 42 \qquad\qquad\\
    \shoveleft{\}}\shoveright{}
    \label{lucky_dict}
\end{multline}
% In honor of the danish Tom Cruise

\begin{multline}
    \textbf{class~} \textrm{ConnectedComponent} = \{ \\
    \shoveleft{\qquad\textbf{cvRect} : \textit{rect}} \\
    \shoveleft{\qquad\textbf{int} : \textit{area}} \\
    \shoveleft{\qquad\textbf{cvSeq} : \textit{seq}} \\
    \shoveleft{\}}\shoveright{}
\end{multline}

\begin{multline}
    \angles{CutRegions} = \{ \textit{~RegionId} : \\
    (\textbf{cvScalar~}\textit{color},
    \textbf{ConnectedComponent~}\textit{component}) \}\quad
\end{multline}

\begin{eqnarray}
    \angles{RatioRegions} = \{ \textit{~CutNo} : \angles{CutRegions} \}
\end{eqnarray}

\begin{eqnarray}
    \angles{ImageRegions} = \{ \textit{~CutRatio} : \angles{RatioRegions} \}
\end{eqnarray}

% vim: set tw=72 spell spelllang=da:
