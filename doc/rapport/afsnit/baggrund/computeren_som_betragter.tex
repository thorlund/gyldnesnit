{
Giver man tre forskellige mennesker den opgave at afgøre, hvad der er
det interessante i et givet maleri, kan man meget vel få tre forskellige
svar. Spørger man, \emph{hvordan} de er kommet frem til svaret på det
foregående spørgsmål, kastes lys over præcis hvor kompliceret en opgave
det kan være, at finde interessante objekter i et maleri. Én begrunder
måske sit valg med en viden om netop det givne maleri, en anden med
viden om maleriets kunstner eller periode, mens en tredie begrunder det
med æstetiske virkemidler eller subjektive holdninger. Her er alle
muligheder åbne for fejl, da man som regel ikke har kunstneren til
rådighed til at give det rigtige svar, hvis han da overhovedet selv
kender det.  Maleriets motiv kan lede én på sporet af, hvad der er
interessant, men dette kræver måske en viden om motivets bagvedliggende
historie.  Mennesket tager altså en lang række overvejelser og
baggrundsviden i betragtning, når ovenstående opgave skal løses.

Det interessante i et billede afhænger af den enkelte opgave.  Hvis vi
forestiller os en læge, der kigger på et røntgenbillede af en patient
med en brækket arm, er det indlysende, hvad lægen betragter som det
interessante i billedet, nemlig der hvor bruddet sidder. Når vi har med
tilfældige malerier at gøre, så står vi dog stadig tilbage med
spørgsmålet: \emph{``Hvad er egentlig \emph{interessant} i et maleri?''}
Vi lader dette spørgsmål stå åbent for at se på, hvordan computeren
betragter et maleri.

\begin{figure}[b]
    \centering
    \includegraphics[scale=0.3]{afsnit/baggrund/billeder/pixel_lena}
    \caption[]{Pixels i et billede}
    \label{pixel_lena}
\end{figure}

Billedet i figur \ref{pixel_lena} er blevet delt op i små felter kaldet
\textbf{pixels}. Hver pixel har en farve. Computeren opfatter et billede
netop som pixels. Den ser endvidere disse pixels \emph{én ad gangen}.
Det svarer altså til, at den enkelte farve til en pixel bliver læst op
for en person, der ikke kan se selve billedet. For at køre eksemplet
helt ud, så har vi, at en person får følgende at vide:
\begin{quote}
    \emph{``Pixel med koordinater $(0, 0)$ er gul. Pixel med
    koordinater $(1, 0)$ er orange''} etc.
\end{quote}
Dette giver ingen egentlig information om, \emph{hvad} billedet
forestiller. Computeren kan ikke se billedet i sin helhed og har som
udgangspunkt ikke nogen baggrundsviden at basere en vurdering på. Det
eneste, computeren kan gøre, er at gå billedet igennem, pixel for pixel,
og sammenligne dem.

Vi ønsker at bruge computeren til at afgøre, om der ligger noget
interessant i billedet omkring det gyldne snit. Vi vil derfor nu kaste
et blik på den forskning, der allerede er blevet gjort på dette område.

}

% vim: set tw=72 spell spelllang=da:
