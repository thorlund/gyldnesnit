{
\textsf{
Dette kapitel har til formål at give en introduktion til det gyldne snit,
både matematisk og historisk. Der kastes et blik på den forskning, der
allerede er blevet gjort på området, hvilke metoder der er blevet brugt
og de metoder, der er til rådighed for ny forskning.
}

\section{Det gyldne snit\label{section_gyldne_snit}}
{
Det vi kalder det gyldne snit, den gyldne ratio eller det guddommelige
forhold, blev allerede beskrevet i Euklids \emph{Elements} fra ca. 300
f.  Kr. som følger:
\begin{quote}
	\emph{``A straight line is said to have been cut in extreme
	and mean ratio when, as the whole line is to the greater
	segment, so is the greater to the less.''}\cite{Euclid300bc}
\end{quote}

\begin{figure}[h!]
	\begin{center}
		\includegraphics[scale=0.49,angle=0]{afsnit/baggrund/billeder/line_segment_a_c_b}
	\end{center}
	\caption{Euklids opdeling af et linjestykke}
	\label{line_segment}
\end{figure}

Givet et linjestykke $A\ B$, som vist i figur \ref{euclid}, og ud fra
Euklids beskrivelse kan $\varphi$ defineres som
\begin{equation}
	\varphi	= \frac{A\ C}{C\ B} = \frac{A\ B}{A\ C}
	\label{euclid}
\end{equation}
Ved at indsætte variable i ligning \ref{euclid} får vi
\begin{equation}
	\varphi = \frac{\varphi + 1}{\varphi}
	\label{expand_euclid}
\end{equation}
hvilket giver os andengradspolynomiet
\begin{equation}
	\varphi^{2} - \varphi - 1 = 0.
	\label{poly_phi}
\end{equation}
Hvis vi nu løser andengradspolynomiet i ligning \ref{poly_phi}, med
$\varphi > 0$, får vi
\begin{eqnarray*}
	\varphi	& =	& \frac{\sqrt{5} + 1}{2} \\
		& =	& 1.6180\ 3398\ 8749\ 8948\ 4820 \dots
\end{eqnarray*}

Tallet $\varphi$ bemærker sig blandt andet ved, at når det kvadreres, så
lægger man blot 1 til. Dette udledes trivielt fra ligning \ref{poly_phi}
\begin{equation}
	\varphi^{2} = \varphi + 1
	\label{phi_squared}
\end{equation}

Vi kan også finde polynomiets anden rod som angives ved $\varPhi$
\begin{eqnarray*}
	\varPhi & = & \frac{1}{\varphi} \\
		& = & \varphi - 1 \\
		& = & 0.6180\ 3398\ 8749\ 8948\ 4820 \dots 
\end{eqnarray*}
Også tallet $\varPhi$ er interessant idet dets eget kvadrat plus sig
selv giver 1. Vi har at
\begin{equation}
	\varPhi^{2} + \varPhi = 1
	\label{Phi_squared}
\end{equation}
hvilket kun er gældende for $\varPhi$.

Det gyldne snit fremviser endvidere en interessant forbindelse til
Fibonaccis talrække, da forholdet mellem to fibonaccital $F(n)$ og $F(n
- 1)$ konvergerer mod $\varphi$ når $n$ nærmer sig uendelig. Mere
formelt har vi at
\begin{eqnarray*}
	\varphi & =     & \lim_{n \rightarrow
	\infty}{\frac{F(n)}{F(n - 1)}}
\end{eqnarray*}

\subsubsection{Et gyldent rektangel}
På samme måde som vi kan opdele en linjestykke efter det gyldne snit,
kan vi konstruere et rektangel hvor forholdene mellem højde og bredde er
$\varphi$. Vi konstruerer et rektangel hvor alle sider er lig 1 og
tegner en diagonal fra dette rektangelels midte til modsatte hjørne. Med
denne diagonal som radius tegnes en cirkel som et gyldent rektangel kan
tegnes efter. Figur \ref{golden_rectangle} illustrerer denne metode.

\begin{figure}[h!]
	\begin{center}
		\includegraphics[scale=0.35,angle=0]{afsnit/baggrund/billeder/Golden_Rectangle_Construction}
	\end{center}
	\caption{Et gyldent rektangel - \emph{Kilde: Wikipedia}}
	\label{golden_rectangle}
\end{figure}
Det ses at rektanglet har forholdet $\varphi:1$ og at eksemplet er helt
analogt til det linjestykke givet i figur \ref{line_segment}. Dog skal
det bemærkes, at det rektangel der kan konstrueres af linjestykkerne 1
og $\varphi - 1$ også er et gyldent rektangel med forholdet $1:\varphi
-1 = \varphi$. Man kan derved konstruere gyldne rektangler ud i det
uendelige ved hele tiden at lave nye gyldne rektangler.

\subsubsection{Spiraler og det gyldne snit}
Når man som ovenfor, gentagende gange deler et gyldent rektangel kan man
bruge dette til at konstruere en gylden spiral. En gylden spiral kan
skrives ved ligningen for generelle logaritmiske spiraler som
\begin{equation}
	r = ae^{c\theta}
	\label{log_spiral_2}
\end{equation}
eller
\begin{equation}
	\theta = \frac{1}{c}\ln(r/a)
	\label{log_spiral_1}
\end{equation}
hvor $e$ er grundtallet for den naturlige logaritme og $c$ skal have en
speciel værdi for at kunne være en gylden spiral. Den gyldne spiral kan
approksimeres ved at konstruere en fibonaccispiral som vist i figur
\ref{fibonacci_spiral}.
\begin{figure}[h!]
	\begin{center}
		\includegraphics[scale=0.35,angle=0]{afsnit/baggrund/billeder/Fibonacci_spiral}
	\end{center}
	\caption{En fibonaccispiral - \emph{Kilde: Wikipedia}}
	\label{fibonacci_spiral}
\end{figure}
Da fibonaccispiralen er konstrueret efter Fibonaccis talrække nærmer
denne spiral sig en gylden spiral, men kan ikke betegnes som værende en
ægte gylden spiral.

Med den matematiske definition på plads kan vi kigge på den forskning
der er blevet gjort i forbindelse med det gyldne snit.

% vim: set tw=72 spell spelllang=da:


\section{Computeren som betragter\label{section_computer_betragter}}
{
Giver man tre forskellige mennesker den opgave at afgøre, hvad der er
det interessante i et givet maleri, kan man meget vel få tre forskellige
svar. Spørger man, \emph{hvordan} de er kommet frem til svaret på det
foregående spørgsmål, kastes lys over præcis hvor kompliceret en opgave
det kan være, at finde interessante objekter i et maleri. Én begrunder
måske sit valg med en viden om netop det givne maleri, en anden med
viden om maleriets kunstner eller periode, mens en tredie begrunder det
med æstetiske virkemidler eller subjektive holdninger. Her er alle
muligheder åbne for fejl, da man som regel ikke har kunstneren til
rådighed til at give det rigtige svar, hvis han da overhovedet selv
kender det.  Maleriets motiv kan lede én på sporet af, hvad der er
interessant, men dette kræver måske en viden om motivets bagvedliggende
historie.  Mennesket tager altså en lang række overvejelser og
baggrundsviden i betragtning, når ovenstående opgave skal løses.

Det interessante i et billede afhænger af den enkelte opgave.  Hvis vi
forestiller os en læge, der kigger på et røntgenbillede af en patient
med en brækket arm, er det indlysende, hvad lægen betragter som det
interessante i billedet, nemlig der hvor bruddet sidder. Når vi har med
tilfældige malerier at gøre, så står vi dog stadig tilbage med
spørgsmålet: \emph{``Hvad er egentlig \emph{interessant} i et maleri?''}
Vi lader dette spørgsmål stå åbent for at se på, hvordan computeren
betragter et maleri.

\begin{figure}[b]
    \centering
    \includegraphics[scale=0.3]{afsnit/baggrund/billeder/pixel_lena}
    \caption[]{Pixels i et billede}
    \label{pixel_lena}
\end{figure}

Billedet i figur \ref{pixel_lena} er blevet delt op i små felter kaldet
\textbf{pixels}. Hver pixel har en farve. Computeren opfatter et billede
netop som pixels. Den ser endvidere disse pixels \emph{én ad gangen}.
Det svarer altså til, at den enkelte farve til en pixel bliver læst op
for en person, der ikke kan se selve billedet. For at køre eksemplet
helt ud, så har vi, at en person får følgende at vide:
\begin{quote}
    \emph{``Pixel med koordinater $(0, 0)$ er gul. Pixel med
    koordinater $(1, 0)$ er orange''} etc.
\end{quote}
Dette giver ingen egentlig information om, \emph{hvad} billedet
forestiller. Computeren kan ikke se billedet i sin helhed og har som
udgangspunkt ikke nogen baggrundsviden at basere en vurdering på. Det
eneste, computeren kan gøre, er at gå billedet igennem, pixel for pixel,
og sammenligne dem.

Vi ønsker at bruge computeren til at afgøre, om der ligger noget
interessant i billedet omkring det gyldne snit. Vi vil derfor nu kaste
et blik på den forskning, der allerede er blevet gjort på dette område.

}

% vim: set tw=72 spell spelllang=da:


\section{Historisk perspektiv\label{section_forskning}}
{
Som allerede nævnt, kendte de gamle grækere til tallet $\varphi$, som
Euklid kaldte for \emph{the division in extreme and mean ratios},
forkortet DEMR. Luca Pacioli udgiver i 1509 bogen \emph{De divina
proportione}, som beskriver samme fænomen omtalt som ``den guddommelige
propertion''. Udtrykket ``det gyldne snit'' kan spores tilbage til
tyskeren Martin Ohm (1792 -- 1872), der første gang betegner Euklids DEMR
som \emph{der Goldener Schnitt} i sin bog \emph{Die reine
Elementar-Mathematik}, fra 1835\cite{Markowsky1992}. Det gyldne snit er
nu blevet den foretrukne betegnelse.

Nu bliver det påstået fra flere kilder, at det gyldne snit bruges i
malerier, arkitektur og musik, da dette forhold har specielt tiltalende
æstetiske
egenskaber\cite{GoldenNumber}\cite{RatioArt}\cite{Putz1995}\cite{Stakhov2006490}\cite{Boussora2004}.
Især \cite{GoldenNumber} er særlig ivrig og finder det gyldne snit i alt
lige fra cigaretpakker til skallen fra en nautil. Netop nautilskallen
bliver meget ofte brugt som argument for, at det gyldne snit findes i
naturen i form af en gylden spiral lignende den fra figur
\ref{fibonacci_spiral}. Hvis man rent faktisk måler efter,
viser det sig, at dette ikke er tilfældet. Som argumenteret i
\cite{Sharp2002} er spiralen i nautilskallen rigtigt nok logaritmisk,
men ikke med en faktor $c$ som ville gøre den til en gylden spiral.

Det græske tempel Parthenon spiller også en central rolle i rygterne om
det gyldne snit. Billeder af templet bliver tit vist med et rektangel
tegnet over. Der florerer forskellige udgaver af disse billeder, hvor
der åbenbart ikke tages højde for, at dele af templet ikke indgår i
rektanglet eller fotografiets perspektiv. George Markowsky gør i
\cite{Markowsky1992} op med mange af disse vrangforestillinger. Påstande
om, at Keopspyramiden skulle være bygget efter det gyldne snit --- som
angivet i \cite{Stakhov2006490} --- at nogle af Leonardo da Vincis malerier
er malet efter det gyldne snit, og at det gyldne rektangel er det mest
æstetisk tiltrækkende format\cite{GoldenNumber}\cite{RatioArt}, bliver i
hans artikel afvist, da mange af disse undersøgelser lider under det, han
kalder \emph{the Pyramidology Fallacy}. Dette udtrykt henter Markowsky
fra \cite{Gardner1952_2} og beskriver de personer, der forsker i
pseudovidenskab såsom pyramidernes arkitektur. Her har forskerne ofte
mange forskellige tal at ``jonglere'' med og kan frit vælge netop dem,
der giver det ønskede resultat. En anden forfatter, Roger
Hertz-Fischler, er ligeledes optaget af den specielle værdi, det gyldne
snit er blevet tillagt. Det som Markowsky og Gardner kalder for
\emph{Pyramidology}, betegner han som \emph{golden numberism}, og han har
fulgt dette fænomens historie tilbage til en tysk mand ved navn
Adolph Zeising (1810 -- 1876)\cite{Herz-Fischler2005}. Hertz-Fischler
hævder, at stort set alle undersøgelser inden for \emph{golden numberism}
kan spores tilbage til Zeising.

Hypotesen om det gyldne rektangels æstetiske egenskaber bliver også
taget op i \cite{Boselie1984} og \cite{Plug1980}, hvor det ikke kan
konkluderes, at det gyldne rektangel skulle have nogen æstetisk
signifikans. En lignende hypotese bliver sat på prøve i
\cite{McManus1995}, hvor det undersøges, om et billedes geometriske
komposition har indflydelse på dets æstetiske effekt. I. C. McManus
kunne, gennem tre eksperimenter, ikke finde noget grundlag for at et
billedes geometriske komposition påvirkede testpersonernes bedømmelse.
Det siges i konklusionen:

\begin{quote}
	\emph{``Together the results of these experiments throw
	considerable doubt upon the hypothesis of the implicit detection
	of latent compositional geometry as a major component of
	aesthetic judgements, at least for relatively unsophisticated
	observers[\dots]''}
\end{quote}

Selvom ovenstående ikke eksplicit nævner det gyldne snit, er der
alligevel en klar relevans til billeder opbygget geometrisk efter det
gyldne snit.

Der ses dog et klart problem, hvilket Markowsky også nævner, idet det
ikke er defineret, hvornår et stykke kunst er konstrueret efter det
gyldne snit. Ej heller er det defineret, præcis hvordan man finder det
gyldne snit: Der findes mange billeder, hvor snittet er illustreret vha.
linjer, men størstedelen af disse har ikke de fornødne mål, og det
gyldne snit findes gerne helt arbitrære steder uden nogen som helst
identificerbar fremgangsmåde.  Én undersøgelse, omhandlende billeders
komposition, har dog lavet statistik på 565 malerier, men kun forholdet
mellem lærredets dimensioner er blevet registreret\cite{Olariu1999}.
Denne undersøgelse kunne ikke konkludere, at kunstnere foretrækker det
gyldne snit i lærredet. Det er derfor interessant at forsøge at
automatisere søgningen efter det gyldne snit i malerier og fastsætte
klare kriterier for, hvornår et billede kan siges at være konstrueret
efter det gyldne snit.  Til denne opgave er det oplagt at udnytte
regnekraften fra computeren, hvilket også giver anledning til  at
analysere langt større datasæt.

\subsection{Eksisterende datalogisk forskning}
Den tidligere datalogiske forskning ligger i generelle teorier og algoritmer
indenfor billedbehandling end en tidligere forskning præcis indefor analyse af
interessante regioner i nærheden af det gyldne snit.
Det mest relaterede forskning er et eksperiment, som analysere æstetik i fotografier\cite{DattaWang}.
Det væsenlige ved dette eksperiment er hvorledes billederne
bedømmes. Her arbejdes der med 56 forskellige kandidater til at definere
et billede, som mere eller mindre æstetisk korrekt og originalt. 
Kandidaterne bliver udvalgt i forskellige kategorier, tre af kandidater 
bliver vurderet ud fra ``The rule of Thirds'', som er beskrevet ved
\begin{quote}
	``The rule can be considered as a sloppy approxmination of to the
	golden ratio (about 0.618)''
\end{quote}

%fyld? Meget fedt at få sagt det men jeg ved ikke om det er specielt
%vigtigt
Deres kandidater dækker over noget ganske centralt i store dele af
billedbehandling nemlig detektion af
interessante regioner i billeder, problemet er at begrebet slet ikke er
entydigt defineret, men skifter mening fra fagområde til fagområde.
Hvad vi opfatter som interessant vil blive diskuteret i \ref{section_kort_intro}

De forskellige metoder til at detektere interessante regioner, indeholder nogle avancerede tekniker.
Potentielt kunne disse tekniker forbedre udtrækning af interessante regioner.

Den første er maskineindlæring, der stammer fra feltet kunstig
intelligens indenfor datalogi. I billedbehandling bliver det dog
brugt til at finde interessante regioner med stor præcision, og formålet
med brugen er bland andet at finde ansigter, mennesker og
biler\cite{ViolaJones01}\cite{SchneidermanKanade00}\cite{Gabor}. Præcisionen på
algoritmernes detektion koster dog meget kompleksiteten i udvikligen 
af algoritmerne. Programmet skal også have en base for at blive oplært
til at finde interessante regioner, og derved låser programmet fast til
kun at finde det som oplæringsbasen er.

Den anden er opdelinger af billeder efter tekstur. Ideen bag denne type
af metoder er at opdele efter
overfladetekstur\cite{218442}\cite{CarsonBelongie02}\cite{PapageorgiouPoggio}, de passer meget bedre på
den type region der ledes efter. De besidder dog adskillige
negative egenskaber. En af disse er at de fleste algoritmer har svært
ved at fungere på støj i billedet, der er selvfølgelig veje at reducere
dette problem, dog er de meget beregningstunge.\cite{PalPal}

}
% vim: set tw=72 spell spelllang=da:


\section*{Opsummering}
Den matematiske definition, giver os nogle værktøjer, til at analysere
malerier for brug af det gyldne snit.  Den eksisterende forskning, eller
mangel på samme, fortæller os, at det er en meget svær opgave, at bruge
computeren til at fortolke malerier, og at det er tvivlsomt, hvorvidt
det gyldne snit kan tillægges nogen speciel værdi.  I næste afsnit vil
vi se på, hvilke metoder vi vil bruge, for at undersøge det gyldne snit
i digitale gengivelser af malerier.

}

% vim: set tw=72 spell spelllang=da:
