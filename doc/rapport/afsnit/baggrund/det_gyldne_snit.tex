{
Det vi kalder det gyldne snit, den gyldne ratio eller det guddommelige
forhold, blev allerede beskrevet i Euklids \emph{Elements} fra ca. 300
f.  Kr. som følger:
\begin{quote}
	\emph{``A straight line is said to have been cut in extreme
	and mean ratio when, as the whole line is to the greater
	segment, so is the greater to the less.''}\cite{Euclid300bc}
\end{quote}

\begin{figure}[h!]
	\begin{center}
		\includegraphics[scale=0.49,angle=0]{afsnit/baggrund/billeder/line_segment_a_c_b}
	\end{center}
	\caption{Euklids opdeling af et linjestykke}
	\label{line_segment}
\end{figure}

Givet et linjestykke $A\ B$, som vist i figur \ref{euclid}, og ud fra
Euklids beskrivelse kan $\varphi$ defineres som
\begin{equation}
	\varphi	= \frac{A\ C}{C\ B} = \frac{A\ B}{A\ C}
	\label{euclid}
\end{equation}
Ved at indsætte variable i ligning \ref{euclid} får vi
\begin{equation}
	\varphi = \frac{\varphi + 1}{\varphi}
	\label{expand_euclid}
\end{equation}
hvilket giver os andengradspolynomiet
\begin{equation}
	\varphi^{2} - \varphi - 1 = 0.
	\label{poly_phi}
\end{equation}
Hvis vi nu løser andengradspolynomiet i ligning \ref{poly_phi}, med
$\varphi > 0$, får vi
\begin{eqnarray*}
	\varphi	& =	& \frac{\sqrt{5} + 1}{2} \\
		& =	& 1.6180\ 3398\ 8749\ 8948\ 4820 \dots
\end{eqnarray*}

Tallet $\varphi$ bemærker sig blandt andet ved, at når det kvadreres, så
lægger man blot 1 til. Dette udledes trivielt fra ligning \ref{poly_phi}
\begin{equation}
	\varphi^{2} = \varphi + 1
	\label{phi_squared}
\end{equation}

Vi kan også finde polynomiets anden rod som angives ved $\varPhi$
\begin{eqnarray*}
	\varPhi & = & \frac{1}{\varphi} \\
		& = & \varphi - 1 \\
		& = & 0.6180\ 3398\ 8749\ 8948\ 4820 \dots 
\end{eqnarray*}
Også tallet $\varPhi$ er interessant idet dets eget kvadrat plus sig
selv giver 1. Vi har at
\begin{equation}
	\varPhi^{2} + \varPhi = 1
	\label{Phi_squared}
\end{equation}
hvilket kun er gældende for $\varPhi$.

Det gyldne snit fremviser endvidere en interessant forbindelse til
Fibonaccis talrække, da forholdet mellem to fibonaccital $F(n)$ og $F(n
- 1)$ konvergerer mod $\varphi$ når $n$ nærmer sig uendelig. Mere
formelt har vi at
\begin{eqnarray*}
	\varphi & =     & \lim_{n \rightarrow
	\infty}{\frac{F(n)}{F(n - 1)}}
\end{eqnarray*}

\subsubsection{Et gyldent rektangel}
På samme måde som vi kan opdele en linjestykke efter det gyldne snit,
kan vi konstruere et rektangel hvor forholdene mellem højde og bredde er
$\varphi$. Vi konstruerer et rektangel hvor alle sider er lig 1 og
tegner en diagonal fra dette rektangelels midte til modsatte hjørne. Med
denne diagonal som radius tegnes en cirkel som et gyldent rektangel kan
tegnes efter. Figur \ref{golden_rectangle} illustrerer denne metode.

\begin{figure}[h!]
	\begin{center}
		\includegraphics[scale=0.35,angle=0]{afsnit/baggrund/billeder/Golden_Rectangle_Construction}
	\end{center}
	\caption{Et gyldent rektangel - \emph{Kilde: Wikipedia}}
	\label{golden_rectangle}
\end{figure}
Det ses at rektanglet har forholdet $\varphi:1$ og at eksemplet er helt
analogt til det linjestykke givet i figur \ref{line_segment}. Dog skal
det bemærkes, at det rektangel der kan konstrueres af linjestykkerne 1
og $\varphi - 1$ også er et gyldent rektangel med forholdet $1:\varphi
-1 = \varphi$. Man kan derved konstruere gyldne rektangler ud i det
uendelige ved hele tiden at lave nye gyldne rektangler.

Med den matematiske definition på plads kan vi kigge på den forskning
der er blevet gjort i forbindelse med det gyldne snit.

% vim: set tw=72 spell spelllang=da:
