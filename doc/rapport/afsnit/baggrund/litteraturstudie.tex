{
Som allerede nævnt, kendte de gamle grækere til tallet $\varphi$, som
Euklid kaldte for \emph{the division in extreme and mean ratios},
forkortet DEMR. Luca Pacioli udgiver i 1509 bogen \emph{De divina
proportione}, som beskriver samme fænomen omtalt som ``den guddommelige
propertion''. Udtrykket ``det gyldne snit'' kan spores tilbage til
tyskeren Martin Ohm (1792 -- 1872), der første gang betegner Euklids DEMR
som \emph{der Goldener Schnitt} i sin bog \emph{Die reine
Elementar-Mathematik}, fra 1835\cite{Markowsky1992}. Det gyldne snit er
nu blevet den foretrukne betegnelse.

Nu bliver det påstået fra flere kilder, at det gyldne snit bruges i
malerier, arkitektur og musik, da dette forhold har specielt tiltalende
æstetiske
egenskaber\cite{GoldenNumber}\cite{RatioArt}\cite{Putz1995}\cite{Stakhov2006490}\cite{Boussora2004}.
Især \cite{GoldenNumber} er særlig ivrig og finder det gyldne snit i alt
lige fra cigaretpakker til skallen fra en nautil. Netop nautilskallen
bliver meget ofte brugt som argument for, at det gyldne snit findes i
naturen i form af en gylden spiral lignende den fra figur
\ref{fibonacci_spiral}. Hvis man rent faktisk måler efter,
viser det sig, at dette ikke er tilfældet. Som argumenteret i
\cite{Sharp2002} er spiralen i nautilskallen rigtigt nok logaritmisk,
men ikke med en faktor $c$ som ville gøre den til en gylden spiral.

Det græske tempel Parthenon spiller også en central rolle i rygterne om
det gyldne snit. Billeder af templet bliver tit vist med et rektangel
tegnet over. Der florerer forskellige udgaver af disse billeder, hvor
der åbenbart ikke tages højde for, at dele af templet ikke indgår i
rektanglet eller fotografiets perspektiv. George Markowsky gør i
\cite{Markowsky1992} op med mange af disse vrangforestillinger. Påstande
om, at Keopspyramiden skulle være bygget efter det gyldne snit --- som
angivet i \cite{Stakhov2006490} --- at nogle af Leonardo da Vincis malerier
er malet efter det gyldne snit, og at det gyldne rektangel er det mest
æstetisk tiltrækkende format\cite{GoldenNumber}\cite{RatioArt}, bliver i
hans artikel afvist, da mange af disse undersøgelser lider under det, han
kalder \emph{the Pyramidology Fallacy}. Dette udtrykt henter Markowsky
fra \cite{Gardner1952_2} og beskriver de personer, der forsker i
pseudovidenskab såsom pyramidernes arkitektur. Her har forskerne ofte
mange forskellige tal at ``jonglere'' med og kan frit vælge netop dem,
der giver det ønskede resultat. En anden forfatter, Roger
Hertz-Fischler, er ligeledes optaget af den specielle værdi, det gyldne
snit er blevet tillagt. Det som Markowsky og Gardner kalder for
\emph{Pyramidology}, betegner han som \emph{golden numberism}, og han har
fulgt dette fænomens historie tilbage til en tysk mand ved navn
Adolph Zeising (1810 -- 1876)\cite{Herz-Fischler2005}. Hertz-Fischler
hævder, at stort set alle undersøgelser inden for \emph{golden numberism}
kan spores tilbage til Zeising.

Hypotesen om det gyldne rektangels æstetiske egenskaber bliver også
taget op i \cite{Boselie1984} og \cite{Plug1980}, hvor det ikke kan
konkluderes, at det gyldne rektangel skulle have nogen æstetisk
signifikans. En lignende hypotese bliver sat på prøve i
\cite{McManus1995}, hvor det undersøges, om et billedes geometriske
komposition har indflydelse på dets æstetiske effekt. I. C. McManus
kunne, gennem tre eksperimenter, ikke finde noget grundlag for at et
billedes geometriske komposition påvirkede testpersonernes bedømmelse.
Det siges i konklusionen:

\begin{quote}
	\emph{``Together the results of these experiments throw
	considerable doubt upon the hypothesis of the implicit detection
	of latent compositional geometry as a major component of
	aesthetic judgements, at least for relatively unsophisticated
	observers[\dots]''}
\end{quote}

Selvom ovenstående ikke eksplicit nævner det gyldne snit, er der
alligevel en klar relevans til billeder opbygget geometrisk efter det
gyldne snit.

Der ses dog et klart problem, hvilket Markowsky også nævner, idet det
ikke er defineret, hvornår et stykke kunst er konstrueret efter det
gyldne snit. Ej heller er det defineret, præcis hvordan man finder det
gyldne snit: Der findes mange billeder, hvor snittet er illustreret vha.
linjer, men størstedelen af disse har ikke de fornødne mål, og det
gyldne snit findes gerne helt arbitrære steder uden nogen som helst
identificerbar fremgangsmåde.  Én undersøgelse, omhandlende billeders
komposition, har dog lavet statistik på 565 malerier, men kun forholdet
mellem lærredets dimensioner er blevet registreret\cite{Olariu1999}.
Denne undersøgelse kunne ikke konkludere, at kunstnere foretrækker det
gyldne snit i lærredet. Det er derfor interessant at forsøge at
automatisere søgningen efter det gyldne snit i malerier og fastsætte
klare kriterier for, hvornår et billede kan siges at være konstrueret
efter det gyldne snit.  Til denne opgave er det oplagt at udnytte
regnekraften fra computeren, hvilket også giver anledning til  at
analysere langt større datasæt.

\subsection{Billedbehandling}
Indenfor billedbehandlingsfeltet er begrebet "feature detection" eller
interessante regioner detektion, ikke klart defineret.
Forskellige gerne af billedbehandling definere begrebet indenfor
deres egen gren, som et eksempel kan en interessant region i film
er et punkt, der let kan genkendes.
Dette punkt kan bruges til at skabe 3d modeller af et objekt ud fra 2d billeder.

Selvom der er skrevet utrolig meget litteratur om interessante regioner
detektion, så er der to grunde til at forkaste hovedparten af teorierne.

Den første er "Machine Learning", dette stammer faktisk fra Kunstig
Intelligens feltet af datalogi. Indenfor billedbehandling bliver det dog
brugt til at finde interessante regioner med stor præcision. Formålet
med brugen er dog med at finde ansigter, mennesker og
biler\cite{ViolaJones01}\cite{SchneidermanKanade00}\cite{Gabor}. Præcisionen på
algoritmernes detektion koster dog meget kompleksiteten i udvikligen 
af algoritmerne. Programmet skal også have en base for at blive oplært
til at finde interessante regioner, og derved låser programmet fast til
kun at finde det som oplæringsbasen er.

Den anden er opdelinger af billeder efter tekstur. Ideen bag denne type
af metoder er at opdele efter
overfladetekstur\cite{218442}\cite{CarsonBelongie02}\cite{PapageorgiouPoggio}, de passer meget bedre på
den type region der ledes efter. De besidder dog adskillige
negative egenskaber. En af disse er at de fleste algoritmer har svært
ved at fungere på støj i billedet, der er selvfølgelig veje at reducere
dette problem, dog er de meget beregningstunge.\cite{PalPal}

Få eksperimenter er gjort med algoritmer, der finder det spændende i
billeder. Det tætteste er et forsøg på at analysere æstetik i
fotografier.\cite{DattaWang} Det gyldne snit bliver faktisk nævnt, det
er dog en sidebemærkning under et kapitel om "The rule of Thirds".
Forsøget benytter sig af en statistik tilgang til problemet, dog
kombineret med "Machine Learning"

}
% vim: set tw=72 spell spelllang=da:
