{
{\sffamily Dette dokument er den endelige rapport udarbejdet i forbindelse med
kurset ``Bachelorprojekt'' som udbydes på Datalogisk Institut ved
Københavns Universitet. Det forventes at læseren har en basal viden
inden for datalogi og kendskab til begreber der bruges i forbindelse med
billedbehandling.
}
}

%\subsection{Forventninger til læser} Dette opgave omhandler metoder og
%udledninger af akademiske problemer som opstår ved programmering inde for
%feltet billedbehandling, Derfor regner vi med at læseren af denne rapport har
%en basal vide inde for datalogi i retninger af billeders opbygning, det vil
%siger, viden om hvad en pixel er, hvad betjener RGB favre, osv. Dog ligger der
%vægt på at forklaring af vores problemer, Det vi er kommet frem til og hvad man
%kan bruge vores fund til. Ligger på en nivo som en kunst studerende kan
%relatere sig til og forstå.  Vi har tænkt på at lave en lille under afsnit til
%vær del i vores opgave, som forklare det vi laver med meget udpenslet sprog.

% vim: set tw=72 spell spelllang=da:
