{
{\sffamily Dette dokument er den endelige rapport, udarbejdet i
forbindelse med kurset ``Bachelorprojekt'', som udbydes på Datalogisk
Institut ved Københavns Universitet. I de mere tekniske afsnit, især med
henblik på kapitel \ref{chap_implementation}, men også dele af kapitel
\ref{chap_detektion}, forventes det, at læseren har en basal viden inden
for datalogi, svarende til at have bestået de obligatoriske kurser på
bacheloruddannelsen af datalogistudiet\cite{DIKUkurser}. Kendskab til de
grundlæggende begreber i forbindelse med billedbehandling, vil endvidere
være en fordel. For en introduktion til billedbehandling henvises til
\cite{SIOlsen}. Kapitel \ref{chap_afproevning}, hvor vi ser på de
grafiske resultater fra programmet, og kapitel \ref{chap_resultater},
hvor vi præsenterer de videnskabelige resultater, kan umiddelbart læses
af alle, uden videre forudsætninger end ren og skær interesse, for emnet
omhandlende det gyldne snit og analyse af digitale gengivelser af
malerier. I det umiddelbart følgende, forventes det, at læseren kender
til simpel slang, alá Malk De Koijn.

\section*{Skud}
Vi sender skud ud til vores hund, Jim Daggerthuggert og alle fra
Langestrand. Ligeledes sendes skud til vores vejleder, Jakob Grue
Simonsen, som modtager store mængder gadekredit. Stephanie Bekkar,
Franck Franck, Emma Haxen og Lisbeth Steenstrup skal have props
og respekt for korrekturlæsning. Fætter Jon og Ida Monrad Graunbøl gives
thumbs-up, for at hænge ud på kontoret i cribben og supplere snacks.

}
}

% vim: set tw=72 spell spelllang=da:
