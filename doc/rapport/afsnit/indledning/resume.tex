\resume{
Det er den generelle opfattelse, at det gyldne snit er specielt æstetisk
tiltalende, og at man i kunstmalerier finder flest interessante regioner
her.  Litteraturen kan ikke entydigt bekræfte denne påstand, og den
største undersøgelse på området talte 565 malerier.  Selvom der findes
metoder fra billedbehandling til systematisk analyse af billeders
komposition, er disse ikke blevet taget i brug med henblik på at
undersøge hypotesen.

Vi har udviklet et program, som kan afgøre, hvorvidt en digital
gengivelse af et maleri har interessante regioner i det gyldne snit, og
kørt dette på 17,364 digitaliserede malerier. En interessant region
defineres som et ensfarvet område i et billede, der er større end 0.2 \%
af billedets størrelse og optager mere end 1/4 af alle pixel i dens
begrænsende rektangel. Med to forskellige metoder vurderes, hvorvidt
interessante regioner ligger i det gyldne snit.
Vi kan ikke afvise, at snittet er specielt æstetisk tiltalende, da over
87 \% af de analyserede malerier har mindst én interessant region i det
gyldne snit.
Der ses en tendens til at kunstnere foretrækker at placere interessante
regioner i de midterste snit, da vi mindst har 1.9 \% flere regioner i
disse, end i det gyldne snit. Den øverste halvdel og kanterne af
maleriet ikke er at foretrække.
Det gennemsnitlige antal regioner i det gyldne snit, afviger med mere
end 10 \% i tidsperioder á 50 år. Med en lignende opdeling af malerier i
skole afviger antallet også med mere end 10 \%. Dette antyder at brugen
af det gyldne snit afhænger af kultur og tidsperiode.

}

% vim: set tw=72 spell spelllang=da:
