\resume{
Det er den generelle opfattelse at det gyldne snit er specielt æstetisk
tiltalende og det derfor er at finde i mange kunstmalerier. Litteraturen
på området kan ikke entydigt bekræfte denne påstand.  Selvom der findes
metoder fra billedbehandling til systematisk analyse af billeders
komposition, er disse endnu ikke blevet taget i brug for at undersøge
hypotesen om det gyldne snit. Der er derfor et reelt grundlag for at
kombinere disse metoder med det formål at kunne afgøre om et givet
billede er komponeret efter det gyldne snit. Man vil derved kunne
analysere langt større datasæt end hidtil muligt.

Vi har udviklet et program som kan trække sammenhængende regioner i
nærheden af det gyldne snit ud af en digital gengivelse af et maleri. De
sammenhængende regioner tages derefter ud og vurderes efter nogle simple
kriterier for at afgøre om de ligger i det gyldne snit. Fra analysen kan
vi få at vide hvor mange regioner vi har fundet, hvor mange der
er sorteret fra og der er regioner der ligger i det gyldne snit. Når
denne analyse automatiseres skal vi gemme resultaterne i en
database, så vi kan lave videre analyse på denne data.
}

{
\section*{Diff}
\begin{itemize}
	\item Reformulering og opstramning af afsnit om den naive
		algoritme (afsnit 3.4).\\
		Hovedpointer
		\begin{itemize}
			\item Regioner kan ligge i det gyldne snit
			\item Regioner kan være interessante
			\item Et billede opfylder det gyldne snit hvis
				det har mindst én interessant region der
				ligger i det gyldne snit.
		\end{itemize}
	\item Vi er blevet i tvivl om vores margin. Vi opererer sådan
		set med to. Et til bedre at kunne finde regioner med
		floodfill og et hvori vi vil acceptere regioner. Vi har
		endnu ikke fået regnet ordentligt igennem hvor stort det
		accepterende rektangel skal være.
\end{itemize}

}

% vim: set tw=72 spell spelllang=da:


% vim: set tw=72 spell spelllang=da:
