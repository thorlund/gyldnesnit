\resume{
Det er den generelle opfattelse, at det gyldne snit er specielt æstetisk
tiltalende, og at det derfor er at finde i mange kunstmalerier.
Litteraturen på området kan ikke entydigt bekræfte denne påstand.
Selvom der findes metoder fra billedbehandling til systematisk analyse
af billeders komposition, er disse endnu ikke blevet taget i brug med
henblik på at undersøge hypotesen om det gyldne snit. Der er derfor et
reelt grundlag for at kombinere disse metoder med det formål at kunne
afgøre, om et givet billede er komponeret efter det gyldne snit. Den
hidtil største undersøgelse talte 565 malerier.

Vi har udviklet et program, som kan afgøre, om en digital gengivelse af
et maleri, har interessante regioner liggende i det gyldne snit, og kørt
dette program på 17,364 digitale billeder af malerier.  Resultaterne fra
analysen opbevares i en database, som tillader videre arbejde med data.
Programmet kan desuden analysere andre snit i billedet end blot det
gyldne, og det bliver således muligt at skelne mellem og sammenligne
resultater fra andre snit --- f.eks, kan vi skelne mellem resultater for
det gyldne snit og for snit ved to tredjedele.

Med vores metoder til udtrækning af regioner og ved naiv vurdering af
disse, kan vi ikke afvise, at det gyldne snit er specielt æstetisk
tiltalende. Hele $91.43\%$ af de analyserede malerier viste sig at have
én eller flere regioner liggende i det gyldne snit. Dog viser
resultaterne ingen umiddelbar indikation på, at det gyldne snit
adskiller sig signifikant fra andre snit i malerier. Der vises nærmere
en tendens til at kunstnere foretrækker at placere interessante regioner
i midten, mens den øverste halvdel og kanterne af maleriet ikke er at
foretrække.

%Vi har udviklet et program, som kan trække sammenhængende regioner i
%nærheden af det gyldne snit, ud af en digital gengivelse af et maleri.
%Disse tages derefter ud og vurderes efter nogle simple kriterier for at
%afgøre, om de ligger i det gyldne snit. En automatisering af denne
%analyse er blevet sammensat, hvor en database gemmer resultaterne som
%tillader videre arbejde med data. Programmet kan desuden analysere andre
%snit i billedet end blot det gyldne, og det bliver således muligt at
%skelne mellem og sammenligne resultater fra forskellige snit --- f.eks,
%vil man kunne skelne mellem resultater for det gyldene snit og for snit
%ved en tredjedel.

}

{
\section*{Diff}
\begin{itemize}
	\item Reformulering og opstramning af afsnit om den naive
		algoritme (afsnit 3.4).\\
		Hovedpointer
		\begin{itemize}
			\item Regioner kan ligge i det gyldne snit
			\item Regioner kan være interessante
			\item Et billede opfylder det gyldne snit hvis
				det har mindst én interessant region der
				ligger i det gyldne snit.
		\end{itemize}
	\item Vi er blevet i tvivl om vores margin. Vi opererer sådan
		set med to. Et til bedre at kunne finde regioner med
		floodfill og et hvori vi vil acceptere regioner. Vi har
		endnu ikke fået regnet ordentligt igennem hvor stort det
		accepterende rektangel skal være.
\end{itemize}

}

% vim: set tw=72 spell spelllang=da:


% vim: set tw=72 spell spelllang=da:
