\resume{
Det er den generelle opfattelse at det gyldne snit er specielt æstetisk
tiltalende og det derfor er at finde i mange kunstmalerier. Litteraturen
på området kan ikke entydigt bekræfte denne påstand.  Selvom der findes
metoder fra billedbehandling til systematisk analyse af billeders
komposition, er disse endnu ikke blevet taget i brug for at undersøge
hypotesen om det gyldne snit. Der er derfor et reelt grundlag for at
kombinere disse metoder med det formål at kunne afgøre om et givet
billede er komponeret efter det gyldne snit. Man vil derved kunne
analysere langt større datasæt end hidtil muligt.

Vi har udviklet et program som kan trække sammenhængende regioner i
nærheden af det gyldne snit ud af en digital gengivelse af et maleri. De
sammenhængende regioner kan senere analyseres så man kan finde
de interessante regioner i billedet. Vi har opstillet nogle simple
kriterier en region skal opfylde før den kan karakteriseres som
interessant og liggende i det gyldne snit. Vi har udarbejdet et
databaseskema til opbevaring af maleriers metadata, herunder resultater
fra en automatiseret analyse.
}

{
\section*{Diff}
\begin{itemize}
    \item Skrevet om to udvidelse (den første er implementeret)
    \item Lidt resultater
\end{itemize}
}

% vim: set tw=72 spell spelllang=da:


% vim: set tw=72 spell spelllang=da:
