\resume{
Det er den generelle opfattelse, at det gyldne snit er specielt æstetisk
tiltalende, og at man i kunstmalerier finder flest interessante regioner
her.  Litteraturen kan ikke entydigt bekræfte denne påstand, og den
største undersøgelse på området talte 565 malerier.  Selvom der findes
metoder fra billedbehandling til systematisk analyse af billeders
komposition, er disse ikke blevet taget i brug med henblik på at
undersøge hypotesen.

Vi har udviklet et program, som kan afgøre, hvorvidt en digital
gengivelse af et maleri har interessante regioner i det gyldne snit, og
kørt dette på 17,364 digitaliserede malerier. Regioner bliver først
kontrolleret for, hvorvidt de er interessante, dernæst for om de ligger
i det gyldne snit. En interessant region defineres som et ensfarvet
område i et billede, der er større end 0.2 \% af billedets størrelse og
optager mere end 1/4 af alle pixel i dens begrænsende rektangel. Med
vores udtrækning af regioner og to forskellige metoder til at vurdere,
hvorvidt interessante regioner ligger i det gyldne snit, kan vi ikke
afvise, at snittet er specielt æstetisk tiltalende. Ved naiv vurdering
havde $\mathsf{91.43\%}$ af de analyserede malerier mindst én
interessant region i det gyldne snit.  Resultaterne viste dog ingen
indikation på, at det gyldne snit adskiller sig signifikant fra andre
snit i malerier.  Der vises en tendens til at kunstnere foretrækker at
placere interessante regioner i midten, mens den øverste halvdel og
kanterne af maleriet ikke er at foretrække.

%Resultaterne fra
%analysen opbevares i en database, som tillader videre arbejde med data.
%Programmet kan desuden analysere andre snit i billedet end blot det
%gyldne, og det bliver således muligt at skelne mellem og sammenligne
%resultater fra andre snit --- f.eks, kan vi skelne mellem resultater for
%det gyldne snit og for snit ved to tredjedele.

%Vi har udviklet et program, som kan trække sammenhængende regioner i
%nærheden af det gyldne snit, ud af en digital gengivelse af et maleri.
%Disse tages derefter ud og vurderes efter nogle simple kriterier for at
%afgøre, om de ligger i det gyldne snit. En automatisering af denne
%analyse er blevet sammensat, hvor en database gemmer resultaterne som
%tillader videre arbejde med data. Programmet kan desuden analysere andre
%snit i billedet end blot det gyldne, og det bliver således muligt at
%skelne mellem og sammenligne resultater fra forskellige snit --- f.eks,
%vil man kunne skelne mellem resultater for det gyldene snit og for snit
%ved en tredjedel.

}

{
\section*{Diff}
\begin{itemize}
    \item Skrevet om to udvidelse (den første er implementeret)
    \item Lidt resultater
\end{itemize}
}

% vim: set tw=72 spell spelllang=da:


% vim: set tw=72 spell spelllang=da:
