I de afprøvninger vi har gjort, har vi brugt 9 malerier, til at være
repræsentativ for resten af malerierne i kunstdatabase. Vi har også selv
valt de 9 malerier, ud for kriteriet at der skulle være alsidighed og at
man skal kunne vise problem stilinger visuelt ud af afprøvning på
malerierne.

Denne metode virker også godt, nå vi gerne vil ud med nogle pointer.
Men for at lave en mere statistiske signifikant undersøgelse for det
datasæt. Burte vi teste en del flere billeder, lad os sige 50 og
så bagefter teste på 100, og så se om der dannet sig et mønster. 

Vi burte også delle billederne op i grupperinger, f.eks. maleriets
oprindelse og så udvælge test malerierne efter ca samme procents vise
fordeling, så hvis kunstdatabasen består af $15 \%$ malerier fra
Italien, så skal der også være ca $15 \%$ Italienske malerier i test
malerierne.

Til sidest burte malerierne også vælges tilfældig ud inde for deres
grupperinger

Grunden til at vi ikke har gjort det på denne måde er, af 3 grunde

\begin{enumerate}
	\item Det vil stadig kun være en undersøgelse for lige det sæt
	malerier som vi arbejder på og ikke andre data materiale.
	\item Det vil krave afprøvning på mange malerier, som vi ikke har
	resurser til. 
	\item Selv med en gennemsnit tærskelværdi fra et statistisk
	signifikant undersøgelse vil der stadig være en stor rigge malerier
	hvor gennemsnit tærskelværdien ikke vil være særlige god for. 
\end{enumerate}
