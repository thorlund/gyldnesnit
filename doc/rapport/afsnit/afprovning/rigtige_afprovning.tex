I de afprøvninger, vi har foretaget, har vi brugt ni malerier til at være
repræsentative for resten af malerierne i kunstdatabasen. Vi har også selv
valgt de ni malerier ud fra kriteriet om, at der skal være alsidighed, og at
man skal kunne vise problemstilinger visuelt.

Denne metode fungerer også fint til at understrege visse pointer.
For at lave en statistisk signifikant undersøgelse for datasættet, burde
vi teste på en del flere billeder. Vi kunne f. eks. starte med 50 billeder
og derefter teste på 100, for så at observere om der danner sig et mønster.

Vi burde også dele billederne op i grupperinger, f. eks. på baggrund af maleriets
oprindelse, og så udvælge testmalerierne efter ca. samme procentvise
fordeling. Dette vil sige, at hvis kunstdatabasen består af $15 \%$ malerier fra
Italien, så skal der også være ca $15 \%$ italienske malerier i testmalerierne.

Endelig burde malerierne også vælges tilfældigt ud inde for deres
grupperinger.

Når vi har valgt ikke at gøres det således, skyldes det tre ting:

\begin{enumerate}
	\item Der vil stadig kun være \'{e}n undersøgelse for lige det sæt
	malerier som vi arbejder på og ikke andet datamateriale.
	\item Det vil kræve afprøvning på mange malerier, hvilket vi ikke har
	ressourcer til. 
	\item Selv med en gennemsnitlig tærskelværdi fra en statistisk
	signifikant undersøgelse, vil der stadig være en stor række
	malerier,
	hvor den gennemsnitslig tærskelværdi ikke vil være særlig god. 
\end{enumerate}
