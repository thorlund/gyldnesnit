I de afprøvninger vi har gjort, har vi brugt ni malerier, til at være
repræsentativ for resten af malerierne i kunstdatabasen. Vi har også selv
valgt de ni malerier, udfra kriteriet om, at der skulle være alsidighed og at
man skal kunne vise problemstilinger visuelt.

Denne metode fungere også fint, til at understrege visse pointer.
For at lave en statistisk signifikant undersøgelse for datasættet, burde
vi test en del flere billeder. Vi kuunne f.eks. starte med 50 billeder
og derefter test på 100, og så observere om der danner sig et mønster.
Men for at lave en mere statistiske signifikant undersøgelse for det
datasæt. Burte vi teste en del flere billeder, lad os sige 50 og
så bagefter teste på 100, og så se om der dannet sig et mønster. 

Vi burde også dele billederne op i grupperinger, f.eks. på baggrund af maleriets
oprindelse og så udvælge testmalerierne efter ca. samme procentvise
fordeling, så hvis kunstdatabasen består af $15 \%$ malerier fra
Italien, så skal der også være ca $15 \%$ italienske malerier i testmalerierne.

Til sidest burte malerierne også vælges tilfældig ud indefor deres
grupperinger

Årsagen til at vi har valgt ikke at gøres det således, skyldes tre ting.

\begin{enumerate}
	\item Det vil stadig kun være en undersøgelse for lige det sæt
	malerier som vi arbejder på og ikke andet data materiale.
	\item Det vil kræve afprøvning på mange malerier, som vi ikke har
	resurser til. 
	\item Selv med en gennemsnitslig tærskelværdi fra en statistisk
	signifikant undersøgelse vil der stadig være en stor række malerier
	hvor gennemsnitslig tærskelværdien ikke vil være særlige god. 
\end{enumerate}
