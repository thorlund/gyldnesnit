%% Bemærk:
%%          Resten af rapporten følger en stil hvor indledninger skrives
%%          med \sffamlily-typen. Denne stil bør også følges her.
%%
{\sffamily
I vores program er der fire forskellige tærskelværdier som påvirker
hvordan maleriet analyseres, nå regioner udtrækning af maleriet.
Margionbredde, kandtdetektion, flootfill og sløring. I dette sektion vil
vi finde værdier for de fire tærskelværdier, samt forklaring hvorfor vi
har valt lige de tal. Malerierne som er brugt i afprøvning er specielt
udvalgt for deres illustrative aspekt. 
}

\subsection{Størrelsen på margin}
Vi regner marginbredde ud fra en procent stats, som vi kalder $\Delta$, af
billedets $B$ og $H$. I defination \ref{margin_min} definere vi den
minimale størrelse på margin til $2.1 \%$. Så $\Delta$ skal være støre en
$2.1 \%$. 

I defination \ref{margin_max}
Den minimale forskel på to snit vi foretager os, er forskellen mellem
det gyldne snit og $\frac{2}{3}$. I definition \ref{margin_max} er den
maksimale bredde margin kan have $2.43\%$. For at marginen ikke
overlapper. Så $\Delta$ skal være mindre en $2.43\%$. 

Det vil sige at $\Delta \in [2.1, 2.43]$. Vi har valt at
sætte $\Delta = 2.4$, da vi derved kan tage højde for uforresette
usikkerhed.

\subsection{Afvigelsen af farver i kantdetektion}
Vi bruger kantdetektion til, er af finde en kanter rundt om de regioner
som vi mener er interessante, og vi vil gerne undgå kanter inde i
regionerne. Begge mål kan ikke altid opfyldes, men vi kan komme så tæt
på et krompromis mellem en perfekt kant rundt om region og ingen kanter
inde i en region som muliget. Dette gøres ved at ændre to tærskelværdier
i kantdetektionen. Vi har valgt at dele billederne som vi observerer op
i ni kategorier, se tabel \ref{thressholdsTabelKant}. Kategoriernr er en
grov opdeling af billederne efter detaljer og farveintensitet, som
bruges til at give en bedre indblik i billedets opbygning.

\subsubsection{Sammenligninger}
Vi har undersøgt på ni malerier og har fundet de tærskelværdier, vi mener
passer bedst på malerierne. Vi vil illustrere, hvordan vi har fastsat
tærskelværdierne, på maleriet \ref{kDetalier}.

Maleriet er malet med mange farver og med mange detaljer. Vi ser først
på tærskelværdierne $(0,0),(100,100).....(600,600),(700,700)$. I
figurene \ref{allesammen1} og \ref{allesammen2} kan man se hvordan de
forskellige tærskelværdier påvirker hvad kantdetektoren finde af kanter i
billedet, læg mærke til at nu højre tærskelværdierne er, nu færre kanter
bliver der også fundet, og det blive svære og svære at genkende maleriet i kanterne. 

Vi finde det billedet i \ref{allesammen1} og \ref{allesammen2}, med
højest tærskelværdi, som ikke har mistet for mange kanterne rundt om
regionerne endnu. Vi vurderer at billedet \ref{300-300} passer bedst på
den beskrivelse, da regionerne som vingerne, hovet og kroben stadig er
omkranset af en kant. Så tærskelværdien $(300-300)$ er den værdi som for
vores kantdetektor til at virke bedst, ind til vidre.

Ved at sætte en af tærskelværdierne lidt op af gangen, kan vi finde en
bedre tærskelværdi for kantdetektoren. Vi opstille igen en rigge
billeder med forskellige tærskelværdier og vurderer billederne ud fra
samme kriterier. Vi vurderer her at billedet \ref{300-750} passe bedst,
da vingerne, hoved og kroben stadig har mange af deres kanter. Derfor er
(300-750) den tærskelværdi som vi vurdere til at være bedst til at
opfylder krompromis meller kanter rundet om regioner og få kanter inde i
regionerne.

Det brugte maleri er ikke særlig repræsentativt for hele vores
billeddatabase. Så vi har også brugt metoden på otte andre billeder. Et
lille udsnit af de otte billeder kan ses i figur \ref{en}, \ref{to} og
\ref{tre}, hvor man skal ligge mærke til at det kandetekteret billedet
liner et omrids af det originale billet regioner. 

\clearpage
\begin{figure}[p]
    \centering
    \subfloat[Det original maleri. Navn: The Archangel Michae. År: Ca 1490 Af: Abadia, Juan de la]{
        \includegraphics[angle=0,width=0.45\textwidth]{afsnit/afprovning/billeder/thressholds/krafitige_farver/krafite_detalier/kDetalier.jpg}
        \label{kDetalier}}
    \subfloat[(100,100)]{
        \includegraphics[angle=0,width=0.45\textwidth]{afsnit/afprovning/billeder/thressholds/krafitige_farver/krafite_detalier/1_iteration/100-100.png}
        \label{100-100}}\\
    \subfloat[(200,200)]{
        \includegraphics[angle=0,width=0.45\textwidth]{afsnit/afprovning/billeder/thressholds/krafitige_farver/krafite_detalier/1_iteration/200-200.png}
        \label{200-200}}
    \subfloat[(300,300)]{
        \includegraphics[angle=0,width=0.45\textwidth]{afsnit/afprovning/billeder/thressholds/krafitige_farver/krafite_detalier/1_iteration/300-300.png}
        \label{300-300}}\hspace{1em}
    \caption{Kantdetektion på maleriet \ref{kDetalier} som har mange detaljer og kraftige farver, med tærskelværdierne fra (100-100) til (300-300)}
	\label{allesammen1}
\end{figure}

\clearpage

\begin{figure}[!h]
	\centering
    \subfloat[(400,400)]{
        \includegraphics[angle=0,width=0.45\textwidth]{afsnit/afprovning/billeder/thressholds/krafitige_farver/krafite_detalier/1_iteration/400-400.png}
        \label{400-400}}    
	\subfloat[(500,500)]{
        \includegraphics[angle=0,width=0.45\textwidth]{afsnit/afprovning/billeder/thressholds/krafitige_farver/krafite_detalier/1_iteration/500-500.png}
        \label{500-500}}\\
    \subfloat[(600,600)]{
        \includegraphics[angle=0,width=0.45\textwidth]{afsnit/afprovning/billeder/thressholds/krafitige_farver/krafite_detalier/1_iteration/600-600.png}
        \label{600-600}}
    \subfloat[(700,700)]{
        \includegraphics[angle=0,width=0.45\textwidth]{afsnit/afprovning/billeder/thressholds/krafitige_farver/krafite_detalier/1_iteration/700-700.png}
        \label{700-700}}
    \caption[]{Kantdetektion på maleriet \ref{kDetalier} som har mange detaljer og kraftige farver, med tærskelværdierne fra (400-400) til (700-700)}
     \label{allesammen2}
\end{figure}

\begin{figure}[!h]
    \centering
    \subfloat[(300,700)]{
        \includegraphics[angle=0,width=0.45\textwidth]{afsnit/afprovning/billeder/thressholds/krafitige_farver/krafite_detalier/2_iteration/300-700.png}
        \label{300-700}}\hspace{1em}
    \subfloat[(300,750)]{
        \includegraphics[angle=0,width=0.45\textwidth]{afsnit/afprovning/billeder/thressholds/krafitige_farver/krafite_detalier/2_iteration/300-750.png}
        \label{300-750}}\\
    \subfloat[(300,800)]{
        \includegraphics[angle=0,width=0.45\textwidth]{afsnit/afprovning/billeder/thressholds/krafitige_farver/krafite_detalier/2_iteration/300-800.png}
        \label{300-800}}\hspace{1em}
    \subfloat[(300,850)]{
        \includegraphics[angle=0,width=0.45\textwidth]{afsnit/afprovning/billeder/thressholds/krafitige_farver/krafite_detalier/2_iteration/300-850.png}
        \label{300-850}}
        \caption[]{Kantdetektion hvor de fire billeder som er interessante er taget med}
     \label{allesammen3}
\end{figure}
 
\begin{figure}[!h]
    \centering
    \subfloat[Det original maleri. Navn: The Lamentation over St Francis. År: 1440. Af: Angelico, Fra. ]{
        \includegraphics[angle=0,width=0.45\textwidth]{afsnit/afprovning/billeder/thressholds/svage_farver/svage_detalier/sDetalier.jpg}
        \label{Orginal1}}
    \subfloat[(100,250)]{
        \includegraphics[angle=0,width=0.45\textwidth]{afsnit/afprovning/billeder/thressholds/svage_farver/svage_detalier/2_iteration/100-250.png}
        \label{100-250}}\hspace{1em}
        \caption[]{Kantdetektion på et billedet med svage farver og få detaljer, hvor tærskelværdien (100,250) er den bedste}
     \label{en}
\end{figure}

\begin{figure}[!h]
    \centering
    \subfloat[Det Original maleri. Navn:The Ninth Wave. År:1850. Af:Aivazovsky, Ivan Konstantinovich.]{
        \includegraphics[angle=0,width=0.85\textwidth]{afsnit/afprovning/billeder/thressholds/medium_farver/svage_detalier/sDetalier1.jpg}
        \label{Orginal2}}\\
    \subfloat[(100,240)]{
        \includegraphics[angle=0,width=0.85\textwidth]{afsnit/afprovning/billeder/thressholds/medium_farver/svage_detalier/2_iteration/100-240.png}
        \label{100-240}}
        \caption[]{Kantdetektion på et billede med medium farver og få detaljer, hvor tærskelværdien (100,240) er den bedste}
     \label{to}
\end{figure}

\begin{figure}[!h]
    \centering
    \subfloat[Det original maleri. Name: The last supper. År: 1498. Af: Leonardo da Vinci]{
        \includegraphics[angle=0,width=0.85\textwidth]{afsnit/afprovning/billeder/thressholds/medium_farver/medium_detalier/mDetalier1.jpg}
        \label{Orginal3}}\\
    \subfloat[(200,460)]{
        \includegraphics[angle=0,width=0.85\textwidth]{afsnit/afprovning/billeder/thressholds/medium_farver/medium_detalier/2_iteration/200-460.png}
        \label{200-460}}

        \caption[]{Kantdetektion på et billedet med medium farver og medium detaljer, hvor tærskelværdien [200,460] er den bedste}
     \label{tre}
\end{figure}

\begin{table}[!h]
    \centering
    \begin{tabular}{| l | l | l | l |} \hline
        & Svage farver 	& Medium farver & Kraftige farver \\ \hline
        Få detaljer 		& (100,250)		& (100,240)		& (200,320)\\ \hline
        Medium detaljer 	& (100,280)		& (200,460)		& (200,380)\\ \hline
        Mange detaljer		& (200,400)		& (200,380)		& (300,750)\\ \hline
    \end{tabular}
    \caption{Tabel over kantdetektionstærskelværdier for ni malerier}
    \label{thressholdsTabelKant}
\end{table}

Det ses i tabel \ref{thressholdsTabelKant} går tærskelværdierne
fra $(100,240)$ til $(300,750)$, så vi tager en gennemsnit af værdierne
og få at de to tærskelværdier skal være $(177,384)$. 

Vi har dog i vores forsøg regnet med værdierne $(78,194)$, da vores
indledende afprøvninger afviger en smule fra den denne afprøvning.

\subsection{Afvigelsen af farver i floodfill}
Floodfill har to tærskelværdier $lo$ og $up$, som betegner hvor mange
pixelværdier en nabopixels farver må afvige. En
fyldestgørelses beskrivelse af floodfill findes i afsnit
\ref{section_opdeling}. 

Vi har tænkt os at finde en fælles tærskelværdi til brug i vores
program. Vi observere hvordan floodfill virker med forskellige
tærskelværdier og finder dem, som passer bedst til maleriet. Resultatet
for observationen kan ses i tabel \ref{thressholdsTabelFF}, hvor de
sammen ni kategorier er brugt og afprøvet på de samme ni malerier. Et af
de malerier hvor den optimale tærskelværdi er fundet kan ses i figur
\ref{Floodfillbilledet}, lig mærke til hvordan træet, samt sne
baggrunden, er fyldt næsten helt ud, uden at nogle af menneskerne er
farvet over.

\begin{figure}[!h]
    \centering
    \subfloat[Det Originale maleri. Navn: Winter Landscape. År: Ukent. Af:Avercamp, Hendrick.]{
        \includegraphics[angle=0,width=0.9\textwidth]{afsnit/afprovning/billeder/thressholds/svage_farver/kraftige_detalier/kDetalier.jpg}
        \label{Orginal4}}\\
    \subfloat[8,8]{
        \includegraphics[angle=0,width=0.9\textwidth]{afsnit/afprovning/billeder/thressholds/svage_farver/kraftige_detalier/floodfill/8-8.png}
        \label{8-8}}
    \caption[]{tærskelværdierne på et billedet med svage farver og kraftige detaljer hvor tærskelværdien (8,8) passer bedst}
    \label{Floodfillbilledet}
\end{figure}

\begin{table}[!h]
    \centering
    \begin{tabular}{| l | l | l | l |} \hline
        & Svage farver 		& Medium farver & kraftige farver \\ \hline
        Få detalier 		& \textbf{(2,2)}	& (3,3)			& (4,4)\\ \hline
        Medium detalier 	& \textbf{(2,2)}	& \textbf{(5,5)}& \textbf{(2,2)}\\ \hline
        Mange detalier		& (8,8)				& (4,4)			& (7,7)\\ \hline
    \end{tabular}
    \caption{Tabel over floodfill tærskelværdier på ni malerier}
    \label{thressholdsTabelFF}
\end{table}

\note{hjælp med formulering, bliver ved med at lave det samme krab}
Det ses i tabel \ref{thressholdsTabelFF} at nogle af værdierne er med
fed, grundet til det, er at, vadierne, er de beste, vi kan finde for
billedet, men at de vadier stadig ikke giver noget som er særlige
brugbart. 

Værdigerne i tabel fluktuere også en del, så vi tager
gennemsnittet af værdierne uden fed og får en tærskelværdierne på (5,5).
