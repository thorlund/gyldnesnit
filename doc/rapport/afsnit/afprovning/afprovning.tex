%% Bemærk:
%%          Resten af rapporten følger en stil hvor indledninger skrives
%%          med \sffamlily-typen. Denne stil bør også følges her.
%%
{
{\sffamily
I dette Kapitel vil vi afprøve vores 2 metoder test billeder og malerier
fra vores kunst database og se hvordan de virker. Udover det vil vi
finde tærskelværdier til kandetection, floodfill og margin, ud fra
observationer gjort under afprøvning. Vi vil også afprøve den generalle
region detektor og kommer ind på dens fordele og ulemper, undervejs vil
vi lave nogle konklutioner på de 2 metoder, samen en sammenligning i
slutningen af afsnittet. }

\section{Tærskelværdier\label{terskelverdi}}
%% Bemærk:
%%          Resten af rapporten følger en stil hvor indledninger skrives
%%          med \sffamlily-typen. Denne stil bør også følges her.
%%
{\sffamily
I vores program er der 4 forskellige tærskelværdier som påvirker hvordan
regions detektoren analysere billedet. Alle 4 tærskelværdier er blevet
introduceret i deres respektive afsnit, men ingen konkrate tal er
opgivet. I dette afsnit vil vi opgive de tal, samt en forklaring på
hvorfor vi har valt lige de tal. Malerierne som er brugt i afprøvning er
spicældt udvalgt for deres illustrative aspekt. 
}

\subsection{Marginens brede}
Vi regner marginens brede ud fra en procent stats $\psi$ af billedets
$B$ og $H$. I afsnittet \ref{section_opdeling} kom vi frem til en
usikkerhed på $2.1 \%$. Så $\psi$ skal være støre en $2.1 \%$. 

Den minimale forskel på 2 snit vi foretager os, er forskellen mellem det
gyldne snit og $\frac{2}{3}$, margin brade udregnet i section
\ref{section_opdeling} til maksimal at være $2.43\%$. For at marginen
ikke krudser. 

Det vil sige at $\psi \in [2.1, 2.43]$. Vi har valt at
sætte $\psi = 2.4$, da vi derved kan tage højde for uforresette
usikkerhed.

\subsection{Afvigelsen af farver i kandtdetection}
Vi bruger kandtdetection til, er af finde en kanter rundt om de regioner
som vi mener er interessante, og undgå de kanter som ligger inde i
regioner. Begge de 2 mål kan ikke altid opfyldes, men vi kan komme så
tæt på et krompromi mellem en perfekt kant rund om region og ingen
kanter inde i region som mulige. Dette gøres ved at ændre 2
tærskelværdier i kantdetectionen. Vi har valt at dele billederne som vi
observere op i 9 kategorier, se tabel \ref{thressholdsTabelKant},
Kategorier er en grove opdeling af billederne efter detaljer og farve
intensitet, som bruges til at give en bedre indblik på billedets
opbygning. 

\subsubsection{Sammenligninger}
Vi har set på 9 malerier og har fundet de tærskelværdier som vi mener
passer bæst på malerierne. Vi vil illustrerer hvordan vi har fastsat
tærskelværdierne på maleriet \ref{kDetalier}.

Maleriet er malet med mange farver og med masser af detaljer. Vi ser
først på tærskelværdierne $(0,0),(100,100).....(600,600),(700,700)$ se
figur \ref{allesammen1} og \ref{allesammen2}, og finde det interval hvor
malerriet ikke har mistet nogle af kanterne rundt om regionerne endnu,
men vil det, i næste interval. I illustration vurdere vi det til
billedet \ref{300-300}, da billedet \ref{400-400} har mistet for mange
af de kanter, som vi gerne vil beholde.

Ved at sætte en af tærskelværdierne op lidt af gangen, kan vi igen få en
række billeder at vælge imellem. Se sammenligningen i figur
\ref{allesammen3}. Man kan se at det begynder at være svær at skelne
figurene i \ref{300-850} og der er lidt for mange kanter i
\ref{300-700}.

Vi har valgt at fastsætte tærskeværdigerne til $(300,750)$. Det maleri
vi lige har brugt er ikke særlige repræsentativ for helle vores maleri
database. så vi har taget 8 andre billeder og brugt samme metode på dem og fastsat en
middel tærskelværdi. Vi viser her en lille udsnit af dem, se figur
\ref{en}, \ref{to} og \ref{tre}.
\clearpage
\begin{figure}[p]
    \centering
    \subfloat[(100,100)]{
        \includegraphics[angle=0,width=0.45\textwidth]{afsnit/afprovning/billeder/thressholds/krafitige_farver/krafite_detalier/1_iteration/100-100.png}
        \label{100-100}}\hspace{1em}
    \subfloat[(200,200)]{
        \includegraphics[angle=0,width=0.45\textwidth]{afsnit/afprovning/billeder/thressholds/krafitige_farver/krafite_detalier/1_iteration/200-200.png}
        \label{200-200}}\\
    \subfloat[(300,300)]{
        \includegraphics[angle=0,width=0.45\textwidth]{afsnit/afprovning/billeder/thressholds/krafitige_farver/krafite_detalier/1_iteration/300-300.png}
        \label{300-300}}\hspace{1em}
    \subfloat[(400,400)]{
        \includegraphics[angle=0,width=0.45\textwidth]{afsnit/afprovning/billeder/thressholds/krafitige_farver/krafite_detalier/1_iteration/400-400.png}
        \label{400-400}}
    \label{allesammen1}
    \caption{Edgedetection på maleriet \ref{kDetalier} som har mange detaliger og kraftige farver, med tærskelværdierne fra (100-100) til (400-400)}
\end{figure}

\clearpage

\begin{figure}[!h]
	\centering
	\subfloat[(500,500)]{
        \includegraphics[angle=0,width=0.45\textwidth]{afsnit/afprovning/billeder/thressholds/krafitige_farver/krafite_detalier/1_iteration/500-500.png}
        \label{500-500}}\hspace{1em}
    \subfloat[(600,600)]{
        \includegraphics[angle=0,width=0.45\textwidth]{afsnit/afprovning/billeder/thressholds/krafitige_farver/krafite_detalier/1_iteration/600-600.png}
        \label{600-600}}\\
    \subfloat[(700,700)]{
        \includegraphics[angle=0,width=0.45\textwidth]{afsnit/afprovning/billeder/thressholds/krafitige_farver/krafite_detalier/1_iteration/700-700.png}
        \label{700-700}}\hspace{1em}
    \subfloat[Original. Navn: The Archangel Michae. År: Ca 1490 Af: Abadia, Juan de la]{
        \includegraphics[angle=0,width=0.45\textwidth]{afsnit/afprovning/billeder/thressholds/krafitige_farver/krafite_detalier/kDetalier.jpg}
        \label{kDetalier}}
    \caption[]{Edgedetection på maleriet \ref{kDetalier} som har mange detaliger og kraftige farver, med tærskelværdierne fra (500-500) til (700-700)}
     \label{allesammen2}
\end{figure}

\begin{figure}[!h]
    \centering
    \subfloat[(300,700)]{
        \includegraphics[angle=0,width=0.45\textwidth]{afsnit/afprovning/billeder/thressholds/krafitige_farver/krafite_detalier/2_iteration/300-700.png}
        \label{300-700}}\hspace{1em}
    \subfloat[(300,750)]{
        \includegraphics[angle=0,width=0.45\textwidth]{afsnit/afprovning/billeder/thressholds/krafitige_farver/krafite_detalier/2_iteration/300-750.png}
        \label{300-750}}\\
    \subfloat[(300,800)]{
        \includegraphics[angle=0,width=0.45\textwidth]{afsnit/afprovning/billeder/thressholds/krafitige_farver/krafite_detalier/2_iteration/300-800.png}
        \label{300-800}}\hspace{1em}
    \subfloat[(300,850)]{
        \includegraphics[angle=0,width=0.45\textwidth]{afsnit/afprovning/billeder/thressholds/krafitige_farver/krafite_detalier/2_iteration/300-850.png}
        \label{300-850}}
        \caption[]{Edgedetection hvor de 4 billeder som er intrasante taget med}
     \label{allesammen3}
\end{figure}
 
\begin{figure}[!h]
    \centering
    \subfloat[(100,250)]{
        \includegraphics[angle=0,width=0.45\textwidth]{afsnit/afprovning/billeder/thressholds/svage_farver/svage_detalier/2_iteration/100-250.png}
        \label{100-250}}\hspace{1em}
    \subfloat[Orginal. Navn: The Lamentation over St Francis. År: 1440. Af: Angelico, Fra. ]{
        \includegraphics[angle=0,width=0.45\textwidth]{afsnit/afprovning/billeder/thressholds/svage_farver/svage_detalier/sDetalier.jpg}
        \label{Orginal1}}
        \caption[]{Edgedetection på et billedet med svage farver og få detalier, hvor tærskenværdigern [100,250] er den beste}
     \label{en}
\end{figure}

\begin{figure}[!h]
    \centering
    \subfloat[(100,240)]{
        \includegraphics[angle=0,width=0.85\textwidth]{afsnit/afprovning/billeder/thressholds/medium_farver/svage_detalier/2_iteration/100-240.png}
        \label{100-240}}\\
    \subfloat[Orginal. Navn:The Ninth Wave. År:1850. Af:Aivazovsky, Ivan Konstantinovich.]{
        \includegraphics[angle=0,width=0.85\textwidth]{afsnit/afprovning/billeder/thressholds/medium_farver/svage_detalier/sDetalier1.jpg}
        \label{Orginal2}}
        \caption[]{Edgedetection på et billedet med medium farver og få detalier, hvor tærskenværdigern [100,240] er den beste}
     \label{to}
\end{figure}

\begin{figure}[!h]
    \centering
    \subfloat[(200,460)]{
        \includegraphics[angle=0,width=0.85\textwidth]{afsnit/afprovning/billeder/thressholds/medium_farver/medium_detalier/2_iteration/200-460.png}
        \label{200-460}}\\
    \subfloat[Orginal. Name: The last supper. År: 1498. Af: Leonardo da Vinci]{
        \includegraphics[angle=0,width=0.85\textwidth]{afsnit/afprovning/billeder/thressholds/medium_farver/medium_detalier/mDetalier1.jpg}
        \label{Orginal3}}
        \caption[]{Edgedetection på et billedet med medium farver og medium detalier, hvor tærskenværdigern [200,460] er den beste}
     \label{tre}
\end{figure}

\begin{table}[!h]
    \centering
    \begin{tabular}{| l | l | l | l |} \hline
        & Svage farver 	& Medium farver & Kraftige farver \\ \hline
        Få detaljer 		& (100,250)		& (100,240)		& (200,320)\\ \hline
        Medium detaljer 	& (100,280)		& (200,460)		& (200,380)\\ \hline
        Mange detaljer		& (200,400)		& (200,380)		& (300,750)\\ \hline
    \end{tabular}
    \caption{Tabel over kantdetektionstærskelværdier for ni malerier}
    \label{thressholdsTabelKant}
\end{table}

Som man kan se af tabel \ref{thressholdsTabelKant} gå tærskelværdierne
fra $(100,240)$ til $(300,750)$, så vi tager en gennemsnit af værdierne
og få at de to tærskelværdier skal være (177 og 384). 

Vi har dog i vores forsøg regnet med værdierne 78 og 194, da vores
indledende afprøvninger afviger en smugle fra den denne afprøvning.

\subsection{Afvigelsen af farver i floodfill}
Floodfill har 2 tærskelværdier $lo$ og $up$, som betegner hvor mange
pixel værdier en nabo pixel farver må variere, ned og op. En
fyldestgørelses beskrivelse af floodfill findes i afsnit
\ref{section_opdeling}. 

Vi har tænkt os at finde en fældes tærskelværdi til brug i vores
program. Måde vi gør det på at ved at observere hvordan floodfill virker
med forskellige tærskelværdier og finde de tærskelværdier som passer
bæst til maleriet. Resultatet for observationen kan ses i tabel
\ref{thressholdsTabelFF}, hvor de 9 sammen kategorier er vist og
afprøvet på de samme 9 malerier. Et af de malerierne kan se i
figur \ref{Floodfillbilledet}, hvor man kan se hvad vi har valt til at
være de optimale værdier.

\begin{figure}[!h]
    \centering
    \subfloat[8,8]{
        \includegraphics[angle=0,width=0.9\textwidth]{afsnit/afprovning/billeder/thressholds/svage_farver/kraftige_detalier/floodfill/8-8.png}
        \label{8-8}}\\
    \subfloat[Orgina. Navn: Winter Landscape. År: Ukent. Af:Avercamp, Hendrick.]{
        \includegraphics[angle=0,width=0.9\textwidth]{afsnit/afprovning/billeder/thressholds/svage_farver/kraftige_detalier/kDetalier.jpg}
        \label{Orginal4}}
    \caption[]{tærskelværdierne på et billedet med svage farver og kraftige detaljer hvor tærskelværdien [8,8] passer best}
    \label{Floodfillbilledet}
\end{figure}

\begin{table}[!h]
    \centering
    \begin{tabular}{| l | l | l | l |} \hline
        & Svage farver 		& Medium farver & kraftige farver \\ \hline
        Få detalier 		& \textbf{[2,2]}	& [3,3]			& [4,4]\\ \hline
        Medium detalier 	& \textbf{[2,2]}	& \textbf{[5,5]}& \textbf{[2,2]}\\ \hline
        Mange detalier		& [8,8]				& [4,4]			& [7,7]\\ \hline
    \end{tabular}
    \caption{Tabel over floodfill tærskelværdier på 9 malerier}
    \label{thressholdsTabelFF}
\end{table}

Som man kan se af tabel \ref{thressholdsTabelFF}, er nogle af vadierne
med fed, begrundelsen for det, er at vadierne, er de beste, vi kan finde
for billedet, men at de vadier stadig ikke giver noget som er særlige
brugbart. værdigerne i tabel fluktuere også en del, så vi tager
gennemsnittet og får tærskelværdierne til at være (5,5).

\clearpage

\section{Sløring}
I forbindelse med sløring vil vi gerne test tre forskellige størrelser på foldningsmatricen der bruges. Vi vil teste
på de sammen ni malerier som i afsnit \ref{terskelverdi}.

\begin{table}[!h]
    \centering
    \begin{tabular}{| l | l | l | l |} \hline
                            & Svage farver	& Medium farver	& Kraftige farver 	\\ \hline
        Få detaljer 		& $(7 \times 7)$	& $(7 \times 7)$	& $(7 \times 7)$		\\ \hline
        Medium detaljer 	& $(5 \times 5)$	& $(7 \times 7)$	& $(7 \times 7)$		\\ \hline
        Mange detaljer		& $(5 \times 5)$	& $(5 \times 5)$	& $(7 \times 7)$		\\ \hline
    \end{tabular}
    \caption{Tabel over hvilke sløringer der er bedst til vert af sine billeder}
    \label{sloringTabel}
\end{table}

Som man kan se af tabel \ref{sloringTabel} ligger tærskelværdierne på
$(5 \times 5)$ og $(7 \times 7)$, med en svær overvægt af værdierne hen
mod $(7 \times 7)$. I vores program har vi brugt $(3 \times 3)$ da denne
værdi, passes bedst til de billeder vi så på først. \note{Kunne vi ikke bare give 1 af eksemplet fra slørings afsnittet}

\clearpage

\section{Regions detektor\label{region_detektor}}
%% Bemærk:
%%          Resten af rapporten følger en stil hvor indledninger skrives
%%          med \sffamlily-typen. Denne stil bør også følges her.
%%

{\sffamily I dette afsnit vil vi afprøve den generelle metode for udtrækning af
regioner. Selve metodens fremgangsmåde står beskrevet i afsnit
\ref{section_udtraek}. Tærskelværdierne er sat til de fundne
tærskelværdier i afsnit \ref{terskelverdi}. Første del af afsnittet vil
omhandle afprøvning af metoden på testbilleder, og anden del afprøvning
ad metoden på udvalgte billeder fra databasen.}

\subsection{Afprøvning på testbilleder}
I dette følgende afprøves metoden på billeder konstrueret med en hvid
baggrund og sorte regioner. Snittet, som der vil blive kigget på, vil
blive tegnet med rødt på billederne. Snittes margin vil blive tegnet med blåt.

I billedet \ref{GRD_test1} er der fem regioner. Tre af dem bliver
fundet, da de ligger i snittet. De to sidste regioner som ikke er fundet, er
stadig sorte. Som man kan se, bliver både den, der krydser snittet og den,
der tangere snittet, taget med. 

\begin{figure}[!h]
    \centering
    	\subfloat[Det originale maleri.]{
	       	\includegraphics[angle=0,width=0.7\textwidth]{afsnit/afprovning/billeder/region_selector/blob_section.png}
	       	\label{GRD_test1_original}}\hspace{1em}
		\subfloat[Her ses fire regioner samt en baggrund. To af figurerne og baggrunden er blevet fundet af metoden.]{
        	\includegraphics[angle=0,width=0.7\textwidth]{afsnit/afprovning/billeder/region_selector/blob_region_section.png}
        	\label{GRD_test1}}\hspace{1em}
        \caption[]{}
     \label{GRD_test1_sammen}
\end{figure}

I billedet \ref{GRD_test2} er der tre regioner, som
alle bliver fundet; baggrunden, den lille region som ligger indenfor
marginen, og den store region som kun har en lille del af sin masse
i snittet. 

\begin{figure}[!h]
    \centering
    	\subfloat[Det originale maleri.]{
	       	\includegraphics[angle=0,width=0.7\textwidth]{afsnit/afprovning/billeder/region_selector/lille_tvers.png}
	       	\label{GRD_test2_original}}\hspace{1em}
		\subfloat[På billedet er der fundet en stor og en mindre region.]{
        	\includegraphics[angle=0,width=0.7\textwidth]{afsnit/afprovning/billeder/region_selector/blob2_region_section.png}
        	\label{GRD_test2}}\hspace{1em}
        \caption[]{}
     \label{GRD_test2_sammen}
\end{figure}

I billedet \ref{GRD_test3} er der en horisont, som ligger oven på
snittet. Begge sider af horisontlinjen bliver udtrukket som en region.

\begin{figure}[!h]
    \centering
    	\subfloat[Det originale maleri.]{
	       	\includegraphics[angle=0,width=0.7\textwidth]{afsnit/afprovning/billeder/region_selector/hoisont.png}
	       	\label{GRD_test3_original}}\hspace{1em}
		\subfloat[På billedet er der fundet to store regioner.]{
        	\includegraphics[angle=0,width=0.7\textwidth]{afsnit/afprovning/billeder/region_selector/hoisont_region_section.png}
        	\label{GRD_test3}}\hspace{1em}
        \caption[]{}
     \label{GRD_test3_sammen}
\end{figure}

I de tre figurer \ref{GRD_test1_sammen}, \ref{GRD_test2_sammen} og
\ref{GRD_test3_sammen} kan ses før og efter metoden er blevet anvendt.
Metoden opfører sig efter de standarder, som vi fremsatte i afsnit \ref{section_naiv}, og
virker efter vores forventninger.
\clearpage

\subsection{Afprøvning på malerier}
Vi vil afprøve metoden til udtrækning af regioner på seks udvalgte
malerier. Malerierne skal demonstrere, hvordan metoden på nogle malerier
fungerer godt, mens den fungerer dårligt på andre.

\begin{figure}[!h]
    \centering
		\subfloat[Maleri med kraftige farver og få detaljer. Navn: Scenes from the Story of Joseph: The Arrest of His Brethren. År: 1515-16. Af: Bacchiacca.]{
        	\includegraphics[angle=270,width=1.0\textwidth]{afsnit/afprovning/billeder/thressholds/krafitige_farver/svage_detalier/floodfill/4-4.png}
        	\label{GRD_virker1}}\hspace{1em}
		\subfloat[Maleri med middel kraftige farver og medium antal detaljer.]{
        	\includegraphics[angle=0,width=1.0\textwidth]{afsnit/afprovning/billeder/4-4.png}
        	\label{GRD_virker2}}\hspace{1em}
        \caption[]{To malerier, hvor den generelle metode virker efter vores ønsker.}
     \label{generelde_region_detektor_virker}
\end{figure}

I figuren \ref{generelde_region_detektor_virker} er der to malerier, hvor
vores regionsudtrækning virker rigtig godt. 

I det første maleri \ref{GRD_virker1} finder metoden syv store regioner
samt en del små. Den skelner mellem de forskellige paneler om kaminen,
og hvert stykke tøj på personen i maleriet opfattes som en særskilt
region. De små regioner er samlet, og forstyrre ikke de store regioner.
Man kunne måske have ønsket sig, at metoden ville fylde mere af
personens kappe ud, men bortset fra det er figuren et godt eksempel på,
hvordan det ser ud, når den generelle metode fungerer.

I maleriet \ref{GRD_virker2} finder vi mange af de samme positive ting:
drengen i midten af maleriet er helt udfyldt, en sko, drengens
badebukser og et håndklæde er også fundet. Det er dog vigtigt at lægge
mærke
til, at de andre personer i vandet, flyder i et med baggrunden. Dette
ville normalt være uheldigt, men eftersom metoden kun ser efter regioner
i snittet -repræsenteret af den røde linje- gør det ikke noget i dette
tilfælde.

\begin{figure}[!h]
    \centering
	\subfloat[Maleri med kraftige farver og mange detaljer, hvor de tærskelværdierne fundet i afsnit \ref{terskelverdi} er brugt.]{
   	 	\includegraphics[angle=270,width=0.90\textwidth]{afsnit/afprovning/billeder/thressholds/krafitige_farver/krafite_detalier/floodfill/4-4.png}
	    \label{GRD_virker_nesten1}}\hspace{1em}
    \subfloat[Det samme maleri, hvor Tærskelværdier er valt specifikt for det her maleri]{
        \includegraphics[angle=270,width=0.95\textwidth]{afsnit/afprovning/billeder/thressholds/krafitige_farver/krafite_detalier/s7_e200_f5.png}
        \label{GRD_virker_nesten1_super}}\\
     \caption[]{Et malerier, hvor den generelle metode ikke helt virker efter vores ønske, men med nye tærskelværider vil virker meget bedre}
     \label{generelde_region_detektor_virker_nesten1}
\end{figure}

\begin{figure}[!h]
    \centering
    \subfloat[Maleri med middel kraftige farver og medium antal detaljer.]{
        \includegraphics[angle=0,width=0.95\textwidth]{afsnit/afprovning/billeder/thressholds/medium_farver/medium_detalier/floodfill/4-4.png}
        \label{GRD_virker_nesten2}}\\
	\subfloat[Det samme maleri, hvor Tærskelværdier er valt specifikt for det her maleri]{
   	 	\includegraphics[angle=0,width=0.90\textwidth]{afsnit/afprovning/billeder/thressholds/medium_farver/medium_detalier/s5_e90_e200_f4.png}
	    \label{GRD_virker_nesten2_super}}\hspace{1em}
     \caption[]{Et malerier, hvor den generelle metode ikke helt virker efter vores ønske, men med nye tærskelværdier vil virker meget bedre}
     \label{generelde_region_detektor_virker_nesten2}
\end{figure}

I figur \ref{generelde_region_detektor_virker_nesten1} og \ref{generelde_region_detektor_virker_nesten2} er der to
malerier, hvor metoden ikke fungerer optimalt efter hensigten. Dog kan
vi stadig bruge figurerne til noget

I maleriet \ref{GRD_virker_nesten1} bliver der hovedsageligt fundet små
regioner. En sko, en skulder og en flise trækkes ud, hvor vi hellere
ville have haft, at personens kappe og arm blev fundet. Dette skyldes
primært, at tærskelværdierne for dette billede er sat for lavt. 


Maleriet i den anden figur \ref{GRD_virker_nesten2} rummer nogle af de
samme problemer: Metoden finder også her en masse små dele af figurer,
der burde hænge sammen. Det vil også kunne løses ved nogle andre
tærskelværdier, men som man også kan se på maleriet er vægge og loftet -
som egentlig er ret ensfarvede -stadig svære at finde for metoden. Det
kunne tyde på, at en højere grad af sløring ville løse problemet, og få
algoritmen til at virke i maleriet.

Når netop de to figure \ref{GRD_virker_nesten1} og
\ref{GRD_virker_nesten2} er interessante, er det fordi, at en ændring af
tærskelværdier, samt en højre grad af sløring, ville bevirke, at den
generelle metode kom til at fungere bedre. I billedet
\ref{GRD_virker_nesten1_super} og \ref{GRD_virker_nesten1_super} er
protretere de samme malerier, men hvor tærskelværdierne passer bedre til
maleriet, man kan se det ved f.eks at karben på anglen samt det meste af
loftet bliver fundet.

\clearpage

\begin{figure}[!h]
    \centering
     \subfloat[Det originale maleri. Navn: Vase of Flower. År: ukent.
	 Af: Arellano, Juan de.]{
        \includegraphics[angle=0,width=0.46\textwidth]{afsnit/afprovning/billeder/thressholds/krafitige_farver/medium_detalier/mDetalier}
        \label{GRD_virker_ikke1_orginal}}
    \subfloat[Maleri med kraftige farver og medium detaljer.]{
        \includegraphics[angle=0,width=0.46\textwidth]{afsnit/afprovning/billeder/thressholds/krafitige_farver/medium_detalier/floodfill/4-4.png}
        \label{GRD_virker_ikke1}}
     \caption{Maleri af blomster, hvor den generelle metode ikke
	 virker.}
     \label{generelde_region_detektor_virker_ikke}
\end{figure}

\begin{figure}[!h]
	\begin{center}
	    \includegraphics[angle=0,width=0.65\textwidth]{afsnit/afprovning/billeder/thressholds/svage_farver/svage_detalier/floodfill/4-4.png}
	\end{center}    
	\caption{Maleri med svage farver og få detaljer.}
    \label{GRD_virker_ikke2}
\end{figure}

I figur \ref{generelde_region_detektor_virker_ikke} og
\ref{GRD_virker_ikke2} er der to malerier hvor
vores generelle metode ikke virker optimalt. 

I maleri \ref{GRD_virker_ikke1} er noget af buketten og baggrunden gået
i et. Desuden er resten af blomsterne i snittet ikke fyldt ud, og selv
en ændring i tærskelværdierne vil ikke hjælpe, da en forøgelse af disse
blot vil resultere i, at flere af blomsterne går i ét med baggrunden. En
sænkning vil resultere i, at ingen af blomsterne bliver trukket ud. 

Maleriet \ref{GRD_virker_ikke2} repræsenterer en anden problemstilling,
som vi ikke kan komme udenom: farverne er så mørke og svage, at en
ændring i tærskelværdien i floodfill på bare en vil få metoden til at gå
fra at finde ingen regioner til at finde for store regioner. det kan ses
i figur \ref{ff_munke}, hvor de 2 malerier, hvor floodfills tærskelværdi
er sat til den lavest og næst laves værdi. Som man kan se, finde metoden
mange snå regioner som ikke er særlige intrasant i maleri
\ref{munk_etff}, men i \ref{munk_toff}, bliver der fundet 2 meget store
regioner, som er blevet alt for store. Dette er et problem da vi så ikke
kan have en tærskelværdi som virker.

\begin{figure}[!h]
    \centering
     \subfloat[Tærskelværdierne for floordfill på maleriet er sat til en]{
        \includegraphics[angle=0,width=0.46\textwidth]{afsnit/afprovning/billeder/thressholds/svage_farver/svage_detalier/1-1.png}
        \label{munk_etff}}
    \subfloat[Tærskelværdierne for floodfill på maleriet er sat til to]{
        \includegraphics[angle=0,width=0.46\textwidth]{afsnit/afprovning/billeder/thressholds/svage_farver/svage_detalier/2-2.png}
        \label{munk_toff}}
     \caption{Et maleri hvor tærskelværdien for floodfill er sat til den laveste værdi den kan have, og et hvor denne næst laves tærskelværdien er sat.}
     \label{ff_munke}
\end{figure}

Dette kan også skyldes, at kantdetektionen tilføjer en mørk kant rundt
om regionerne, men da maleriet er mørkt i forvejen, hjælper
kandetektionen ikke. I maleri \ref{bla} er kanten tegnet med blåt og man
kan se at munken to fra venstre, ikke får sit skæg med, men hvor han går
det i maleriet med sort kant, se maleri \ref{sort}.

\begin{figure}[!h]
    \centering
     \subfloat[Maleri med blå kanter i kantdetektionen]{
        \includegraphics[angle=0,width=0.46\textwidth]{afsnit/afprovning/billeder/thressholds/svage_farver/svage_detalier/blueE.png}
        \label{bla}}
    \subfloat[Maleri med sorte kanter i kantdetektionen]{
        \includegraphics[angle=0,width=0.46\textwidth]{afsnit/afprovning/billeder/thressholds/svage_farver/svage_detalier/floodfill/4-4.png}
        \label{sort}}
     \caption{To forskellige farver brugt til at ligge kanter på maleriernes regioner}
     \label{fleremunke}
\end{figure}


\clearpage

\section{Sortering i interessante regioner}
{\sffamily 
Nå regions detekter har fundet alle regioner vil vi gerne sortere ud i
dem, da små regioner og regioner med en for lille masse ikke er
interessante for vores 2 metoder
}
\subsection{Masse}
For at en region bliver taget i betragtning skal den have en masse som
er støre en vis tærskelværdi, som forklare i section \ref{naiv_regler}. På samme
fremgangsmåde som i afsnit \ref{region_stoerlse} sammenligner vi fem
forskellige tærskelværdier. Vi kommer frem til at hvis figurens masse er
under $\frac{1}{4}$ af afgrænsende rektangel, vil vi ikke tage den
med. En illustrere af hvad vi vælger at tage med kan ses i figur
\ref{masse}, hvor region detektoren er kørt på billedet
\ref{naiv_masse_original} og resultatet kan ses i figur \ref{naiv_masse}

\begin{figure}[!h]
    \centering
		\subfloat[2 regioner hvor den ende er sorteret vær på grund af dens masse]{
	        \includegraphics[angle=0,width=0.55\textwidth]{afsnit/afprovning/billeder/naive_losning/naiv_mass.png}
	       	\label{naiv_masse}}\hspace{1em}
	    \subfloat[Original]{
	        \includegraphics[angle=0,width=0.55\textwidth]{afsnit/afprovning/billeder/naive_losning/mass.png}
	       	\label{naiv_masse_original}}\hspace{1em}
		\label{masse}
		\caption{2 billeder som ilustrere forskellen på den region vi tager med og en vi ikke tager med}
\end{figure}

\clearpage

\subsection{Regionstørrelse \label{region_stoerlse}}
%% Bemærk:
%%          Resten af rapporten følger en stil hvor indledninger skrives
%%          med \sffamlily-typen. Denne stil bør ikke bruges her, da dette ikke er \section
%%
{
Ved udtrækning af regioner sætter vi en værdi for, hvor stor en region skal
være, for at blive tage i betragtning, i denne sektion vil vi teste, hvor stor
en regionen skal være for at vi vil tage den med. Problematikken med for små
regioner er illustreret i maleriet i figur \ref{alt_med}. Her er alle de grønne
kasser regioner, som vores naive algoritme mener er interessant, og da
alle regioner er taget med, bliver 939 regioner vurderet til at ligge i
snittet.

\begin{figure}[¡h]
    \setlength\fboxsep{0pt}
    \setlength\fboxrule{0.5pt}
    \begin{center}
        \fbox{\includegraphics[angle=0,width=0.45\textwidth]{afsnit/afprovning/billeder/stoerelse/alt_med.png}}
    \end{center}
    \caption{Maleri hvor alle størrelse regioner er godtaget, der er fundet $939$ intrasante regioner.}
	\label{alt_med}
\end{figure}

For at teste hvilken størrelse vi bør godtage, er der fremstillet et
testbillede i figur \ref{original_stoerelse}, hvor ni regioner er opstillet,
således at de alle ligger i samme snit. Den øverste region er den mindste, og
regionernes størrelse stiger gradvist ned langs snittet.

I hver af de andre fem testbilleder i figur \ref{stoerelse_sammenlining}, er
tærskelværdierne gradvist sat op. De regioner der godtages er repræsenteret med grønne kasser. 
Vi mener, at der bliver taget for mange regioner med i billede \ref{0,0}, \ref{0,005}, \ref{0,001} og
\ref{0,0015}. I billedet \ref{0,002} vurderer vi, at det kun er regioner
af en tilstrækkelig størrelse, som er blevet udvalgt, og vi har derfor
valgt at sætte tærskelværdien til $0.002$.

\begin{figure}[!h]
    \setlength\fboxsep{0pt}
    \setlength\fboxrule{0.5pt}
    \centering
    \subfloat[Det originale billede.]{
        \fbox{\includegraphics[angle=0,width=0.45\textwidth]{afsnit/afprovning/billeder/stoerelse/stoerelse.png}}
       \label{original_stoerelse}}
    \subfloat[Tærskelværdien sat til $0$.]{
        \fbox{\includegraphics[angle=0,width=0.45\textwidth]{afsnit/afprovning/billeder/stoerelse/0.png}}
       \label{0,0}}\\
    \subfloat[Tærskelværdien sat til $0.0005$.]{
        \fbox{\includegraphics[angle=0,width=0.45\textwidth]{afsnit/afprovning/billeder/stoerelse/0,0005.png}}
       \label{0,005}}
    \subfloat[Tærskelværdien sat til $0.001$.]{
        \fbox{\includegraphics[angle=0,width=0.45\textwidth]{afsnit/afprovning/billeder/stoerelse/0,001.png}}
       \label{0,001}}\\
    \subfloat[Tærskelværdien sat til $0.0015$.]{
        \fbox{\includegraphics[angle=0,width=0.45\textwidth]{afsnit/afprovning/billeder/stoerelse/0,0015.png}}
       \label{0,0015}}
    \subfloat[Tærskelværdien sat til $0.002$.]{
        \fbox{\includegraphics[angle=0,width=0.45\textwidth]{afsnit/afprovning/billeder/stoerelse/0,002.png}}
       \label{0,002}}
    \caption{Testbillede, med ni regioner som bliver større og større.
	Det originale billede, samt fem forskellige tærskelværdier, er afbilledet.}
    \label{stoerelse_sammenlining}
\end{figure}
}

\clearpage

\subsection{overvejlser}
Udvælgelsen af interessante regioner kunne godt være mere
sofistikeret.  Vi undersøger kun regioner for deres størrelse og masse,
hvilket stadig tillader mange regioner, som egentlig er uinteressante
pga. deres form.  I udvælgelsen af interessante regioner kunne man
derfor kigge på regionens form eller udstrækning, ved at undersøge
dennes massemidtpunkt.  Hvis massen er koncentreret langt væk fra
snittet, er denne region ikke interessant. Vi skal dog passe på, at vi
ikke tager beslutninger, som egentlig vedrører, om regionen er placeret
i snittet. Vi ønsker, i denne udvælgelse af regioner, udelukkende at
bestemme, hvorvidt regionen skal tages op til videre overvejelse, mht.
om den ligger i snittet.

Kun hvis vores søgning efter objekter i billedet bliver mere specifik,
giver det mening at undersøge regionerne nærmere i udvælgelsen af
interessante regioner. Vi kan forestille os en situation, hvor man
udelukkende vil finde ansigter placeret i det gyldne snit. I dette
tilfælde skal vi selvfølgelig ikke sende en region til videre
vurdering, hvis den \emph{ikke} er et ansigt. Vi har dog ikke en sådan
specifik søgning, hvorfor vi kun kan frasortere regioner ud fra
informationen om deres størrelse og masse.





\section{Naive løsning}
%% Bemærk:
%%          Resten af rapporten følger en stil hvor indledninger skrives
%%          med \sffamlily-typen. Denne stil bør også følges her.
%%
{\sffamily
I dette sektion vil vi teste den naive løsning, ved at se om den sortere
de rigtige regioner væk og om løsningen opføre sig på samme måde som vi
har håbet på. Det vil vi gøre ved ført at se på nogle fabrikeret test
billeder få at se om den naive løsning virker efter meningen og bag
efter vil vi teste på malerierne for at se om den naive løsning kan
bruges i praktisk.
}
  
\subsection{Afprøvning på testbilleder}
Vi vil teste på de samme billeder som var i figur
\ref{region_detektor_test}, samt nogle af de test billeder som blev
brugt i forklaringen af den naive metodes, de 4 billeder som vi har valt
at test kan se i figur \ref{naiv_detektion_test} hvor en grån kasse
rundt om en region, betyder at den er valt til at ligge i det gyldne
snit af den naive metode. Det første billedet \ref{naiv_blob1}, har 5
regioner og hvor 3 af dem blev fundet af reginon detektoren, vores naive
løsning har så sorteret baggrundens regionen og den øverste region i
snitte vær, da de begge krydser marginer og derfor ikke overholder regel
\ref{2.2.4, c}. Det andet billedet \ref{naiv_blob2}, er alle blevet
sorteret vær, også den lille region som ligger lige i miden af snittet.
Grunde til det er at den er for lille og derfor ikke overholder regel
\ref{2.2.4, a}. I test billedet \ref{naive_hoisont1}, sortere algoritmen
himlen fra, da den krydser margin lidt, men tager jorden med. I test
billedet \ref{naiv_masse} er der kun en af den 3 regioner som ikke
bliver sorteret væk, grunden til at den nederste region ikke bliver
tager med er at den ikke har en masse der er stor nok, som forklaret i
\ref{2.2.3} og derfor ikke overholder regel \ref{2.2.4 b}. Alle de test
billeder som vi har vist her opføre sig præcist på den måde som vi havde
regnet med.

\begin{figure}[!h]
    \centering
		\subfloat[Naive algoritme finder 1 ud af 5 regioner]{
        	\includegraphics[angle=0,width=0.55\textwidth]{afsnit/afprovning/billeder/naive_losning/naiv_blob1.png}
        	\label{naiv_blob1}}\hspace{1em}
    		\subfloat[Værgen den lille region eller den store er fundet]{
	        	\includegraphics[angle=0,width=0.55\textwidth]{afsnit/afprovning/billeder/naive_losning/naiv_blob2.png}
	       	\label{naiv_blob2}}\hspace{1em}
    		\subfloat[Kun den nederste højrisondt er fundet]{
	        	\includegraphics[angle=0,width=0.55\textwidth]{afsnit/afprovning/billeder/naive_losning/naiv_hoisont1.png}
		    \label{naiv_hoisont1}}\hspace{1em}
		    \subfloat[2 regioner hvor den ende er sorteret vær på grund af dens masse]{
	        	\includegraphics[angle=0,width=0.55\textwidth]{afsnit/afprovning/billeder/naive_losning/naiv_mass.png}
	       	\label{naiv_masse}}\hspace{1em}
        \caption[]{4 test billeder som også blev brugt til at illustrere den naive løsnings fremgangs måde, grån kasse rund om region, betyder at den er taget med af dem naive løsning}
     \label{naiv_detektion_test}
\end{figure}
\clearpage

\subsection{Afprøvning på malerier}
For at se på hvordan den naive metode virker på malerier afprøver vi den
på 6 malerier, først på 3 malerier, hvor regions detektoren virker
efter vores hensigt og så på 3 malerier, hvor region detektoren ikke
virker. Beskrivelsen af hvad der sker i billedet vil stå i caption


\begin{figure}[h!!]
	\begin{center}
		\includegraphics[scale=0.3,angle=0]{afsnit/afprovning/billeder/naive_losning/naiv_kfarver_sdetaljer.png}
	\end{center}
	\caption[]{5 ud af de 6 store regioner fra figur \ref{GRD_virker1} valt til at ligge i snittet, skoene er få små til at blive taget i betragtning}
	\label{naiv_kfarver_sdetaljer}
\end{figure}

\begin{figure}[h!!]
	\begin{center}
		\includegraphics[scale=0.3,angle=0]{afsnit/afprovning/billeder/naive_losning/naiv_mfarver_mdetaljer.png}
	\end{center}
	\caption[]{Bukserne og skoene er tager med af den naive løsning, men drengen er sorteret vær da har krydser snittet}
	\label{naiv_mfarver_mdetaljer}
\end{figure}

\begin{figure}[h!!]
	\begin{center}
		\includegraphics[scale=0.3,angle=0]{afsnit/afprovning/billeder/naive_losning/naiv_kfarver_kdetaljer.png}
	\end{center}
	\caption[]{Et billedet med mange hoder i snittet, hvor 2 af dem bliver godtaget af den naive metode til at ligger i snittet, en trøje bliver desværre også taget med »»(måske noget med at der bliver soteret få mange hover fra)}
	\label{naiv_kfarver_kdetaljer}
\end{figure}

\begin{figure}[h!!]
	\begin{center}
		\includegraphics[scale=0.3,angle=0]{afsnit/afprovning/billeder/naive_losning/naiv_virker_ikke1.png}
	\end{center}
	\caption[]{Mallerie hvor region detektor ikke virker, den naive løsning godtager tager en region som ligger helt forkert }
	\label{naiv_virker_ikke1}
\end{figure}

\begin{figure}[h!!]
	\begin{center}
		\includegraphics[scale=0.3,angle=0]{afsnit/afprovning/billeder/naive_losning/naiv_virker_ikke2.png}
	\end{center}
	\caption[]{3 regioner bliver godtaget, selv om de ikke er særlige intresante ««(er det ikke en farlige ting at sige )}
	\label{naiv_virker_ikke2}
\end{figure}

\begin{figure}[h!!]
	\begin{center}
		\includegraphics[scale=0.3,angle=0]{afsnit/afprovning/billeder/naive_losning/naiv_virker_ikke3.png}
	\end{center}
	\caption[]{Der bliver fundet 3 region, hvor kun en af dem passer på en ting i billedet}
	\label{naiv_virker_ikke3}
\end{figure}
\clearpage

\subsection{konkulution}
Det virker som om den naive løsning virker efter vores entationer dog
med nogle få falske positive, hvis region detektoren virker på
malerierne, dog fejler den på malerier hvor region detektoren fejler, og
kommer med en masse falske positive.

\clearpage

\section{Udvidet løsning}
%% Bemærk:
%%          Resten af rapporten følger en stil hvor indledninger skrives
%%          med \sffamlily-typen. Denne stil bør også følges her.
%%
{\sffamily
I denne sektion vil vi test den udvidet løsning, om den virker som vi
hade planlagt på test billeder og om hvordan den virker i praktisk på
udvalgte malerier.
}
\subsection{Afprøvning på testbilleder}
Vi teste metoden på 4 test billeder. Billedet \ref{hus_virker} er et hus
hvor snittet og massemidtpunktet næsten ligge oven i hinanden. 

\begin{figure}[h!!]
	\begin{center}
		\includegraphics[scale=0.3,angle=0]{afsnit/afprovning/billeder/udvidet_losning/udvidet_hus1_test.png}
	\end{center}
	\caption[]{Et hus der er symetrisk og der har massemidtpunkt i spisen af taget, massemidtpunktet er inde for margin så huset bliver udvalgt til at ligge i snittet.}
	\label{hus_virker}
\end{figure}

I det andet billedet \ref{hus_virker_ikke} er huset forskubbet så masse
midtpunktet lige lige uden for margin, og derved ikke skulle blive tage
med af algoritmen. 

\begin{figure}[h!!]
	\begin{center}
		\includegraphics[scale=0.3,angle=0]{afsnit/afprovning/billeder/udvidet_losning/udvidet_hus2_test.png}
	\end{center}
	\caption[]{Hus hvor masse midtpunktet er flyttet få pixels ud for margin, den udvide løsning vælger ikke at tage huset med.}
	\label{hus_virker_ikke}
\end{figure}

Det tredje billedet \ref{udvidet_blob_test} har en region som er meget lille
og derfor bliver fra sorteret på grund af den størrelse, den stor
region, bliver taget med at dens massemidtpunkt ligger inden for margin. 

\begin{figure}[h!!]
	\begin{center}
		\includegraphics[scale=0.3,angle=0]{afsnit/afprovning/billeder/udvidet_losning/udvidet_blob2_test.png}
	\end{center}
	\caption[]{2 regioner hvor den nederste har et massemidtpunkt inde for margin.}
	\label{udvidet_blob_test}
\end{figure}

Det fjerre billedet \ref{bleksprutte_test} er der en region med som har
et massemidtpunkt inde i margin, men som har en skæv fordeling af pixels
i forhold til snittet og derfor bliver sorteret fra. Ud fra de fire
observationer, virker det som om, den udvidet løsning virker efter
forventningerne.

\begin{figure}[h!!]
	\begin{center}
		\includegraphics[scale=0.3,angle=0]{afsnit/afprovning/billeder/udvidet_losning/udvidet_bleksprutte_test.png}
	\end{center}
	\caption[]{Margin er sat meget op, så man kan se at regionen har et massemidtpunkt ind for margin, men da størstedelen af regionens pixels er på højre side, vælger den fra.}
	\label{bleksprutte_test}
\end{figure}

Som man kan se af testbillederne vælger den udvidet løsning vid
forskellige regioner til at ligger i snittet. 
\clearpage


\subsection{Afprøvning på malerier}
Afprøvningen af den udvidet løsning på malerier fra vores database,
foregå på samme måde som den naive test.

\begin{figure}[h!!]
	\begin{center}
		\includegraphics[scale=0.3,angle=0]{afsnit/afprovning/billeder/udvidet_losning/udvidet_kfarver_sdetaljer.png}
	\end{center}
	\caption[]{2 ud af de 6 region er godtager som ligene i det gyldne snit af den udvidet løsning}
	\label{udvidet_virker1}
\end{figure}

\begin{figure}[h!!]
	\begin{center}
		\includegraphics[scale=0.3,angle=0]{afsnit/afprovning/billeder/udvidet_losning/udvidet_dreng.png}
	\end{center}
	\caption[]{baggrunden er den nedeste region som bliver godtaget, da den har et massemidtpunkt som liger inde for marginen}
	\label{udvidet_virker2}
\end{figure}

\begin{figure}[!h]
    \centering
    	\subfloat[Udvidet løsning]{
        	\includegraphics[angle=0,width=0.45\textwidth]{afsnit/afprovning/billeder/udvidet_losning/udvidet_pige.png}
        	\label{udvidet_pige}}\hspace{1em}
    	\subfloat[naiv løsning]{
        	\includegraphics[angle=0,width=0.45\textwidth]{afsnit/afprovning/billeder/udvidet_losning/naiv_pige.png}
        	\label{naiv_pige}}\hspace{1em}        	    			
        \caption[]{Et maleri hvor resultatet for den udvidet og den naive løsning er protrateret. Der bliver fundet forskellige regioner for vær af algoritmerne. Navn: Angel Announcing. År ca. 1500. Af: Bellini, Giovanni}
     \label{udvidet_virker3}
\end{figure}

\begin{figure}[h!!]
	\begin{center}
		\includegraphics[scale=0.3,angle=0]{afsnit/afprovning/billeder/udvidet_losning/udvidet_kfarver_kdetaljer.png}
	\end{center}
	\caption[]{En sko og en flise bliver taget med at algoritmen}
	\label{udvidet_virker_ikke1}
\end{figure}

\begin{figure}[h!!]
	\begin{center}
		\includegraphics[scale=0.3,angle=0]{afsnit/afprovning/billeder/udvidet_losning/udvidet_mfarver_mdetaljer.png}
	\end{center}
	\caption[]{Den udvidet løsning sortere alle regioner væk}
	\label{udvidet_virker_ikke2}
\end{figure}

\begin{figure}[h!!]
	\begin{center}
		\includegraphics[scale=0.3,angle=0]{afsnit/afprovning/billeder/udvidet_losning/udvidet_sfarver_mdetaljer.png}
	\end{center}
	\caption[]{2 regioner som ikke høre nogle steder hende bliver fundet}
	\label{udvidet_virker_ikke3}
\end{figure}
\clearpage

\subsection{konklution}
Ud for teste billederne ser det ud til at den udvidet metode virker på
den måde som vi havde planlangt at den skulle virke. I den praktiske
brug på malerierne, finder den udvidet løsning mange gode regioner hvis
regions detektoren virker se figur \ref{udvidet_virker1}, men kommer
desværre stadig med få regioner som ikke er særlige gode, f.eks i maleri
\ref{udvidet_virker2}. Der udvidet løsning finder også regioner som den
naive ikke vil finde se en sammenlining i figur \ref{udvidet_virker3}. I
de tilfælde hvor regions detektoren ikke virker godt, er den udvide
løsning god til at sorter regioner væk og kommer derfor ikke med særlige
mange regioner som ikke skulle være der, se figur
\ref{udvidet_virker_ikke2} og \ref{udvidet_virker_ikke3}. Dog kommer vi
stadig ud i problemer, hvor der kun er regioner som er forkerte med,
se figur \ref{udvidet_virker_ikke1}.

\clearpage

\section{Sammelining af de 2 metoder?}
Ud for de 2 metoders konklution laver vi en sammenligning af de 2 metoder. Den første metode finder mange gode regioner, men også mange ikke så gode regioner, Den anden metode finder en del fære gode regioner, men meget få ikke så gode regioner
den her vil jeg have hjælp til






\clearpage


}

% vim: set tw=72 spell spelllang=da:
