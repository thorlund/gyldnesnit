%% Bemærk:
%%          Resten af rapporten følger en stil hvor indledninger skrives
%%          med \sffamlily-typen. Denne stil bør også følges her.
%%
{\sffamily
I dette sektion vil vi teste den naive løsning, ved at se om den sortere
de rigtige regioner væk og om løsningen opføre sig på samme måde som vi
har håbet på. Det vil vi gøre ved ført at se på nogle fabrikeret test
billeder få at se om den naive løsning virker efter meningen og bag
efter vil vi teste på malerierne for at se om den naive løsning kan
bruges i praktisk.
}
  
\subsection{Afprøvning på test billeder}
Vi vil teste på de samme billeder som var i figur
\ref{region_detektor_test}, samt nogle af de test billeder som blev
brugt i forklaringen af den naive metodes, de 4 billeder som vi har valt
at test kan se i figur \ref{naiv_detektion_test} hvor en grån kasse
rundt om en region, betyder at den er valt til at ligge i det gyldne
snit af den naive metode. Det første billedet \ref{naiv_blob1}, har 5
regioner og hvor 3 af dem blev fundet af reginon detektoren, vores naive
løsning har så sorteret baggrundens regionen og den øverste region i
snitte vær, da de begge krydser marginer og derfor ikke overholder regel
\ref{2.2.4, c}. Det andet billedet \ref{naiv_blob2}, er alle blevet
sorteret vær, også den lille region som ligger lige i miden af snittet.
Grunde til det er at den er for lille og derfor ikke overholder regel
\ref{2.2.4, a}. I test billedet \ref{naive_hoisont1}, sortere algoritmen
himlen fra, da den krydser margin lidt, men tager jorden med. I test
billedet \ref{naiv_masse} er der kun en af den 3 regioner som ikke
bliver sorteret væk, grunden til at den nederste region ikke bliver
tager med er at den ikke har en masse der er stor nok, som forklaret i
\ref{2.2.3} og derfor ikke overholder regel \ref{2.2.4 b}. Alle de test
billeder som vi har vist her opføre sig præcist på den måde som vi havde
regnet med.

\begin{figure}[!h]
    \centering
		\subfloat[Naive algoritme finder 1 ud af 5 regioner]{
        	\includegraphics[angle=0,width=0.55\textwidth]{afsnit/afprovning/billeder/naive_losning/naiv_blob1.png}
        	\label{naiv_blob1}}\hspace{1em}
    		\subfloat[Værgen den lille region eller den store er fundet]{
	        	\includegraphics[angle=0,width=0.55\textwidth]{afsnit/afprovning/billeder/naive_losning/naiv_blob2.png}
	       	\label{naiv_blob2}}\hspace{1em}
    		\subfloat[Kun den nederste højrisondt er fundet]{
	        	\includegraphics[angle=0,width=0.55\textwidth]{afsnit/afprovning/billeder/naive_losning/naiv_hoisont1.png}
		    \label{naiv_hoisont1}}\hspace{1em}
		    \subfloat[2 regioner hvor den ende er sorteret vær på grund af dens masse]{
	        	\includegraphics[angle=0,width=0.55\textwidth]{afsnit/afprovning/billeder/naive_losning/naiv_mass.png}
	       	\label{naiv_masse}}\hspace{1em}
        \caption[]{4 test billeder som også blev brugt til at illustrere den naive løsnings fremgangs måde, grån kasse rund om region, betyder at den er taget med af dem naive løsning}
     \label{naiv_detektion_test}
\end{figure}
\clearpage

\subsection{Afprøvning på malerier}
For at se på hvordan den naive metode virker på malerier afprøver vi den
på 6 malerier, først på 3 malerier, hvor regions detektoren virker
efter vores hensigt og så på 3 malerier, hvor region detektoren ikke
virker. Beskrivelsen af hvad der sker i billedet vil stå i caption


\begin{figure}[h!!]
	\begin{center}
		\includegraphics[scale=0.3,angle=0]{afsnit/afprovning/billeder/naive_losning/naiv_kfarver_sdetaljer.png}
	\end{center}
	\caption[]{5 ud af de 6 store regioner fra figur \ref{GRD_virker1} valt til at ligge i snittet skoende er få små til at blive taget i betragtning}
	\label{naiv_kfarver_sdetaljer}
\end{figure}

\begin{figure}[h!!]
	\begin{center}
		\includegraphics[scale=0.3,angle=0]{afsnit/afprovning/billeder/naive_losning/naiv_mfarver_mdetaljer.png}
	\end{center}
	\caption[]{}
	\label{naiv_mfarver_mdetaljer}
\end{figure}

\begin{figure}[h!!]
	\begin{center}
		\includegraphics[scale=0.3,angle=0]{afsnit/afprovning/billeder/naive_losning/naiv_kfarver_kdetaljer.png}
	\end{center}
	\caption[]{}
	\label{naiv_kfarver_kdetaljer}
\end{figure}

\clearpage
\subsection{hvad kan vi gøre bedre}
I det første billedet \ref{ff}, mener vi at den ting som virkelige kan
forbedre måde vores metode arbejder på. Er ved at ændre på den
definition som sortere regioner fra. Så en region kan krydse begge
marginer. Den anden ting man kunne se på, omhandler det andet billedet
\ref{nicofill}, hvor vi få for meget med, dette kunne løses ved at se på
individuelde billeder og regne ud hvor meget algoritmerne må tage med.
Begge disse to forslag vil blive diskuteret yderligt i den udvidet
løsning af problemet, hvor vi vil gå ind på hvad vi gør for at kommer
over nogle at de problemer som vi har.
