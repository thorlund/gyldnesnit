%% Bemærk:
%%          Resten af rapporten følger en stil hvor indledninger skrives
%%          med \sffamlily-typen. Denne stil bør også følges her.
%%
{\sffamily
I denne sektion vil vi teste den naive løsning. V vil se på, om den
sorterer de rigtige regioner, væk og om løsningen opfører sig som vi
har håbet på. Det vil vi gøre ved at se på nogle fabrikerede
testbilleder og bagefter vil vi teste på malerierne for at se, om den
naive metode kan bruges i praksis.
}
  
\subsection{Afprøvning på testbilleder}
Vi vil teste på de samme testbilleder, som i sidste afsnit, samt på et nyt
sæt testbilleder, som blev brugt i forklaringen af den naive metode. De fire
billeder, som vi har valgt at teste, kan ses i afsnit \ref{region_detektor}
og billedet \ref{naiv_masse_original}.

Det første billede \ref{naiv_blob1} har fem regioner. Kun tre af
dem blev fundet under udtrækningen af regioner. Vores naive løsning har så
sorteret baggrundensregionen og den øverste region i snittet væk, da
ingen af de to regioners kanter ligger inden for snittets marginer og
derfor overholder regionerne ikke defination \ref{def_naiv} . 

\begin{figure}[!h]
	\begin{center}
        \includegraphics[angle=0,width=0.55\textwidth]{afsnit/afprovning/billeder/naive_losning/naiv_blob1.png}
	\end{center}       
	\caption{Naive algoritme finder en ud af fem regioner, den grå kasse
	betyder at den region er blevet udvalgt.}	
	\label{naiv_blob1}
\end{figure}

På det andet billede, \ref{naiv_blob2}, er alle regionerne blevet sorteret væk, også
den mindste, da den er for lille og derfor ikke overholder definition
\ref{def_interessant} b. 

\begin{figure}[!h]
	\begin{center}
       	\includegraphics[angle=0,width=0.55\textwidth]{afsnit/afprovning/billeder/naive_losning/naiv_blob2.png}
	\end{center}
	\caption{Hverken den lille region eller den store er fundet.} 
   	\label{naiv_blob2}
\end{figure}

I testbillede, \ref{naive_hoisont1} sorterer metoden
himlen fra, da den krydser marginen en smule, mens jorden tages med. 

\begin{figure}[!h]
	\begin{center}
       	\includegraphics[angle=0,width=0.55\textwidth]{afsnit/afprovning/billeder/naive_losning/naiv_hoisont1.png}
	\end{center}
	\caption{Kun den nederste horisont er fundet.} 
   	\label{naive_hoisont1}
\end{figure}

\clearpage

\subsection{Afprøvning på malerier}
Vi afprøver den naive algoritme på seks malerier. Først på tre malerier,
hvor regiondetektoren virker efter hensigten, og så på tre
malerier, hvor regiondetektoren ikke virker. Beskrivelsen af, hvad der
sker i billedet, vil stå i billedbeskrivelse.


\begin{figure}[h!!]
	\begin{center}
		\includegraphics[scale=0.3,angle=0]{afsnit/afprovning/billeder/naive_losning/naiv_kfarver_sdetaljer.png}
	\end{center}
	\caption[]{Fem ud af de seks store regioner, fra figur
	\ref{GRD_virker1}, valgt til at ligge i snittet. Skoene er for små til
	at blive taget med i betragtning.}
	\label{naiv_kfarver_sdetaljer}
\end{figure}

\begin{figure}[h!!]
	\begin{center}
		\includegraphics[scale=0.3,angle=0]{afsnit/afprovning/billeder/naive_losning/naiv_mfarver_mdetaljer.png}
	\end{center}
	\caption[]{Bukserne og skoene er tager med af den naive metode, mens
	drengen er sorteret væk, da han krydser snittet.}
	\label{naiv_mfarver_mdetaljer}
\end{figure}

\begin{figure}[h!!]
	\begin{center}
		\includegraphics[scale=0.3,angle=0]{afsnit/afprovning/billeder/naive_losning/naiv_kfarver_kdetaljer.png}
	\end{center}
	\caption[]{Et billedet med mange hoveder i snittet, hvoraf to
	bliver godtaget af den naive metode, en
	trøje bliver desværre også taget med. Navn: Christ Carrying the
	Cross. År: 1480. Af: Bosch Hieronymus.}
	\label{naiv_kfarver_kdetaljer}
\end{figure}

\begin{figure}[h!!]
	\begin{center}
		\includegraphics[scale=0.3,angle=0]{afsnit/afprovning/billeder/naive_losning/naiv_virker_ikke1.png}
	\end{center}
	\caption[]{Maleri hvor udtrækningen af regioner ikke virker. Den naive
	løsning godtager en region, som ligger helt forkert. Navn:
	Peasants in an Inn Playing "La Main Chaude". År: Ukendt. Af:
	Molenaer, Jan Miense.}
	\label{naiv_virker_ikke1}
\end{figure}

\begin{figure}[h!!]
	\begin{center}
		\includegraphics[scale=0.3,angle=0]{afsnit/afprovning/billeder/naive_losning/naiv_virker_ikke2.png}
	\end{center}
	\caption[]{Tre regioner bliver godtaget, selv om de ikke er særlig
	interessante.}
	\label{naiv_virker_ikke2}
\end{figure}

\begin{figure}[h!!]
	\begin{center}
		\includegraphics[scale=0.3,angle=0]{afsnit/afprovning/billeder/naive_losning/naiv_virker_ikke3.png}
	\end{center}
	\caption[]{Der bliver fundet tre regioner, hvoraf kun \'{e}n rent
	faktisk er en figur i billedet.}
	\label{naiv_virker_ikke3}
\end{figure}
\clearpage

For at give et bedre overblik over hvor mange regioner, vi mener er
interessante, og hvor mange, der er falske positiver, har vi optalt dem på
de samme ni malerier, som vi brugte til tærskelværdiafprøvningen. I tabel
\ref{naiv_good} ses antallet af interessante regioner, og i tabel
\ref{naiv_bad} ses antallet af falske positiver. Der bliver fundet 12
interessante regioner og 10 falske positive, hvilket vil sige at der er $20
\%$ flere interessante regioner end falske.

\begin{table}[H]
    \centering
    \begin{tabular}{|c|l|l|l|}
			\hline
            & Kraftige farver & Medium farver & Svage farver \\\hline
		Mange detaljer	& 2 & 0 & 7 \\\hline
        Medium detaljer  & 1 & 0 & 0 \\\hline
        Få detaljer     & 1 & 0 & 1 \\\hline
    \end{tabular}
    \caption[]{Tabel over antal interessante regioner fundet i ni testmalerier i to gyldne snit.}
    \label{naiv_good}
\end{table}

\begin{table}[H]
    \centering
    \begin{tabular}{|c|l|l|l|}
			\hline
            & Kraftige farver & Medium farver & Svage farver \\\hline
		Mange detaljer	& 3 & 0 & 0 \\\hline
        Medium detaljer  & 0 & 3 & 0 \\\hline
        Få detaljer     & 1 & 3 & 0 \\\hline
    \end{tabular}
    \caption[]{Tabel over antal falske positiver i ni testmalerier i to gyldne snit.}
    \label{naiv_bad}
\end{table}

\subsection{Konklusion}
Den naive metode virker efter intentionen på vores testbilleder, med
ensfarvede regioner, men i praksis finder den mange regioner som vi helst
ville være foruden f. eks. i maleri \ref{naiv_virker_ikke3}. På
testmalerierne findes der altså kun $20\%$ flere interessante
regioner, som vi finder korrekt detekteret, end falske positiver. Der er
ikke særlig godt og kommer desværre til at påvirke de resultater, vi
kommer frem til.
