I forbindelse med sløring vil vi gerne test tre forskellige størrelser på foldningsmatricen der bruges. Vi vil teste
på de sammen ni malerier som i afsnit \ref{terskelverdi}.

\begin{table}[!h]
    \centering
    \begin{tabular}{| l | l | l | l |} \hline
                            & Svage farver	& Medium farver	& Kraftige farver 	\\ \hline
        Få detaljer 		& $(7 \times 7)$	& $(7 \times 7)$	& $(7 \times 7)$		\\ \hline
        Medium detaljer 	& $(5 \times 5)$	& $(7 \times 7)$	& $(7 \times 7)$		\\ \hline
        Mange detaljer		& $(5 \times 5)$	& $(5 \times 5)$	& $(7 \times 7)$		\\ \hline
    \end{tabular}
    \caption{Tabel over hvilke sløringer der er bedst til vert af sine billeder}
    \label{sloringTabel}
\end{table}

Som man kan se af tabel \ref{sloringTabel} ligger tærskelværdierne på
$(5 \times 5)$ og $(7 \times 7)$, med en svær overvægt af værdierne hen
mod $(7 \times 7)$. I vores program har vi brugt $(3 \times 3)$ da denne
værdi, passes bedst til de billeder vi så på først. \note{Kunne vi ikke bare give 1 af eksemplet fra slørings afsnittet}
