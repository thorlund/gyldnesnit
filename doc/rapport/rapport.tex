% vim: set tw=72 spell spelllang=da:
\documentclass[a4paper, 10pt, danish, final]{report}
%%%%%%%%%%%%%%%%%%%%%%%%%%%%%%%%%%%%%%%%%%%%%%%%%%%%%%%%%%%%%%%%%
% Pakker
%%%%%%%%%%%%%%%%%%%%%%%%%%%%%%%%%%%%%%%%%%%%%%%%%%%%%%%%%%%%%%%%%
%\usepackage{a4wide}
\usepackage[danish]{babel}
\usepackage[utf8]{inputenc}
\usepackage[T1]{fontenc}

\usepackage{charter}
\usepackage{verbatim}
\usepackage{amsfonts}
\usepackage{amsmath}
\usepackage{amssymb}
%\usepackage{mathrsfs}
%\usepackage[mathcal]{euscript}
\usepackage{listings}
\usepackage{graphicx}
\usepackage{multirow}
\usepackage{hyperref}
\usepackage{cite}
\usepackage{float}
\usepackage[small,bf]{caption}
\usepackage{xypic}

%%%%%%%%%%%%%%%%%%%%%%%%%%%%%%%%%%%%%%%%%%%%%%%%%%%%%%%%%%%%%%%%
% Indstillinger
%%%%%%%%%%%%%%%%%%%%%%%%%%%%%%%%%%%%%%%%%%%%%%%%%%%%%%%%%%%%%%%%
\parindent=5pt
\lstset{language=Python, basicstyle=\scriptsize, showstringspaces=false,
numbers=none, stepnumber=1, numberstyle=\tiny}

%%%%%%%%%%%%%%%%%%%%%%%%%%%%%%%%%%%%%%%%%%%%%%%%%%%%%%%%%%%%%%%%
% Kommandoer
%%%%%%%%%%%%%%%%%%%%%%%%%%%%%%%%%%%%%%%%%%%%%%%%%%%%%%%%%%%%%%%%
%\newcommand{\old}[1]{\oldstylenums{#1}}
%\newcommand{\old}[1]{{#1}}
\newcommand{\mailto}[1]{\href{mailto:#1}{#1}}
\newcommand{\resume}[1]{\begin{abstract}{\sffamily #1}\end{abstract}}
%\renewcommand{\thefigure}{\thechapter.\thesection.\arabic{figure}}
%\renewcommand{\thetable}{\thechapter.\thesection.\arabic{table}}

%%%%%%%%%%%%%%%%%%%%%%%%%%%%%%%%%%%%%%%%%%%%%%%%%%%%%%%%%%%%%%%
% Titel, forfatter og dato
%%%%%%%%%%%%%%%%%%%%%%%%%%%%%%%%%%%%%%%%%%%%%%%%%%%%%%%%%%%%%%%
\title{Detektion af ``Det Gyldne Snit'' i digitaliserede malerier}

\author{Ulrik Bonde - \mailto{bonde@diku.dk}\\
Kasper Steenstrup - \mailto{khsj@diku.dk}\\
Morten Thorlund - \mailto{thorlund@diku.dk}\\
\\
Vejleder\\Jakob Grue Simonsen}
\date{\today}

\hypersetup{
colorlinks,%
citecolor=black,%
filecolor=black,%
linkcolor=black,%
urlcolor=black,%
bookmarksopen=false,
pdftitle={Detektion af Det Gyldne Snit i digitaliserede malerier},
pdfauthor={Ulrik Bonde, Kasper Steenstrup og Morten Thorlund},
pdfsubject={Computer Vision},
pdfkeywords={Det gyldne snit, computer vision}
}

%%%%%%%%%%%%%%%%%%%%%%%%%%%%%%%%%%%%%%%%%%%%%%%%%%%%%%%%%%%%%%
% Indhold
%%%%%%%%%%%%%%%%%%%%%%%%%%%%%%%%%%%%%%%%%%%%%%%%%%%%%%%%%%%%%%
\begin{document}
\maketitle
\thispagestyle{empty}
%\pagestyle{headings}

\resume{
Det er den generelle opfattelse, at det gyldne snit er specielt æstetisk
tiltalende, og at man i kunstmalerier finder flest interessante regioner
her.  Litteraturen kan ikke entydigt bekræfte denne påstand, og den
største undersøgelse på området talte 565 malerier.  Selvom der findes
metoder fra billedbehandling til systematisk analyse af billeders
komposition, er disse ikke blevet taget i brug med henblik på at
undersøge hypotesen.

Vi har udviklet et program, som kan afgøre, hvorvidt en digital
gengivelse af et maleri har interessante regioner i det gyldne snit, og
kørt dette på 17,364 digitaliserede malerier. Regioner bliver først
kontrolleret for, hvorvidt de er interessante, dernæst for om de ligger
i det gyldne snit. En interessant region defineres som et ensfarvet
område i et billede, der er større end 0.2 \% af billedets størrelse og
optager mere end 1/4 af alle pixel i dens begrænsende rektangel. Med
vores udtrækning af regioner og to forskellige metoder til at vurdere,
hvorvidt interessante regioner ligger i det gyldne snit, kan vi ikke
afvise, at snittet er specielt æstetisk tiltalende. Ved naiv vurdering
havde $\mathsf{91.43\%}$ af de analyserede malerier mindst én
interessant region i det gyldne snit.  Resultaterne viste dog ingen
indikation på, at det gyldne snit adskiller sig signifikant fra andre
snit i malerier.  Der vises en tendens til at kunstnere foretrækker at
placere interessante regioner i midten, mens den øverste halvdel og
kanterne af maleriet ikke er at foretrække.

%Resultaterne fra
%analysen opbevares i en database, som tillader videre arbejde med data.
%Programmet kan desuden analysere andre snit i billedet end blot det
%gyldne, og det bliver således muligt at skelne mellem og sammenligne
%resultater fra andre snit --- f.eks, kan vi skelne mellem resultater for
%det gyldne snit og for snit ved to tredjedele.

%Vi har udviklet et program, som kan trække sammenhængende regioner i
%nærheden af det gyldne snit, ud af en digital gengivelse af et maleri.
%Disse tages derefter ud og vurderes efter nogle simple kriterier for at
%afgøre, om de ligger i det gyldne snit. En automatisering af denne
%analyse er blevet sammensat, hvor en database gemmer resultaterne som
%tillader videre arbejde med data. Programmet kan desuden analysere andre
%snit i billedet end blot det gyldne, og det bliver således muligt at
%skelne mellem og sammenligne resultater fra forskellige snit --- f.eks,
%vil man kunne skelne mellem resultater for det gyldene snit og for snit
%ved en tredjedel.

}

{
\section*{Diff}
\begin{itemize}
    \item Skrevet om to udvidelse (den første er implementeret)
    \item Lidt resultater
\end{itemize}
}

% vim: set tw=72 spell spelllang=da:


% vim: set tw=72 spell spelllang=da:


\tableofcontents
\listoftables
\listoffigures

\parskip=8pt plus 2pt minus 4pt

\chapter{Forord}
{
{\sffamily Dette dokument er den endelige rapport, udarbejdet i
forbindelse med kurset ``Bachelorprojekt'', som udbydes på Datalogisk
Institut ved Københavns Universitet. I de mere tekniske afsnit, især med
henblik på kapitel \ref{chap_implementation}, men også dele af kapitel
\ref{chap_detektion}, forventes det, at læseren har en basal viden inden
for datalogi, svarende til at have bestået de obligatoriske kurser på
bacheloruddannelsen af datalogistudiet\cite{DIKUkurser}. Kendskab til de
grundlæggende begreber i forbindelse med billedbehandling, vil endvidere
være en fordel. For en introduktion til billedbehandling henvises til
\cite{SIOlsen}. Kapitel \ref{chap_afproevning}, hvor vi ser på de
grafiske resultater fra programmet, og kapitel \ref{chap_resultater},
hvor vi præsenterer de videnskabelige resultater, kan umiddelbart læses
af alle, uden videre forudsætninger end ren og skær interesse, for emnet
omhandlende det gyldne snit og analyse af digitale gengivelser af
malerier.

\section*{Tak}
Der skal først sendes en tak ud til vores vejleder, Jakob Grue Simonsen.
Stephanie Bekkar, Franck Franck, Emma Haxen og Lisbeth Steenstrup skal
også have mange tak for korrekturlæsning. Fætter Jon og Ida Monrad
Graunbøl gives thumbs-up, for at hænge ud på kontoret og supplere
snacks. Endeligt vil vi takke vores hund, Jim Daggerthuggert og alle fra
Langestrand.

}
}

% vim: set tw=72 spell spelllang=da:


\chapter{Indledning}
\textsf{
Indsæt lam indledning til afsnit.
}

\subsection{Det gyldne snit}
{
Det vi kalder det gyldne snit, den gyldne ratio eller det guddommelige
forhold, blev allerede beskrevet i Euklids \emph{Elements} fra ca. 300
f.  Kr. som følger:
\begin{quote}
	\emph{``A straight line is said to have been cut in extreme
	and mean ratio when, as the whole line is to the greater
	segment, so is the greater to the less.''}\cite{Euclid300bc}
\end{quote}

\begin{figure}[h!]
	\begin{center}
		\includegraphics[scale=0.49,angle=0]{afsnit/baggrund/billeder/line_segment_a_c_b}
	\end{center}
	\caption{Euklids opdeling af et linjestykke}
	\label{line_segment}
\end{figure}

Givet et linjestykke $A\ B$, som vist i figur \ref{euclid}, og ud fra
Euklids beskrivelse kan $\varphi$ defineres som
\begin{equation}
	\varphi	= \frac{A\ C}{C\ B} = \frac{A\ B}{A\ C}
	\label{euclid}
\end{equation}
Ved at indsætte variable i ligning \ref{euclid} får vi
\begin{equation}
	\varphi = \frac{\varphi + 1}{\varphi}
	\label{expand_euclid}
\end{equation}
hvilket giver os andengradspolynomiet
\begin{equation}
	\varphi^{2} - \varphi - 1 = 0.
	\label{poly_phi}
\end{equation}
Hvis vi nu løser andengradspolynomiet i ligning \ref{poly_phi}, med
$\varphi > 0$, får vi
\begin{eqnarray*}
	\varphi	& =	& \frac{\sqrt{5} + 1}{2} \\
		& =	& 1.6180\ 3398\ 8749\ 8948\ 4820 \dots
\end{eqnarray*}

Tallet $\varphi$ bemærker sig blandt andet ved, at når det kvadreres, så
lægger man blot 1 til. Dette udledes trivielt fra ligning \ref{poly_phi}
\begin{equation}
	\varphi^{2} = \varphi + 1
	\label{phi_squared}
\end{equation}

Vi kan også finde polynomiets anden rod som angives ved $\varPhi$
\begin{eqnarray*}
	\varPhi & = & \frac{1}{\varphi} \\
		& = & \varphi - 1 \\
		& = & 0.6180\ 3398\ 8749\ 8948\ 4820 \dots 
\end{eqnarray*}
Også tallet $\varPhi$ er interessant idet dets eget kvadrat plus sig
selv giver 1. Vi har at
\begin{equation}
	\varPhi^{2} + \varPhi = 1
	\label{Phi_squared}
\end{equation}
hvilket kun er gældende for $\varPhi$.

Det gyldne snit fremviser endvidere en interessant forbindelse til
Fibonaccis talrække, da forholdet mellem to fibonaccital $F(n)$ og $F(n
- 1)$ konvergerer mod $\varphi$ når $n$ nærmer sig uendelig. Mere
formelt har vi at
\begin{eqnarray*}
	\varphi & =     & \lim_{n \rightarrow
	\infty}{\frac{F(n)}{F(n - 1)}}
\end{eqnarray*}

\subsubsection{Et gyldent rektangel}
På samme måde som vi kan opdele en linjestykke efter det gyldne snit,
kan vi konstruere et rektangel hvor forholdene mellem højde og bredde er
$\varphi$. Vi konstruerer et rektangel hvor alle sider er lig 1 og
tegner en diagonal fra dette rektangelels midte til modsatte hjørne. Med
denne diagonal som radius tegnes en cirkel som et gyldent rektangel kan
tegnes efter. Figur \ref{golden_rectangle} illustrerer denne metode.

\begin{figure}[h!]
	\begin{center}
		\includegraphics[scale=0.35,angle=0]{afsnit/baggrund/billeder/Golden_Rectangle_Construction}
	\end{center}
	\caption{Et gyldent rektangel - \emph{Kilde: Wikipedia}}
	\label{golden_rectangle}
\end{figure}
Det ses at rektanglet har forholdet $\varphi:1$ og at eksemplet er helt
analogt til det linjestykke givet i figur \ref{line_segment}. Dog skal
det bemærkes, at det rektangel der kan konstrueres af linjestykkerne 1
og $\varphi - 1$ også er et gyldent rektangel med forholdet $1:\varphi
-1 = \varphi$. Man kan derved konstruere gyldne rektangler ud i det
uendelige ved hele tiden at lave nye gyldne rektangler.

\subsubsection{Spiraler og det gyldne snit}
Når man som ovenfor, gentagende gange deler et gyldent rektangel kan man
bruge dette til at konstruere en gylden spiral. En gylden spiral kan
skrives ved ligningen for generelle logaritmiske spiraler som
\begin{equation}
	r = ae^{c\theta}
	\label{log_spiral_2}
\end{equation}
eller
\begin{equation}
	\theta = \frac{1}{c}\ln(r/a)
	\label{log_spiral_1}
\end{equation}
hvor $e$ er grundtallet for den naturlige logaritme og $c$ skal have en
speciel værdi for at kunne være en gylden spiral. Den gyldne spiral kan
approksimeres ved at konstruere en fibonaccispiral som vist i figur
\ref{fibonacci_spiral}.
\begin{figure}[h!]
	\begin{center}
		\includegraphics[scale=0.35,angle=0]{afsnit/baggrund/billeder/Fibonacci_spiral}
	\end{center}
	\caption{En fibonaccispiral - \emph{Kilde: Wikipedia}}
	\label{fibonacci_spiral}
\end{figure}
Da fibonaccispiralen er konstrueret efter Fibonaccis talrække nærmer
denne spiral sig en gylden spiral, men kan ikke betegnes som værende en
ægte gylden spiral.

Med den matematiske definition på plads kan vi kigge på den forskning
der er blevet gjort i forbindelse med det gyldne snit.

% vim: set tw=72 spell spelllang=da:


\subsection{Litteraturstudie}
{
Som allerede nævnt, kendte de gamle grækere til tallet $\varphi$, som
Euklid kaldte for \emph{the division in extreme and mean ratios},
forkortet DEMR. Luca Pacioli udgiver i 1509 bogen \emph{De divina
proportione}, som beskriver samme fænomen omtalt som ``den guddommelige
propertion''. Udtrykket ``det gyldne snit'' kan spores tilbage til
tyskeren Martin Ohm (1792 -- 1872), der første gang betegner Euklids DEMR
som \emph{der Goldener Schnitt} i sin bog \emph{Die reine
Elementar-Mathematik}, fra 1835\cite{Markowsky1992}. Det gyldne snit er
nu blevet den foretrukne betegnelse.

Nu bliver det påstået fra flere kilder, at det gyldne snit bruges i
malerier, arkitektur og musik, da dette forhold har specielt tiltalende
æstetiske
egenskaber\cite{GoldenNumber}\cite{RatioArt}\cite{Putz1995}\cite{Stakhov2006490}\cite{Boussora2004}.
Især \cite{GoldenNumber} er særlig ivrig og finder det gyldne snit i alt
lige fra cigaretpakker til skallen fra en nautil. Netop nautilskallen
bliver meget ofte brugt som argument for, at det gyldne snit findes i
naturen i form af en gylden spiral lignende den fra figur
\ref{fibonacci_spiral}. Hvis man rent faktisk måler efter,
viser det sig, at dette ikke er tilfældet. Som argumenteret i
\cite{Sharp2002} er spiralen i nautilskallen rigtigt nok logaritmisk,
men ikke med en faktor $c$ som ville gøre den til en gylden spiral.

Det græske tempel Parthenon spiller også en central rolle i rygterne om
det gyldne snit. Billeder af templet bliver tit vist med et rektangel
tegnet over. Der florerer forskellige udgaver af disse billeder, hvor
der åbenbart ikke tages højde for, at dele af templet ikke indgår i
rektanglet eller fotografiets perspektiv. George Markowsky gør i
\cite{Markowsky1992} op med mange af disse vrangforestillinger. Påstande
om, at Keopspyramiden skulle være bygget efter det gyldne snit --- som
angivet i \cite{Stakhov2006490} --- at nogle af Leonardo da Vincis malerier
er malet efter det gyldne snit, og at det gyldne rektangel er det mest
æstetisk tiltrækkende format\cite{GoldenNumber}\cite{RatioArt}, bliver i
hans artikel afvist, da mange af disse undersøgelser lider under det, han
kalder \emph{the Pyramidology Fallacy}. Dette udtrykt henter Markowsky
fra \cite{Gardner1952_2} og beskriver de personer, der forsker i
pseudovidenskab såsom pyramidernes arkitektur. Her har forskerne ofte
mange forskellige tal at ``jonglere'' med og kan frit vælge netop dem,
der giver det ønskede resultat. En anden forfatter, Roger
Hertz-Fischler, er ligeledes optaget af den specielle værdi, det gyldne
snit er blevet tillagt. Det som Markowsky og Gardner kalder for
\emph{Pyramidology}, betegner han som \emph{golden numberism}, og han har
fulgt dette fænomens historie tilbage til en tysk mand ved navn
Adolph Zeising (1810 -- 1876)\cite{Herz-Fischler2005}. Hertz-Fischler
hævder, at stort set alle undersøgelser inden for \emph{golden numberism}
kan spores tilbage til Zeising.

Hypotesen om det gyldne rektangels æstetiske egenskaber bliver også
taget op i \cite{Boselie1984} og \cite{Plug1980}, hvor det ikke kan
konkluderes, at det gyldne rektangel skulle have nogen æstetisk
signifikans. En lignende hypotese bliver sat på prøve i
\cite{McManus1995}, hvor det undersøges, om et billedes geometriske
komposition har indflydelse på dets æstetiske effekt. I. C. McManus
kunne, gennem tre eksperimenter, ikke finde noget grundlag for at et
billedes geometriske komposition påvirkede testpersonernes bedømmelse.
Det siges i konklusionen:

\begin{quote}
	\emph{``Together the results of these experiments throw
	considerable doubt upon the hypothesis of the implicit detection
	of latent compositional geometry as a major component of
	aesthetic judgements, at least for relatively unsophisticated
	observers[\dots]''}
\end{quote}

Selvom ovenstående ikke eksplicit nævner det gyldne snit, er der
alligevel en klar relevans til billeder opbygget geometrisk efter det
gyldne snit.

Der ses dog et klart problem, hvilket Markowsky også nævner, idet det
ikke er defineret, hvornår et stykke kunst er konstrueret efter det
gyldne snit. Ej heller er det defineret, præcis hvordan man finder det
gyldne snit: Der findes mange billeder, hvor snittet er illustreret vha.
linjer, men størstedelen af disse har ikke de fornødne mål, og det
gyldne snit findes gerne helt arbitrære steder uden nogen som helst
identificerbar fremgangsmåde.  Én undersøgelse, omhandlende billeders
komposition, har dog lavet statistik på 565 malerier, men kun forholdet
mellem lærredets dimensioner er blevet registreret\cite{Olariu1999}.
Denne undersøgelse kunne ikke konkludere, at kunstnere foretrækker det
gyldne snit i lærredet. Det er derfor interessant at forsøge at
automatisere søgningen efter det gyldne snit i malerier og fastsætte
klare kriterier for, hvornår et billede kan siges at være konstrueret
efter det gyldne snit.  Til denne opgave er det oplagt at udnytte
regnekraften fra computeren, hvilket også giver anledning til  at
analysere langt større datasæt.

\subsection{Eksisterende datalogisk forskning}
Den tidligere datalogiske forskning ligger i generelle teorier og algoritmer
indenfor billedbehandling end en tidligere forskning præcis indefor analyse af
interessante regioner i nærheden af det gyldne snit.
Det mest relaterede forskning er et eksperiment, som analysere æstetik i fotografier\cite{DattaWang}.
Det væsenlige ved dette eksperiment er hvorledes billederne
bedømmes. Her arbejdes der med 56 forskellige kandidater til at definere
et billede, som mere eller mindre æstetisk korrekt og originalt. 
Kandidaterne bliver udvalgt i forskellige kategorier, tre af kandidater 
bliver vurderet ud fra ``The rule of Thirds'', som er beskrevet ved
\begin{quote}
	``The rule can be considered as a sloppy approxmination of to the
	golden ratio (about 0.618)''
\end{quote}

%fyld? Meget fedt at få sagt det men jeg ved ikke om det er specielt
%vigtigt
Deres kandidater dækker over noget ganske centralt i store dele af
billedbehandling nemlig detektion af
interessante regioner i billeder, problemet er at begrebet slet ikke er
entydigt defineret, men skifter mening fra fagområde til fagområde.
Hvad vi opfatter som interessant vil blive diskuteret i \ref{section_kort_intro}

De forskellige metoder til at detektere interessante regioner, indeholder nogle avancerede tekniker.
Potentielt kunne disse tekniker forbedre udtrækning af interessante regioner.

Den første er maskineindlæring, der stammer fra feltet kunstig
intelligens indenfor datalogi. I billedbehandling bliver det dog
brugt til at finde interessante regioner med stor præcision, og formålet
med brugen er bland andet at finde ansigter, mennesker og
biler\cite{ViolaJones01}\cite{SchneidermanKanade00}\cite{Gabor}. Præcisionen på
algoritmernes detektion koster dog meget kompleksiteten i udvikligen 
af algoritmerne. Programmet skal også have en base for at blive oplært
til at finde interessante regioner, og derved låser programmet fast til
kun at finde det som oplæringsbasen er.

Den anden er opdelinger af billeder efter tekstur. Ideen bag denne type
af metoder er at opdele efter
overfladetekstur\cite{218442}\cite{CarsonBelongie02}\cite{PapageorgiouPoggio}, de passer meget bedre på
den type region der ledes efter. De besidder dog adskillige
negative egenskaber. En af disse er at de fleste algoritmer har svært
ved at fungere på støj i billedet, der er selvfølgelig veje at reducere
dette problem, dog er de meget beregningstunge.\cite{PalPal}

}
% vim: set tw=72 spell spelllang=da:


% vim: set tw=72 spell spelllang=da:


\chapter{Detektion af det gyldne snit}
\subsection{Opdeling af billeder}
\subsection*{Trunkerings- og afrundingsproblemer}
Mange af de metoder, vi bruger til at udregne, det gyldne
snits position eller størrelsen af en margin, udregnes med brøker. Hvorimod et
billede opbygges af pixels. Dette gør, at vi bliver nødt til at tage
approksimationer af udregningerne for at få dem tilbage til et helt tal.
Men hvor meget af data er gået tabt, og hvor mange af resultaterne kan
man stole på?

\subsubsection{Acceptabel afvigelse}
\note{Nogle referanse}Som beskrevet i afsnit \ref{mange_tal}, udregnes
det gyldne snit med mange decimaler. En kunstner, hvor god han end er,
har ingen chance for at male så præcist at man kan sige at strøget
ligger nøjagtigt oven på snittet selv om hans ententioner er at ramme
snittet. Vi kommer derfor til at have en vis uprecis hed på de data vi
få fra billedet selv. Vi starter med at se på alle de ting, som kan
skabe en usikkerhed fra malerens side. Man kan gå ud fra at den
procentvise afvigelse ikke er særlig stor, da vi ikke har bestræbelser
på at arbejde på abstrakt malerier, dog har vi sat den procentvise
afvigelse til 0.5 \%. Det vil sige at en maler med et lærred på 100
cm,maksimalt vil male $0.5$ cm forkert.

Når maleren vælger en ramme og et lærred, har vi igen problematikken,
selv om maleren spicifikt gå efter at bygge maleriet op efter det gyldne
snit, kan snittets placering i maleriet have forskubbet sig, ved dårlige
valg at ramme eller lærred. Derfor sætter vi den afvigelse til $1\%$. Da
vi igen mener at dette er den maksimale afvigelse, der kan opstå.

Når maleren maler en region i et maleri, forekommerer der normalt en
lille kant rundt om objektet, et omrids. Dette omrids kan vores
algoritmer ikke tage højde for, og vi må derfor modregne omridset, så vi
er sikre på at vi ser på regionen og ikke dens omrids. Da et omrids ikke
er særligt stort, har vi sat denne procentsats til $0.5\%$. 

Alt i alt giver det en afvigelse på vores aktuelle udtrækning af data
fra malerierne på $2\%$ det vil sige at finder vi en region, som ligger
på pixel 200, i et billedet, der er $500$ cm bredt, befinder den sig
faktisk i intervallet [190,210]. Måden hvorpå vi tager højde for den
forskel, er ved hjælp af marginer, som nævnt i afsnit
\ref{section_naiv}.

\subsection{Inddeling af billede efter snit}

I et billede betegnes højden og bredden som hhv. $H$ og $B$, se figur
\ref{cut}. Der er 4 gyldne snit, 2 vertikale og 2 horisontale, som vist
i figur \ref{lenasnit2}. For at finde ud af, hvor de 4 snit skal ligge i
billedet, multipliceres B og H med $\varPhi$,og man får to tal. Disse
betegner, hvor mange pixels, det gyldne snit befinder sig fra hhv. $H$
og $B$ f.eks vil de 2 horisontale snit, i et billedet, som har $B =
4000$ pixel, ligge hhv. $4000 \cdot \varPhi \approx 2472$ pixels fra
billedets øvre og nedre kant.

\begin{figure}[h]
	\begin{center}
		\includegraphics[scale=0.42,angle=0]{afsnit/vores_implementation/billeder/naiv_algoritme/Lenagolden}
	\end{center}
	\caption[]{Billedet som har indtegnet de fire gyldne snit}
	\label{lenasnit2}
\end{figure}

\begin{figure}[h]
	\begin{center}
		\includegraphics[scale=0.42,angle=0]{afsnit/vores_implementation/billeder/naiv_algoritme/Cut}
	\end{center}
	\caption[]{Billedets højde og bredde betegnes hvv. H og B. De 4 snit er navngivet.}
	\label{cut}
\end{figure}

De 4 snit tildeles hvert deres Id, "snit 0,1,2 og 3" så vi kan kende
forskel på de individuelde snit, Id'erne placering kan ses i figur
\ref{cut}. Vi vil i resten af rapporten kalde snittene efter deres Id.
Hvis vi gerne vil finde snittet som ligger i miden kommer der kun 2
snit, med vær deres Id "snit 0 og 1" som kan ses i figur \ref{Cut2}

\begin{figure}[h]
	\begin{center}
		\includegraphics[scale=0.42,angle=0]{afsnit/vores_implementation/billeder/naiv_algoritme/2Cut}
	\end{center}
	\caption[]{Billedet skæres her kun af 2 snit}
	\label{2Cut}
\end{figure}

\subsubsection{Heltal i det gyldne snit}

I eksemplet med 4000 pixels ovenfor, approksimerer vi antal pixels ved
at afrunde resultatet $2472.13595 \approx 2472$, se udregning
\ref{afrundning}. Det betyder at vi mister 0.13595 pixels i præction,
hvilket svarer til en misvisning af punktet på 0.00339875 $\%$ i forholdt
til $B$ på billedet. Se udregning \ref{afrundning2}.

\begin{equation}
	4000 \cdot \varPhi = 4000(\sqrt{5}-1)/2 = 2472.13595 \approx 2472 \label{afrundning}
\end{equation}

\begin{equation}
	0.13595/4000 \cdot 100 = 0.00339875 \label{afrundning2}
\end{equation}

Det er en meget lille del af selve billedet og skulle ikke give nogle
misvisninger i forhold til udregningen. For at gøre det lidt mere
generelt, sætter vi trunkeringsfejlen til $0.5$, da det er den maksimale
afrundingsfacktor som kan forekomme. Hvis billedet har en størrelse på
500 pixels, hvilket er det mindste billedet vi har, giver dette en fejlmargin
på $0.1 \%$. Dette tal bliver adderet til fejlsatsen ovenfor, og giver
en samlet afvigelse på $2.1\%$.

\subsubsection{Snitratio}
End til vider har vi kun arbejder med det gyldne snit, men andre snit i
billedet kan godt optræde, derfor indføre vi en nu betegnelse
snitratio, som betegner en procents sats for hvor lang inde i billedet
snittet befinder sig. Det vil sige at hvis en snitration er på $0.2$. Et
billedet har $B$ på 4000 vil et snit befinde sig i pixel $4000*0.2 =
800$.

\subsubsection{Heltal ved udregning af Margin}
Når vi har 2 forskellige snitratioer, f.eks. $\varPhi$ og $\frac{2}{3}$,
som ligger meget tæt på hinanden, og vi gerne vil sammenligne hvilken
regioner der ligger i snitratioernens snit, er det vigtigt at margin for
vært af de 2 snitratioers snit ikke krydser hinanden. 

Hvis margin krydser. Vil det indebære, at den samme region bliver fundet
af begge snit. Dette vil give et skævt billedet af forskellen på de to.
Derfor må vi sørge for at marginerne ikke krydser. Hvis $x$ betegner
antal pixels i $B$ eller $H$, og vi vil se på, hvor mange pixels, der er
mellem snitratio $\frac{2}{3}$ og $\varPhi$, multiplicerer vi $x$ med de
to snitratioen for at finde deres placering. Derefter subtraheres vi et
af de snit som befinder sig tetest på hinanden i vær snitratio med
hinanden.

\begin{eqnarray}
	\frac{x2}{3} - \frac{x2}{\sqrt{5}+1} & = & x(\frac{2}{3} - \frac{2}{\sqrt{5} + 1}) \nonumber \\
	& = & x(0.666667-0.618034) \\ \nonumber
	& = & x(0.048633)
\end{eqnarray}

Vi har nu fundet antal pixels mellem de to snit. Vi vil gerne undgå at
de to marginens ikke krydser hinanden, så der dividere vi med 2 og
afrunder værdien.

\begin{equation}
	\left\lfloor \frac{0.048633x}{2}\right\rfloor = \left\lfloor0.024316x \right\rfloor
\end{equation}

Tallet $2.4316$ er altså den minimale procentvise størrelse som vores
margin må have, nå vi sammen liner det gyldne snit og $\frac{2}{3}$.
Det betyder også at vi ikke må sammen line snit som ligger særlig
meget tætter på hinanden, da 0.021 er den minimale procent margin som vi må have.
$\lfloor 0.024316x \rfloor$ giver os et antal pixels som skildre de to snit. For at vise
hvor stort marginen egentlige kan være, bruger jeg denne formel på to
billeder, et som svare til vores mindste billedet, 500 pixels, og et
som svare til vores største billedet, 4000 pixels. Ved 500 pixels
bliver resultatet

\begin{equation}
	 \lfloor 500(0.024316)\rfloor = 12
\end{equation}

Det er en fint margin, da vores fejl på udregningerne ligger på 2.1 \%,
som svare til $\lceil 500*0.021 \rceil = 11$ pixels, som er 1 pixels fra vores
margin

Ved 4000 pixels giver det.

\begin{equation}
	 \left\lfloor 4000(0.024316)\right\rfloor = 97
\end{equation}

Som også er god nok da $4000*0.021 = 84$ pixels.
% vim: set tw=72 spell spelllang=da:



\chapter{Afprøvning}
\subsection{kort introduktion til prototype}
(hvorfor har vi valt lige disse billeder, eventuel noget med, der er mange hovder så der er nok også en vis cance for at der finder en hovde. lala
)
Ud fra de karakteriseringer som er beskravet i rapporten. få vi
produceret noget data, som omhandler tilblivelsen af region i et snit og
hvor mange der regioner der befinder sig. I dette afsnit vil vi vise
hvad vores prototype er kommet frem til. Ved at se på et vis række billeder
som illustrer hvad der sker i programmet, i grove trak. Der efter vil vi
konkludere, metode god og dårlige eneskaber, og om der kan gøres noget for at
forbedre på fremgangsmåden.


\subsection{Noget smart, kan bare ikke komme på det}
Nå et billedet bearbejdes, i vores program, er der 2 naturlige steder
hvor det vil være fornuftige at se om hvad der sker. Det første er efter vi har
iteraret hen over et af de 4 cut og fundet alle de regioner som ligger
på eller lige omkring snittet.

\begin{figure}[h!!]
	\begin{center}
		\includegraphics[scale=0.42,angle=0]{afsnit/afprovning/billeder/floodfillbilledet.png}
	\end{center}
	\caption[]{Billedet hvor vores algoritme har fundet x antal regioner}
	\label{ff}
\end{figure}

I billedet \ref{ff} er vær region, indtegnet med vær sin farve, og snittet
som er beregnet ud fra er det horisontale gyldne snit som ligger mest
til højre. Som man kan se finder algoritmen en masse regioner, hvor
nogle af dem er ret spændene at se på, som f.eks. drengen som er farvet
helt lyserød. Der bliver også fundet en masse regioner som ikke er
særlige spændene, f.eks. helle baggrunden som er malet mørkelilla i
billedet. Som man kan af billedet, er der ikke særlige mange regioner, i
selve snittet som bliver tværet ud og kommer til at falde i et med et af
de andre regioner i billedet, det er et godt tegn, da sammensmeltningen af
regioner vil give en forkert fortolkning af billedet.

Det andet naturlige sted at se på, er nå vi tager alle de regioner som
blev fundet og sortere, dem fra, som ikke passer overens med vores
definition, af om regionen ligger i snittet. De regionerne som er taget
med, bliver inremmet i en bos og kan ses på billedet \ref{ff}. 

\begin{figure}[h!!]
	\begin{center}
		\includegraphics[scale=0.42,angle=0]{afsnit/afprovning/billeder/boindingboxbilledet.png}
	\end{center}
	\caption[]{Billedet de regioner som vi har tilbage, er i en box}
	\label{blob}
\end{figure}

Som man kan se er der kun 1 region tilbage, som omkredser en sko i
billedet. Men tilgængel er alle de store regioner så som baggrunden og
græsset ikke blevet taget med. Det kan både siges at være godt og skidt.
det gode er at vi nu har en god ide om, hvad der bliver fundet. hvis vi
samme liner vores beskrivelse af vores fremgangsmåde, og hvordan
billedet ser ud. Stemmer det over ens, så vores algoritme virker skulle virke. Det
dårlige er, at vi fjerner ret meget. F.eks. kunne drengen være en
region som man gerne vil have med.

For at give lidt parspiktic, vil jeg også visser vi et par andet billedet, hvor
vores metode opføre sig en kende anderledes.
	
\begin{figure}[h!!]
	\begin{center}
		\includegraphics[scale=0.20,angle=0]{afsnit/afprovning/billeder/nicofloodfillbilledet.png}
	\end{center}
	\caption[]{Som man kan se på billedet, er der kun en stor region.
	           Grunden til dette er at billedet er af en anden kunstner
	           og er blevet lidt gråligt, efter alle de år på muserum,
	           så farverne ligger ikke særlige lang for hinanden. Dette
	           få vores metode til at fejle en kende.}
	\label{nicofill}
\end{figure}

\begin{figure}[h!!]
	\begin{center}
		\includegraphics[scale=0.30,angle=0]{afsnit/afprovning/billeder/BB1annunc2.png}
	\end{center}
	\caption[]{Billedet af en engel, hvor noget af glorien og hinden
	           hånd er fundet. som det kan ses ligger hånden efter vores
	           defination i det gyldne snit, og er en ret interasat
	           ting, da hånden viser hvad engelen hendtyr, selve glorien
	           er også i interasant, men ligger ikke i snittet, så det
	           er lidt uheldig at vi finder en fraktion af dem.}
	\label{BB1annunc2}
\end{figure}

\begin{figure}[h!!]
	\begin{center}
		\includegraphics[scale=0.30,angle=0]{afsnit/afprovning/billeder/BB09crucif.png}
	\end{center}
	\caption[]{Metoden finder i dette billedet svanen, som er en meget
	           framtradene ting i billedet, metoden finder også skildet,
	           selve om det er svært at se, dette er et ret god eksempel
	           på at vore metode faktisk finder nogle lunde pracice
	           ragioner, dog kan man se at metoden ikke har det så god
	           nå vi arbejder på firkandet billeder}
	\label{BB09crucif}
\end{figure}

\begin{figure}[h!!]
	\begin{center}
		\includegraphics[scale=0.3,angle=0]{afsnit/afprovning/billeder/FFadorat.png}
	\end{center}
	\caption[]{Dette floodfillet billedet, vise at selv om vi har en
	           metode som virker ok, så har vi stadig problemer nå
	           billedet er rundt og ikke gå helt ud til kanten af
	           rammen, få det ser faktisk ud som om kvinden og manden
	           ligger i snittet. Men dette kan ændre sig meget, da det
	           snit som er tegnet ind, er udregnet ud fra kanten af
	           rammen, og ikke billedet kant.}
	\label{FFadorat}
\end{figure}

\begin{figure}[h!!]
	\begin{center}
		\includegraphics[scale=0.50,angle=0]{afsnit/afprovning/billeder/BBallegory.png}
	\end{center}
	\caption[]{Dette er et fantastisk eksempel på at der ikke ligger
	           noget som helst i dette snit, og der er heller ikke
	           findet nogle ragioner i snittet.}
	\label{BBallegory}
\end{figure}

\begin{figure}[h!!]
	\begin{center}
		\includegraphics[scale=0.35,angle=0]{afsnit/afprovning/billeder/BBCarruingcut2.png}
	\end{center}
	\caption[]{I dette billedet finder vi 4 ragioner som ligger i det
	           gyldne snit, 3 hovder og et barne omris. Dette passer god}
	\label{BBCarruingcut2}
\end{figure}


\begin{figure}[h!!]
	\begin{center}
		\includegraphics[scale=0.50,angle=0]{afsnit/afprovning/billeder/BBMonalisacut1.png}
	\end{center}
	\caption[]{Monalisa med ved cut 1 , hvor der bliver fundet hindes overbryst og hindes pande, dette tyder på at noget af monalisas krop befinder sig inde i det gyldne snit}
	\label{BBMonalisacut1}
\end{figure}


\subsection{hvad kan vi gøre bedre}
I det første billedet \ref{ff}, mener vi at den ting som virkelige kan
forbedre måde vores metode arbejder på. Er ved at ændre på den
definition som sortere regioner fra. Så en region kan krydse begge
marginer. Den anden ting man kunne se på, omhandler det andet billedet
\ref{nicofill}, hvor vi få for meget med, dette kunne løses ved at se på
individuelde billeder og regne ud hvor meget algoritmerne må tage med.
Begge disse to forslag vil blive diskuteret yderligt i den udvidet
løsning af problemet, hvor vi vil gå ind på hvad vi gør for at kommer
over nogle at de problemer som vi har.


\chapter{Videnskabelige resultater}

% sort in citation order
\bibliographystyle{unsrt}
\bibliography{litteraturliste}
\addcontentsline{toc}{chapter}{Litteratur}

\chapter{Bilag}
\appendix

%\section{Kildekode}
%\input{materialer.tex}
%\newpage

%kan ikke kommer på nettet så har lavet nogle spørgsmål her
% kan det passe at ting i bagrundet er mere sløret og falder mere i et med helle malerriet?
% vi skal huske at snakke om tressholds

\end{document}
