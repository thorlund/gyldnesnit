% vim: set tw=72 spell spelllang=da:
\documentclass[a4paper, 10pt, danish, final]{report}
%%%%%%%%%%%%%%%%%%%%%%%%%%%%%%%%%%%%%%%%%%%%%%%%%%%%%%%%%%%%%%%%%
% Pakker
%%%%%%%%%%%%%%%%%%%%%%%%%%%%%%%%%%%%%%%%%%%%%%%%%%%%%%%%%%%%%%%%%
%\usepackage{a4wide}
\usepackage[danish]{babel}
\usepackage[utf8]{inputenc}
\usepackage[T1]{fontenc}

\usepackage{charter}
\usepackage{verbatim}
\usepackage{amsfonts}
\usepackage{amsmath}
\usepackage{amssymb}
%\usepackage{mathrsfs}
%\usepackage[mathcal]{euscript}
\usepackage{listings}
\usepackage{graphicx}
\usepackage{multirow}
\usepackage{hyperref}
\usepackage{cite}
\usepackage{float}
\usepackage[small,bf]{caption}
\usepackage{xypic}
\usepackage[table]{xcolor}
\usepackage{subfig}

%%%%%%%%%%%%%%%%%%%%%%%%%%%%%%%%%%%%%%%%%%%%%%%%%%%%%%%%%%%%%%%%
% Indstillinger
%%%%%%%%%%%%%%%%%%%%%%%%%%%%%%%%%%%%%%%%%%%%%%%%%%%%%%%%%%%%%%%%
\parindent=5pt
\lstset{language=Python, basicstyle=\scriptsize, showstringspaces=false, numbers=none, stepnumber=1, numberstyle=\tiny}
\setcounter{tocdepth}{2}

%%%%%%%%%%%%%%%%%%%%%%%%%%%%%%%%%%%%%%%%%%%%%%%%%%%%%%%%%%%%%%%%
% Kommandoer
%%%%%%%%%%%%%%%%%%%%%%%%%%%%%%%%%%%%%%%%%%%%%%%%%%%%%%%%%%%%%%%%
%\newcommand{\old}[1]{\oldstylenums{#1}}
%\newcommand{\old}[1]{{#1}}
\newcommand{\mailto}[1]{\href{mailto:#1}{#1}}
\newcommand{\resume}[1]{\begin{abstract}{\sffamily #1}\end{abstract}}
%\renewcommand{\thefigure}{\thechapter.\thesection.\arabic{figure}}
%\renewcommand{\thetable}{\thechapter.\thesection.\arabic{table}}

%%%%%%%%%%%%%%%%%%%%%%%%%%%%%%%%%%%%%%%%%%%%%%%%%%%%%%%%%%%%%%%
% Titel, forfatter og dato
%%%%%%%%%%%%%%%%%%%%%%%%%%%%%%%%%%%%%%%%%%%%%%%%%%%%%%%%%%%%%%%
\title{Detektion af ``Det Gyldne Snit'' i digitaliserede malerier}

\author{Ulrik Bonde - \mailto{bonde@diku.dk}\\
Kasper Steenstrup - \mailto{khsj@diku.dk}\\
Morten Thorlund - \mailto{thorlund@diku.dk}\\
\\
Vejleder\\Jakob Grue Simonsen}
\date{\today}

\hypersetup{
colorlinks,%
citecolor=black,%
filecolor=black,%
linkcolor=black,%
urlcolor=black,%
bookmarksopen=false,
pdftitle={Detektion af Det Gyldne Snit i digitaliserede malerier},
pdfauthor={Ulrik Bonde, Kasper Steenstrup og Morten Thorlund},
pdfsubject={Computer Vision},
pdfkeywords={Det gyldne snit, computer vision}
}

%%%%%%%%%%%%%%%%%%%%%%%%%%%%%%%%%%%%%%%%%%%%%%%%%%%%%%%%%%%%%%
% Indhold
%%%%%%%%%%%%%%%%%%%%%%%%%%%%%%%%%%%%%%%%%%%%%%%%%%%%%%%%%%%%%%
\begin{document}
\maketitle
\pagenumbering{roman}
\thispagestyle{empty}
%\pagestyle{headings}
\resume{
Det er den generelle opfattelse, at det gyldne snit er specielt æstetisk
tiltalende, og at man i kunstmalerier finder flest interessante regioner
her.  Litteraturen kan ikke entydigt bekræfte denne påstand, og den
største undersøgelse på området talte 565 malerier.  Selvom der findes
metoder fra billedbehandling til systematisk analyse af billeders
komposition, er disse ikke blevet taget i brug med henblik på at
undersøge hypotesen.

Vi har udviklet et program, som kan afgøre, hvorvidt en digital
gengivelse af et maleri har interessante regioner i det gyldne snit, og
kørt dette på 17,364 digitaliserede malerier. Regioner bliver først
kontrolleret for, hvorvidt de er interessante, dernæst for om de ligger
i det gyldne snit. En interessant region defineres som et ensfarvet
område i et billede, der er større end 0.2 \% af billedets størrelse og
optager mere end 1/4 af alle pixel i dens begrænsende rektangel. Med
vores udtrækning af regioner og to forskellige metoder til at vurdere,
hvorvidt interessante regioner ligger i det gyldne snit, kan vi ikke
afvise, at snittet er specielt æstetisk tiltalende. Ved naiv vurdering
havde $\mathsf{91.43\%}$ af de analyserede malerier mindst én
interessant region i det gyldne snit.  Resultaterne viste dog ingen
indikation på, at det gyldne snit adskiller sig signifikant fra andre
snit i malerier.  Der vises en tendens til at kunstnere foretrækker at
placere interessante regioner i midten, mens den øverste halvdel og
kanterne af maleriet ikke er at foretrække.

%Resultaterne fra
%analysen opbevares i en database, som tillader videre arbejde med data.
%Programmet kan desuden analysere andre snit i billedet end blot det
%gyldne, og det bliver således muligt at skelne mellem og sammenligne
%resultater fra andre snit --- f.eks, kan vi skelne mellem resultater for
%det gyldne snit og for snit ved to tredjedele.

%Vi har udviklet et program, som kan trække sammenhængende regioner i
%nærheden af det gyldne snit, ud af en digital gengivelse af et maleri.
%Disse tages derefter ud og vurderes efter nogle simple kriterier for at
%afgøre, om de ligger i det gyldne snit. En automatisering af denne
%analyse er blevet sammensat, hvor en database gemmer resultaterne som
%tillader videre arbejde med data. Programmet kan desuden analysere andre
%snit i billedet end blot det gyldne, og det bliver således muligt at
%skelne mellem og sammenligne resultater fra forskellige snit --- f.eks,
%vil man kunne skelne mellem resultater for det gyldene snit og for snit
%ved en tredjedel.

}

{
\section*{Diff}
\begin{itemize}
    \item Skrevet om to udvidelse (den første er implementeret)
    \item Lidt resultater
\end{itemize}
}

% vim: set tw=72 spell spelllang=da:


% vim: set tw=72 spell spelllang=da:


\chapter*{Forord}
\addcontentsline{toc}{chapter}{Forord}
{
{\sffamily Dette dokument er den endelige rapport, udarbejdet i
forbindelse med kurset ``Bachelorprojekt'', som udbydes på Datalogisk
Institut ved Københavns Universitet. I de mere tekniske afsnit, især med
henblik på kapitel \ref{chap_implementation}, men også dele af kapitel
\ref{chap_detektion}, forventes det, at læseren har en basal viden inden
for datalogi, svarende til at have bestået de obligatoriske kurser på
bacheloruddannelsen af datalogistudiet\cite{DIKUkurser}. Kendskab til de
grundlæggende begreber i forbindelse med billedbehandling, vil endvidere
være en fordel. For en introduktion til billedbehandling henvises til
\cite{SIOlsen}. Kapitel \ref{chap_afproevning}, hvor vi ser på de
grafiske resultater fra programmet, og kapitel \ref{chap_resultater},
hvor vi præsenterer de videnskabelige resultater, kan umiddelbart læses
af alle, uden videre forudsætninger end ren og skær interesse, for emnet
omhandlende det gyldne snit og analyse af digitale gengivelser af
malerier.

\section*{Tak}
Der skal først sendes en tak ud til vores vejleder, Jakob Grue Simonsen.
Stephanie Bekkar, Franck Franck, Emma Haxen og Lisbeth Steenstrup skal
også have mange tak for korrekturlæsning. Fætter Jon og Ida Monrad
Graunbøl gives thumbs-up, for at hænge ud på kontoret og supplere
snacks. Endeligt vil vi takke vores hund, Jim Daggerthuggert og alle fra
Langestrand.

}
}

% vim: set tw=72 spell spelllang=da:


\tableofcontents
%\listoftables
%\listoffigures

\parskip=8pt plus 2pt minus 4pt

\chapter{Indledning}
\pagenumbering{arabic}
\textsf{
Indsæt lam indledning til afsnit.
}

\subsection{Det gyldne snit}
{
Det vi kalder det gyldne snit, den gyldne ratio eller det guddommelige
forhold, blev allerede beskrevet i Euklids \emph{Elements} fra ca. 300
f.  Kr. som følger:
\begin{quote}
	\emph{``A straight line is said to have been cut in extreme
	and mean ratio when, as the whole line is to the greater
	segment, so is the greater to the less.''}\cite{Euclid300bc}
\end{quote}

\begin{figure}[h!]
	\begin{center}
		\includegraphics[scale=0.49,angle=0]{afsnit/baggrund/billeder/line_segment_a_c_b}
	\end{center}
	\caption{Euklids opdeling af et linjestykke}
	\label{line_segment}
\end{figure}

Givet et linjestykke $A\ B$, som vist i figur \ref{euclid}, og ud fra
Euklids beskrivelse kan $\varphi$ defineres som
\begin{equation}
	\varphi	= \frac{A\ C}{C\ B} = \frac{A\ B}{A\ C}
	\label{euclid}
\end{equation}
Ved at indsætte variable i ligning \ref{euclid} får vi
\begin{equation}
	\varphi = \frac{\varphi + 1}{\varphi}
	\label{expand_euclid}
\end{equation}
hvilket giver os andengradspolynomiet
\begin{equation}
	\varphi^{2} - \varphi - 1 = 0.
	\label{poly_phi}
\end{equation}
Hvis vi nu løser andengradspolynomiet i ligning \ref{poly_phi}, med
$\varphi > 0$, får vi
\begin{eqnarray*}
	\varphi	& =	& \frac{\sqrt{5} + 1}{2} \\
		& =	& 1.6180\ 3398\ 8749\ 8948\ 4820 \dots
\end{eqnarray*}

Tallet $\varphi$ bemærker sig blandt andet ved, at når det kvadreres, så
lægger man blot 1 til. Dette udledes trivielt fra ligning \ref{poly_phi}
\begin{equation}
	\varphi^{2} = \varphi + 1
	\label{phi_squared}
\end{equation}

Vi kan også finde polynomiets anden rod som angives ved $\varPhi$
\begin{eqnarray*}
	\varPhi & = & \frac{1}{\varphi} \\
		& = & \varphi - 1 \\
		& = & 0.6180\ 3398\ 8749\ 8948\ 4820 \dots 
\end{eqnarray*}
Også tallet $\varPhi$ er interessant idet dets eget kvadrat plus sig
selv giver 1. Vi har at
\begin{equation}
	\varPhi^{2} + \varPhi = 1
	\label{Phi_squared}
\end{equation}
hvilket kun er gældende for $\varPhi$.

Det gyldne snit fremviser endvidere en interessant forbindelse til
Fibonaccis talrække, da forholdet mellem to fibonaccital $F(n)$ og $F(n
- 1)$ konvergerer mod $\varphi$ når $n$ nærmer sig uendelig. Mere
formelt har vi at
\begin{eqnarray*}
	\varphi & =     & \lim_{n \rightarrow
	\infty}{\frac{F(n)}{F(n - 1)}}
\end{eqnarray*}

\subsubsection{Et gyldent rektangel}
På samme måde som vi kan opdele en linjestykke efter det gyldne snit,
kan vi konstruere et rektangel hvor forholdene mellem højde og bredde er
$\varphi$. Vi konstruerer et rektangel hvor alle sider er lig 1 og
tegner en diagonal fra dette rektangelels midte til modsatte hjørne. Med
denne diagonal som radius tegnes en cirkel som et gyldent rektangel kan
tegnes efter. Figur \ref{golden_rectangle} illustrerer denne metode.

\begin{figure}[h!]
	\begin{center}
		\includegraphics[scale=0.35,angle=0]{afsnit/baggrund/billeder/Golden_Rectangle_Construction}
	\end{center}
	\caption{Et gyldent rektangel - \emph{Kilde: Wikipedia}}
	\label{golden_rectangle}
\end{figure}
Det ses at rektanglet har forholdet $\varphi:1$ og at eksemplet er helt
analogt til det linjestykke givet i figur \ref{line_segment}. Dog skal
det bemærkes, at det rektangel der kan konstrueres af linjestykkerne 1
og $\varphi - 1$ også er et gyldent rektangel med forholdet $1:\varphi
-1 = \varphi$. Man kan derved konstruere gyldne rektangler ud i det
uendelige ved hele tiden at lave nye gyldne rektangler.

\subsubsection{Spiraler og det gyldne snit}
Når man som ovenfor, gentagende gange deler et gyldent rektangel kan man
bruge dette til at konstruere en gylden spiral. En gylden spiral kan
skrives ved ligningen for generelle logaritmiske spiraler som
\begin{equation}
	r = ae^{c\theta}
	\label{log_spiral_2}
\end{equation}
eller
\begin{equation}
	\theta = \frac{1}{c}\ln(r/a)
	\label{log_spiral_1}
\end{equation}
hvor $e$ er grundtallet for den naturlige logaritme og $c$ skal have en
speciel værdi for at kunne være en gylden spiral. Den gyldne spiral kan
approksimeres ved at konstruere en fibonaccispiral som vist i figur
\ref{fibonacci_spiral}.
\begin{figure}[h!]
	\begin{center}
		\includegraphics[scale=0.35,angle=0]{afsnit/baggrund/billeder/Fibonacci_spiral}
	\end{center}
	\caption{En fibonaccispiral - \emph{Kilde: Wikipedia}}
	\label{fibonacci_spiral}
\end{figure}
Da fibonaccispiralen er konstrueret efter Fibonaccis talrække nærmer
denne spiral sig en gylden spiral, men kan ikke betegnes som værende en
ægte gylden spiral.

Med den matematiske definition på plads kan vi kigge på den forskning
der er blevet gjort i forbindelse med det gyldne snit.

% vim: set tw=72 spell spelllang=da:


\subsection{Litteraturstudie}
{
Som allerede nævnt, kendte de gamle grækere til tallet $\varphi$, som
Euklid kaldte for \emph{the division in extreme and mean ratios},
forkortet DEMR. Luca Pacioli udgiver i 1509 bogen \emph{De divina
proportione}, som beskriver samme fænomen omtalt som ``den guddommelige
propertion''. Udtrykket ``det gyldne snit'' kan spores tilbage til
tyskeren Martin Ohm (1792 -- 1872), der første gang betegner Euklids DEMR
som \emph{der Goldener Schnitt} i sin bog \emph{Die reine
Elementar-Mathematik}, fra 1835\cite{Markowsky1992}. Det gyldne snit er
nu blevet den foretrukne betegnelse.

Nu bliver det påstået fra flere kilder, at det gyldne snit bruges i
malerier, arkitektur og musik, da dette forhold har specielt tiltalende
æstetiske
egenskaber\cite{GoldenNumber}\cite{RatioArt}\cite{Putz1995}\cite{Stakhov2006490}\cite{Boussora2004}.
Især \cite{GoldenNumber} er særlig ivrig og finder det gyldne snit i alt
lige fra cigaretpakker til skallen fra en nautil. Netop nautilskallen
bliver meget ofte brugt som argument for, at det gyldne snit findes i
naturen i form af en gylden spiral lignende den fra figur
\ref{fibonacci_spiral}. Hvis man rent faktisk måler efter,
viser det sig, at dette ikke er tilfældet. Som argumenteret i
\cite{Sharp2002} er spiralen i nautilskallen rigtigt nok logaritmisk,
men ikke med en faktor $c$ som ville gøre den til en gylden spiral.

Det græske tempel Parthenon spiller også en central rolle i rygterne om
det gyldne snit. Billeder af templet bliver tit vist med et rektangel
tegnet over. Der florerer forskellige udgaver af disse billeder, hvor
der åbenbart ikke tages højde for, at dele af templet ikke indgår i
rektanglet eller fotografiets perspektiv. George Markowsky gør i
\cite{Markowsky1992} op med mange af disse vrangforestillinger. Påstande
om, at Keopspyramiden skulle være bygget efter det gyldne snit --- som
angivet i \cite{Stakhov2006490} --- at nogle af Leonardo da Vincis malerier
er malet efter det gyldne snit, og at det gyldne rektangel er det mest
æstetisk tiltrækkende format\cite{GoldenNumber}\cite{RatioArt}, bliver i
hans artikel afvist, da mange af disse undersøgelser lider under det, han
kalder \emph{the Pyramidology Fallacy}. Dette udtrykt henter Markowsky
fra \cite{Gardner1952_2} og beskriver de personer, der forsker i
pseudovidenskab såsom pyramidernes arkitektur. Her har forskerne ofte
mange forskellige tal at ``jonglere'' med og kan frit vælge netop dem,
der giver det ønskede resultat. En anden forfatter, Roger
Hertz-Fischler, er ligeledes optaget af den specielle værdi, det gyldne
snit er blevet tillagt. Det som Markowsky og Gardner kalder for
\emph{Pyramidology}, betegner han som \emph{golden numberism}, og han har
fulgt dette fænomens historie tilbage til en tysk mand ved navn
Adolph Zeising (1810 -- 1876)\cite{Herz-Fischler2005}. Hertz-Fischler
hævder, at stort set alle undersøgelser inden for \emph{golden numberism}
kan spores tilbage til Zeising.

Hypotesen om det gyldne rektangels æstetiske egenskaber bliver også
taget op i \cite{Boselie1984} og \cite{Plug1980}, hvor det ikke kan
konkluderes, at det gyldne rektangel skulle have nogen æstetisk
signifikans. En lignende hypotese bliver sat på prøve i
\cite{McManus1995}, hvor det undersøges, om et billedes geometriske
komposition har indflydelse på dets æstetiske effekt. I. C. McManus
kunne, gennem tre eksperimenter, ikke finde noget grundlag for at et
billedes geometriske komposition påvirkede testpersonernes bedømmelse.
Det siges i konklusionen:

\begin{quote}
	\emph{``Together the results of these experiments throw
	considerable doubt upon the hypothesis of the implicit detection
	of latent compositional geometry as a major component of
	aesthetic judgements, at least for relatively unsophisticated
	observers[\dots]''}
\end{quote}

Selvom ovenstående ikke eksplicit nævner det gyldne snit, er der
alligevel en klar relevans til billeder opbygget geometrisk efter det
gyldne snit.

Der ses dog et klart problem, hvilket Markowsky også nævner, idet det
ikke er defineret, hvornår et stykke kunst er konstrueret efter det
gyldne snit. Ej heller er det defineret, præcis hvordan man finder det
gyldne snit: Der findes mange billeder, hvor snittet er illustreret vha.
linjer, men størstedelen af disse har ikke de fornødne mål, og det
gyldne snit findes gerne helt arbitrære steder uden nogen som helst
identificerbar fremgangsmåde.  Én undersøgelse, omhandlende billeders
komposition, har dog lavet statistik på 565 malerier, men kun forholdet
mellem lærredets dimensioner er blevet registreret\cite{Olariu1999}.
Denne undersøgelse kunne ikke konkludere, at kunstnere foretrækker det
gyldne snit i lærredet. Det er derfor interessant at forsøge at
automatisere søgningen efter det gyldne snit i malerier og fastsætte
klare kriterier for, hvornår et billede kan siges at være konstrueret
efter det gyldne snit.  Til denne opgave er det oplagt at udnytte
regnekraften fra computeren, hvilket også giver anledning til  at
analysere langt større datasæt.

\subsection{Eksisterende datalogisk forskning}
Den tidligere datalogiske forskning ligger i generelle teorier og algoritmer
indenfor billedbehandling end en tidligere forskning præcis indefor analyse af
interessante regioner i nærheden af det gyldne snit.
Det mest relaterede forskning er et eksperiment, som analysere æstetik i fotografier\cite{DattaWang}.
Det væsenlige ved dette eksperiment er hvorledes billederne
bedømmes. Her arbejdes der med 56 forskellige kandidater til at definere
et billede, som mere eller mindre æstetisk korrekt og originalt. 
Kandidaterne bliver udvalgt i forskellige kategorier, tre af kandidater 
bliver vurderet ud fra ``The rule of Thirds'', som er beskrevet ved
\begin{quote}
	``The rule can be considered as a sloppy approxmination of to the
	golden ratio (about 0.618)''
\end{quote}

%fyld? Meget fedt at få sagt det men jeg ved ikke om det er specielt
%vigtigt
Deres kandidater dækker over noget ganske centralt i store dele af
billedbehandling nemlig detektion af
interessante regioner i billeder, problemet er at begrebet slet ikke er
entydigt defineret, men skifter mening fra fagområde til fagområde.
Hvad vi opfatter som interessant vil blive diskuteret i \ref{section_kort_intro}

De forskellige metoder til at detektere interessante regioner, indeholder nogle avancerede tekniker.
Potentielt kunne disse tekniker forbedre udtrækning af interessante regioner.

Den første er maskineindlæring, der stammer fra feltet kunstig
intelligens indenfor datalogi. I billedbehandling bliver det dog
brugt til at finde interessante regioner med stor præcision, og formålet
med brugen er bland andet at finde ansigter, mennesker og
biler\cite{ViolaJones01}\cite{SchneidermanKanade00}\cite{Gabor}. Præcisionen på
algoritmernes detektion koster dog meget kompleksiteten i udvikligen 
af algoritmerne. Programmet skal også have en base for at blive oplært
til at finde interessante regioner, og derved låser programmet fast til
kun at finde det som oplæringsbasen er.

Den anden er opdelinger af billeder efter tekstur. Ideen bag denne type
af metoder er at opdele efter
overfladetekstur\cite{218442}\cite{CarsonBelongie02}\cite{PapageorgiouPoggio}, de passer meget bedre på
den type region der ledes efter. De besidder dog adskillige
negative egenskaber. En af disse er at de fleste algoritmer har svært
ved at fungere på støj i billedet, der er selvfølgelig veje at reducere
dette problem, dog er de meget beregningstunge.\cite{PalPal}

}
% vim: set tw=72 spell spelllang=da:


% vim: set tw=72 spell spelllang=da:


\chapter{Detektion af det gyldne snit}
\subsection{Opdeling af billeder}
\subsection*{Trunkerings- og afrundingsproblemer}
Mange af de metoder, vi bruger til at udregne, det gyldne
snits position eller størrelsen af en margin, udregnes med brøker. Hvorimod et
billede opbygges af pixels. Dette gør, at vi bliver nødt til at tage
approksimationer af udregningerne for at få dem tilbage til et helt tal.
Men hvor meget af data er gået tabt, og hvor mange af resultaterne kan
man stole på?

\subsubsection{Acceptabel afvigelse}
\note{Nogle referanse}Som beskrevet i afsnit \ref{mange_tal}, udregnes
det gyldne snit med mange decimaler. En kunstner, hvor god han end er,
har ingen chance for at male så præcist at man kan sige at strøget
ligger nøjagtigt oven på snittet selv om hans ententioner er at ramme
snittet. Vi kommer derfor til at have en vis uprecis hed på de data vi
få fra billedet selv. Vi starter med at se på alle de ting, som kan
skabe en usikkerhed fra malerens side. Man kan gå ud fra at den
procentvise afvigelse ikke er særlig stor, da vi ikke har bestræbelser
på at arbejde på abstrakt malerier, dog har vi sat den procentvise
afvigelse til 0.5 \%. Det vil sige at en maler med et lærred på 100
cm,maksimalt vil male $0.5$ cm forkert.

Når maleren vælger en ramme og et lærred, har vi igen problematikken,
selv om maleren spicifikt gå efter at bygge maleriet op efter det gyldne
snit, kan snittets placering i maleriet have forskubbet sig, ved dårlige
valg at ramme eller lærred. Derfor sætter vi den afvigelse til $1\%$. Da
vi igen mener at dette er den maksimale afvigelse, der kan opstå.

Når maleren maler en region i et maleri, forekommerer der normalt en
lille kant rundt om objektet, et omrids. Dette omrids kan vores
algoritmer ikke tage højde for, og vi må derfor modregne omridset, så vi
er sikre på at vi ser på regionen og ikke dens omrids. Da et omrids ikke
er særligt stort, har vi sat denne procentsats til $0.5\%$. 

Alt i alt giver det en afvigelse på vores aktuelle udtrækning af data
fra malerierne på $2\%$ det vil sige at finder vi en region, som ligger
på pixel 200, i et billedet, der er $500$ cm bredt, befinder den sig
faktisk i intervallet [190,210]. Måden hvorpå vi tager højde for den
forskel, er ved hjælp af marginer, som nævnt i afsnit
\ref{section_naiv}.

\subsection{Inddeling af billede efter snit}

I et billede betegnes højden og bredden som hhv. $H$ og $B$, se figur
\ref{cut}. Der er 4 gyldne snit, 2 vertikale og 2 horisontale, som vist
i figur \ref{lenasnit2}. For at finde ud af, hvor de 4 snit skal ligge i
billedet, multipliceres B og H med $\varPhi$,og man får to tal. Disse
betegner, hvor mange pixels, det gyldne snit befinder sig fra hhv. $H$
og $B$ f.eks vil de 2 horisontale snit, i et billedet, som har $B =
4000$ pixel, ligge hhv. $4000 \cdot \varPhi \approx 2472$ pixels fra
billedets øvre og nedre kant.

\begin{figure}[h]
	\begin{center}
		\includegraphics[scale=0.42,angle=0]{afsnit/vores_implementation/billeder/naiv_algoritme/Lenagolden}
	\end{center}
	\caption[]{Billedet som har indtegnet de fire gyldne snit}
	\label{lenasnit2}
\end{figure}

\begin{figure}[h]
	\begin{center}
		\includegraphics[scale=0.42,angle=0]{afsnit/vores_implementation/billeder/naiv_algoritme/Cut}
	\end{center}
	\caption[]{Billedets højde og bredde betegnes hvv. H og B. De 4 snit er navngivet.}
	\label{cut}
\end{figure}

De 4 snit tildeles hvert deres Id, "snit 0,1,2 og 3" så vi kan kende
forskel på de individuelde snit, Id'erne placering kan ses i figur
\ref{cut}. Vi vil i resten af rapporten kalde snittene efter deres Id.
Hvis vi gerne vil finde snittet som ligger i miden kommer der kun 2
snit, med vær deres Id "snit 0 og 1" som kan ses i figur \ref{Cut2}

\begin{figure}[h]
	\begin{center}
		\includegraphics[scale=0.42,angle=0]{afsnit/vores_implementation/billeder/naiv_algoritme/2Cut}
	\end{center}
	\caption[]{Billedet skæres her kun af 2 snit}
	\label{2Cut}
\end{figure}

\subsubsection{Heltal i det gyldne snit}

I eksemplet med 4000 pixels ovenfor, approksimerer vi antal pixels ved
at afrunde resultatet $2472.13595 \approx 2472$, se udregning
\ref{afrundning}. Det betyder at vi mister 0.13595 pixels i præction,
hvilket svarer til en misvisning af punktet på 0.00339875 $\%$ i forholdt
til $B$ på billedet. Se udregning \ref{afrundning2}.

\begin{equation}
	4000 \cdot \varPhi = 4000(\sqrt{5}-1)/2 = 2472.13595 \approx 2472 \label{afrundning}
\end{equation}

\begin{equation}
	0.13595/4000 \cdot 100 = 0.00339875 \label{afrundning2}
\end{equation}

Det er en meget lille del af selve billedet og skulle ikke give nogle
misvisninger i forhold til udregningen. For at gøre det lidt mere
generelt, sætter vi trunkeringsfejlen til $0.5$, da det er den maksimale
afrundingsfacktor som kan forekomme. Hvis billedet har en størrelse på
500 pixels, hvilket er det mindste billedet vi har, giver dette en fejlmargin
på $0.1 \%$. Dette tal bliver adderet til fejlsatsen ovenfor, og giver
en samlet afvigelse på $2.1\%$.

\subsubsection{Snitratio}
End til vider har vi kun arbejder med det gyldne snit, men andre snit i
billedet kan godt optræde, derfor indføre vi en nu betegnelse
snitratio, som betegner en procents sats for hvor lang inde i billedet
snittet befinder sig. Det vil sige at hvis en snitration er på $0.2$. Et
billedet har $B$ på 4000 vil et snit befinde sig i pixel $4000*0.2 =
800$.

\subsubsection{Heltal ved udregning af Margin}
Når vi har 2 forskellige snitratioer, f.eks. $\varPhi$ og $\frac{2}{3}$,
som ligger meget tæt på hinanden, og vi gerne vil sammenligne hvilken
regioner der ligger i snitratioernens snit, er det vigtigt at margin for
vært af de 2 snitratioers snit ikke krydser hinanden. 

Hvis margin krydser. Vil det indebære, at den samme region bliver fundet
af begge snit. Dette vil give et skævt billedet af forskellen på de to.
Derfor må vi sørge for at marginerne ikke krydser. Hvis $x$ betegner
antal pixels i $B$ eller $H$, og vi vil se på, hvor mange pixels, der er
mellem snitratio $\frac{2}{3}$ og $\varPhi$, multiplicerer vi $x$ med de
to snitratioen for at finde deres placering. Derefter subtraheres vi et
af de snit som befinder sig tetest på hinanden i vær snitratio med
hinanden.

\begin{eqnarray}
	\frac{x2}{3} - \frac{x2}{\sqrt{5}+1} & = & x(\frac{2}{3} - \frac{2}{\sqrt{5} + 1}) \nonumber \\
	& = & x(0.666667-0.618034) \\ \nonumber
	& = & x(0.048633)
\end{eqnarray}

Vi har nu fundet antal pixels mellem de to snit. Vi vil gerne undgå at
de to marginens ikke krydser hinanden, så der dividere vi med 2 og
afrunder værdien.

\begin{equation}
	\left\lfloor \frac{0.048633x}{2}\right\rfloor = \left\lfloor0.024316x \right\rfloor
\end{equation}

Tallet $2.4316$ er altså den minimale procentvise størrelse som vores
margin må have, nå vi sammen liner det gyldne snit og $\frac{2}{3}$.
Det betyder også at vi ikke må sammen line snit som ligger særlig
meget tætter på hinanden, da 0.021 er den minimale procent margin som vi må have.
$\lfloor 0.024316x \rfloor$ giver os et antal pixels som skildre de to snit. For at vise
hvor stort marginen egentlige kan være, bruger jeg denne formel på to
billeder, et som svare til vores mindste billedet, 500 pixels, og et
som svare til vores største billedet, 4000 pixels. Ved 500 pixels
bliver resultatet

\begin{equation}
	 \lfloor 500(0.024316)\rfloor = 12
\end{equation}

Det er en fint margin, da vores fejl på udregningerne ligger på 2.1 \%,
som svare til $\lceil 500*0.021 \rceil = 11$ pixels, som er 1 pixels fra vores
margin

Ved 4000 pixels giver det.

\begin{equation}
	 \left\lfloor 4000(0.024316)\right\rfloor = 97
\end{equation}

Som også er god nok da $4000*0.021 = 84$ pixels.
% vim: set tw=72 spell spelllang=da:



\chapter{Afprøvning}
{
I dette Kapitel vil vi afprøve vores 2 metoder på XX(hvormange) billeder og se hvordan de virker. Udover det vil vi finde tærskelværdier til kandetection, floodfill og margin, ud fra observationer gjort under afprøvning. Vi vil også afprøve den generalle region detektor og kommer ind på dens fordele og ulemper. 
\section{Tærskelværdier}
%% Bemærk:
%%          Resten af rapporten følger en stil hvor indledninger skrives
%%          med \sffamlily-typen. Denne stil bør også følges her.
%%
{\sffamily
I vores program er der 4 forskellige tærskelværdier som påvirker hvordan
regions detektoren analysere billedet. Alle 4 tærskelværdier er blevet
introduceret i deres respektive afsnit, men ingen konkrate tal er
opgivet. I dette afsnit vil vi opgive de tal, samt en forklaring på
hvorfor vi har valt lige de tal. Malerierne som er brugt i afprøvning er
spicældt udvalgt for deres illustrative aspekt. 
}

\subsection{Marginens brede}
Vi regner marginens brede ud fra en procent stats $\psi$ af billedets
$B$ og $H$. I afsnittet \ref{section_opdeling} kom vi frem til en
usikkerhed på $2.1 \%$. Så $\psi$ skal være støre en $2.1 \%$. 

Den minimale forskel på 2 snit vi foretager os, er forskellen mellem det
gyldne snit og $\frac{2}{3}$, margin brade udregnet i section
\ref{section_opdeling} til maksimal at være $2.43\%$. For at marginen
ikke krudser. 

Det vil sige at $\psi \in [2.1, 2.43]$. Vi har valt at
sætte $\psi = 2.4$, da vi derved kan tage højde for uforresette
usikkerhed.

\subsection{Afvigelsen af farver i kandtdetection}
Vi bruger kandtdetection til, er af finde en kanter rundt om de regioner
som vi mener er interessante, og undgå de kanter som ligger inde i
regioner. Begge de 2 mål kan ikke altid opfyldes, men vi kan komme så
tæt på et krompromi mellem en perfekt kant rund om region og ingen
kanter inde i region som mulige. Dette gøres ved at ændre 2
tærskelværdier i kantdetectionen. Vi har valt at dele billederne som vi
observere op i 9 kategorier, se tabel \ref{thressholdsTabelKant},
Kategorier er en grove opdeling af billederne efter detaljer og farve
intensitet, som bruges til at give en bedre indblik på billedets
opbygning. 

\subsubsection{Sammenligninger}
Vi har set på 9 malerier og har fundet de tærskelværdier som vi mener
passer bæst på malerierne. Vi vil illustrerer hvordan vi har fastsat
tærskelværdierne på maleriet \ref{kDetalier}.

Maleriet er malet med mange farver og med masser af detaljer. Vi ser
først på tærskelværdierne $(0,0),(100,100).....(600,600),(700,700)$ se
figur \ref{allesammen1} og \ref{allesammen2}, og finde det interval hvor
malerriet ikke har mistet nogle af kanterne rundt om regionerne endnu,
men vil det, i næste interval. I illustration vurdere vi det til
billedet \ref{300-300}, da billedet \ref{400-400} har mistet for mange
af de kanter, som vi gerne vil beholde.

Ved at sætte en af tærskelværdierne op lidt af gangen, kan vi igen få en
række billeder at vælge imellem. Se sammenligningen i figur
\ref{allesammen3}. Man kan se at det begynder at være svær at skelne
figurene i \ref{300-850} og der er lidt for mange kanter i
\ref{300-700}.

Vi har valgt at fastsætte tærskeværdigerne til $(300,750)$. Det maleri
vi lige har brugt er ikke særlige repræsentativ for helle vores maleri
database. så vi har taget 8 andre billeder og brugt samme metode på dem og fastsat en
middel tærskelværdi. Vi viser her en lille udsnit af dem, se figur
\ref{en}, \ref{to} og \ref{tre}.
\clearpage
\begin{figure}[p]
    \centering
    \subfloat[(100,100)]{
        \includegraphics[angle=0,width=0.45\textwidth]{afsnit/afprovning/billeder/thressholds/krafitige_farver/krafite_detalier/1_iteration/100-100.png}
        \label{100-100}}\hspace{1em}
    \subfloat[(200,200)]{
        \includegraphics[angle=0,width=0.45\textwidth]{afsnit/afprovning/billeder/thressholds/krafitige_farver/krafite_detalier/1_iteration/200-200.png}
        \label{200-200}}\\
    \subfloat[(300,300)]{
        \includegraphics[angle=0,width=0.45\textwidth]{afsnit/afprovning/billeder/thressholds/krafitige_farver/krafite_detalier/1_iteration/300-300.png}
        \label{300-300}}\hspace{1em}
    \subfloat[(400,400)]{
        \includegraphics[angle=0,width=0.45\textwidth]{afsnit/afprovning/billeder/thressholds/krafitige_farver/krafite_detalier/1_iteration/400-400.png}
        \label{400-400}}
    \label{allesammen1}
    \caption{Edgedetection på maleriet \ref{kDetalier} som har mange detaliger og kraftige farver, med tærskelværdierne fra (100-100) til (400-400)}
\end{figure}

\clearpage

\begin{figure}[!h]
	\centering
	\subfloat[(500,500)]{
        \includegraphics[angle=0,width=0.45\textwidth]{afsnit/afprovning/billeder/thressholds/krafitige_farver/krafite_detalier/1_iteration/500-500.png}
        \label{500-500}}\hspace{1em}
    \subfloat[(600,600)]{
        \includegraphics[angle=0,width=0.45\textwidth]{afsnit/afprovning/billeder/thressholds/krafitige_farver/krafite_detalier/1_iteration/600-600.png}
        \label{600-600}}\\
    \subfloat[(700,700)]{
        \includegraphics[angle=0,width=0.45\textwidth]{afsnit/afprovning/billeder/thressholds/krafitige_farver/krafite_detalier/1_iteration/700-700.png}
        \label{700-700}}\hspace{1em}
    \subfloat[Original. Navn: The Archangel Michae. År: Ca 1490 Af: Abadia, Juan de la]{
        \includegraphics[angle=0,width=0.45\textwidth]{afsnit/afprovning/billeder/thressholds/krafitige_farver/krafite_detalier/kDetalier.jpg}
        \label{kDetalier}}
    \caption[]{Edgedetection på maleriet \ref{kDetalier} som har mange detaliger og kraftige farver, med tærskelværdierne fra (500-500) til (700-700)}
     \label{allesammen2}
\end{figure}

\begin{figure}[!h]
    \centering
    \subfloat[(300,700)]{
        \includegraphics[angle=0,width=0.45\textwidth]{afsnit/afprovning/billeder/thressholds/krafitige_farver/krafite_detalier/2_iteration/300-700.png}
        \label{300-700}}\hspace{1em}
    \subfloat[(300,750)]{
        \includegraphics[angle=0,width=0.45\textwidth]{afsnit/afprovning/billeder/thressholds/krafitige_farver/krafite_detalier/2_iteration/300-750.png}
        \label{300-750}}\\
    \subfloat[(300,800)]{
        \includegraphics[angle=0,width=0.45\textwidth]{afsnit/afprovning/billeder/thressholds/krafitige_farver/krafite_detalier/2_iteration/300-800.png}
        \label{300-800}}\hspace{1em}
    \subfloat[(300,850)]{
        \includegraphics[angle=0,width=0.45\textwidth]{afsnit/afprovning/billeder/thressholds/krafitige_farver/krafite_detalier/2_iteration/300-850.png}
        \label{300-850}}
        \caption[]{Edgedetection hvor de 4 billeder som er intrasante taget med}
     \label{allesammen3}
\end{figure}
 
\begin{figure}[!h]
    \centering
    \subfloat[(100,250)]{
        \includegraphics[angle=0,width=0.45\textwidth]{afsnit/afprovning/billeder/thressholds/svage_farver/svage_detalier/2_iteration/100-250.png}
        \label{100-250}}\hspace{1em}
    \subfloat[Orginal. Navn: The Lamentation over St Francis. År: 1440. Af: Angelico, Fra. ]{
        \includegraphics[angle=0,width=0.45\textwidth]{afsnit/afprovning/billeder/thressholds/svage_farver/svage_detalier/sDetalier.jpg}
        \label{Orginal1}}
        \caption[]{Edgedetection på et billedet med svage farver og få detalier, hvor tærskenværdigern [100,250] er den beste}
     \label{en}
\end{figure}

\begin{figure}[!h]
    \centering
    \subfloat[(100,240)]{
        \includegraphics[angle=0,width=0.85\textwidth]{afsnit/afprovning/billeder/thressholds/medium_farver/svage_detalier/2_iteration/100-240.png}
        \label{100-240}}\\
    \subfloat[Orginal. Navn:The Ninth Wave. År:1850. Af:Aivazovsky, Ivan Konstantinovich.]{
        \includegraphics[angle=0,width=0.85\textwidth]{afsnit/afprovning/billeder/thressholds/medium_farver/svage_detalier/sDetalier1.jpg}
        \label{Orginal2}}
        \caption[]{Edgedetection på et billedet med medium farver og få detalier, hvor tærskenværdigern [100,240] er den beste}
     \label{to}
\end{figure}

\begin{figure}[!h]
    \centering
    \subfloat[(200,460)]{
        \includegraphics[angle=0,width=0.85\textwidth]{afsnit/afprovning/billeder/thressholds/medium_farver/medium_detalier/2_iteration/200-460.png}
        \label{200-460}}\\
    \subfloat[Orginal. Name: The last supper. År: 1498. Af: Leonardo da Vinci]{
        \includegraphics[angle=0,width=0.85\textwidth]{afsnit/afprovning/billeder/thressholds/medium_farver/medium_detalier/mDetalier1.jpg}
        \label{Orginal3}}
        \caption[]{Edgedetection på et billedet med medium farver og medium detalier, hvor tærskenværdigern [200,460] er den beste}
     \label{tre}
\end{figure}

\begin{table}[!h]
    \centering
    \begin{tabular}{| l | l | l | l |} \hline
        & Svage farver 	& Medium farver & Kraftige farver \\ \hline
        Få detaljer 		& (100,250)		& (100,240)		& (200,320)\\ \hline
        Medium detaljer 	& (100,280)		& (200,460)		& (200,380)\\ \hline
        Mange detaljer		& (200,400)		& (200,380)		& (300,750)\\ \hline
    \end{tabular}
    \caption{Tabel over kantdetektionstærskelværdier for ni malerier}
    \label{thressholdsTabelKant}
\end{table}

Som man kan se af tabel \ref{thressholdsTabelKant} gå tærskelværdierne
fra $(100,240)$ til $(300,750)$, så vi tager en gennemsnit af værdierne
og få at de to tærskelværdier skal være (177 og 384). 

Vi har dog i vores forsøg regnet med værdierne 78 og 194, da vores
indledende afprøvninger afviger en smugle fra den denne afprøvning.

\subsection{Afvigelsen af farver i floodfill}
Floodfill har 2 tærskelværdier $lo$ og $up$, som betegner hvor mange
pixel værdier en nabo pixel farver må variere, ned og op. En
fyldestgørelses beskrivelse af floodfill findes i afsnit
\ref{section_opdeling}. 

Vi har tænkt os at finde en fældes tærskelværdi til brug i vores
program. Måde vi gør det på at ved at observere hvordan floodfill virker
med forskellige tærskelværdier og finde de tærskelværdier som passer
bæst til maleriet. Resultatet for observationen kan ses i tabel
\ref{thressholdsTabelFF}, hvor de 9 sammen kategorier er vist og
afprøvet på de samme 9 malerier. Et af de malerierne kan se i
figur \ref{Floodfillbilledet}, hvor man kan se hvad vi har valt til at
være de optimale værdier.

\begin{figure}[!h]
    \centering
    \subfloat[8,8]{
        \includegraphics[angle=0,width=0.9\textwidth]{afsnit/afprovning/billeder/thressholds/svage_farver/kraftige_detalier/floodfill/8-8.png}
        \label{8-8}}\\
    \subfloat[Orgina. Navn: Winter Landscape. År: Ukent. Af:Avercamp, Hendrick.]{
        \includegraphics[angle=0,width=0.9\textwidth]{afsnit/afprovning/billeder/thressholds/svage_farver/kraftige_detalier/kDetalier.jpg}
        \label{Orginal4}}
    \caption[]{tærskelværdierne på et billedet med svage farver og kraftige detaljer hvor tærskelværdien [8,8] passer best}
    \label{Floodfillbilledet}
\end{figure}

\begin{table}[!h]
    \centering
    \begin{tabular}{| l | l | l | l |} \hline
        & Svage farver 		& Medium farver & kraftige farver \\ \hline
        Få detalier 		& \textbf{[2,2]}	& [3,3]			& [4,4]\\ \hline
        Medium detalier 	& \textbf{[2,2]}	& \textbf{[5,5]}& \textbf{[2,2]}\\ \hline
        Mange detalier		& [8,8]				& [4,4]			& [7,7]\\ \hline
    \end{tabular}
    \caption{Tabel over floodfill tærskelværdier på 9 malerier}
    \label{thressholdsTabelFF}
\end{table}

Som man kan se af tabel \ref{thressholdsTabelFF}, er nogle af vadierne
med fed, begrundelsen for det, er at vadierne, er de beste, vi kan finde
for billedet, men at de vadier stadig ikke giver noget som er særlige
brugbart. værdigerne i tabel fluktuere også en del, så vi tager
gennemsnittet og får tærskelværdierne til at være (5,5).

\section{Regions detektor}
%% Bemærk:
%%          Resten af rapporten følger en stil hvor indledninger skrives
%%          med \sffamlily-typen. Denne stil bør også følges her.
%%

{\sffamily I dette afsnit vil vi afprøve den generelle metode for udtrækning af
regioner. Selve metodens fremgangsmåde står beskrevet i afsnit
\ref{section_udtraek}. Tærskelværdierne er sat til de fundne
tærskelværdier i afsnit \ref{terskelverdi}. Første del af afsnittet vil
omhandle afprøvning af metoden på testbilleder, og anden del afprøvning
ad metoden på udvalgte billeder fra databasen.}

\subsection{Afprøvning på testbilleder}
I dette følgende afprøves metoden på billeder konstrueret med en hvid
baggrund og sorte regioner. Snittet, som der vil blive kigget på, vil
blive tegnet med rødt på billederne. Snittes margin vil blive tegnet med blåt.

I billedet \ref{GRD_test1} er der fem regioner. Tre af dem bliver
fundet, da de ligger i snittet. De to sidste regioner som ikke er fundet, er
stadig sorte. Som man kan se, bliver både den, der krydser snittet og den,
der tangere snittet, taget med. 

\begin{figure}[!h]
    \centering
    	\subfloat[Det originale maleri.]{
	       	\includegraphics[angle=0,width=0.7\textwidth]{afsnit/afprovning/billeder/region_selector/blob_section.png}
	       	\label{GRD_test1_original}}\hspace{1em}
		\subfloat[Her ses fire regioner samt en baggrund. To af figurerne og baggrunden er blevet fundet af metoden.]{
        	\includegraphics[angle=0,width=0.7\textwidth]{afsnit/afprovning/billeder/region_selector/blob_region_section.png}
        	\label{GRD_test1}}\hspace{1em}
        \caption[]{}
     \label{GRD_test1_sammen}
\end{figure}

I billedet \ref{GRD_test2} er der tre regioner, som
alle bliver fundet; baggrunden, den lille region som ligger indenfor
marginen, og den store region som kun har en lille del af sin masse
i snittet. 

\begin{figure}[!h]
    \centering
    	\subfloat[Det originale maleri.]{
	       	\includegraphics[angle=0,width=0.7\textwidth]{afsnit/afprovning/billeder/region_selector/lille_tvers.png}
	       	\label{GRD_test2_original}}\hspace{1em}
		\subfloat[På billedet er der fundet en stor og en mindre region.]{
        	\includegraphics[angle=0,width=0.7\textwidth]{afsnit/afprovning/billeder/region_selector/blob2_region_section.png}
        	\label{GRD_test2}}\hspace{1em}
        \caption[]{}
     \label{GRD_test2_sammen}
\end{figure}

I billedet \ref{GRD_test3} er der en horisont, som ligger oven på
snittet. Begge sider af horisontlinjen bliver udtrukket som en region.

\begin{figure}[!h]
    \centering
    	\subfloat[Det originale maleri.]{
	       	\includegraphics[angle=0,width=0.7\textwidth]{afsnit/afprovning/billeder/region_selector/hoisont.png}
	       	\label{GRD_test3_original}}\hspace{1em}
		\subfloat[På billedet er der fundet to store regioner.]{
        	\includegraphics[angle=0,width=0.7\textwidth]{afsnit/afprovning/billeder/region_selector/hoisont_region_section.png}
        	\label{GRD_test3}}\hspace{1em}
        \caption[]{}
     \label{GRD_test3_sammen}
\end{figure}

I de tre figurer \ref{GRD_test1_sammen}, \ref{GRD_test2_sammen} og
\ref{GRD_test3_sammen} kan ses før og efter metoden er blevet anvendt.
Metoden opfører sig efter de standarder, som vi fremsatte i afsnit \ref{section_naiv}, og
virker efter vores forventninger.
\clearpage

\subsection{Afprøvning på malerier}
Vi vil afprøve metoden til udtrækning af regioner på seks udvalgte
malerier. Malerierne skal demonstrere, hvordan metoden på nogle malerier
fungerer godt, mens den fungerer dårligt på andre.

\begin{figure}[!h]
    \centering
		\subfloat[Maleri med kraftige farver og få detaljer. Navn: Scenes from the Story of Joseph: The Arrest of His Brethren. År: 1515-16. Af: Bacchiacca.]{
        	\includegraphics[angle=270,width=1.0\textwidth]{afsnit/afprovning/billeder/thressholds/krafitige_farver/svage_detalier/floodfill/4-4.png}
        	\label{GRD_virker1}}\hspace{1em}
		\subfloat[Maleri med middel kraftige farver og medium antal detaljer.]{
        	\includegraphics[angle=0,width=1.0\textwidth]{afsnit/afprovning/billeder/4-4.png}
        	\label{GRD_virker2}}\hspace{1em}
        \caption[]{To malerier, hvor den generelle metode virker efter vores ønsker.}
     \label{generelde_region_detektor_virker}
\end{figure}

I figuren \ref{generelde_region_detektor_virker} er der to malerier, hvor
vores regionsudtrækning virker rigtig godt. 

I det første maleri \ref{GRD_virker1} finder metoden syv store regioner
samt en del små. Den skelner mellem de forskellige paneler om kaminen,
og hvert stykke tøj på personen i maleriet opfattes som en særskilt
region. De små regioner er samlet, og forstyrre ikke de store regioner.
Man kunne måske have ønsket sig, at metoden ville fylde mere af
personens kappe ud, men bortset fra det er figuren et godt eksempel på,
hvordan det ser ud, når den generelle metode fungerer.

I maleriet \ref{GRD_virker2} finder vi mange af de samme positive ting:
drengen i midten af maleriet er helt udfyldt, en sko, drengens
badebukser og et håndklæde er også fundet. Det er dog vigtigt at lægge
mærke
til, at de andre personer i vandet, flyder i et med baggrunden. Dette
ville normalt være uheldigt, men eftersom metoden kun ser efter regioner
i snittet -repræsenteret af den røde linje- gør det ikke noget i dette
tilfælde.

\begin{figure}[!h]
    \centering
	\subfloat[Maleri med kraftige farver og mange detaljer, hvor de tærskelværdierne fundet i afsnit \ref{terskelverdi} er brugt.]{
   	 	\includegraphics[angle=270,width=0.90\textwidth]{afsnit/afprovning/billeder/thressholds/krafitige_farver/krafite_detalier/floodfill/4-4.png}
	    \label{GRD_virker_nesten1}}\hspace{1em}
    \subfloat[Det samme maleri, hvor Tærskelværdier er valt specifikt for det her maleri]{
        \includegraphics[angle=270,width=0.95\textwidth]{afsnit/afprovning/billeder/thressholds/krafitige_farver/krafite_detalier/s7_e200_f5.png}
        \label{GRD_virker_nesten1_super}}\\
     \caption[]{Et malerier, hvor den generelle metode ikke helt virker efter vores ønske, men med nye tærskelværider vil virker meget bedre}
     \label{generelde_region_detektor_virker_nesten1}
\end{figure}

\begin{figure}[!h]
    \centering
    \subfloat[Maleri med middel kraftige farver og medium antal detaljer.]{
        \includegraphics[angle=0,width=0.95\textwidth]{afsnit/afprovning/billeder/thressholds/medium_farver/medium_detalier/floodfill/4-4.png}
        \label{GRD_virker_nesten2}}\\
	\subfloat[Det samme maleri, hvor Tærskelværdier er valt specifikt for det her maleri]{
   	 	\includegraphics[angle=0,width=0.90\textwidth]{afsnit/afprovning/billeder/thressholds/medium_farver/medium_detalier/s5_e90_e200_f4.png}
	    \label{GRD_virker_nesten2_super}}\hspace{1em}
     \caption[]{Et malerier, hvor den generelle metode ikke helt virker efter vores ønske, men med nye tærskelværdier vil virker meget bedre}
     \label{generelde_region_detektor_virker_nesten2}
\end{figure}

I figur \ref{generelde_region_detektor_virker_nesten1} og \ref{generelde_region_detektor_virker_nesten2} er der to
malerier, hvor metoden ikke fungerer optimalt efter hensigten. Dog kan
vi stadig bruge figurerne til noget

I maleriet \ref{GRD_virker_nesten1} bliver der hovedsageligt fundet små
regioner. En sko, en skulder og en flise trækkes ud, hvor vi hellere
ville have haft, at personens kappe og arm blev fundet. Dette skyldes
primært, at tærskelværdierne for dette billede er sat for lavt. 


Maleriet i den anden figur \ref{GRD_virker_nesten2} rummer nogle af de
samme problemer: Metoden finder også her en masse små dele af figurer,
der burde hænge sammen. Det vil også kunne løses ved nogle andre
tærskelværdier, men som man også kan se på maleriet er vægge og loftet -
som egentlig er ret ensfarvede -stadig svære at finde for metoden. Det
kunne tyde på, at en højere grad af sløring ville løse problemet, og få
algoritmen til at virke i maleriet.

Når netop de to figure \ref{GRD_virker_nesten1} og
\ref{GRD_virker_nesten2} er interessante, er det fordi, at en ændring af
tærskelværdier, samt en højre grad af sløring, ville bevirke, at den
generelle metode kom til at fungere bedre. I billedet
\ref{GRD_virker_nesten1_super} og \ref{GRD_virker_nesten1_super} er
protretere de samme malerier, men hvor tærskelværdierne passer bedre til
maleriet, man kan se det ved f.eks at karben på anglen samt det meste af
loftet bliver fundet.

\clearpage

\begin{figure}[!h]
    \centering
     \subfloat[Det originale maleri. Navn: Vase of Flower. År: ukent.
	 Af: Arellano, Juan de.]{
        \includegraphics[angle=0,width=0.46\textwidth]{afsnit/afprovning/billeder/thressholds/krafitige_farver/medium_detalier/mDetalier}
        \label{GRD_virker_ikke1_orginal}}
    \subfloat[Maleri med kraftige farver og medium detaljer.]{
        \includegraphics[angle=0,width=0.46\textwidth]{afsnit/afprovning/billeder/thressholds/krafitige_farver/medium_detalier/floodfill/4-4.png}
        \label{GRD_virker_ikke1}}
     \caption{Maleri af blomster, hvor den generelle metode ikke
	 virker.}
     \label{generelde_region_detektor_virker_ikke}
\end{figure}

\begin{figure}[!h]
	\begin{center}
	    \includegraphics[angle=0,width=0.65\textwidth]{afsnit/afprovning/billeder/thressholds/svage_farver/svage_detalier/floodfill/4-4.png}
	\end{center}    
	\caption{Maleri med svage farver og få detaljer.}
    \label{GRD_virker_ikke2}
\end{figure}

I figur \ref{generelde_region_detektor_virker_ikke} og
\ref{GRD_virker_ikke2} er der to malerier hvor
vores generelle metode ikke virker optimalt. 

I maleri \ref{GRD_virker_ikke1} er noget af buketten og baggrunden gået
i et. Desuden er resten af blomsterne i snittet ikke fyldt ud, og selv
en ændring i tærskelværdierne vil ikke hjælpe, da en forøgelse af disse
blot vil resultere i, at flere af blomsterne går i ét med baggrunden. En
sænkning vil resultere i, at ingen af blomsterne bliver trukket ud. 

Maleriet \ref{GRD_virker_ikke2} repræsenterer en anden problemstilling,
som vi ikke kan komme udenom: farverne er så mørke og svage, at en
ændring i tærskelværdien i floodfill på bare en vil få metoden til at gå
fra at finde ingen regioner til at finde for store regioner. det kan ses
i figur \ref{ff_munke}, hvor de 2 malerier, hvor floodfills tærskelværdi
er sat til den lavest og næst laves værdi. Som man kan se, finde metoden
mange snå regioner som ikke er særlige intrasant i maleri
\ref{munk_etff}, men i \ref{munk_toff}, bliver der fundet 2 meget store
regioner, som er blevet alt for store. Dette er et problem da vi så ikke
kan have en tærskelværdi som virker.

\begin{figure}[!h]
    \centering
     \subfloat[Tærskelværdierne for floordfill på maleriet er sat til en]{
        \includegraphics[angle=0,width=0.46\textwidth]{afsnit/afprovning/billeder/thressholds/svage_farver/svage_detalier/1-1.png}
        \label{munk_etff}}
    \subfloat[Tærskelværdierne for floodfill på maleriet er sat til to]{
        \includegraphics[angle=0,width=0.46\textwidth]{afsnit/afprovning/billeder/thressholds/svage_farver/svage_detalier/2-2.png}
        \label{munk_toff}}
     \caption{Et maleri hvor tærskelværdien for floodfill er sat til den laveste værdi den kan have, og et hvor denne næst laves tærskelværdien er sat.}
     \label{ff_munke}
\end{figure}

Dette kan også skyldes, at kantdetektionen tilføjer en mørk kant rundt
om regionerne, men da maleriet er mørkt i forvejen, hjælper
kandetektionen ikke. I maleri \ref{bla} er kanten tegnet med blåt og man
kan se at munken to fra venstre, ikke får sit skæg med, men hvor han går
det i maleriet med sort kant, se maleri \ref{sort}.

\begin{figure}[!h]
    \centering
     \subfloat[Maleri med blå kanter i kantdetektionen]{
        \includegraphics[angle=0,width=0.46\textwidth]{afsnit/afprovning/billeder/thressholds/svage_farver/svage_detalier/blueE.png}
        \label{bla}}
    \subfloat[Maleri med sorte kanter i kantdetektionen]{
        \includegraphics[angle=0,width=0.46\textwidth]{afsnit/afprovning/billeder/thressholds/svage_farver/svage_detalier/floodfill/4-4.png}
        \label{sort}}
     \caption{To forskellige farver brugt til at ligge kanter på maleriernes regioner}
     \label{fleremunke}
\end{figure}


\section{Naive løsning}
%% Bemærk:
%%          Resten af rapporten følger en stil hvor indledninger skrives
%%          med \sffamlily-typen. Denne stil bør også følges her.
%%
{\sffamily
I dette sektion vil vi teste den naive løsning, ved at se om den sortere
de rigtige regioner væk og om løsningen opføre sig på samme måde som vi
har håbet på. Det vil vi gøre ved ført at se på nogle fabrikeret test
billeder få at se om den naive løsning virker efter meningen og bag
efter vil vi teste på malerierne for at se om den naive løsning kan
bruges i praktisk.
}
  
\subsection{Afprøvning på testbilleder}
Vi vil teste på de samme billeder som var i figur
\ref{region_detektor_test}, samt nogle af de test billeder som blev
brugt i forklaringen af den naive metodes, de 4 billeder som vi har valt
at test kan se i figur \ref{naiv_detektion_test} hvor en grån kasse
rundt om en region, betyder at den er valt til at ligge i det gyldne
snit af den naive metode. Det første billedet \ref{naiv_blob1}, har 5
regioner og hvor 3 af dem blev fundet af reginon detektoren, vores naive
løsning har så sorteret baggrundens regionen og den øverste region i
snitte vær, da de begge krydser marginer og derfor ikke overholder regel
\ref{2.2.4, c}. Det andet billedet \ref{naiv_blob2}, er alle blevet
sorteret vær, også den lille region som ligger lige i miden af snittet.
Grunde til det er at den er for lille og derfor ikke overholder regel
\ref{2.2.4, a}. I test billedet \ref{naive_hoisont1}, sortere algoritmen
himlen fra, da den krydser margin lidt, men tager jorden med. I test
billedet \ref{naiv_masse} er der kun en af den 3 regioner som ikke
bliver sorteret væk, grunden til at den nederste region ikke bliver
tager med er at den ikke har en masse der er stor nok, som forklaret i
\ref{2.2.3} og derfor ikke overholder regel \ref{2.2.4 b}. Alle de test
billeder som vi har vist her opføre sig præcist på den måde som vi havde
regnet med.

\begin{figure}[!h]
    \centering
		\subfloat[Naive algoritme finder 1 ud af 5 regioner]{
        	\includegraphics[angle=0,width=0.55\textwidth]{afsnit/afprovning/billeder/naive_losning/naiv_blob1.png}
        	\label{naiv_blob1}}\hspace{1em}
    		\subfloat[Værgen den lille region eller den store er fundet]{
	        	\includegraphics[angle=0,width=0.55\textwidth]{afsnit/afprovning/billeder/naive_losning/naiv_blob2.png}
	       	\label{naiv_blob2}}\hspace{1em}
    		\subfloat[Kun den nederste højrisondt er fundet]{
	        	\includegraphics[angle=0,width=0.55\textwidth]{afsnit/afprovning/billeder/naive_losning/naiv_hoisont1.png}
		    \label{naiv_hoisont1}}\hspace{1em}
		    \subfloat[2 regioner hvor den ende er sorteret vær på grund af dens masse]{
	        	\includegraphics[angle=0,width=0.55\textwidth]{afsnit/afprovning/billeder/naive_losning/naiv_mass.png}
	       	\label{naiv_masse}}\hspace{1em}
        \caption[]{4 test billeder som også blev brugt til at illustrere den naive løsnings fremgangs måde, grån kasse rund om region, betyder at den er taget med af dem naive løsning}
     \label{naiv_detektion_test}
\end{figure}
\clearpage

\subsection{Afprøvning på malerier}
For at se på hvordan den naive metode virker på malerier afprøver vi den
på 6 malerier, først på 3 malerier, hvor regions detektoren virker
efter vores hensigt og så på 3 malerier, hvor region detektoren ikke
virker. Beskrivelsen af hvad der sker i billedet vil stå i caption


\begin{figure}[h!!]
	\begin{center}
		\includegraphics[scale=0.3,angle=0]{afsnit/afprovning/billeder/naive_losning/naiv_kfarver_sdetaljer.png}
	\end{center}
	\caption[]{5 ud af de 6 store regioner fra figur \ref{GRD_virker1} valt til at ligge i snittet, skoene er få små til at blive taget i betragtning}
	\label{naiv_kfarver_sdetaljer}
\end{figure}

\begin{figure}[h!!]
	\begin{center}
		\includegraphics[scale=0.3,angle=0]{afsnit/afprovning/billeder/naive_losning/naiv_mfarver_mdetaljer.png}
	\end{center}
	\caption[]{Bukserne og skoene er tager med af den naive løsning, men drengen er sorteret vær da har krydser snittet}
	\label{naiv_mfarver_mdetaljer}
\end{figure}

\begin{figure}[h!!]
	\begin{center}
		\includegraphics[scale=0.3,angle=0]{afsnit/afprovning/billeder/naive_losning/naiv_kfarver_kdetaljer.png}
	\end{center}
	\caption[]{Et billedet med mange hoder i snittet, hvor 2 af dem bliver godtaget af den naive metode til at ligger i snittet, en trøje bliver desværre også taget med »»(måske noget med at der bliver soteret få mange hover fra)}
	\label{naiv_kfarver_kdetaljer}
\end{figure}

\begin{figure}[h!!]
	\begin{center}
		\includegraphics[scale=0.3,angle=0]{afsnit/afprovning/billeder/naive_losning/naiv_virker_ikke1.png}
	\end{center}
	\caption[]{Mallerie hvor region detektor ikke virker, den naive løsning godtager tager en region som ligger helt forkert }
	\label{naiv_virker_ikke1}
\end{figure}

\begin{figure}[h!!]
	\begin{center}
		\includegraphics[scale=0.3,angle=0]{afsnit/afprovning/billeder/naive_losning/naiv_virker_ikke2.png}
	\end{center}
	\caption[]{3 regioner bliver godtaget, selv om de ikke er særlige intresante ««(er det ikke en farlige ting at sige )}
	\label{naiv_virker_ikke2}
\end{figure}

\begin{figure}[h!!]
	\begin{center}
		\includegraphics[scale=0.3,angle=0]{afsnit/afprovning/billeder/naive_losning/naiv_virker_ikke3.png}
	\end{center}
	\caption[]{Der bliver fundet 3 region, hvor kun en af dem passer på en ting i billedet}
	\label{naiv_virker_ikke3}
\end{figure}
\clearpage

\subsection{konkulution}
Det virker som om den naive løsning virker efter vores entationer dog
med nogle få falske positive, hvis region detektoren virker på
malerierne, dog fejler den på malerier hvor region detektoren fejler, og
kommer med en masse falske positive.

\section{Udvidet løsning}


}

% vim: set tw=72 spell spelllang=da:


\chapter{Videnskabelige resultater}

% sort in citation order
\bibliographystyle{unsrt}
\bibliography{litteraturliste}
\addcontentsline{toc}{chapter}{Litteratur}

\chapter{Bilag}
\appendix

%\section{Kildekode}
%\input{materialer.tex}
%\newpage

\end{document}
