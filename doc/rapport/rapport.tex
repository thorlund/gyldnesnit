% vim: set tw=72 spell spelllang=da:
\documentclass[a4paper, 10pt, danish, final]{report}
%\documentclass[a4paper, twoside, openright, titlepage, 10pt, danish, final]{report}
%%%%%%%%%%%%%%%%%%%%%%%%%%%%%%%%%%%%%%%%%%%%%%%%%%%%%%%%%%%%%%%%%
% Pakker
%%%%%%%%%%%%%%%%%%%%%%%%%%%%%%%%%%%%%%%%%%%%%%%%%%%%%%%%%%%%%%%%%
%\usepackage{a4wide}
\usepackage[danish]{babel}
\usepackage[utf8]{inputenc}
\usepackage[T1]{fontenc}

\usepackage{charter}
\usepackage{verbatim}
\usepackage{amsfonts}
\usepackage{amsmath}
\usepackage{amsthm}
\usepackage{amssymb}
%\usepackage{mathrsfs}
%\usepackage[mathcal]{euscript}
\usepackage{listings}
\usepackage{graphicx}
\usepackage{multirow}
\usepackage{hyperref}
\usepackage{cite}
\usepackage{float}
\usepackage[small,bf]{caption}
\usepackage{xypic}
\usepackage[table]{xcolor}
\usepackage{subfig}
\usepackage{dirtree}
\usepackage{ulem}
\usepackage{korrektur}

%%%%%%%%%%%%%%%%%%%%%%%%%%%%%%%%%%%%%%%%%%%%%%%%%%%%%%%%%%%%%%%%
% Indstillinger
%%%%%%%%%%%%%%%%%%%%%%%%%%%%%%%%%%%%%%%%%%%%%%%%%%%%%%%%%%%%%%%%
\parindent=5pt
\renewcommand*\lstlistingname{Kodeboks}
\lstset{language=Python, basicstyle=\scriptsize, showstringspaces=false, numbers=none, stepnumber=1, numberstyle=\tiny}
\setcounter{tocdepth}{2}
% Orddeling
\hyphenation{hvis rekt-ang-el ud-regn-ing-en george lin-je-styk-ke
lin-je-styk-ket smel-te smel-tes}

%\theoremstyle{definition}
\newtheorem{hypotese}{Hypotese}

%%%%%%%%%%%%%%%%%%%%%%%%%%%%%%%%%%%%%%%%%%%%%%%%%%%%%%%%%%%%%%%%
% Kommandoer
%%%%%%%%%%%%%%%%%%%%%%%%%%%%%%%%%%%%%%%%%%%%%%%%%%%%%%%%%%%%%%%%
%\newcommand{\old}[1]{\oldstylenums{#1}}
%\newcommand{\old}[1]{{#1}}
\newcommand{\mailto}[1]{\href{mailto:#1}{#1}}
\newcommand{\resume}[1]{\begin{abstract}{\sffamily #1}\end{abstract}}
\newcommand{\angles}[1]{\langle\textrm{#1}\rangle}
\newcommand{\colbox}[2]{\fcolorbox{black}{#1}{#2}}
%\renewcommand{\thefigure}{\thechapter.\thesection.\arabic{figure}}
%\renewcommand{\thetable}{\thechapter.\thesection.\arabic{table}}

%%%%%%%%%%%%%%%%%%%%%%%%%%%%%%%%%%%%%%%%%%%%%%%%%%%%%%%%%%%%%%%
% Titel, forfatter og dato
%%%%%%%%%%%%%%%%%%%%%%%%%%%%%%%%%%%%%%%%%%%%%%%%%%%%%%%%%%%%%%%
\title{Detektion af ``Det Gyldne Snit'' i digitaliserede malerier}

\author{Ulrik Bonde - \mailto{bonde@diku.dk}\\
Kasper Steenstrup - \mailto{khsj@diku.dk}\\
Morten Thorlund - \mailto{thorlund@diku.dk}\\
\\
Vejleder\\Jakob Grue Simonsen}
\date{\today}

\hypersetup{
colorlinks,%
citecolor=black,%
filecolor=black,%
linkcolor=black,%
urlcolor=black,%
bookmarksopen=false,
pdftitle={Detektion af Det Gyldne Snit i digitaliserede malerier},
pdfauthor={Ulrik Bonde, Kasper Steenstrup og Morten Thorlund},
pdfsubject={Computer Vision},
pdfkeywords={Det gyldne snit, computer vision}
}

%%%%%%%%%%%%%%%%%%%%%%%%%%%%%%%%%%%%%%%%%%%%%%%%%%%%%%%%%%%%%%
% Indhold
%%%%%%%%%%%%%%%%%%%%%%%%%%%%%%%%%%%%%%%%%%%%%%%%%%%%%%%%%%%%%%
\begin{document}
\normalem
\maketitle
\pagenumbering{roman}
\thispagestyle{empty}
%\pagestyle{headings}
\resume{
Det er den generelle opfattelse, at det gyldne snit er specielt æstetisk
tiltalende, og at man i kunstmalerier finder flest interessante regioner
her.  Litteraturen kan ikke entydigt bekræfte denne påstand, og den
største undersøgelse på området talte 565 malerier.  Selvom der findes
metoder fra billedbehandling til systematisk analyse af billeders
komposition, er disse ikke blevet taget i brug med henblik på at
undersøge hypotesen.

Vi har udviklet et program, som kan afgøre, hvorvidt en digital
gengivelse af et maleri har interessante regioner i det gyldne snit, og
kørt dette på 17,364 digitaliserede malerier. Regioner bliver først
kontrolleret for, hvorvidt de er interessante, dernæst for om de ligger
i det gyldne snit. En interessant region defineres som et ensfarvet
område i et billede, der er større end 0.2 \% af billedets størrelse og
optager mere end 1/4 af alle pixel i dens begrænsende rektangel. Med
vores udtrækning af regioner og to forskellige metoder til at vurdere,
hvorvidt interessante regioner ligger i det gyldne snit, kan vi ikke
afvise, at snittet er specielt æstetisk tiltalende. Ved naiv vurdering
havde $\mathsf{91.43\%}$ af de analyserede malerier mindst én
interessant region i det gyldne snit.  Resultaterne viste dog ingen
indikation på, at det gyldne snit adskiller sig signifikant fra andre
snit i malerier.  Der vises en tendens til at kunstnere foretrækker at
placere interessante regioner i midten, mens den øverste halvdel og
kanterne af maleriet ikke er at foretrække.

%Resultaterne fra
%analysen opbevares i en database, som tillader videre arbejde med data.
%Programmet kan desuden analysere andre snit i billedet end blot det
%gyldne, og det bliver således muligt at skelne mellem og sammenligne
%resultater fra andre snit --- f.eks, kan vi skelne mellem resultater for
%det gyldne snit og for snit ved to tredjedele.

%Vi har udviklet et program, som kan trække sammenhængende regioner i
%nærheden af det gyldne snit, ud af en digital gengivelse af et maleri.
%Disse tages derefter ud og vurderes efter nogle simple kriterier for at
%afgøre, om de ligger i det gyldne snit. En automatisering af denne
%analyse er blevet sammensat, hvor en database gemmer resultaterne som
%tillader videre arbejde med data. Programmet kan desuden analysere andre
%snit i billedet end blot det gyldne, og det bliver således muligt at
%skelne mellem og sammenligne resultater fra forskellige snit --- f.eks,
%vil man kunne skelne mellem resultater for det gyldene snit og for snit
%ved en tredjedel.

}

{
\section*{Diff}
\begin{itemize}
    \item Skrevet om to udvidelse (den første er implementeret)
    \item Lidt resultater
\end{itemize}
}

% vim: set tw=72 spell spelllang=da:


% vim: set tw=72 spell spelllang=da:


\chapter*{Forord}
\addcontentsline{toc}{chapter}{Forord}
{
{\sffamily Dette dokument er den endelige rapport, udarbejdet i
forbindelse med kurset ``Bachelorprojekt'', som udbydes på Datalogisk
Institut ved Københavns Universitet. I de mere tekniske afsnit, især med
henblik på kapitel \ref{chap_implementation}, men også dele af kapitel
\ref{chap_detektion}, forventes det, at læseren har en basal viden inden
for datalogi, svarende til at have bestået de obligatoriske kurser på
bacheloruddannelsen af datalogistudiet\cite{DIKUkurser}. Kendskab til de
grundlæggende begreber i forbindelse med billedbehandling, vil endvidere
være en fordel. For en introduktion til billedbehandling henvises til
\cite{SIOlsen}. Kapitel \ref{chap_afproevning}, hvor vi ser på de
grafiske resultater fra programmet, og kapitel \ref{chap_resultater},
hvor vi præsenterer de videnskabelige resultater, kan umiddelbart læses
af alle, uden videre forudsætninger end ren og skær interesse, for emnet
omhandlende det gyldne snit og analyse af digitale gengivelser af
malerier.

\section*{Tak}
Der skal først sendes en tak ud til vores vejleder, Jakob Grue Simonsen.
Stephanie Bekkar, Franck Franck, Emma Haxen og Lisbeth Steenstrup skal
også have mange tak for korrekturlæsning. Fætter Jon og Ida Monrad
Graunbøl gives thumbs-up, for at hænge ud på kontoret og supplere
snacks. Endeligt vil vi takke vores hund, Jim Daggerthuggert og alle fra
Langestrand.

}
}

% vim: set tw=72 spell spelllang=da:


\tableofcontents
%\listoftables
%\listoffigures

\parskip=8pt plus 2pt minus 4pt

\chapter{Indledning\label{chap_indledning}}
\pagenumbering{arabic}
\textsf{
Indsæt lam indledning til afsnit.
}

\subsection{Det gyldne snit}
{
Det vi kalder det gyldne snit, den gyldne ratio eller det guddommelige
forhold, blev allerede beskrevet i Euklids \emph{Elements} fra ca. 300
f.  Kr. som følger:
\begin{quote}
	\emph{``A straight line is said to have been cut in extreme
	and mean ratio when, as the whole line is to the greater
	segment, so is the greater to the less.''}\cite{Euclid300bc}
\end{quote}

\begin{figure}[h!]
	\begin{center}
		\includegraphics[scale=0.49,angle=0]{afsnit/baggrund/billeder/line_segment_a_c_b}
	\end{center}
	\caption{Euklids opdeling af et linjestykke}
	\label{line_segment}
\end{figure}

Givet et linjestykke $A\ B$, som vist i figur \ref{euclid}, og ud fra
Euklids beskrivelse kan $\varphi$ defineres som
\begin{equation}
	\varphi	= \frac{A\ C}{C\ B} = \frac{A\ B}{A\ C}
	\label{euclid}
\end{equation}
Ved at indsætte variable i ligning \ref{euclid} får vi
\begin{equation}
	\varphi = \frac{\varphi + 1}{\varphi}
	\label{expand_euclid}
\end{equation}
hvilket giver os andengradspolynomiet
\begin{equation}
	\varphi^{2} - \varphi - 1 = 0.
	\label{poly_phi}
\end{equation}
Hvis vi nu løser andengradspolynomiet i ligning \ref{poly_phi}, med
$\varphi > 0$, får vi
\begin{eqnarray*}
	\varphi	& =	& \frac{\sqrt{5} + 1}{2} \\
		& =	& 1.6180\ 3398\ 8749\ 8948\ 4820 \dots
\end{eqnarray*}

Tallet $\varphi$ bemærker sig blandt andet ved, at når det kvadreres, så
lægger man blot 1 til. Dette udledes trivielt fra ligning \ref{poly_phi}
\begin{equation}
	\varphi^{2} = \varphi + 1
	\label{phi_squared}
\end{equation}

Vi kan også finde polynomiets anden rod som angives ved $\varPhi$
\begin{eqnarray*}
	\varPhi & = & \frac{1}{\varphi} \\
		& = & \varphi - 1 \\
		& = & 0.6180\ 3398\ 8749\ 8948\ 4820 \dots 
\end{eqnarray*}
Også tallet $\varPhi$ er interessant idet dets eget kvadrat plus sig
selv giver 1. Vi har at
\begin{equation}
	\varPhi^{2} + \varPhi = 1
	\label{Phi_squared}
\end{equation}
hvilket kun er gældende for $\varPhi$.

Det gyldne snit fremviser endvidere en interessant forbindelse til
Fibonaccis talrække, da forholdet mellem to fibonaccital $F(n)$ og $F(n
- 1)$ konvergerer mod $\varphi$ når $n$ nærmer sig uendelig. Mere
formelt har vi at
\begin{eqnarray*}
	\varphi & =     & \lim_{n \rightarrow
	\infty}{\frac{F(n)}{F(n - 1)}}
\end{eqnarray*}

\subsubsection{Et gyldent rektangel}
På samme måde som vi kan opdele en linjestykke efter det gyldne snit,
kan vi konstruere et rektangel hvor forholdene mellem højde og bredde er
$\varphi$. Vi konstruerer et rektangel hvor alle sider er lig 1 og
tegner en diagonal fra dette rektangelels midte til modsatte hjørne. Med
denne diagonal som radius tegnes en cirkel som et gyldent rektangel kan
tegnes efter. Figur \ref{golden_rectangle} illustrerer denne metode.

\begin{figure}[h!]
	\begin{center}
		\includegraphics[scale=0.35,angle=0]{afsnit/baggrund/billeder/Golden_Rectangle_Construction}
	\end{center}
	\caption{Et gyldent rektangel - \emph{Kilde: Wikipedia}}
	\label{golden_rectangle}
\end{figure}
Det ses at rektanglet har forholdet $\varphi:1$ og at eksemplet er helt
analogt til det linjestykke givet i figur \ref{line_segment}. Dog skal
det bemærkes, at det rektangel der kan konstrueres af linjestykkerne 1
og $\varphi - 1$ også er et gyldent rektangel med forholdet $1:\varphi
-1 = \varphi$. Man kan derved konstruere gyldne rektangler ud i det
uendelige ved hele tiden at lave nye gyldne rektangler.

\subsubsection{Spiraler og det gyldne snit}
Når man som ovenfor, gentagende gange deler et gyldent rektangel kan man
bruge dette til at konstruere en gylden spiral. En gylden spiral kan
skrives ved ligningen for generelle logaritmiske spiraler som
\begin{equation}
	r = ae^{c\theta}
	\label{log_spiral_2}
\end{equation}
eller
\begin{equation}
	\theta = \frac{1}{c}\ln(r/a)
	\label{log_spiral_1}
\end{equation}
hvor $e$ er grundtallet for den naturlige logaritme og $c$ skal have en
speciel værdi for at kunne være en gylden spiral. Den gyldne spiral kan
approksimeres ved at konstruere en fibonaccispiral som vist i figur
\ref{fibonacci_spiral}.
\begin{figure}[h!]
	\begin{center}
		\includegraphics[scale=0.35,angle=0]{afsnit/baggrund/billeder/Fibonacci_spiral}
	\end{center}
	\caption{En fibonaccispiral - \emph{Kilde: Wikipedia}}
	\label{fibonacci_spiral}
\end{figure}
Da fibonaccispiralen er konstrueret efter Fibonaccis talrække nærmer
denne spiral sig en gylden spiral, men kan ikke betegnes som værende en
ægte gylden spiral.

Med den matematiske definition på plads kan vi kigge på den forskning
der er blevet gjort i forbindelse med det gyldne snit.

% vim: set tw=72 spell spelllang=da:


\subsection{Litteraturstudie}
{
Som allerede nævnt, kendte de gamle grækere til tallet $\varphi$, som
Euklid kaldte for \emph{the division in extreme and mean ratios},
forkortet DEMR. Luca Pacioli udgiver i 1509 bogen \emph{De divina
proportione}, som beskriver samme fænomen omtalt som ``den guddommelige
propertion''. Udtrykket ``det gyldne snit'' kan spores tilbage til
tyskeren Martin Ohm (1792 -- 1872), der første gang betegner Euklids DEMR
som \emph{der Goldener Schnitt} i sin bog \emph{Die reine
Elementar-Mathematik}, fra 1835\cite{Markowsky1992}. Det gyldne snit er
nu blevet den foretrukne betegnelse.

Nu bliver det påstået fra flere kilder, at det gyldne snit bruges i
malerier, arkitektur og musik, da dette forhold har specielt tiltalende
æstetiske
egenskaber\cite{GoldenNumber}\cite{RatioArt}\cite{Putz1995}\cite{Stakhov2006490}\cite{Boussora2004}.
Især \cite{GoldenNumber} er særlig ivrig og finder det gyldne snit i alt
lige fra cigaretpakker til skallen fra en nautil. Netop nautilskallen
bliver meget ofte brugt som argument for, at det gyldne snit findes i
naturen i form af en gylden spiral lignende den fra figur
\ref{fibonacci_spiral}. Hvis man rent faktisk måler efter,
viser det sig, at dette ikke er tilfældet. Som argumenteret i
\cite{Sharp2002} er spiralen i nautilskallen rigtigt nok logaritmisk,
men ikke med en faktor $c$ som ville gøre den til en gylden spiral.

Det græske tempel Parthenon spiller også en central rolle i rygterne om
det gyldne snit. Billeder af templet bliver tit vist med et rektangel
tegnet over. Der florerer forskellige udgaver af disse billeder, hvor
der åbenbart ikke tages højde for, at dele af templet ikke indgår i
rektanglet eller fotografiets perspektiv. George Markowsky gør i
\cite{Markowsky1992} op med mange af disse vrangforestillinger. Påstande
om, at Keopspyramiden skulle være bygget efter det gyldne snit --- som
angivet i \cite{Stakhov2006490} --- at nogle af Leonardo da Vincis malerier
er malet efter det gyldne snit, og at det gyldne rektangel er det mest
æstetisk tiltrækkende format\cite{GoldenNumber}\cite{RatioArt}, bliver i
hans artikel afvist, da mange af disse undersøgelser lider under det, han
kalder \emph{the Pyramidology Fallacy}. Dette udtrykt henter Markowsky
fra \cite{Gardner1952_2} og beskriver de personer, der forsker i
pseudovidenskab såsom pyramidernes arkitektur. Her har forskerne ofte
mange forskellige tal at ``jonglere'' med og kan frit vælge netop dem,
der giver det ønskede resultat. En anden forfatter, Roger
Hertz-Fischler, er ligeledes optaget af den specielle værdi, det gyldne
snit er blevet tillagt. Det som Markowsky og Gardner kalder for
\emph{Pyramidology}, betegner han som \emph{golden numberism}, og han har
fulgt dette fænomens historie tilbage til en tysk mand ved navn
Adolph Zeising (1810 -- 1876)\cite{Herz-Fischler2005}. Hertz-Fischler
hævder, at stort set alle undersøgelser inden for \emph{golden numberism}
kan spores tilbage til Zeising.

Hypotesen om det gyldne rektangels æstetiske egenskaber bliver også
taget op i \cite{Boselie1984} og \cite{Plug1980}, hvor det ikke kan
konkluderes, at det gyldne rektangel skulle have nogen æstetisk
signifikans. En lignende hypotese bliver sat på prøve i
\cite{McManus1995}, hvor det undersøges, om et billedes geometriske
komposition har indflydelse på dets æstetiske effekt. I. C. McManus
kunne, gennem tre eksperimenter, ikke finde noget grundlag for at et
billedes geometriske komposition påvirkede testpersonernes bedømmelse.
Det siges i konklusionen:

\begin{quote}
	\emph{``Together the results of these experiments throw
	considerable doubt upon the hypothesis of the implicit detection
	of latent compositional geometry as a major component of
	aesthetic judgements, at least for relatively unsophisticated
	observers[\dots]''}
\end{quote}

Selvom ovenstående ikke eksplicit nævner det gyldne snit, er der
alligevel en klar relevans til billeder opbygget geometrisk efter det
gyldne snit.

Der ses dog et klart problem, hvilket Markowsky også nævner, idet det
ikke er defineret, hvornår et stykke kunst er konstrueret efter det
gyldne snit. Ej heller er det defineret, præcis hvordan man finder det
gyldne snit: Der findes mange billeder, hvor snittet er illustreret vha.
linjer, men størstedelen af disse har ikke de fornødne mål, og det
gyldne snit findes gerne helt arbitrære steder uden nogen som helst
identificerbar fremgangsmåde.  Én undersøgelse, omhandlende billeders
komposition, har dog lavet statistik på 565 malerier, men kun forholdet
mellem lærredets dimensioner er blevet registreret\cite{Olariu1999}.
Denne undersøgelse kunne ikke konkludere, at kunstnere foretrækker det
gyldne snit i lærredet. Det er derfor interessant at forsøge at
automatisere søgningen efter det gyldne snit i malerier og fastsætte
klare kriterier for, hvornår et billede kan siges at være konstrueret
efter det gyldne snit.  Til denne opgave er det oplagt at udnytte
regnekraften fra computeren, hvilket også giver anledning til  at
analysere langt større datasæt.

\subsection{Eksisterende datalogisk forskning}
Den tidligere datalogiske forskning ligger i generelle teorier og algoritmer
indenfor billedbehandling end en tidligere forskning præcis indefor analyse af
interessante regioner i nærheden af det gyldne snit.
Det mest relaterede forskning er et eksperiment, som analysere æstetik i fotografier\cite{DattaWang}.
Det væsenlige ved dette eksperiment er hvorledes billederne
bedømmes. Her arbejdes der med 56 forskellige kandidater til at definere
et billede, som mere eller mindre æstetisk korrekt og originalt. 
Kandidaterne bliver udvalgt i forskellige kategorier, tre af kandidater 
bliver vurderet ud fra ``The rule of Thirds'', som er beskrevet ved
\begin{quote}
	``The rule can be considered as a sloppy approxmination of to the
	golden ratio (about 0.618)''
\end{quote}

%fyld? Meget fedt at få sagt det men jeg ved ikke om det er specielt
%vigtigt
Deres kandidater dækker over noget ganske centralt i store dele af
billedbehandling nemlig detektion af
interessante regioner i billeder, problemet er at begrebet slet ikke er
entydigt defineret, men skifter mening fra fagområde til fagområde.
Hvad vi opfatter som interessant vil blive diskuteret i \ref{section_kort_intro}

De forskellige metoder til at detektere interessante regioner, indeholder nogle avancerede tekniker.
Potentielt kunne disse tekniker forbedre udtrækning af interessante regioner.

Den første er maskineindlæring, der stammer fra feltet kunstig
intelligens indenfor datalogi. I billedbehandling bliver det dog
brugt til at finde interessante regioner med stor præcision, og formålet
med brugen er bland andet at finde ansigter, mennesker og
biler\cite{ViolaJones01}\cite{SchneidermanKanade00}\cite{Gabor}. Præcisionen på
algoritmernes detektion koster dog meget kompleksiteten i udvikligen 
af algoritmerne. Programmet skal også have en base for at blive oplært
til at finde interessante regioner, og derved låser programmet fast til
kun at finde det som oplæringsbasen er.

Den anden er opdelinger af billeder efter tekstur. Ideen bag denne type
af metoder er at opdele efter
overfladetekstur\cite{218442}\cite{CarsonBelongie02}\cite{PapageorgiouPoggio}, de passer meget bedre på
den type region der ledes efter. De besidder dog adskillige
negative egenskaber. En af disse er at de fleste algoritmer har svært
ved at fungere på støj i billedet, der er selvfølgelig veje at reducere
dette problem, dog er de meget beregningstunge.\cite{PalPal}

}
% vim: set tw=72 spell spelllang=da:


% vim: set tw=72 spell spelllang=da:


\chapter{Detektion af det gyldne snit\label{chap_detektion}}
\subsection{Opdeling af billeder}
\subsection*{Trunkerings- og afrundingsproblemer}
Mange af de metoder, vi bruger til at udregne, det gyldne
snits position eller størrelsen af en margin, udregnes med brøker. Hvorimod et
billede opbygges af pixels. Dette gør, at vi bliver nødt til at tage
approksimationer af udregningerne for at få dem tilbage til et helt tal.
Men hvor meget af data er gået tabt, og hvor mange af resultaterne kan
man stole på?

\subsubsection{Acceptabel afvigelse}
\note{Nogle referanse}Som beskrevet i afsnit \ref{mange_tal}, udregnes
det gyldne snit med mange decimaler. En kunstner, hvor god han end er,
har ingen chance for at male så præcist at man kan sige at strøget
ligger nøjagtigt oven på snittet selv om hans ententioner er at ramme
snittet. Vi kommer derfor til at have en vis uprecis hed på de data vi
få fra billedet selv. Vi starter med at se på alle de ting, som kan
skabe en usikkerhed fra malerens side. Man kan gå ud fra at den
procentvise afvigelse ikke er særlig stor, da vi ikke har bestræbelser
på at arbejde på abstrakt malerier, dog har vi sat den procentvise
afvigelse til 0.5 \%. Det vil sige at en maler med et lærred på 100
cm,maksimalt vil male $0.5$ cm forkert.

Når maleren vælger en ramme og et lærred, har vi igen problematikken,
selv om maleren spicifikt gå efter at bygge maleriet op efter det gyldne
snit, kan snittets placering i maleriet have forskubbet sig, ved dårlige
valg at ramme eller lærred. Derfor sætter vi den afvigelse til $1\%$. Da
vi igen mener at dette er den maksimale afvigelse, der kan opstå.

Når maleren maler en region i et maleri, forekommerer der normalt en
lille kant rundt om objektet, et omrids. Dette omrids kan vores
algoritmer ikke tage højde for, og vi må derfor modregne omridset, så vi
er sikre på at vi ser på regionen og ikke dens omrids. Da et omrids ikke
er særligt stort, har vi sat denne procentsats til $0.5\%$. 

Alt i alt giver det en afvigelse på vores aktuelle udtrækning af data
fra malerierne på $2\%$ det vil sige at finder vi en region, som ligger
på pixel 200, i et billedet, der er $500$ cm bredt, befinder den sig
faktisk i intervallet [190,210]. Måden hvorpå vi tager højde for den
forskel, er ved hjælp af marginer, som nævnt i afsnit
\ref{section_naiv}.

\subsection{Inddeling af billede efter snit}

I et billede betegnes højden og bredden som hhv. $H$ og $B$, se figur
\ref{cut}. Der er 4 gyldne snit, 2 vertikale og 2 horisontale, som vist
i figur \ref{lenasnit2}. For at finde ud af, hvor de 4 snit skal ligge i
billedet, multipliceres B og H med $\varPhi$,og man får to tal. Disse
betegner, hvor mange pixels, det gyldne snit befinder sig fra hhv. $H$
og $B$ f.eks vil de 2 horisontale snit, i et billedet, som har $B =
4000$ pixel, ligge hhv. $4000 \cdot \varPhi \approx 2472$ pixels fra
billedets øvre og nedre kant.

\begin{figure}[h]
	\begin{center}
		\includegraphics[scale=0.42,angle=0]{afsnit/vores_implementation/billeder/naiv_algoritme/Lenagolden}
	\end{center}
	\caption[]{Billedet som har indtegnet de fire gyldne snit}
	\label{lenasnit2}
\end{figure}

\begin{figure}[h]
	\begin{center}
		\includegraphics[scale=0.42,angle=0]{afsnit/vores_implementation/billeder/naiv_algoritme/Cut}
	\end{center}
	\caption[]{Billedets højde og bredde betegnes hvv. H og B. De 4 snit er navngivet.}
	\label{cut}
\end{figure}

De 4 snit tildeles hvert deres Id, "snit 0,1,2 og 3" så vi kan kende
forskel på de individuelde snit, Id'erne placering kan ses i figur
\ref{cut}. Vi vil i resten af rapporten kalde snittene efter deres Id.
Hvis vi gerne vil finde snittet som ligger i miden kommer der kun 2
snit, med vær deres Id "snit 0 og 1" som kan ses i figur \ref{Cut2}

\begin{figure}[h]
	\begin{center}
		\includegraphics[scale=0.42,angle=0]{afsnit/vores_implementation/billeder/naiv_algoritme/2Cut}
	\end{center}
	\caption[]{Billedet skæres her kun af 2 snit}
	\label{2Cut}
\end{figure}

\subsubsection{Heltal i det gyldne snit}

I eksemplet med 4000 pixels ovenfor, approksimerer vi antal pixels ved
at afrunde resultatet $2472.13595 \approx 2472$, se udregning
\ref{afrundning}. Det betyder at vi mister 0.13595 pixels i præction,
hvilket svarer til en misvisning af punktet på 0.00339875 $\%$ i forholdt
til $B$ på billedet. Se udregning \ref{afrundning2}.

\begin{equation}
	4000 \cdot \varPhi = 4000(\sqrt{5}-1)/2 = 2472.13595 \approx 2472 \label{afrundning}
\end{equation}

\begin{equation}
	0.13595/4000 \cdot 100 = 0.00339875 \label{afrundning2}
\end{equation}

Det er en meget lille del af selve billedet og skulle ikke give nogle
misvisninger i forhold til udregningen. For at gøre det lidt mere
generelt, sætter vi trunkeringsfejlen til $0.5$, da det er den maksimale
afrundingsfacktor som kan forekomme. Hvis billedet har en størrelse på
500 pixels, hvilket er det mindste billedet vi har, giver dette en fejlmargin
på $0.1 \%$. Dette tal bliver adderet til fejlsatsen ovenfor, og giver
en samlet afvigelse på $2.1\%$.

\subsubsection{Snitratio}
End til vider har vi kun arbejder med det gyldne snit, men andre snit i
billedet kan godt optræde, derfor indføre vi en nu betegnelse
snitratio, som betegner en procents sats for hvor lang inde i billedet
snittet befinder sig. Det vil sige at hvis en snitration er på $0.2$. Et
billedet har $B$ på 4000 vil et snit befinde sig i pixel $4000*0.2 =
800$.

\subsubsection{Heltal ved udregning af Margin}
Når vi har 2 forskellige snitratioer, f.eks. $\varPhi$ og $\frac{2}{3}$,
som ligger meget tæt på hinanden, og vi gerne vil sammenligne hvilken
regioner der ligger i snitratioernens snit, er det vigtigt at margin for
vært af de 2 snitratioers snit ikke krydser hinanden. 

Hvis margin krydser. Vil det indebære, at den samme region bliver fundet
af begge snit. Dette vil give et skævt billedet af forskellen på de to.
Derfor må vi sørge for at marginerne ikke krydser. Hvis $x$ betegner
antal pixels i $B$ eller $H$, og vi vil se på, hvor mange pixels, der er
mellem snitratio $\frac{2}{3}$ og $\varPhi$, multiplicerer vi $x$ med de
to snitratioen for at finde deres placering. Derefter subtraheres vi et
af de snit som befinder sig tetest på hinanden i vær snitratio med
hinanden.

\begin{eqnarray}
	\frac{x2}{3} - \frac{x2}{\sqrt{5}+1} & = & x(\frac{2}{3} - \frac{2}{\sqrt{5} + 1}) \nonumber \\
	& = & x(0.666667-0.618034) \\ \nonumber
	& = & x(0.048633)
\end{eqnarray}

Vi har nu fundet antal pixels mellem de to snit. Vi vil gerne undgå at
de to marginens ikke krydser hinanden, så der dividere vi med 2 og
afrunder værdien.

\begin{equation}
	\left\lfloor \frac{0.048633x}{2}\right\rfloor = \left\lfloor0.024316x \right\rfloor
\end{equation}

Tallet $2.4316$ er altså den minimale procentvise størrelse som vores
margin må have, nå vi sammen liner det gyldne snit og $\frac{2}{3}$.
Det betyder også at vi ikke må sammen line snit som ligger særlig
meget tætter på hinanden, da 0.021 er den minimale procent margin som vi må have.
$\lfloor 0.024316x \rfloor$ giver os et antal pixels som skildre de to snit. For at vise
hvor stort marginen egentlige kan være, bruger jeg denne formel på to
billeder, et som svare til vores mindste billedet, 500 pixels, og et
som svare til vores største billedet, 4000 pixels. Ved 500 pixels
bliver resultatet

\begin{equation}
	 \lfloor 500(0.024316)\rfloor = 12
\end{equation}

Det er en fint margin, da vores fejl på udregningerne ligger på 2.1 \%,
som svare til $\lceil 500*0.021 \rceil = 11$ pixels, som er 1 pixels fra vores
margin

Ved 4000 pixels giver det.

\begin{equation}
	 \left\lfloor 4000(0.024316)\right\rfloor = 97
\end{equation}

Som også er god nok da $4000*0.021 = 84$ pixels.
% vim: set tw=72 spell spelllang=da:



\chapter{Afprøvning\label{chap_afproevning}}
{
I dette Kapitel vil vi afprøve vores 2 metoder på XX(hvormange) billeder og se hvordan de virker. Udover det vil vi finde tærskelværdier til kandetection, floodfill og margin, ud fra observationer gjort under afprøvning. Vi vil også afprøve den generalle region detektor og kommer ind på dens fordele og ulemper. 
\section{Tærskelværdier}
%% Bemærk:
%%          Resten af rapporten følger en stil hvor indledninger skrives
%%          med \sffamlily-typen. Denne stil bør også følges her.
%%
{\sffamily
I vores program er der 4 forskellige tærskelværdier som påvirker hvordan
regions detektoren analysere billedet. Alle 4 tærskelværdier er blevet
introduceret i deres respektive afsnit, men ingen konkrate tal er
opgivet. I dette afsnit vil vi opgive de tal, samt en forklaring på
hvorfor vi har valt lige de tal. Malerierne som er brugt i afprøvning er
spicældt udvalgt for deres illustrative aspekt. 
}

\subsection{Marginens brede}
Vi regner marginens brede ud fra en procent stats $\psi$ af billedets
$B$ og $H$. I afsnittet \ref{section_opdeling} kom vi frem til en
usikkerhed på $2.1 \%$. Så $\psi$ skal være støre en $2.1 \%$. 

Den minimale forskel på 2 snit vi foretager os, er forskellen mellem det
gyldne snit og $\frac{2}{3}$, margin brade udregnet i section
\ref{section_opdeling} til maksimal at være $2.43\%$. For at marginen
ikke krudser. 

Det vil sige at $\psi \in [2.1, 2.43]$. Vi har valt at
sætte $\psi = 2.4$, da vi derved kan tage højde for uforresette
usikkerhed.

\subsection{Afvigelsen af farver i kandtdetection}
Vi bruger kandtdetection til, er af finde en kanter rundt om de regioner
som vi mener er interessante, og undgå de kanter som ligger inde i
regioner. Begge de 2 mål kan ikke altid opfyldes, men vi kan komme så
tæt på et krompromi mellem en perfekt kant rund om region og ingen
kanter inde i region som mulige. Dette gøres ved at ændre 2
tærskelværdier i kantdetectionen. Vi har valt at dele billederne som vi
observere op i 9 kategorier, se tabel \ref{thressholdsTabelKant},
Kategorier er en grove opdeling af billederne efter detaljer og farve
intensitet, som bruges til at give en bedre indblik på billedets
opbygning. 

\subsubsection{Sammenligninger}
Vi har set på 9 malerier og har fundet de tærskelværdier som vi mener
passer bæst på malerierne. Vi vil illustrerer hvordan vi har fastsat
tærskelværdierne på maleriet \ref{kDetalier}.

Maleriet er malet med mange farver og med masser af detaljer. Vi ser
først på tærskelværdierne $(0,0),(100,100).....(600,600),(700,700)$ se
figur \ref{allesammen1} og \ref{allesammen2}, og finde det interval hvor
malerriet ikke har mistet nogle af kanterne rundt om regionerne endnu,
men vil det, i næste interval. I illustration vurdere vi det til
billedet \ref{300-300}, da billedet \ref{400-400} har mistet for mange
af de kanter, som vi gerne vil beholde.

Ved at sætte en af tærskelværdierne op lidt af gangen, kan vi igen få en
række billeder at vælge imellem. Se sammenligningen i figur
\ref{allesammen3}. Man kan se at det begynder at være svær at skelne
figurene i \ref{300-850} og der er lidt for mange kanter i
\ref{300-700}.

Vi har valgt at fastsætte tærskeværdigerne til $(300,750)$. Det maleri
vi lige har brugt er ikke særlige repræsentativ for helle vores maleri
database. så vi har taget 8 andre billeder og brugt samme metode på dem og fastsat en
middel tærskelværdi. Vi viser her en lille udsnit af dem, se figur
\ref{en}, \ref{to} og \ref{tre}.
\clearpage
\begin{figure}[p]
    \centering
    \subfloat[(100,100)]{
        \includegraphics[angle=0,width=0.45\textwidth]{afsnit/afprovning/billeder/thressholds/krafitige_farver/krafite_detalier/1_iteration/100-100.png}
        \label{100-100}}\hspace{1em}
    \subfloat[(200,200)]{
        \includegraphics[angle=0,width=0.45\textwidth]{afsnit/afprovning/billeder/thressholds/krafitige_farver/krafite_detalier/1_iteration/200-200.png}
        \label{200-200}}\\
    \subfloat[(300,300)]{
        \includegraphics[angle=0,width=0.45\textwidth]{afsnit/afprovning/billeder/thressholds/krafitige_farver/krafite_detalier/1_iteration/300-300.png}
        \label{300-300}}\hspace{1em}
    \subfloat[(400,400)]{
        \includegraphics[angle=0,width=0.45\textwidth]{afsnit/afprovning/billeder/thressholds/krafitige_farver/krafite_detalier/1_iteration/400-400.png}
        \label{400-400}}
    \label{allesammen1}
    \caption{Edgedetection på maleriet \ref{kDetalier} som har mange detaliger og kraftige farver, med tærskelværdierne fra (100-100) til (400-400)}
\end{figure}

\clearpage

\begin{figure}[!h]
	\centering
	\subfloat[(500,500)]{
        \includegraphics[angle=0,width=0.45\textwidth]{afsnit/afprovning/billeder/thressholds/krafitige_farver/krafite_detalier/1_iteration/500-500.png}
        \label{500-500}}\hspace{1em}
    \subfloat[(600,600)]{
        \includegraphics[angle=0,width=0.45\textwidth]{afsnit/afprovning/billeder/thressholds/krafitige_farver/krafite_detalier/1_iteration/600-600.png}
        \label{600-600}}\\
    \subfloat[(700,700)]{
        \includegraphics[angle=0,width=0.45\textwidth]{afsnit/afprovning/billeder/thressholds/krafitige_farver/krafite_detalier/1_iteration/700-700.png}
        \label{700-700}}\hspace{1em}
    \subfloat[Original. Navn: The Archangel Michae. År: Ca 1490 Af: Abadia, Juan de la]{
        \includegraphics[angle=0,width=0.45\textwidth]{afsnit/afprovning/billeder/thressholds/krafitige_farver/krafite_detalier/kDetalier.jpg}
        \label{kDetalier}}
    \caption[]{Edgedetection på maleriet \ref{kDetalier} som har mange detaliger og kraftige farver, med tærskelværdierne fra (500-500) til (700-700)}
     \label{allesammen2}
\end{figure}

\begin{figure}[!h]
    \centering
    \subfloat[(300,700)]{
        \includegraphics[angle=0,width=0.45\textwidth]{afsnit/afprovning/billeder/thressholds/krafitige_farver/krafite_detalier/2_iteration/300-700.png}
        \label{300-700}}\hspace{1em}
    \subfloat[(300,750)]{
        \includegraphics[angle=0,width=0.45\textwidth]{afsnit/afprovning/billeder/thressholds/krafitige_farver/krafite_detalier/2_iteration/300-750.png}
        \label{300-750}}\\
    \subfloat[(300,800)]{
        \includegraphics[angle=0,width=0.45\textwidth]{afsnit/afprovning/billeder/thressholds/krafitige_farver/krafite_detalier/2_iteration/300-800.png}
        \label{300-800}}\hspace{1em}
    \subfloat[(300,850)]{
        \includegraphics[angle=0,width=0.45\textwidth]{afsnit/afprovning/billeder/thressholds/krafitige_farver/krafite_detalier/2_iteration/300-850.png}
        \label{300-850}}
        \caption[]{Edgedetection hvor de 4 billeder som er intrasante taget med}
     \label{allesammen3}
\end{figure}
 
\begin{figure}[!h]
    \centering
    \subfloat[(100,250)]{
        \includegraphics[angle=0,width=0.45\textwidth]{afsnit/afprovning/billeder/thressholds/svage_farver/svage_detalier/2_iteration/100-250.png}
        \label{100-250}}\hspace{1em}
    \subfloat[Orginal. Navn: The Lamentation over St Francis. År: 1440. Af: Angelico, Fra. ]{
        \includegraphics[angle=0,width=0.45\textwidth]{afsnit/afprovning/billeder/thressholds/svage_farver/svage_detalier/sDetalier.jpg}
        \label{Orginal1}}
        \caption[]{Edgedetection på et billedet med svage farver og få detalier, hvor tærskenværdigern [100,250] er den beste}
     \label{en}
\end{figure}

\begin{figure}[!h]
    \centering
    \subfloat[(100,240)]{
        \includegraphics[angle=0,width=0.85\textwidth]{afsnit/afprovning/billeder/thressholds/medium_farver/svage_detalier/2_iteration/100-240.png}
        \label{100-240}}\\
    \subfloat[Orginal. Navn:The Ninth Wave. År:1850. Af:Aivazovsky, Ivan Konstantinovich.]{
        \includegraphics[angle=0,width=0.85\textwidth]{afsnit/afprovning/billeder/thressholds/medium_farver/svage_detalier/sDetalier1.jpg}
        \label{Orginal2}}
        \caption[]{Edgedetection på et billedet med medium farver og få detalier, hvor tærskenværdigern [100,240] er den beste}
     \label{to}
\end{figure}

\begin{figure}[!h]
    \centering
    \subfloat[(200,460)]{
        \includegraphics[angle=0,width=0.85\textwidth]{afsnit/afprovning/billeder/thressholds/medium_farver/medium_detalier/2_iteration/200-460.png}
        \label{200-460}}\\
    \subfloat[Orginal. Name: The last supper. År: 1498. Af: Leonardo da Vinci]{
        \includegraphics[angle=0,width=0.85\textwidth]{afsnit/afprovning/billeder/thressholds/medium_farver/medium_detalier/mDetalier1.jpg}
        \label{Orginal3}}
        \caption[]{Edgedetection på et billedet med medium farver og medium detalier, hvor tærskenværdigern [200,460] er den beste}
     \label{tre}
\end{figure}

\begin{table}[!h]
    \centering
    \begin{tabular}{| l | l | l | l |} \hline
        & Svage farver 	& Medium farver & Kraftige farver \\ \hline
        Få detaljer 		& (100,250)		& (100,240)		& (200,320)\\ \hline
        Medium detaljer 	& (100,280)		& (200,460)		& (200,380)\\ \hline
        Mange detaljer		& (200,400)		& (200,380)		& (300,750)\\ \hline
    \end{tabular}
    \caption{Tabel over kantdetektionstærskelværdier for ni malerier}
    \label{thressholdsTabelKant}
\end{table}

Som man kan se af tabel \ref{thressholdsTabelKant} gå tærskelværdierne
fra $(100,240)$ til $(300,750)$, så vi tager en gennemsnit af værdierne
og få at de to tærskelværdier skal være (177 og 384). 

Vi har dog i vores forsøg regnet med værdierne 78 og 194, da vores
indledende afprøvninger afviger en smugle fra den denne afprøvning.

\subsection{Afvigelsen af farver i floodfill}
Floodfill har 2 tærskelværdier $lo$ og $up$, som betegner hvor mange
pixel værdier en nabo pixel farver må variere, ned og op. En
fyldestgørelses beskrivelse af floodfill findes i afsnit
\ref{section_opdeling}. 

Vi har tænkt os at finde en fældes tærskelværdi til brug i vores
program. Måde vi gør det på at ved at observere hvordan floodfill virker
med forskellige tærskelværdier og finde de tærskelværdier som passer
bæst til maleriet. Resultatet for observationen kan ses i tabel
\ref{thressholdsTabelFF}, hvor de 9 sammen kategorier er vist og
afprøvet på de samme 9 malerier. Et af de malerierne kan se i
figur \ref{Floodfillbilledet}, hvor man kan se hvad vi har valt til at
være de optimale værdier.

\begin{figure}[!h]
    \centering
    \subfloat[8,8]{
        \includegraphics[angle=0,width=0.9\textwidth]{afsnit/afprovning/billeder/thressholds/svage_farver/kraftige_detalier/floodfill/8-8.png}
        \label{8-8}}\\
    \subfloat[Orgina. Navn: Winter Landscape. År: Ukent. Af:Avercamp, Hendrick.]{
        \includegraphics[angle=0,width=0.9\textwidth]{afsnit/afprovning/billeder/thressholds/svage_farver/kraftige_detalier/kDetalier.jpg}
        \label{Orginal4}}
    \caption[]{tærskelværdierne på et billedet med svage farver og kraftige detaljer hvor tærskelværdien [8,8] passer best}
    \label{Floodfillbilledet}
\end{figure}

\begin{table}[!h]
    \centering
    \begin{tabular}{| l | l | l | l |} \hline
        & Svage farver 		& Medium farver & kraftige farver \\ \hline
        Få detalier 		& \textbf{[2,2]}	& [3,3]			& [4,4]\\ \hline
        Medium detalier 	& \textbf{[2,2]}	& \textbf{[5,5]}& \textbf{[2,2]}\\ \hline
        Mange detalier		& [8,8]				& [4,4]			& [7,7]\\ \hline
    \end{tabular}
    \caption{Tabel over floodfill tærskelværdier på 9 malerier}
    \label{thressholdsTabelFF}
\end{table}

Som man kan se af tabel \ref{thressholdsTabelFF}, er nogle af vadierne
med fed, begrundelsen for det, er at vadierne, er de beste, vi kan finde
for billedet, men at de vadier stadig ikke giver noget som er særlige
brugbart. værdigerne i tabel fluktuere også en del, så vi tager
gennemsnittet og får tærskelværdierne til at være (5,5).

\section{Regions detektor}
%% Bemærk:
%%          Resten af rapporten følger en stil hvor indledninger skrives
%%          med \sffamlily-typen. Denne stil bør også følges her.
%%

{\sffamily I dette afsnit vil vi afprøve den generelle metode for udtrækning af
regioner. Selve metodens fremgangsmåde står beskrevet i afsnit
\ref{section_udtraek}. Tærskelværdierne er sat til de fundne
tærskelværdier i afsnit \ref{terskelverdi}. Første del af afsnittet vil
omhandle afprøvning af metoden på testbilleder, og anden del afprøvning
ad metoden på udvalgte billeder fra databasen.}

\subsection{Afprøvning på testbilleder}
I dette følgende afprøves metoden på billeder konstrueret med en hvid
baggrund og sorte regioner. Snittet, som der vil blive kigget på, vil
blive tegnet med rødt på billederne. Snittes margin vil blive tegnet med blåt.

I billedet \ref{GRD_test1} er der fem regioner. Tre af dem bliver
fundet, da de ligger i snittet. De to sidste regioner som ikke er fundet, er
stadig sorte. Som man kan se, bliver både den, der krydser snittet og den,
der tangere snittet, taget med. 

\begin{figure}[!h]
    \centering
    	\subfloat[Det originale maleri.]{
	       	\includegraphics[angle=0,width=0.7\textwidth]{afsnit/afprovning/billeder/region_selector/blob_section.png}
	       	\label{GRD_test1_original}}\hspace{1em}
		\subfloat[Her ses fire regioner samt en baggrund. To af figurerne og baggrunden er blevet fundet af metoden.]{
        	\includegraphics[angle=0,width=0.7\textwidth]{afsnit/afprovning/billeder/region_selector/blob_region_section.png}
        	\label{GRD_test1}}\hspace{1em}
        \caption[]{}
     \label{GRD_test1_sammen}
\end{figure}

I billedet \ref{GRD_test2} er der tre regioner, som
alle bliver fundet; baggrunden, den lille region som ligger indenfor
marginen, og den store region som kun har en lille del af sin masse
i snittet. 

\begin{figure}[!h]
    \centering
    	\subfloat[Det originale maleri.]{
	       	\includegraphics[angle=0,width=0.7\textwidth]{afsnit/afprovning/billeder/region_selector/lille_tvers.png}
	       	\label{GRD_test2_original}}\hspace{1em}
		\subfloat[På billedet er der fundet en stor og en mindre region.]{
        	\includegraphics[angle=0,width=0.7\textwidth]{afsnit/afprovning/billeder/region_selector/blob2_region_section.png}
        	\label{GRD_test2}}\hspace{1em}
        \caption[]{}
     \label{GRD_test2_sammen}
\end{figure}

I billedet \ref{GRD_test3} er der en horisont, som ligger oven på
snittet. Begge sider af horisontlinjen bliver udtrukket som en region.

\begin{figure}[!h]
    \centering
    	\subfloat[Det originale maleri.]{
	       	\includegraphics[angle=0,width=0.7\textwidth]{afsnit/afprovning/billeder/region_selector/hoisont.png}
	       	\label{GRD_test3_original}}\hspace{1em}
		\subfloat[På billedet er der fundet to store regioner.]{
        	\includegraphics[angle=0,width=0.7\textwidth]{afsnit/afprovning/billeder/region_selector/hoisont_region_section.png}
        	\label{GRD_test3}}\hspace{1em}
        \caption[]{}
     \label{GRD_test3_sammen}
\end{figure}

I de tre figurer \ref{GRD_test1_sammen}, \ref{GRD_test2_sammen} og
\ref{GRD_test3_sammen} kan ses før og efter metoden er blevet anvendt.
Metoden opfører sig efter de standarder, som vi fremsatte i afsnit \ref{section_naiv}, og
virker efter vores forventninger.
\clearpage

\subsection{Afprøvning på malerier}
Vi vil afprøve metoden til udtrækning af regioner på seks udvalgte
malerier. Malerierne skal demonstrere, hvordan metoden på nogle malerier
fungerer godt, mens den fungerer dårligt på andre.

\begin{figure}[!h]
    \centering
		\subfloat[Maleri med kraftige farver og få detaljer. Navn: Scenes from the Story of Joseph: The Arrest of His Brethren. År: 1515-16. Af: Bacchiacca.]{
        	\includegraphics[angle=270,width=1.0\textwidth]{afsnit/afprovning/billeder/thressholds/krafitige_farver/svage_detalier/floodfill/4-4.png}
        	\label{GRD_virker1}}\hspace{1em}
		\subfloat[Maleri med middel kraftige farver og medium antal detaljer.]{
        	\includegraphics[angle=0,width=1.0\textwidth]{afsnit/afprovning/billeder/4-4.png}
        	\label{GRD_virker2}}\hspace{1em}
        \caption[]{To malerier, hvor den generelle metode virker efter vores ønsker.}
     \label{generelde_region_detektor_virker}
\end{figure}

I figuren \ref{generelde_region_detektor_virker} er der to malerier, hvor
vores regionsudtrækning virker rigtig godt. 

I det første maleri \ref{GRD_virker1} finder metoden syv store regioner
samt en del små. Den skelner mellem de forskellige paneler om kaminen,
og hvert stykke tøj på personen i maleriet opfattes som en særskilt
region. De små regioner er samlet, og forstyrre ikke de store regioner.
Man kunne måske have ønsket sig, at metoden ville fylde mere af
personens kappe ud, men bortset fra det er figuren et godt eksempel på,
hvordan det ser ud, når den generelle metode fungerer.

I maleriet \ref{GRD_virker2} finder vi mange af de samme positive ting:
drengen i midten af maleriet er helt udfyldt, en sko, drengens
badebukser og et håndklæde er også fundet. Det er dog vigtigt at lægge
mærke
til, at de andre personer i vandet, flyder i et med baggrunden. Dette
ville normalt være uheldigt, men eftersom metoden kun ser efter regioner
i snittet -repræsenteret af den røde linje- gør det ikke noget i dette
tilfælde.

\begin{figure}[!h]
    \centering
	\subfloat[Maleri med kraftige farver og mange detaljer, hvor de tærskelværdierne fundet i afsnit \ref{terskelverdi} er brugt.]{
   	 	\includegraphics[angle=270,width=0.90\textwidth]{afsnit/afprovning/billeder/thressholds/krafitige_farver/krafite_detalier/floodfill/4-4.png}
	    \label{GRD_virker_nesten1}}\hspace{1em}
    \subfloat[Det samme maleri, hvor Tærskelværdier er valt specifikt for det her maleri]{
        \includegraphics[angle=270,width=0.95\textwidth]{afsnit/afprovning/billeder/thressholds/krafitige_farver/krafite_detalier/s7_e200_f5.png}
        \label{GRD_virker_nesten1_super}}\\
     \caption[]{Et malerier, hvor den generelle metode ikke helt virker efter vores ønske, men med nye tærskelværider vil virker meget bedre}
     \label{generelde_region_detektor_virker_nesten1}
\end{figure}

\begin{figure}[!h]
    \centering
    \subfloat[Maleri med middel kraftige farver og medium antal detaljer.]{
        \includegraphics[angle=0,width=0.95\textwidth]{afsnit/afprovning/billeder/thressholds/medium_farver/medium_detalier/floodfill/4-4.png}
        \label{GRD_virker_nesten2}}\\
	\subfloat[Det samme maleri, hvor Tærskelværdier er valt specifikt for det her maleri]{
   	 	\includegraphics[angle=0,width=0.90\textwidth]{afsnit/afprovning/billeder/thressholds/medium_farver/medium_detalier/s5_e90_e200_f4.png}
	    \label{GRD_virker_nesten2_super}}\hspace{1em}
     \caption[]{Et malerier, hvor den generelle metode ikke helt virker efter vores ønske, men med nye tærskelværdier vil virker meget bedre}
     \label{generelde_region_detektor_virker_nesten2}
\end{figure}

I figur \ref{generelde_region_detektor_virker_nesten1} og \ref{generelde_region_detektor_virker_nesten2} er der to
malerier, hvor metoden ikke fungerer optimalt efter hensigten. Dog kan
vi stadig bruge figurerne til noget

I maleriet \ref{GRD_virker_nesten1} bliver der hovedsageligt fundet små
regioner. En sko, en skulder og en flise trækkes ud, hvor vi hellere
ville have haft, at personens kappe og arm blev fundet. Dette skyldes
primært, at tærskelværdierne for dette billede er sat for lavt. 


Maleriet i den anden figur \ref{GRD_virker_nesten2} rummer nogle af de
samme problemer: Metoden finder også her en masse små dele af figurer,
der burde hænge sammen. Det vil også kunne løses ved nogle andre
tærskelværdier, men som man også kan se på maleriet er vægge og loftet -
som egentlig er ret ensfarvede -stadig svære at finde for metoden. Det
kunne tyde på, at en højere grad af sløring ville løse problemet, og få
algoritmen til at virke i maleriet.

Når netop de to figure \ref{GRD_virker_nesten1} og
\ref{GRD_virker_nesten2} er interessante, er det fordi, at en ændring af
tærskelværdier, samt en højre grad af sløring, ville bevirke, at den
generelle metode kom til at fungere bedre. I billedet
\ref{GRD_virker_nesten1_super} og \ref{GRD_virker_nesten1_super} er
protretere de samme malerier, men hvor tærskelværdierne passer bedre til
maleriet, man kan se det ved f.eks at karben på anglen samt det meste af
loftet bliver fundet.

\clearpage

\begin{figure}[!h]
    \centering
     \subfloat[Det originale maleri. Navn: Vase of Flower. År: ukent.
	 Af: Arellano, Juan de.]{
        \includegraphics[angle=0,width=0.46\textwidth]{afsnit/afprovning/billeder/thressholds/krafitige_farver/medium_detalier/mDetalier}
        \label{GRD_virker_ikke1_orginal}}
    \subfloat[Maleri med kraftige farver og medium detaljer.]{
        \includegraphics[angle=0,width=0.46\textwidth]{afsnit/afprovning/billeder/thressholds/krafitige_farver/medium_detalier/floodfill/4-4.png}
        \label{GRD_virker_ikke1}}
     \caption{Maleri af blomster, hvor den generelle metode ikke
	 virker.}
     \label{generelde_region_detektor_virker_ikke}
\end{figure}

\begin{figure}[!h]
	\begin{center}
	    \includegraphics[angle=0,width=0.65\textwidth]{afsnit/afprovning/billeder/thressholds/svage_farver/svage_detalier/floodfill/4-4.png}
	\end{center}    
	\caption{Maleri med svage farver og få detaljer.}
    \label{GRD_virker_ikke2}
\end{figure}

I figur \ref{generelde_region_detektor_virker_ikke} og
\ref{GRD_virker_ikke2} er der to malerier hvor
vores generelle metode ikke virker optimalt. 

I maleri \ref{GRD_virker_ikke1} er noget af buketten og baggrunden gået
i et. Desuden er resten af blomsterne i snittet ikke fyldt ud, og selv
en ændring i tærskelværdierne vil ikke hjælpe, da en forøgelse af disse
blot vil resultere i, at flere af blomsterne går i ét med baggrunden. En
sænkning vil resultere i, at ingen af blomsterne bliver trukket ud. 

Maleriet \ref{GRD_virker_ikke2} repræsenterer en anden problemstilling,
som vi ikke kan komme udenom: farverne er så mørke og svage, at en
ændring i tærskelværdien i floodfill på bare en vil få metoden til at gå
fra at finde ingen regioner til at finde for store regioner. det kan ses
i figur \ref{ff_munke}, hvor de 2 malerier, hvor floodfills tærskelværdi
er sat til den lavest og næst laves værdi. Som man kan se, finde metoden
mange snå regioner som ikke er særlige intrasant i maleri
\ref{munk_etff}, men i \ref{munk_toff}, bliver der fundet 2 meget store
regioner, som er blevet alt for store. Dette er et problem da vi så ikke
kan have en tærskelværdi som virker.

\begin{figure}[!h]
    \centering
     \subfloat[Tærskelværdierne for floordfill på maleriet er sat til en]{
        \includegraphics[angle=0,width=0.46\textwidth]{afsnit/afprovning/billeder/thressholds/svage_farver/svage_detalier/1-1.png}
        \label{munk_etff}}
    \subfloat[Tærskelværdierne for floodfill på maleriet er sat til to]{
        \includegraphics[angle=0,width=0.46\textwidth]{afsnit/afprovning/billeder/thressholds/svage_farver/svage_detalier/2-2.png}
        \label{munk_toff}}
     \caption{Et maleri hvor tærskelværdien for floodfill er sat til den laveste værdi den kan have, og et hvor denne næst laves tærskelværdien er sat.}
     \label{ff_munke}
\end{figure}

Dette kan også skyldes, at kantdetektionen tilføjer en mørk kant rundt
om regionerne, men da maleriet er mørkt i forvejen, hjælper
kandetektionen ikke. I maleri \ref{bla} er kanten tegnet med blåt og man
kan se at munken to fra venstre, ikke får sit skæg med, men hvor han går
det i maleriet med sort kant, se maleri \ref{sort}.

\begin{figure}[!h]
    \centering
     \subfloat[Maleri med blå kanter i kantdetektionen]{
        \includegraphics[angle=0,width=0.46\textwidth]{afsnit/afprovning/billeder/thressholds/svage_farver/svage_detalier/blueE.png}
        \label{bla}}
    \subfloat[Maleri med sorte kanter i kantdetektionen]{
        \includegraphics[angle=0,width=0.46\textwidth]{afsnit/afprovning/billeder/thressholds/svage_farver/svage_detalier/floodfill/4-4.png}
        \label{sort}}
     \caption{To forskellige farver brugt til at ligge kanter på maleriernes regioner}
     \label{fleremunke}
\end{figure}


\section{Naive løsning}
%% Bemærk:
%%          Resten af rapporten følger en stil hvor indledninger skrives
%%          med \sffamlily-typen. Denne stil bør også følges her.
%%
{\sffamily
I dette sektion vil vi teste den naive løsning, ved at se om den sortere
de rigtige regioner væk og om løsningen opføre sig på samme måde som vi
har håbet på. Det vil vi gøre ved ført at se på nogle fabrikeret test
billeder få at se om den naive løsning virker efter meningen og bag
efter vil vi teste på malerierne for at se om den naive løsning kan
bruges i praktisk.
}
  
\subsection{Afprøvning på testbilleder}
Vi vil teste på de samme billeder som var i figur
\ref{region_detektor_test}, samt nogle af de test billeder som blev
brugt i forklaringen af den naive metodes, de 4 billeder som vi har valt
at test kan se i figur \ref{naiv_detektion_test} hvor en grån kasse
rundt om en region, betyder at den er valt til at ligge i det gyldne
snit af den naive metode. Det første billedet \ref{naiv_blob1}, har 5
regioner og hvor 3 af dem blev fundet af reginon detektoren, vores naive
løsning har så sorteret baggrundens regionen og den øverste region i
snitte vær, da de begge krydser marginer og derfor ikke overholder regel
\ref{2.2.4, c}. Det andet billedet \ref{naiv_blob2}, er alle blevet
sorteret vær, også den lille region som ligger lige i miden af snittet.
Grunde til det er at den er for lille og derfor ikke overholder regel
\ref{2.2.4, a}. I test billedet \ref{naive_hoisont1}, sortere algoritmen
himlen fra, da den krydser margin lidt, men tager jorden med. I test
billedet \ref{naiv_masse} er der kun en af den 3 regioner som ikke
bliver sorteret væk, grunden til at den nederste region ikke bliver
tager med er at den ikke har en masse der er stor nok, som forklaret i
\ref{2.2.3} og derfor ikke overholder regel \ref{2.2.4 b}. Alle de test
billeder som vi har vist her opføre sig præcist på den måde som vi havde
regnet med.

\begin{figure}[!h]
    \centering
		\subfloat[Naive algoritme finder 1 ud af 5 regioner]{
        	\includegraphics[angle=0,width=0.55\textwidth]{afsnit/afprovning/billeder/naive_losning/naiv_blob1.png}
        	\label{naiv_blob1}}\hspace{1em}
    		\subfloat[Værgen den lille region eller den store er fundet]{
	        	\includegraphics[angle=0,width=0.55\textwidth]{afsnit/afprovning/billeder/naive_losning/naiv_blob2.png}
	       	\label{naiv_blob2}}\hspace{1em}
    		\subfloat[Kun den nederste højrisondt er fundet]{
	        	\includegraphics[angle=0,width=0.55\textwidth]{afsnit/afprovning/billeder/naive_losning/naiv_hoisont1.png}
		    \label{naiv_hoisont1}}\hspace{1em}
		    \subfloat[2 regioner hvor den ende er sorteret vær på grund af dens masse]{
	        	\includegraphics[angle=0,width=0.55\textwidth]{afsnit/afprovning/billeder/naive_losning/naiv_mass.png}
	       	\label{naiv_masse}}\hspace{1em}
        \caption[]{4 test billeder som også blev brugt til at illustrere den naive løsnings fremgangs måde, grån kasse rund om region, betyder at den er taget med af dem naive løsning}
     \label{naiv_detektion_test}
\end{figure}
\clearpage

\subsection{Afprøvning på malerier}
For at se på hvordan den naive metode virker på malerier afprøver vi den
på 6 malerier, først på 3 malerier, hvor regions detektoren virker
efter vores hensigt og så på 3 malerier, hvor region detektoren ikke
virker. Beskrivelsen af hvad der sker i billedet vil stå i caption


\begin{figure}[h!!]
	\begin{center}
		\includegraphics[scale=0.3,angle=0]{afsnit/afprovning/billeder/naive_losning/naiv_kfarver_sdetaljer.png}
	\end{center}
	\caption[]{5 ud af de 6 store regioner fra figur \ref{GRD_virker1} valt til at ligge i snittet, skoene er få små til at blive taget i betragtning}
	\label{naiv_kfarver_sdetaljer}
\end{figure}

\begin{figure}[h!!]
	\begin{center}
		\includegraphics[scale=0.3,angle=0]{afsnit/afprovning/billeder/naive_losning/naiv_mfarver_mdetaljer.png}
	\end{center}
	\caption[]{Bukserne og skoene er tager med af den naive løsning, men drengen er sorteret vær da har krydser snittet}
	\label{naiv_mfarver_mdetaljer}
\end{figure}

\begin{figure}[h!!]
	\begin{center}
		\includegraphics[scale=0.3,angle=0]{afsnit/afprovning/billeder/naive_losning/naiv_kfarver_kdetaljer.png}
	\end{center}
	\caption[]{Et billedet med mange hoder i snittet, hvor 2 af dem bliver godtaget af den naive metode til at ligger i snittet, en trøje bliver desværre også taget med »»(måske noget med at der bliver soteret få mange hover fra)}
	\label{naiv_kfarver_kdetaljer}
\end{figure}

\begin{figure}[h!!]
	\begin{center}
		\includegraphics[scale=0.3,angle=0]{afsnit/afprovning/billeder/naive_losning/naiv_virker_ikke1.png}
	\end{center}
	\caption[]{Mallerie hvor region detektor ikke virker, den naive løsning godtager tager en region som ligger helt forkert }
	\label{naiv_virker_ikke1}
\end{figure}

\begin{figure}[h!!]
	\begin{center}
		\includegraphics[scale=0.3,angle=0]{afsnit/afprovning/billeder/naive_losning/naiv_virker_ikke2.png}
	\end{center}
	\caption[]{3 regioner bliver godtaget, selv om de ikke er særlige intresante ««(er det ikke en farlige ting at sige )}
	\label{naiv_virker_ikke2}
\end{figure}

\begin{figure}[h!!]
	\begin{center}
		\includegraphics[scale=0.3,angle=0]{afsnit/afprovning/billeder/naive_losning/naiv_virker_ikke3.png}
	\end{center}
	\caption[]{Der bliver fundet 3 region, hvor kun en af dem passer på en ting i billedet}
	\label{naiv_virker_ikke3}
\end{figure}
\clearpage

\subsection{konkulution}
Det virker som om den naive løsning virker efter vores entationer dog
med nogle få falske positive, hvis region detektoren virker på
malerierne, dog fejler den på malerier hvor region detektoren fejler, og
kommer med en masse falske positive.

\section{Udvidet løsning}


}

% vim: set tw=72 spell spelllang=da:


\chapter{Implementation\label{chap_implementation}}
%% Bemærk:
%%          Programmeringssprog skrives med stort begyndelsesbogstav (første gang fed, (hvis gennemgående))
%%          Pakker skrives med kursiv (\emph{})
{
{\sffamily Dette kapitel har til formål at gennemgå, hvordan
principperne fra kapitel \ref{chap_indledning} og metoderne fra kapitel
\ref{chap_detektion} er blevet implementeret, samt hvilke problemer, der
kan opstå i denne forbindelse. Vi vil også komme ind på, hvordan
resultater --- som dem vist i kapitel \ref{chap_afproevning} ---
repræsenteres og gemmes, samt på hvordan vi kører analysen på malerierne
i vores database.  Vi forklarer programmet, og metoderne deri, ``fra
bunden og op'', dvs. at vi bevæger os fra lille kompleksitet, hvor der
lægges ud med de helt grundlæggende strukturer og principper, til stor
kompleksitet, når de enkelte dele sættes sammen.  I bilag
\ref{appendix_struktur} og \ref{appendix_vejledning} er vedlagt en
oversigt over programmets opdeling og en kort brugervejledning.
Indledningsvis kigger vi på, hvilke programmeringssprog og biblioteker
vi benytter os af.
}

\section{Programmeringssprog og biblioteker\label{section_programmeringssprog}}
{
{\sffamily Ved valg af programmeringssprog har vi først og fremmest lagt
vægt på at kunne udarbejde en prototype hurtigt og bruge et sprog, som
er let at gå til. Det valgte sprog skal også gøre det nemt at udvide den
endelige implementation. Vi har også gerne villet undgå at skulle
konstruere komplicerede datastrukturer for relativt simple metoder, både
af hensyn til tidspresset og til implementationens kompleksitet. Af
ovenstående grunde har vi besluttet at udarbejde vores løsning i
programmeringssproget \textbf{Python}, da netop dette sprog er yderst
velegnet at skrive forholdsvis avancerede prototyper i. Python er
ydermere meget fleksibelt med hensyn til datastrukturer og byder
umiddelbart på en lang række, for problemstillingen relevante, pakker.

(Skal man skrive noget om Pythons udbredelse, anerkendelse og brug?)
}

\subsection{OpenCV}
Til udførelse af billedmanipulationer benytter vi os af et bibliotek skrevet
i C og C++, der hedder \emph{OpenCV}. Biblioteket er udviklet af Intel
og tilbyder, udover et solidt udvalg af algoritmer, bindinger til
Python.  Endelig er det meget veldokumenteret og giver referencer til
publikationer om bibliotekets algoritmer. Biblioteket er udviklet med
specielt henblik på real-tids behandling af billeder, f.eks. med et
videokamera som kilde, men det egner sig også til brug på enkelte
billeder.  \emph{OpenCV} tilbyder mange brugbare datastrukturer med
hensyn til arbejdet med billeder i Python.

Der er også andre biblioteker til billedbehandling i Python. Her kan
nævnes \emph{PIL} (Python Image Library) og \emph{PythonMagick}
(ImageMagick bindings), men de er ikke nær så grundige som
\emph{OpenCV}.

\subsubsection{Andre muligheder}
Der er to helt oplagte muligheder, med hensyn til programmeringssprog,
når man taler om billedbehandling, nemlig Matlab og dets Open
Source-alternativ Octave. Disse sprog blev dog valgt fra, da vores
samlede erfaring med udvikling i disse sprog ikke var stor nok.
Endvidere finder vi, at disse sprog, på trods af, at de især egner sig
til den type beregninger, vi skal lave, er besværlige at lave større
programmer med. Matlab og Octave er dog blevet brugt til at sammenligne
resultater og teste alternative metoder med.

Da \emph{OpenCV} er skrevet i C/C++, ville det også være oplagt at bruge
et af disse sprog. Vores erfaring er dog, at man let kommer til at bruge
mere tid på at konstruere de fornødne datastrukturer og hjælpemetoder,
end på at fokusere på opgavens kerne. En senere implementation, med
fokus på køretid, kunne med fordel implementeres i C/C++, da man så
ville have fuld kontrol over, hvilke strukturer der bliver brugt i
programmet.

\subsection{Værktøjer til databasen}
Vi bruger \textbf{SQLite} til selve databasen, hovedsagelig fordi der ikke
kræves nogen videre konfiguration af en sådan database. Den
underliggende database er dog underordnet, da vi bruger Python-pakken
\emph{SQLObject}, som giver et abstraktionslag til en bred vifte af
databaser. Vi opretter blot de tabeller, vi ønsker at have i databasen,
som klasser i Python og får ligeledes en sådan klasse tilbage, når der
laves forespørgsler til databasen. Da \emph{SQLObject} klarer al
kommunikation med databasen, er det derfor muligt at skifte den
underliggende database ud, hvis man ønsker det. SQLite har endvidere den
umiddelbare fordel, at selve databasen eksisterer som en fil i
filsystemet.  Det er derfor en let sag at tage sikkerhedskopier af
databasen uden alt for meget besvær.

\subsection{Andre værktøjer}
Vi gør også brug af statistikprogrammet \textbf{R} til at behandle og
præsentere vores resultater.

}

% vim: set tw=72 spell spelllang=da:


\section{Billedbehandling med OpenCV\label{section_impBilledbehandling}}
{
{\sffamily I det følgende kaster vi et blik på detaljerne i programmets
billedbehandlingsmetoder. Efter en teknisk introduktion til digitale
billeder vises det, hvilke datastrukturer og metoder \emph{OpenCV}
stiller til rådighed, samt hvordan disse bruges til at udtrække regioner
i digitale billeder. Endelig vil vi undersøge, hvilke svagheder vores
implementering til udtrækning af regioner har.
}

\subsection{Digitale billeder}
I afsnit \ref{section_kort_intro} blev der givet en kort introduktion
til den digitale repræsentation af billeder. Det blev antaget, at en
pixel kunne antage værdier i mængden $\{0, 1\}$, men i praksis kan
pixels godt antage andre værdier. Vi arbejder med billeder, hvor værdien
for hver pixel er repræsenteret ved tre 8 bit størrelser, hver med
værdier i mængden $\{0, 1, 2, \cdots, 254, 255\}$. Sammensætningen af de
tre værdier, som beskrives som kanaler eller farvebånd, kaldes for en
RGB-farve, hvor tallene repræsenteret ved $(R,G,B)$ angiver mængden af
hhv. rød, grøn og blå farve i en pixel. Et sådant billede, kaldes for et
RGB-billede. Vores korpus består af sådanne RGB-billeder med 8 bits, og
vi benytter derfor også denne repræsentation internt. Der er enkelte
undtagelser, hvor der kun bruges én kanal, således at vi arbejder med
gråtonebilleder. For en uddybning af billeders repræsentation henvises
til \cite{SIOlsen}.

Vi har også tidligere, i ligning \ref{billede_matrix},  vist, at et
digitalt billede kan skrives som en matrix. Ved denne matrix kan man få
værdien for en pixel med koordinater $(x,y)$ ved elementet
$\mathbf{I}_{xy}$. Vi har altså, at x-aksen går fra øverste venstre
hjørne i nedadgående retning og y-aksen fra øverste venstre hjørne mod
højre. \emph{OpenCV} bruger dog et andet koordinatsystem, hvor akserne
er vendt om, som vist i figur \ref{opencv_koordinatsystem}. Man skal
derfor huske på, at matricen, som\emph{OpenCV} bruger, er transponeret,
og vi skal derfor bytte om på koordinaterne. For at få værdien på
pixelen med koordinaterne $(x,y)$, skal vi altså bruge værdien
$\mathbf{I}_{yx}$. Bemærk de omvendte koordinater.

\begin{figure}[!b]
    \centering
    \begin{picture}(122,55)
        \put(61, 50){$x$}
        \put(-10, 22){$y$}
        \put(0, 45){\circle*{3}}
        \put(-1, 45){\vector(1, 0){120}}
        \put(0, 45){\vector(0, -1){48}}

    \end{picture}
    \caption[]{Koordinatsystemet, som bliver brugt i \emph{OpenCV}.}
    \label{opencv_koordinatsystem}
\end{figure}

Billedmatricen $\mathbf{I}$ repræsenteres i \emph{OpenCV} som et
dobbelt-array. I figur \ref{opencv_billedematrix} er vist, hvordan denne
struktur ser ud, når vi har indlæst billedet givet i figur
\ref{billede_pixels}. Hvis billedet er indlæst i variablen \texttt{I},
kan vi i Python tilgå pixelen med koordinater $(x,y) = (0,2)$ ved
$\textbf{I}_{2,0} =~$\texttt{I[2][0]}.

\begin{figure}[t]
    \begin{verbatim}
                        I = [ [0, 1, 0],
                              [1, 1, 1],
                              [0, 1, 0] ]
    \end{verbatim}
\vspace{-2em}
\caption{Repræsentation af billedmatricen fra ligning
\ref{billede_pixels} i \emph{OpenCV}.}
\label{opencv_billedematrix}
\end{figure}

\subsection{Resultaters struktur\label{resultat_struktur}}
Indledningsvist vil vi introducere to vigtige datastrukturer, som vi
bruger fra \emph{OpenCV}. Alle datastrukturer og metoder fra
\emph{OpenCV} har præfikset ``cv'', hvilket gør det let at skelne vores
egne metoder fra dem, der er givet i \emph{OpenCV}. Vi præsenterer
desuden en notation for datastrukturer, der er underlagt strukturen vist
i \eqref{types_class}.
\begin{multline}
    \textbf{class~} [\textit{name}] = \{ \\
    \shoveleft{\qquad[\textit{type}] : [\textit{varName}]} \\
    \shoveleft{\}}\shoveright{}
    \label{types_class}
\end{multline}
I \eqref{example_class} ses et eksempel på en struktur kaldet
\textbf{ExampleClass}.
\begin{multline}
    \textbf{class~} \textrm{ExampleClass} = \{ \\
    \shoveleft{\qquad\textbf{int} : \textit{intValue}} \\
    \shoveleft{\qquad\textbf{string} : \textit{stringValue}} \\
    \shoveleft{\qquad\textbf{int[2]} : \textit{arrayValues}} \\
    \shoveleft{\}}\shoveright{}
    \label{example_class}
\end{multline}
Bemærk, at en strukturs navn bliver skrevet med fed skrift i
brødteksten, når der refereres til den. Ligeledes vil en strukturs navn
blive skrevet med fed i pseudokoden, hvis der refereres til den. Vi kan
definere en ny struktur, \textbf{NewClass}, som illustrerer dette, i
\eqref{new_class} herunder.
\begin{multline}
    \textbf{class~} \textrm{NewClass} = \{ \\
    \shoveleft{\qquad\textbf{ExampleClass} : \textit{exampleInstance}} \\
    \shoveleft{\qquad\textbf{string} : \textit{stringValue}} \\
    \shoveleft{\}}\shoveright{}
    \label{new_class}
\end{multline}
Med ovenstående notation kan vi nu beskrive to vigtige strukturer fra
\emph{OpenCV} kaldet \textbf{cvRect} og \textbf{cvConnectedComp}.

\subsubsection{cvRect}
Denne struktur beskriver et rektangel ved at angive dets øverste venstre
hjørne og dets dimensioner. Strukturen vises i \eqref{cvRect_class}.
\begin{multline}
    \textbf{class~} \textrm{cvRect} = \{ \\
    \shoveleft{\qquad\textbf{int} : \textit{x}} \\
    \shoveleft{\qquad\textbf{int} : \textit{y}} \\
    \shoveleft{\qquad\textbf{int} : \textit{height}} \\
    \shoveleft{\qquad\textbf{int} : \textit{width}} \\
    \shoveleft{\}}\shoveright{}
    \label{cvRect_class}
\end{multline}
Implementeringen bruger denne struktur til at angive en regions
begrænsende rektangel. Vi vil i afsnit \ref{section_vurdering_regioner}
se, hvordan strukturen \textbf{cvRect} bruges i vurderingen af regioner.

\subsubsection{cvConnectedComp}
\emph{OpenCV} har implementeret floodfill-metoden, beskrevet i afsnit
\ref{subsec_floodfill}, som gør brug af en datastruktur kaldet
\textbf{cvConnectedComp}. Denne struktur bruges til at beskrive den
region i billedet, som metoden fylder ud.  Strukturen er vist i
\eqref{cvConnectedComp_class}.
\begin{multline}
    \textbf{class~} \textrm{cvConnectedComp} = \{ \\
    \shoveleft{\qquad\textbf{double} : \textit{area}} \\
    \shoveleft{\qquad\textbf{float} : \textit{value}} \\
    \shoveleft{\qquad\textbf{cvRect} : \textit{rect}} \\
    \shoveleft{\}}\shoveright{}
    \label{cvConnectedComp_class}
\end{multline}
Strukturen indeholder en regions begrænsende rektangel og regionens
areal.

\subsubsection{Resultater}
Nu introduceres en ny struktur, som repræsenterer Pythons
\emph{dictionary}, forkortet \emph{dict}.  En \emph{dict} viser vi som i
\ref{def_dict}.
\begin{eqnarray}
    \langle[\textit{dictName}]\rangle = \{ [\textit{key}] : [\textit{data}] \}
    \label{def_dict}
\end{eqnarray}
Som et simpelt eksempel kan vi konstruere en \emph{dict}
$\angles{LuckyNumbers}$, som indeholder personers lykketal:
% TeX-Gods, please forgive me :(
\begin{multline}
    \angles{LuckyNumbers} = \{ \qquad \textrm{Tom Cruise} : 4 , \\
    \textrm{Arthur Dent} : 42 \qquad\qquad\\
    \shoveleft{\}}\shoveright{}
    \label{lucky_dict}
\end{multline}
% In honor of the danish Tom Cruise
Data i en \emph{dict}, kan godt være af mere komplicerede typer, f. eks.
kan man have en anden \emph{dict}. Man kan således lave en hierakisk
struktur, som egner sig til opbevaring af fundne regioner i et billede.
Hierakiet er illustreret ved grammatikken i
\eqref{resultat_hieraki}.

\begin{equation}
    \begin{split}
        \textit{ImageRegions}  & \to  (\textit{RatioRegions~ImageRegions}')\\
        \textit{ImageRegions}' & \to  \textit{RatioRegions~ImageRegions}'\\
        \textit{ImageRegions}' & \to  \\
        \textit{RatioRegions}  & \to  (\textit{CutRegions~CutRegions~RatioRegions}')\\
        \textit{RatioRegions}' & \to  \textit{CutRegions~CutRegions}\\
        \textit{RatioRegions}' & \to  \\
        \textit{CutRegions}    & \to  (\textbf{CV\_RGB}, \textbf{cvConnectedComp})\textit{~CutRegions} \\
        \textit{CutRegions}    & \to
    \end{split}
    \label{resultat_hieraki}
\end{equation}

Vi har da, at et billede kan have et endeligt antal snitratioer, som vi
ønsker at undersøge. Disse snitratioer har enten to eller fire snit,
hvor der i hvert snit er et antal fundne regioner med en tilhørende
farve.

Hver region i et snit bliver tildelt et \emph{ID}, som er en
strengrepræsentation af regionens farve.  Hvis det antages, at regionens
farve er hvid, vil den have RGB-værdien $(255, 255, 255)$, og den får
derfor tildelt $ID = \textrm{'255255255'}$. Vi konstruerer nu
$\angles{CutRegions}$, som indeholder de fundne regioner for et snit.
$\angles{CutRegions}$ er vist herunder i \eqref{CutRegions_dict}.
\begin{multline}
    \angles{CutRegions} = \{ \textit{~RegionId} : \\
    (\textbf{CV\_RGB~}\textit{color}, \textbf{cvConnectedComp~}\textit{region}) \}\quad
    \label{CutRegions_dict}
\end{multline}

\noindent I $\angles{CutRegions}$ bruges regionens ID som nøgle. Vi kan nu
konstruere den \emph{dict} på niveauet over, som vi betegner
$\angles{RatioRegions}$. Den ses i \eqref{RatioRegions_dict}.
\begin{eqnarray}
    \angles{RatioRegions} = \{ \textit{~CutNo} : \angles{CutRegions} \}
    \label{RatioRegions_dict}
\end{eqnarray}

\noindent $\angles{RatioRegions}$ holder alle regioner for en givet
snitratio.  Hvert snit, tilknyttet en snitratio, blev i afsnit
\ref{section_opdeling} tildelt et ID i mængden $\{0,1,2,3\}$. Vi bruger
snittets ID som nøgle.  Data i $\angles{RatioRegions}$ er den instans af
$\angles{CutRegions}$, som tilhører snittet.

Det sidste niveau i hierakiet fra figur \ref{resultat_hieraki} er endnu
en \emph{dict}, givet ved $\angles{ImageRegions}$, vist i
\eqref{ImageRegions_dict}.
\begin{eqnarray}
    \angles{ImageRegions} = \{ \textit{~CutRatio} : \angles{RatioRegions} \}
    \label{ImageRegions_dict}
\end{eqnarray}

\noindent $\angles{ImageRegions}$ har nøglen $CutRatio$, som angiver en
snitratio til et billede. Hver snitratio har et antal snit, som endelig
har et antal fundne regioner. Vi har således fået beskrevet den
struktur, som resultater bliver repræsenteret ved.

Resultatet fra en analyse på et billede, med snitratioerne $0.5$ og
$0.618$, ligner da det nedenstående:
%% !!! Bwadr-tex !!!
\begin{multline}
    \angles{ImageRegions} = \{ 0.5 : \{ 0 : \{ \textrm{'012345678'} : (color, region) \},\\
                                            \{ \textrm{'123456789'} : (color, region) \}
                                            \},\\
                                            \{ 1 : \{ \textrm{'012345678'} : ( \cdots ) \}, 
                                            \{ \cdots \} \}, \\
                              0.618 : \{ 0 : \{ \cdots \} \}, \{ 1 :
                              \cdots \}, \{ 2 : \cdots \}, \{ 3 : \cdots
                              \} \\
    \shoveleft{\}}\shoveright{}
    \label{CutRegions_dict}
\end{multline}

\subsection{Udtrækning af regioner}
I figur \ref{graphic_pipeline} er givet den række af manipulationer, et
billede skal igennem i programmet før regionerne i et givet snit kan
trækkes ud.  Resten af dette afsnit, vil omhandle den egentlige
implementering af fremgangmåden samt de problemer, der knytter sig til
den. Indledningsvist vises, hvordan snit i billedet repræsenteres.
\begin{figure}[!h]
    \includegraphics[width=\textwidth]{afsnit/implementation/billeder/billedbehandling/pipeline.png}
    \caption{Skema over grafiske modifikationer i forbindelse med
    udtrækning af regioner.}
    \label{graphic_pipeline}
\end{figure}

\subsubsection{Repræsentation af snit i billedet}
Metoderne i dette afsnit er ikke specifikke for snitratioen $\varPhi$,
men kan overføres til ethvert andet snit i billedet med en arbitrær
snitratio.

Snit i et billede bliver repræsenteret ved et linjestykke. Vi
præsenterer nu datastrukturen \textbf{Line}, som består af et
linjestykkes to endepunkter.
\begin{multline}
    \textbf{class~} \textrm{Line} = \{ \\
    \shoveleft{\qquad\textbf{cvPoint} : \textit{p1}} \\
    \shoveleft{\qquad\textbf{cvPoint} : \textit{p2}} \\
    \shoveleft{\}}\shoveright{}
    \label{Line_class}
\end{multline}

\subsubsection{Præparation af billedet}
Det første skridt i præparation af billedet er, at detektere kanterne på
dets objekter.  Kanter detekteres i et sort/hvid-billede, så vi starter
med at lave en sort/hvid kopi af det originale billede. Inden den
egentlige kantdetektion, sløres det sort/hvide billede, således at vi
kun betragter kanter, som fremstår tydeligt i billedet.

I andet skridt af fremgangsmåden sløres det originale billede. Både i
sløringingen inden kantdetektion og denne bruges simpel sløring.

Sidste skridt er fremhævelse af de detekterede kanter. Det gøres ved at
oprette et nyt billede med samme størrelse som originalbilledet. Alle
pixels i det nye billede farves sorte. Dernæst kopieres alle pixels fra
det originale billede over i det sorte billede, med undtagelse af de
pixels, hvor der er detekteret en kant. Derved forbliver kanterne sorte.
At lade kanterne være sorte kan have nogle utilsigtede konsekvenser. Vi
henviser til afsnit \ref{subsec_svagheder} for en gennemgang af disse.

Præparation af et billede er fremstillet som pseudokode i kodeboks
\ref{pseudo_prepare}.

\begin{lstlisting}[caption={Pseudokode for metoder til præparation af
    billeder.},captionpos=b,label={pseudo_prepare},numbers=left,
    frame=tb, breaklines=false, float=h]
def GetEdges(original, threshold1, threshold2):
    # Create images
    gray = cvCreateImage(cv.cvGetSize(original), 8, 1)
    out  = cvCreateImage(cv.cvGetSize(original), 8, 1)

    # Convert to B/W
    cvCvtColor(original, gray, CV_BGR2GRAY)

    # Simple blur
    cvSmooth(gray, out, CV_BLUR, 3, 3, 0)

    # Edge detect
    cvCanny(gray, out, threshold1, threshold2)

    return out

def EnhanceEdges(img, edges):
    workImage = cvCreateImage(cv.cvGetSize(original), 8, 3)

    # Superimpose the edges onto the image
    cvNot(edgeImage, edgeImage)
    cvCopy(img, workImage, edges)

    # Get the edges back to white
    cvNot(edges, edges)

    return workImage

def Preprocessing(original, thresholds):
    # Find edges
    edgeImage = GetEdges(original, threshold1, threshold2)

    # Blur
    blurImage = cvCreateImage(cv.cvGetSize(original), 8, 3)
    cvSmooth(original, blurImage, CV_BLUR, 3, 3, 0)

    # Enhance edges
    enhanced = EnhanceEdges(blurImage, edges)

    return (enhanced, edges)

\end{lstlisting}

\subsubsection{Segmentering med floodfill-metoden}
% Plan
% Vi optimerer, maler ikke alle pixel
% Vi bruger det kantdetekterede billede
% Vi laver linjestykker
% Vi bemærker at ikke hele regionen bliver fyldt ud
% Vi maler to gange, men det er en fejl
% Vi retter koden
Vi beskriver nu metoden til at trække regioner ud af et præpareret
billede. Metoden bygger videre på beskrivelsen givet i
\ref{sammensaetning_af_metoder}, hvor floodfill-metoden bruges på hver
pixel langs et snit. I praksis er denne fremgangsmåde meget langsom,
især med store billeder. Vi vil derfor komme frem til en metode, som kun
bruger floodfill-metoden på de nødvendige pixels.

Vi bruger de detekterede kanter til at hjælpe med at segmentere
billedet, da vi vil bruge floodfill med en ny farve, når vi passerer en
kant. Vi trækker derfor de punkter ud, hvor en kant krydser det givne
snit, ved metoden vist i kodeboks \ref{pseudo_GetPoints}.

\begin{lstlisting}[caption={Pseudokode for metode til at finde punkter,
    hvor en kant krydser det givne snit.},captionpos=b,label={pseudo_GetPoints},numbers=left,
    frame=tb, breaklines=false, float=h]
def GetPoints(edges, cut):
    points = []
    for pixel on cut:
        if not (color(pixel) == 0):
            points += pixel
\end{lstlisting}

Vi repræsenterer snittet som et linjestykke, og vi kan således dele dette
i mindre segmenter mellem kanter. Et eksempel på et horisontalt snit er
vist i figur \ref{impUdtraek_kantpunkter}.

\begin{figure}[!h]
    \centering
    \begin{picture}(240,30)
        \put(0, 10){$A$}
        \put(3, -5){\line(0, 1){10}}

        \put(82, 10){$e_1$}
        \put(85, 0){\circle*{3}}

        \put(102, 10){$e_2$}
        \put(105, 0){\circle*{3}}

        \put(134, 10){$e_3$}
        \put(137, 0){\circle*{3}}

        \put(158, 10){$e_4$}
        \put(161, 0){\circle*{3}}

        \put(200, 10){$e_5$}
        \put(203, 0){\circle*{3}}

        \put(221, 10){$e_6$}
        \put(224, 0){\circle*{3}}

        \put(233, 10){$B$}
        \put(236, -5){\line(0, 1){10}}

        \put(3, 0){\line(1, 0){233}}
    \end{picture}
    \caption[]{Punkter, hvor der er en kant, der krydser snittet.}
    \label{impUdtraek_kantpunkter}
\end{figure}

Vi vil nu bruge floodfill således, at hvert linjestykke bliver tildelt
en tilfældig farve, som endnu ikke er blevet givet til et andet
linjestykke. Hvert linjestykke males derefter med denne. Naivt ville man
bruge floodfill på midten af hvert linjestykke, men dette er ikke
holdbart, da vi ikke kan være sikre på, at hele linjestykket farves.
Figur \ref{floodfill_taerskel_problem} viser, at det ikke er lige meget
\emph{hvor} på linjestykket, man bruger floodfill, når vi har med
arbitrære billeder at gøre. Vi vil derfor gå langs snittet og bruge
floodfill på alle pixels, som endnu ikke er blevet farvet af
floodfill-metoden.  Hver gang, vi bruger floodfill, returneres en
instans af strukturen \textbf{cvConnectedComp}.

Vi skal også være sikre på, at floodfill returnerer hele den fundne
region. Den returnerede \textbf{cvConnectedComp} er den \emph{senest
farvede} region. I figur \ref{floodfill_return_entire_region} vises en
situation, hvor vi ikke har farvet hele linjestykket, og når vi udvider
den fundne region, returnerer floodfill kun et undersæt af alle pixels i
regionen. Vi skal derfor bruge floodfill mere end én gang for at sikre
os, at hele regionen returneres. Vi har dertil udviklet pseudokoden i
kodeboks \ref{pseudo_udtraek_org}.

\begin{figure}[h]
    \setlength\fboxsep{0pt}
    \setlength\fboxrule{0.5pt}
    \centering
    \fbox{\includegraphics[width=0.8\textwidth]{afsnit/implementation/billeder/billedbehandling/floodfill_color.png}}
    \caption[]{Problemet med floodfill-metoden i billeder med flere
    farver. Hvid farve er ikke blevet malet af floodfill-metoden endnu.
    Den lyseblå region dækker ikke hele linjestykket ned til kanten.}
    \label{floodfill_taerskel_problem}
\end{figure}

\begin{lstlisting}[caption={Original pseudokode til udtrækning af
    regioner. Denne kan returnere den samme region flere
    gange.},captionpos=b,label={pseudo_udtraek_org},numbers=left,
    frame=tb, breaklines=false, float=h]
for lineSegment in Cut:
    # Get a new color that is not in the component dictionary
    color = getRandomColor()
    region = cvConnectedComp()

    for pixel in lineSegment:

        # Check if the color of the pixel equals current color
        if not (color(pixel) ==  color):

            # Check if the color of the pixel are in the saved regions
            if not (color(pixel) in CutRegions):
                cv.cvFloodFill(img, pixel, color, lowerThres, upperThres, region)

    # Color the last pixel again to make sure that
    # the returned component is the entire region
    cv.cvFloodFill(img, pixel, color, lowerThres, upperThres, region)

    # Put the results in the CutRegions-dictionary
    CutRegions[color.toString()] = (color, region)
\end{lstlisting}

\begin{figure}[p]
    \setlength\fboxsep{0pt}
    \setlength\fboxrule{0.5pt}
    \centering
    \subfloat[En tilføjelse til den lyseblå region. Det begrænsende
    rektangel svarer kun til udvidelsen.]{
        \label{new_reg_small_box}
        \fbox{\includegraphics[angle=0,width=0.8\textwidth]{afsnit/implementation/billeder/billedbehandling/floodfill_color_new_reg_small_box}}
        }\\
    \subfloat[Ved anvedelse af floodfill på regionen igen, markeres hele den
    lyseblå region som ønsket.]{
        \label{new_reg_big_box}
        \fbox{\includegraphics[angle=0,width=0.8\textwidth]{afsnit/implementation/billeder/billedbehandling/floodfill_color_new_reg_big_box}}
        }
    \caption[]{
    Floodfills opførsel ved udvidelse af regioner.
    }
    \label{floodfill_return_entire_region}
\end{figure}

Vi har i kodeboks \ref{pseudo_udtraek_org} forsøgt at tage højde for
problemet, nævnt i figur \ref{new_reg_small_box}, ved et ekstra kald til
\texttt{cvFloodFill} i linje 17. Dette vil farve den sidste pixel en
ekstra gang for at sikre, at hele regionen returneres. Kaldet gør dog
også, at der \emph{altid} bliver returneret en region for hvert segment.
Dette er ikke ønskværdigt, specielt ikke hvis vi er sprunget over alle
pixels på et linjestykke. Dette sker netop, når hele regionen allerede
er blevet fyldt ud. I praksis betyder dette, at der kan blive fundet
flere regioner i et billede, end der reelt er. I bilag \ref{appendix_bug}
er givet en håndkørsel af den fejlende kode, og i afsnit
\ref{program_bug} ses på hvordan dette påvirker resultaterne.

Fejlen i linje 17 har eksisteret selv under vores kørte eksperiment i
afsnit \ref{section_naiv_koersel}. Metoden er senere blevet rettet til
at have den korrekte opførsel, og den reviderede metode er vist i
kodeboks \ref{pseudo_udtraek_rev}.

\begin{lstlisting}[caption={Revideret pseudokode til udtrækning af
    regioner. Returnerer ingen
    duplikater.},captionpos=b,label={pseudo_udtraek_rev},numbers=left,
    frame=tb, breaklines=false, float=h]
for lineSegment in Cut:
    # Get a new color that is not in the component dictionary
    color = getRandomColor()

    # Set region to None, as we have not yet found any
    region = None

    for pixel in lineSegment:

        # Check if the color of the pixel equals current color
        if not (color(pixel) ==  color):

            # Check if the color of the pixel are in the saved regions
            if not (color(pixel) in CutRegions):
                # Now we've got a new region
                region = cvConnectedComp()
                cv.cvFloodFill(img, pixel, color, lowerThres, upperThres, region)

                # Color the pixel again to make sure that
                # the returned component is the entire region
                cv.cvFloodFill(img, pixel, color, lowerThres, upperThres, region)

    # If we have found a region, then put the result in the CutRegions-dictionary
    if not (region is None):
        CutRegions[color.toString()] = (color, region)
\end{lstlisting}

Metoden i kodeboks \ref{pseudo_udtraek_rev} bruger \texttt{cvFloodFill}
to gange, hver gang man møder en ikke-farvet pixel. Dette koster lidt
køretid men sikrer, at man altid får hele regionen returneret i
\textbf{cvConnectedComp}.

%Man kan fristes til at flytte kaldet i linje
%21 ind i \texttt{if}-sætningen i linje 24, men dette åbner op for
%svagheden igen, da vi ikke ved, om den sidste pixel tilhører den
%aktuelle region. Derfor er vi nødt til at ofre lidt køretid, for at
%være sikre på resultatet. Metoden benytter et ``først til
%mølle''-princip, hvor en pixel, når den først et blevet tilknyttet en
%region, altid vil tilhøre denne.

Vi trækker dog kun de regioner ud som rører \emph{snittet}. I kapitel
\ref{chap_detektion} indførtes et margin, således at også regioner, som
ligger tæt på snittet, kan trækkes ud. Vi vil nu udvide pseudokoden i
kodeboks \ref{pseudo_udtraek_rev} til også at udtrække regioner, som
krydser margin. Vi navngiver metoden \texttt{GetCutRegions}, og den
tager et snit som argument. Metoden ses i kodeboks
\ref{pseudo_udtraek_margin} og trækker først regioner ud med hensyn til
det nedre margin, så med hensyn til det øvre margin og til sidst med
hensyn til selve snittet. Alle regioner gemmes i den samme instans af
$\angles{CutRatios}$. For selve udregningen af margin henvises til
afsnit \ref{subsec_margin_udregning}.

\begin{lstlisting}[caption={Pseudokode til udtrækning af regioner med
    margin.},captionpos=b,label={pseudo_udtraek_margin},numbers=left,
    frame=tb, breaklines=false, float=h]
def GetCutRegions(img, edges, cut):
    # Calculate lowerMargin and upperMargin
    (lowerMargin, upperMargin) = calculateMargins(cut)
    Cuts = [lowerMargin, upperMargin, cut]

    # Initialize an empty CutRegions-dict
    CutRegions = {}

    for Cut in Cuts:
        # Use the edge image to find the line segments
        for lineSegment in Cut:
            # Get a new color that is not in the component dictionary
            color = getRandomColor()

            # Set region to None, as we have not yet found any
            region = None

            for pixel in lineSegment:

                # Check if the color of the pixel equals current color
                if not (color(pixel) ==  color):

                    # Check if the color of the pixel are in the saved regions
                    if not (color(pixel) in CutRegions):
                        # Now we've got a new region
                        region = cvConnectedComp()
                        cv.cvFloodFill(img, pixel, color,
                                    lowerThres, upperThres, region)

                        # Color the pixel again to make sure that
                        # the returned component is the entire region
                        cv.cvFloodFill(img, pixel, color,
                                    lowerThres, upperThres, region)

            # If we have found a region,
            # then put the result in the CutRegions-dictionary
            if not (region is None):
                CutRegions[color.toString()] = (color, region)

    return CutRegions
\end{lstlisting}

Vi kan nu kombinere alle metoderne, således at vi kan trække regioner ud
af arbitrære billeder. Til dette bruges pseudokoden vist i kodeboks
\ref{pseudo_udtraek_all}.

\begin{lstlisting}[caption={Fuld udtrækning af regioner i et arbitært
    billede.},captionpos=b,label={pseudo_udtraek_all},numbers=left,
    frame=tb, breaklines=false, float=h]
def NaiveExtraction(original):
    # Preprocess the image
    (enhanced, edges) = Preprocessing(original, thresholds)

    # Get regions
    return GetCutRegions(enhanced, edges, cut)
\end{lstlisting}

\subsection{Svagheder\label{subsec_svagheder}}
Den endelige metode for udtrækning af regioner, med hensyn til et givet
snit i billedet, har nogle svagheder. Den første er allerede nævnt:
Metoden trækker kun regioner ud \emph{med hensyn} til et snit. Dette
betyder, at kun regioner, som rører margin eller selve snittet, bliver
trukket ud. Foregående skal tages helt bogstaveligt, i den forstand, at
vi \emph{skal} have, at en region har en pixel enten på det nedre
margin, det øvre margin eller på selve snittet for at blive trukket ud.
Vi kan altså godt have interessante regioner, som faktisk har mindst én
kant af deres begrænsende rektangel inden for margin, men som ikke
bliver trukket ud. Et eksempel er vist i figur \ref{respect_to_cut},
hvor den sorte region ikke vil blive trukket ud, selvom den har to
kanter inden for margin.

\begin{figure}[h]
    \setlength\fboxsep{0pt}
    \setlength\fboxrule{0.5pt}
    \centering
    \fbox{\includegraphics[width=0.8\textwidth]{afsnit/implementation/billeder/billedbehandling/respect_to_cut.png}}
    \caption[]{Sort region, som ikke bliver trukket ud af billedet.
    Selvom regionens begrænsende rektangel ligger inden for margin, så
    krydser regionen hverken nedre margin, øvre margin eller selve
    snittet. Derfor opdages regionen ikke.}
    \label{respect_to_cut}
\end{figure}

\subsubsection{Valg af tilfældige RGB-værdier}
Man skal endvidere være opmærksom på, at hver gang vi markerer en ny
region, tildeles denne en tilfældig farve.  Vi kan, i vores valg af
farve, være uheldige at vælge en, som bliver brugt i det originale
billede. Når vi kalder \texttt{cvFloodFill} igen, for at være sikre på, at
hele regionen bliver returneret, kan vi smelte to regioner sammen, som
egentlig ikke burde hænge sammen. I figur \ref{floodfill_colors} er
denne situation vist på det præparerede billede fra figur \ref{bathers}.

Ligeledes kan vi være uheldige at støde på en pixel, som har en farve
lig med en allerede farvet region. I dette tilfælde vil denne pixel
blive anset som værende del af en eksisterende region, hvilket
resulterer i at vi ikke bruger \texttt{cvFloodFill} på denne. Vi kan dog
håbe, at denne pixel bliver inkluderet i den efterfølgende iteration.
Vi vælger aldrig, at farve en region med en farve, som allerede er brugt
til en anden region, men vi kan være uheldige og vælge en farve, som
ligger inden for floodfill-metodens tilladte afvigelse, således at to
regioner smeltes sammen.

\begin{figure}[h]
    \setlength\fboxsep{0pt}
    \setlength\fboxrule{0.5pt}
    \centering
    \subfloat[Originalt præpareret billede inden vi vælger en tilfældig
    farve til floodfill.]{
        \label{colors_1}
        \fbox{\includegraphics[width=0.3\textwidth]{afsnit/implementation/billeder/billedbehandling/pre_floodfill_1}}}\hspace{1em}
    \subfloat[Region fyldes med farve lig eksisterende pixels i
    billedet. Pixels i drengens hår har nu samme farve som kroppen.]{
        \label{colors_2}
        \fbox{\includegraphics[width=0.3\textwidth]{afsnit/implementation/billeder/billedbehandling/pre_floodfill_2}}}\\
    \subfloat[Når floodfill bruges en ekstra gang, for at være sikker på
    at hele regionen returneres, smeltes drengens krop og hår sammen.]{
        \label{colors_3}
        \fbox{\includegraphics[width=0.3\textwidth]{afsnit/implementation/billeder/billedbehandling/pre_floodfill_3}}}\hspace{1em}
    \subfloat[Hvis den første region havde fået en anden farve, ville vi
    ikke have smeltet to regioner sammen.]{
        \label{colors_4}
        \fbox{\includegraphics[width=0.3\textwidth]{afsnit/implementation/billeder/billedbehandling/pre_floodfill_4}}}\\
    \caption[]{Uheldige valg af tilfældig farve til regioner.}
    \label{floodfill_colors}
\end{figure}

\subsubsection{Ikke-sammenhængende regioner}
Metoden i kodeboks \ref{pseudo_udtraek_margin}, trækker højest én region
ud per segment. Dette giver mening, set i lyset af hvad
udtrækningsmetoden er blevet udviklet til, nemlig udtrækning af
\emph{sammenhængende regioner} afgrænset ved kantdetektion. I afsnit
\ref{section_computer_betragter} antog vi, at interessante regioner i et
billede, er tydeligt afgrænset. En region er ikke tydeligt afgrænset,
hvis vi faktisk \emph{har} flere regioner indenfor et segment. I figur
\ref{usammenhaengende_region} betragter vi en hypotetisk situation, hvor
en region ikke har været tydeligt afgrænset. Billedet i figur
\ref{usammenhaengende_region}, illustrerer et billede, som allerede er
blevet segmenteret ved metoden i kodeboks \ref{pseudo_udtraek_margin}.
De sorte cirkler markerer dér, hvor vi har fundet kanter, som krydser
snittet. Vi har altså fem segmenter på snittet. På det længste segment,
ser vi, at den mørkeblå region er blevet afbrudt af en tidligere fundet
region. Den mørkeblå region er således ikke sammenhængende. Metoden
returnerer kun den nederste mørkeblå region, da vi ikke kan forbinde
de to mørkeblå regioner. Vi mister derfor informationen om at der
faktisk befinder sig en region i midten af billedet.

Denne opførsel, er en konsekvens af vores antagelse, om at interessante
regioner er tydeligt afgrænset. Vi accepterer derfor, at situationen i
figur \ref{usammenhaengende_region} kan forekomme, og at vi derfor kan
undlade at trække enkelte regioner ud.

\begin{figure}[t]
    \setlength\fboxsep{0pt}
    \setlength\fboxrule{0.5pt}
    \centering
    \fbox{\includegraphics[width=0.8\textwidth]{afsnit/implementation/billeder/billedbehandling/usammenhaengende_region.png}}
    \caption[]{Et segmenteret billede. Sorte cirkler angiver steder,
    hvor vi har detekteret en kant, der krydser snittet. Den mørkeblå
    region er usammenhængende over snittet, da den afbrydes af den røde
    region. Kun den nederste mørkeblå region er blevet returneret.}
    \label{usammenhaengende_region}
\end{figure}

\subsubsection{Usikkerhed ved fremhævelse af kanter}
De fremhævede kanter, tegnes i det slørede billede med sort farve, som
vælges uanset hvilke farver, der er brugt i maleriet i forvejen. Vi kan
derfor støde på problemer med mørke malerier eller lokalt i mørke
områder af et billede. Vi kan også være uheldige, at komme til at
forbinde to regioner, ved at markere kanterne med sort.

På grund af, at kanterne fremhæves ved at male i billedet, kan vi, ved
floodfill-metoden, komme til at male kanter, som har samme retning, som
det snit vi betragter. Dvs, at lodrette kanter kan blive malet ved
vertikale snit og omvendt kan vandrette kanter risikere at blive malet
ved horisontale snit.

\subsection{Andre tilgange og forbedringer}
Dette afsnit vil kort nævne andre fremgangsmåder, vi har prøvet, for at
trække regioner ud af et billede. Vi har tidligt i udviklingen af
programmet, eksperimenteret både med Octave og Matlab. Til at starte
med, har vi brugt disse sprog til at komme problemstillingen nærmere,
og udvikle naive implementationer af enkelte algoritmer. Vi ser nu på
hvad der kan gøres for at forbedre udtrækningen af regioner.

\subsubsection{Udtrækning af alle regioner}
At vi kan have regioner som ikke trækkes ud af billedet, som i figur
\ref{respect_to_cut}, er beklagelig, men kan løses ved pseudokoden i
kodeboks \ref{pseudo_fix}.  Her er tanken, at man i stedet for kun at
trække regioner ud på margin og selve snittet, så gøres det på hver
pixel mellem margin. På denne måde kan regioner, som den i figur
\ref{respect_to_cut}, trækkes ud af billedet. Vores implementation gør
dog ikke dette. Når regioner kun trækkes ud, med hensyn til et snit, kan
vi ved denne metode heller ikke garantere en fuld segmentering af
billedet.

\begin{lstlisting}[caption={Pseudokode til udtrækning af regioner med
    margin.},captionpos=b,label={pseudo_fix},numbers=left,
    frame=tb, breaklines=false, float=h]
def FixGetRegions(img, edges, cut):
    # Calculate lowerMargin and upperMargin
    (lowerMargin, upperMargin) = calculateMargins(cut)

    for pixel in (lowerMargin to upperMargin):
        # Extract regions
        pass
\end{lstlisting}

Vi har også eksperimenteret med, først at skalere billedet ned, inden
man trak regioner ud. Dette gør nemlig, at kanterne forbliver intakte,
mens farverne, til en vis grad, bliver mere ensartede. Vi stødte dog på
problemer når billedet skulle skaleres op igen, eller rettere, få
resultaterne skalleret op, så de passede til originalbilledet. Vi mister
også meget præcision, når vi bruger et nedskaleret billede til at finde
regioner i.


I præparationen af billedet, vil vi gerne have, at farverne i billedet
bliver ensartede, men vi ønsker også at bibeholde kanterne. Vi forsøgte
at bruge en metode i \emph{OpenCV}, der hedder
\texttt{cvPyrMeanShiftFiltering}, som netop segmenterer billedet efter
hvilke farver der ligner hinanden, men en fejl i biblioteket forsagede
altid en segmenteringsfejl i det underliggende C-program. Vi blev siden
gjort opmærksom på en sløringsmetode af Perona og
Malik\cite{perona1990scale}, hvor det grafiske resultat tilnærmer sig
det ønskede. Kun i Octaves \emph{Image}-pakke er denne implementeret. Vi
lavede en hurtig afprøvning, hvor resultatet fra denne metode blev
analyseret, men på basis at det endelige resultat, vurderede vi, at vi
godt kunne nøjes med at fremhæve kanterne i billedet efter en simpel
sløring.

\subsubsection{Bedre fremhævelse af kanter}

\emph{OpenCV} gør det faktisk muligt, at bruge en maske, så man kan
kalde \texttt{cvFloodFill} med et kantdetekteret billede, således at man
ikke maler over de steder i billedet, hvor der er detekteret en kant. I
praksis viste det sig, at være yderst besværligt at bruge denne
funktion, da det kantdetekterede billede skal være to pixel større, i
hver dimension --- det kantdetekterede billede skal have en ramme, på én
pixel. En meget naiv løsning, hvor pixels kopieres én ad gangen, var
tidskrævende på store billeder og vi beholdt derfor de sorte kanter.
Alternativt kunne man beskære det originale billede, men vi besluttede,
at vi ikke ville manipulere med dimensionerne på vores inddata og bevare
det originale billede.

\subsubsection{Optimeringer}
Et hurtigt kig på arbejdsgangen i figur \ref{graphic_pipeline}, viser
allerede mindst ét sted, hvor vi kan optimere programmet. Billedet bliver
nemlig sløret to steder, både i forbindelse med kantdetektion og den
generelle sløring. Det er derfor oplagt, kun at sløre billedet én gang
og passere det slørede billede videre, som argument til de metoder, som
skal bruge det. Dette kan dog kun gøres, hvis vi ønsker at bruge den
samme sløringsmetode, til de to billeder.

Endvidere, kan selve udtrækningen af regioner måske forbedres, ved at bruge
sløring ved statistisk median. Denne metode har vi dog ikke fået testet
ordentlig igennem. Ved brug af en anden sløringsmetode, kunne det godt
tænkes, at fremhævelse af kanter bliver helt overflødig

}

% vim: set tw=72 spell spelllang=da:


\section{Vurdering af regioner\label{section_vurdering_regioner}}
{
{\sffamily Vi vender nu opmærksomheden mod selve implementationen af de
metoder, som afgør, hvorvidt en udtrukket regionen kan siges at være
interessant, samt hvordan vi har implementeret den naive fremgangsmåde,
som bedømmer, om en region ligger i det gyldne snit. Alle metoder
vedrørende vurdering af regioner er implementeret i filen
\texttt{regionSelector.py} i mappen \texttt{lib/}.  Vi har igen, at
metoderne ikke er specifikke for det gyldne snit, men kan anvendes på
ethvert snit i billedet. Sidst i afsnittet viser vi, hvordan vi har
implementeret den videregående vurdering, hvor regioner bedømmes ud fra
deres massemidtpunkt. Vi starter med at se på en fælles datastruktur der
bruges, når regioner bliver vurderet.
}

\subsection{Datastruktur til betingelser}
Når vi skal afgøre, hvorvidt et antal udtrukne regioner er interessante
og ligger i snittet, er der en række betingelser, der skal være opfyldt.
Til disse, er der forbundet nogle udregninger, som vil være de samme for
hver region. Vi bruger derfor en struktur, som indholder resultaterne
fra disse udregninger, således at de ikke skal udføres for hver eneste
region vi kontrollerer. Datastrukturen kaldes \textbf{Constraints} og
ses herunder i \eqref{Constraints_class}.
\begin{multline}
    \textbf{class~} \textrm{Constraints} = \{ \\
    \shoveleft{\qquad\textbf{int} : \textit{coordinate}} \\
    \shoveleft{\qquad\textbf{double} : \textit{minSize}} \\
    \shoveleft{\qquad\textbf{double} : \textit{minMass}} \\
    \shoveleft{\qquad\textbf{int[}2\delta + 1\textbf{]} : \textit{acceptRange}} \\
    \shoveleft{\}}\shoveright{}
    \label{Constraints_class}
\end{multline}
Variablene \texttt{minSize} og \texttt{minMass} relaterer sig kun til
klassificering af interessante regioner, mens \texttt{coordinate} og
\texttt{acceptRange} hører til klassificering af regioner, liggende i
snittet. De enkelte variable vil, i det følgende, blive forklaret
nærmere, når det er relevant.

\subsection{Interessante regioner}
I afsnit \ref{section_naiv}, blev det defineret, at for at en region,
kan betegnes som værende \textbf{interessant}, skal den
\begin{enumerate}
        \renewcommand{\labelenumi}{(\alph{enumi})}
    \item have et areal større end en tærskelværdi, der sættes i
        forhold til billedets størrelse
    \item have en masse større end en tærskelværdi, der ligeledes,
        sættes i forhold til billedets størrelse,
\end{enumerate}
Bemærk, at vurderingen af interessante regioner, ikke har noget at gøre
med hverken snitratio eller margin. Vi undersøger udelukkende de
udtrukne regioners areal og masse. Regioner bliver vurderet, umiddelbart
efter de er blevet trukket ud, så vi har regionerne til rådighed som en
instans af $\angles{CutRegions}$. Vi definerer nu to metoder; én til at
kontrollere en regions størrelse og én til at kontrollere dens masse. Vi
kalder disse \texttt{checkSize} og \texttt{checkMass}. De kan ses i
kodeboks \ref{pseudo_size_mass}.

\begin{lstlisting}[caption={Metoder til at konstollere en regions
    størrelse og masse.},captionpos=b,label={pseudo_size_mass},
    frame=tb, breaklines=false, float=b]
def checkSize(component, constraints):
    "Test if the component have size greater than the minumum size
    defined by the constraints."
    return component.area >= constraints.minSize

def checkMass(component, constraints):
    "Check if the component have mass greater than the minimum mass
    defined by the contraints."
    rect = component.rect
    mass = component.area/(rect.width * rect.height)
    return mass >= constraints.minMass
\end{lstlisting}

Med metoderne i kodeboks \ref{pseudo_size_mass}, returnerer begge en
sandhedsværdi for hvorvidt en region lever op til betingelserne for en
interessant region, og vi da kontrollere hver enkelt. Vi laver nu en ny
metode, som returnerer en \emph{dict} med kun de interessante regioner.
Metoden, kaldet \texttt{GetInterestingRegions}, ses i kodeboks
\ref{pseudo_GetInterestingRegions}. Den tager, som argument, den instans
af $\angles{CutRegions}$, som returneres fra \texttt{ExtractRegions} i
kodeboks \ref{pseudo_udtraek_margin}.

\begin{lstlisting}[caption={Metode som returnerer kun de insteressante
    regioner, givet en instans af $\angles{CutRegions}$}, captionpos=b,
    label={pseudo_GetInterestingRegions}, frame=tb, breaklines=false,
    float=t]
def GetInterestingRegions(CutRegions, constraints):
    interestingRegions = {}
    for id in CutRegions:
        component = CutRegions[id][1]
        passSizeCheck = checkSize(component, constraints)
        passMassCheck = checkMass(component, constraints)
        if (passSizeCheck and passMassCheck):
            interestingRegions[id] = CutRegions[id]
    return interestingRegions
\end{lstlisting}

I kodeboks \ref{pseudo_GetInterestingRegions} bruger vi en instans af
\textbf{Constraints} som argument til metoderne, som kontrollerer
regionens størrelse og masse. Vi skal derfor, inden metoden
\texttt{GetInterestingRegions} kaldes, have initialiseret vores
betingelser, så de passer til billedet. Vi har i kapitel
\ref{chap_afproevning}, fastsat en procentsats for en regions
minimumareal i forhold til billedets størrelse, og denne
minimumstørrelse findes ved udregningen i \eqref{region_min_size}
herunder.
\begin{equation}
    \mathtt{minSize} =
    \lfloor
    \mathrm{minSizePercentage}\cdot\mathrm{height}\cdot\mathrm{width}
    \rfloor
    \label{region_min_size}
\end{equation}
Ligeledes, har vi fastsat en procentsats for en regions minimummasse, men denne
skal vi ikke regne videre på, da metoden \texttt{checkMass} også regner
en procentsats ud for den givne region. Vi gemmer den fastsatte
procentsats, for regioners minimummasse direkte i vores instans af
\textbf{Constraints}. I \texttt{checkMass} sammenlignes minimummassen
med regionen masse direkte.

\subsubsection{Overvejelser}
Udvælgelsen af interessante regioner, kunne godt, være mere
sofistikeret.  Vi undersøger kun regioner for deres størrelse og masse,
hvilket stadig tillader mange regioner, som egentlig er uinteressante,
pga. deres form.  I udvælgelsen af interessante regioner, kunne man
derfor kigge på regionens form eller udstrækning, ved at undersøge
dennes massemidtpunkt.  Hvis massen er koncentreret langt væk fra
snittet, er denne region ikke interessant. Vi skal dog passe på, at vi
ikke tager beslutninger, som egentlig vedrører, om regionen er placeret
i snittet. Vi ønsker, i denne udvælgelse af regioner, udelukkende at
bestemme, hvorvidt regionen skal tages op til videre overvejelse, for om
denne ligger i snittet.

Kun hvis vores søgning for objekter i billedet, bliver mere specifik,
giver det mening, at undersøge regionerne nærmere i udvælgelsen af
interessante regioner. Vi kan forestille os, en situation, hvor man
udelukkende vil finde ansigter, placeret i det gyldne snit. I dette
tilfælde, skal vi selvfølgelig ikke sende en region til videre
vurdering, hvis denne \emph{ikke} er et ansigt. Vi har dog ikke en sådan
specifik søgning, hvorfor vi kun kan frasortere regioner, ud fra
informationen om deres størrelse og masse.

\subsection{Naiv vurdering af regioner}
Vi vil i dette afsnit se på, hvordan den naive fremgangsmåde, givet i
kapitel \ref{chap_detektion}, vurderer hvorvidt en region ligger i det
gyldne snit. Vi vil dog først give en forklaring på hvordan vores margin
bliver repræsenteret.

\subsubsection{Udregning af margin\label{subsec_margin_udregning}}
Inden vi ser på, hvordan det vurderes, hvorvidt en region er placeret i
et givet snit, skal vi se på hvordan vi egentlig regner margin ud. Som
nævnt, både i kapitel \ref{chap_detektion} og \ref{chap_afproevning},
bruges en procentsats til at angive vores margin. Denne procentsats vil,
hvis vi kun undersøger ét snit, blive sat til $2.4\%$ af billedets højde
eller bredde, alt efter hvilken orientering det aktuelle snit har.
Procentsatsen er baseret på forskellen mellem det gyldne snit og snittet
ved to tredjedele. Vi har implementeret fastsættelsen af denne
procentsats således, at \emph{hvis} man ønsker at sammenligne snit, som
ligger tættere på hinanden end det gyldne snit og to tredjedele, så
findes den procentsats der gør, at disse snits margin ikke overlapper. I
praksis gives en liste med snitratioer, der ønskes undersøgt, og fra
denne liste, findes den mindste differens mellem ratioerne,. Den mindste
differens mellem snitratioer, bruges da, som procentsats for alle snit i
analysen. Hvis den mindste differens, mellem snitrationerne, er større
end $2.4\%$, sættes procentsatsen for margin til $2.4\%$. Vi antager i
det følgende, at vores procentsats for margin er blevet sat til $2.4\%$.

Når vi skal vurdere regioner med hensyn til vores margin, får vi brug
for den eksakte pixelstørrelse på margin. I filen
\texttt{marginCalculator.py} i mappen \texttt{lib/} er der implementeret
metoder til dette. Her bruges metoden \texttt{getPixels}, som, givet et
billede, et snit og en procentsats for margin, returnerer afstanden fra
snittet til margin i pixels, som vi også skriver som $\delta$. Det er
også i \texttt{marginCalculator.py} som finder den mindste differens
mellem snitratioer, ved metoden \texttt{getPercentage}.

\begin{figure}[b]
    \centering
    \begin{picture}(122,55)
        \put(61, 50){$x$}
        \put(-10, 22){$y$}
        \put(0, 45){\circle*{3}}
        \put(-1, 45){\vector(1, 0){120}}
        \put(0, 45){\vector(0, -1){48}}

        \color{red}
        \put(88, 50){\line(0, -1){55}}

        \color{blue}
        \put(84, 50){\line(0, -1){55}}
        \put(92, 50){\line(0, -1){55}}

        \color{black}

        \put(66, 30){$-^{x}$}
        \put(78, 30){\vector(1, 0){20}}

        \put(100, 30){$+^{x} $}
        \put(98, 30){\vector(-1, 0){20}}


    \end{picture}
    \caption[]{Koordinatsystem med indtegnet snit og margin. Bemærk, at
    ved det vertikale snit, skal vi kun betragte regioners $x$-værdi,
    når vi skal bedømme om de ligger inden for margin. Ligeledes kan
    margin repræsenteres kun ved de tilladte $x$-værdier.}
    \label{margin_koordinatsystem}
\end{figure}
Når vi, i den naive vurdering af regioner, skal afgøre om en region
ligger inden for vores margin, kan vi udnytte, at vi enten betragter et
vertikalt eller horisontalt snit.  Når vi har et vertikalt snit, behøver
vi kun at betragte $x$-værdier, da $y$-værdien ikke influerer på det
vertikale snit. Omvendt med horisontale snit, behøver vi kun at betragte
$y$-værdier, da $x$-værdierne, i denne situation, ikke har betydning for
snittet.  Tilfældet, for det vertikale snit, er vist i figur
\ref{margin_koordinatsystem}.  Denne egenskab gør, at vi nu kan oprette
et sæt bestående af netop kun de koordinater som ligger inden for
margin. Hvis vi betragter et vertikalt snit, kan vi i Python oprette
sættet af accepterende $x$-værdier ved at bruge strukturen
\texttt{range}. F.eks. vil \texttt{range(1, 4)} returnere listen
$\{1,2,3\}$.  Vores implementation bruger endvidere variablen
\texttt{coordinate} i \textbf{Constraints}, som, lidt misvisende,
indikerer om snittet vi undersøger er vertikalt eller horisontalt. Er
snittet vertikalt, sættes \texttt{coordinate} til $0$, og $1$ for et
horisontalt snit. For ethvert snit, kan vi da finde de accepterende
$x$-værdier, som vist i kodeboks \ref{pseudo_acceptRange}. Variablen
\texttt{margin}, der tages som argument, er del pixelstørrelse der er
returnet fra metoden \texttt{getPixels} fra
\texttt{marginCalculator.py}.

\begin{lstlisting}[caption={Pseudokode},captionpos=b,label={pseudo_acceptRange},
    frame=tb, breaklines=false, float=t]
GetAcceptRange(cut, margin, coordinate):
    if coordinate:
        # Horizontal cut
        lower_bound = cut.p1.y - margin
        upper_bound = cut.p1.y + margin
        acceptRange = range(lower_bound, upper_bound)
    else:
        # Vertical cut
        lower_bound = cut.p1.x - margin
        upper_bound = cut.p1.x + margin
        acceptRange = range(lower_bound, upper_bound)

    return acceptRange
\end{lstlisting}

\subsubsection{Kontrol på en regions afgrænsende rektangel}
\begin{lstlisting}[caption={Metode, som kontrollerer, hvorvidt en region
    har en kant af det afgrænsende rektangel inden for margin.},
    captionpos=b, label={pseudo_position}, frame=tb, breaklines=false,
    float=b]
def checkPosition(component, constraints):
    "Test if the component have a bounding box inside the accepting
    rectangle defined in the constraints."
    d = component.rect.width
    p = component.rect.x
    if constraints.coordinate:
        d = component.rect.height
        p = component.rect.y

    lowerInRange = p in constraints.acceptRange
    upperInRange = (p + d) in constraints.acceptRange

    return lowerInRange or upperInRange
\end{lstlisting}
Når vi har fundet værdierne i \texttt{acceptRange}, og dermed også
fastfast \texttt{coordinate}, dvs. vi ved hvilken orientering snittet
har, kan vi endelig sige om en interessant region ligger placeret i
snittet eller ej. Vi har allerede udnyttet, at vi kun behøver at
betragte én koordinat, når vi kun har vertikale og horisontale snit. Det
er derfor ligetil at kontrollere, om en regions afgrænsende rektangel,
har en kant inden for margin. Regionen er repræsenteret som en instans
af \textbf{cvConnectedComp}, hvori der er gemt en instans af
\textbf{cvRect}. Vi har en metode, kaldet \texttt{checkPosition}, som,
alt efter om vi har et vertikalt eller et horisontalt snit, returnerer
en sandhedsværdi for om den relevante koordinat findes i de accepterende
koordinater. Metoden ses i kodeboks \ref{pseudo_position}.

Vi kan nu sammensætte en metode som returnerer alle interessante
regioner, med en kant inden for margin. Metoden tager en instans af
$\angles{CutRegions}$ og en instans af \textbf{Constraints} som
argumenter. Vi genererer altså vores betingelser inden vi begynder at
frasortere regioner, således at alle informationer om snittet og krav
for regioner ligge i instansen af \textbf{Constraints}. Den endelige
metode, for vurdering efter den naive fremgangsmåde, kaldes
\texttt{GetInterestingRegionsInCut} og er vist i kodeboks
\ref{pseudo_GetInterestingRegionsInCut}.

\begin{lstlisting}[caption={Pseudokode, som returnerer alle interessante
    regioner, der har en kant, af deres afgrænsende rektangel, inden for
    margin.},
    captionpos=b, label={pseudo_GetInterestingRegionsInCut}, frame=tb, breaklines=false,
    float=t]
def GetInterestingRegionsInCut(CutRegions, constraints):

    # First remove uninteresting regions
    interestingRegions = GetInterestingRegions(CutRegions, constraints)

    # Initialize an empty dict
    interestingRegionsInCut = {}

    # Check every interesting region if it's bounding box
    # has an edge inside the margin
    for id in interestingRegions:
        component = interestingRegions[id][1]
        if checkPosition(component, constraints):
            interestingRegionsInCut[id] = interestingRegions[id]

    # The resulting dict contains only interesting regions
    # with an edge inside the margin
    return interestingRegionsInCut
\end{lstlisting}

\subsection{Udvidet vurdering af regioner}
\subsubsection{Gitter (Grid)}
\subsubsection{Massemidtpunkt}

% vim: set tw=72 spell spelllang=da:


\section{Gennemgang af en kørsel\label{section_koersel}}
{
{\sffamily
For at beskrive integrationen og samspillet mellem en kørsel og
databasen vil der følge en beskrivelse af de vigtige kald mellem 
programmet og databasen. 
}


Filen start.py er distributør af arbejde til de andre dele af programkoden.
Diagram \ref{start_workflow} og pseudokode \ref{pseudo_workflow} giver et overblik over hvad der bliver
forklaret i resten af dette kapitel. 
\begin{figure}[h!]
	\begin{center}
		\includegraphics[scale=0.5]{afsnit/implementation/billeder/workflow_start_py.png}
	\end{center}
	\caption{De blå pile er ting, som sker en enkelt gang, mens de blå
	\label{start_workflow}
	bliver gentaget indtil der ikke er flere billeder at arbejde på}
\end{figure}
\begin{lstlisting}[caption={Pseudokode for
start},frame=tb,label={pseudo_workflow}]
cuts = experiment.generateCuts()
experiment.setSettings(settings)
experiment.setGlobalSettings(globalSettings)
db = Database(globalSettings)
db.construct(Database)
run = m.createNewRun(settings)
paintings = m.Painting.select(m.Painting.q.form=="painting")
for painting in paintings:
	paintingContainer = Painting(painting)
	paintingContainer.setResults(paintingAnalyzer.analyze(paintingContainer,settings))
	m.saveResults(run.id,paintingContainer)
\end{lstlisting}
\subsection{Eksperimenter}
Eksperimenter kontrollere hvilke snit og indstillinger en given kørsel
skal afvikles med.  \ref{pseudo_experiment} er et eksempel
på hvordan et eksperiment evt. kunne se ud.
\begin{lstlisting}[caption={Pseudokode for et
experiment, som checker på $\varPhi$ og $\frac{2}{3}$},frame=tb,label={pseudo_experiment}]
def generateCuts():
	cuts = [goldenLibrary.PHI,2/3]
	return cuts
def setSettings(settings):
	settings.setMarginPercentage(0.024)
	return 0
def setGlobalSettings(globalSettings):
	return 0
\end{lstlisting}
%Settings 
\subsection{Globale indstillinger}
De globale indstillinger er placeringen på databasen og den
kommasepareredefil. Dette er indstillinger, som ikke ændre sig fra
kørsel til kørsel, dog er de vitale for hele programmet.

\subsection{Kørselsindstillinger}
Kørselsindstilinger kan afvige i de forskellige kørsler. De består af
tærskelværdier for floodfill og kantdetektion, hvilken analysemetode,
samt størrelsen på margin\ref{terskelverdi} og de snit, som
skal undersøges. De her variabler betyder meget for resultaterne og derfor er vigtige
at kunne ændre fra kørsel til kørsel. 

\subsection{Initialisering af databasen}
Databasen er konstrueret ved at gennemløbe den kommaseparedefil fra
wga.hu.
Ved at bruge \emph{SQLObject} er det ligetil at konstruere tabeller i
databasen. Vi har i afsnit \ref{section_database} givet det database
skema som vi opbygger databasen efter. I kodeboks
\ref{code_tabel_artist} er vist hvordan tabellen \texttt{artist}
konstrueres ved brug af \emph{SQLObject} i Python.

\begin{lstlisting}[caption={Pythonkode for oprettelse af tabeller i
    databasen.}, captionpos=b, label={code_tabel_artist}, frame=tb,
    breaklines=false, float=hb]
import sqlobject as s

class Artist(s.SQLObject):
    "
    _id_, name, born, died, school, timeline
    "
    name = s.StringCol()
    born = s.IntCol()
    died = s.IntCol()
    school = s.StringCol()
    timeline = s.StringCol()
\end{lstlisting}

Når man vil oprette en ny kunstner i databasen gøres det som vist
i kodeboks \ref{code_new_artist}.

\begin{lstlisting}[caption={Oprettelse af en kunstner i databasen.},
    captionpos=b, label={code_new_artist}, frame=tb, breaklines=false,
    float=h]
# Init variables
name = "Homer Simpson"
born = 1968
died = 2000
school = "Springfield"
timeline = "1950-2000"

# Create the artist in the database
Artist(name=name, born=born, died=died, school=school, timeline=timeline)
\end{lstlisting}

\emph{SQLObject} opretter automatisk et id-felt til alle tabeller i
databasen. Vi kan udnytte dette til at lave \emph{foreign keys} i
tabellerne. Vi viser i kodeboks \ref{code_tabel_result} hvordan tabellen
\texttt{result} oprettes i databasen, hvor det er interessant at bemærke
hvorledes de to \emph{foreign keys} oprettes.

\begin{lstlisting}[caption={Pythonkode for oprettelse af \emph{foreign
    keys} i databasen.}, captionpos=b, label={code_tabel_result}, frame=tb,
    breaklines=false, float=h]
class Result(s.SQLObject):
    "
    _id_, ^runId, ^paintingId, cutRatio, cutNo, numberOfRegions
    "

    run = s.ForeignKey('Run')
    painting = s.ForeignKey('Painting')
    cutRatio = s.FloatCol()
    cutNo = s.IntCol()
    numberOfRegions = s.IntCol()
\end{lstlisting}

Dette sker kun i det tilfælde at databasen ikke findes ved starten af en
kørsel. Den kommasepareredefil bliver parset før enhver kørsel og bliver
linje for linje sammenlignet med databasen for at finde evt. mangler.
Det er værd at bemærke at databasen ikke kan håndtere at blive afbrudt i dette stadie første gang den bliver kørt.
Grunden til dette er at det er en meget overfladisk sammenligning, kun
er billedes placering på wga.hu, som bliver sammenlignet, da denne er
unik.
Omvendt giver det mulighed for at den kommaseparerede fil kan opdateres uden nogen
problemer. Billederne automatisk bliver hentet hvis de mangler
giver det også mulighed for at flytte databasen til en anden
maskine uden nogen problemer.
Der er kun to tilfælde databasen kan håndtere udvidelser, hvis den
kommasepareredefil bliver opdateret og beholder sin nuværende form, og
hvis en testdatabases \texttt{count} variable bliver sat til en højere
værdi, dette tilfælde forklares senere.
Følgende pseudokode forklarer hvad databasen gennemgår hver gang der
startes en ny kørsel.
\begin{lstlisting}[caption={Pseudokode for database
initialisering},frame=tb,label={pseudo_init_db}]
csvfile = open(Settings.csvfilelocation)
for line in csvfile:
	line = parser.parse(line)
	if not os.path.isfile(line.path):
		download(line.url)
	if database.Painting.select(database.Painting.url==line.url).count() == 0:
		database.Painting.insert(line)
\end{lstlisting}
\subsubsection{Parsing af kommasepareret fil}
I den kommaseparfilen gives endvidere mange oplysninger, om den enkelte artikel samt
dennes kunstner.  Vi har konstrueret en parser, som trækker disse
informationer ud fra filen og lægger dem ind i databasen. Da vi primært
vil beskæftige os med malerier, vil vi nu blot omtale kunstartikler som
malerier.

Den konstruerede parser, til den kommaseparerede fil, er dog ret grov,
da folkene bag hjemmesiden \cite{wgahu} ikke har lagt meget vægt på, at være
konsistente i deres formulering af en kunstners fødsels- og dødsår eller
en genstands dimensioner. En følge deraf er, at nogle kunstnere, hvor
hjemmesiden\cite{wgahu} ikke har en klar indikation af dennes levealder, ikke
bliver registreret i databasen. Vi kan dog stadig slå kunstneren op ved
at bruge feltet ``timeline'', som angiver hvilken periode kunstneren
tilhører. Vi har i enkelte tilfælde, set os nødsaget til at rette i den
kommaseparerede fil, hvor der er blevet indsat tegn, som helt umuliggør
korrekt parsing af filen, såsom ekstra komma eller semikolon.

\subsubsection{Testdatabase}\label{test_db}
For at alle kan arbejde på det samme billeder undervejs i udviklingen er
det muligt at konstruere en lille testdatabase, dette gøres ved at sætte
\texttt{testdatabase} variablen til True og 
\texttt{count} variablen til det antal af billeder der ønskes. En fuld beskrivelse kan findes i \ref{brugervejl_test_db}


\subsection{Analysen}
Analysen bliver kørt på alle billeder, som har typen "painting" i den
kommaseparerede fil. Som beskrevet i linje 8-10 i \ref{pseudo_workflow}
så bliver billedet sendt igennem Painting klassen, hvor det bliver
konverteret til et \emph{Opencv} billede. Når det er konverteret sendes
det til paintingAnalyser, hvis formål er at sende det videre til
udtrækning af regioner. Efter alle regioner er trukket ud gemmes
resultaterne i databasen.

\subsection{Genskabelse af parametre og resultater}
At kunne genskabe de fundne resultater fra en analyse har meget stor
betydning, dels for at kunne udtage stikprøver i udviklingen af hele
programmet, men også for at kunne fremvise grafiske resultater. Vi har
allerede været inde på, at man for at kunne genskabe et resultat, skal
vide hvilke parametre der oprindeligt har været brugt. Ovenstående
databaseskema gør det let at hente disse parametre ud. Hvis vi får et
resultat med overraskende mange regioner og gerne vil undersøge dette
tilfælde, har vi metoder til rådighed der giver os lige nøjagtig de
informationer vi har brug for at vise dette grafisk. Helt konkret har vi
metoderne vist i listing \ref{rekonst_koersel} til rådighed.

\vspace{0.5cm}
\begin{lstlisting}[caption={Metoder til rekonstruktion af kørsler},captionpos=b,label={rekonst_koersel},numbers=none]
def getSettingsForRunId(runId):
    """Return the settings instance for a given run"""
    pass

def getCutRatiosForRunId(runId):
    """Return the list of cut ratios for a given run"""
    pass

def getSettingsForResultId(resultId):
    """Return the settings instance for a given result"""
    pass

def getSettingsForRegionId(regionId):
    """Return the settings instance for a given region"""
    pass

def getCutRatioForRegionId(regionId):
    """Return the list of cut ratios for a given region"""
    pass

def getCutNoForRegionId(regionId):
    """Return the cut number for a given region"""
    pass

def getRegionsForResultId(resultId):
    """Return the list of regions for a given result"""
    pass
\end{lstlisting}

Selvom metoderne i listing \ref{rekonst_koersel} ikke viser noget
egentlig kode, bør det ud fra sammenhængen være klart hvad disse metoder
gør. Alle metoder der starter med \texttt{getSettings} returnerer
klassen \texttt{Settings} som vist i listing \ref{settings_klassen} med
indstillinger tilpasset den enkelte forespørgelse.
\vspace{0.5cm}
\begin{lstlisting}[caption={Settings-klassen med standardindstillinger},captionpos=b,label={settings_klassen},numbers=none]
class Settings:
    """These are the default settings for the analysis"""
    edgeThreshold1 = 78
    edgeThreshold2 = 2.5 * edgeThreshold1
    lo = 4
    up = 4
    cutRatios = None
    marginPercentage = 0.009
    method = 'naive'
    ...
\end{lstlisting}

Det ses at vi har mulighed for at trække de fundne regioner ved et
snit ud og vi behøver derfor ikke at køre nogen analyse på billedet hvis
vi blot ønsker at få de fundne regioners begrænsende areal vist. I dette
tilfælde kan vi nøjes med at forespørge databasen om de regioner der er
tilknyttet et snit vi gerne vil undersøge og traversere gennem den liste
af regioner vi får tilbage. Hver region er repræsenteret som en klasse
hvor vi kan trække rektanglet ud og vi bruger da \emph{OpenCV} til at
tegne rektanglet på det tilknyttede billede.
}
% vim: set tw=72 spell spelllang=da:


\section{Database\label{section_imp_database}}
Databasen er lavet i sqlite.  Formålet med databasen er todelt,
opbevaring af metadata og opsamling af statestik fra kørsler.
Problemet med dette er at når databasen skal konstrueres, er det ikke
muligt at sige noget definitivt omkring statestik delen af databasen.
At splitte databasen op i to, er en mulighed.  En del af de udvidede
løsninger opridset i synopsisen benytter en del af metadataerne omkring
billederne, derfor er denne løsning af tvivlsom anvendlighed.  En
løsning, der tilbyder udvidelse, må derfor være at tilstræbe.
Basisløsningen består udelukkende af opbevaringen af metadata, grunden
til dette er udelukkende planlægning af projektet.  Database skemaet
til basisløsningen bygger endnu videre på den struktur, der er
udleveret af \cite{wgahu}.

\begin{center}
\begin{tabular}{|l||c|c|c|c|c|c|}
    \hline
    \bf{artist} \hspace{0.5cm} & \underline{artistId} & name & born & died & school & timeline \\\hline
\end{tabular}

\begin{tabular}{|l||c|c|c|c|c|c}
    \hline
    \bf{painting} \hspace{0.5cm} & \underline{paintingId} & artistId & title & date & paint & $\cdots$ \\\hline
\end{tabular}\\ \vspace{0.2cm}\hspace{1.2cm}
\begin{tabular}{c|c|c|c|c|c|c}
    \hline
    $\cdots$ & material & location & url & form & type & $\cdots$ \\\hline
\end{tabular}\\ \vspace{0.2cm}\hspace{1.4cm}
\begin{tabular}{c|c|c|c|c|c|}
    \hline
    $\cdots$ & realHeight & realWidth & height & width & filepath \\\hline
\end{tabular}

\begin{tabular}{|l||c|c|c|c|c|c|c|}
    \hline
    \bf{run} \hspace{0.5cm} & \underline{runId} & trsh1 & trsh2 & lo & up & marginPercentage & method \\\hline
\end{tabular}

\begin{tabular}{|l||c|c|c|c|c|c|}
    \hline
    \bf{result} \hspace{0.5cm} & \underline{resultId} & runId & paintingId & cutRatio & cutNo & numberOfRegions \\\hline
\end{tabular}

\begin{tabular}{|l||c|c|c|c|c|c|c|}
    \hline
    \bf{region} \hspace{0.5cm} & \underline{regionId} & resultId & x & y & height & width & area \\\hline
\end{tabular}
\end{center}

\newpage
\includegraphics[scale=0.9]{afsnit/vores_implementation/billeder/ER}

% vim: set tw=72 spell spelllang=da:


}

% vim: set tw=72 spell spelllang=da:


\chapter{Videnskabelige resultater\label{chap_resultater}}
{
{\sffamily Vi vil i dette kapitel se på, hvilke resultater, vi med vores
metoder, er kommet frem til. Vi præsenterer en række hypoteser og
undersøger hvorvidt vores data kan be- eller afkræfte disse hypoteser.
Vi vil også gerne se, om vores data fremvise nogle interessante
observationer vedrørende kunstneres oprindelse eller fødselsår.
}

\section{Hypoteser}
{
{\sffamily Vi vil i det følgende præsentere de hypoteser, som vi, på
baggrund af resultaterne fra en analyse på vores datasæt, vil forsøge at
besvare. De fleste af hypoteserne antager den generelle opfattelse, at
det gyldne snit er specielt æstetisk tiltalende og af denne grund, er
meget brugt i malerkunsten. Vi deler hypoteserne op i nogle kategorier,
alt efter hvilket aspekt hypotesen belyser. Inden vi præsenterer de
omtalte hypoteser, er det dog nødvendigt at kaste et kritisk blik på
vores datasæt.
}

\subsection{Datasæt}
Det korpus, vi kører vores analyse på, består af billeder hentet fra
hjemmesiden wga.hu\cite{wgahu}, som er en online billededatabase, med
europæiske kunstartikler fra år 1001 -- 1900. I kunstartiklerne, hvor
det samlede antal er omkring 23.000, indgår møbler, kalkmalerier,
skulpturer, mosaikker og malerier, hvor sidstnævnte, vil være vores
fokus. Over halvdelen af disse kunstartikler står udstillet på museum.
Databasen blev oprettet i 1996, med det formål at præsentere kunst fra
renæssancen (ca.  14. -- 17.  århundrede), men blev senere udvidet, til
også at inkludere kunst fra andre perioder. Dette betyder, at
størstedelen af malerierne vi undersøger, er fra tidsperioden 1450 --
1650 og er malet af italienske kunstnere. Endvidere er langt de fleste
malerier, klassificeret som religiøse.  Disse informationer er givet fra
wga.hu, men er også suppleret i bilag \ref{appendix_grafer} som
grafer.

Vi må af ovenstående grunde forvente, at resultater, fra en analyse på
vores datasæt, vil være farvet af samlingen af malerier, og det derfor
kan være svært at drage nogen konklusioner for malerkunsten generelt, da
resultaterne vil være begrænset, til kun at gælde for et udsnit af
vestlig kultur. Endvidere findes der ingen nyere malerier i datasættet,
hvilket gør at vi ikke kan udtale os om nyere malerkunst.

Billederne, som suppleres fra databasen, er af høj kvalitet, men der er
visse problemer, som vi nævner nedenfor.

\begin{itemize}
    \item \textbf{Beskæring af billeder}\\
        Vi kan ikke vide os sikre på, om billederne i datasættet er
        ordentligt beskåret. Dvs. vi \emph{kan} have, at noget af
        billedrammen er med i billedet. Dette kan muligvis volde lidt
        problemer med udtrækning af regioner, men hvad værre er, så gør
        det vores mål, for hvor det gyldne snit ligger, upræcist. Dette
        har vi dog taget højde for, i kraft af vores margin.  Endeligt
        er der inkluderet billeder af malerier detaljer i databasen, som
        er udsnit af maleriet, således at målene på billedet ikke
        passer.
    \item \textbf{Forvrængning og perspektiv}\\
        Billederne af malerier er taget med et kamera, hvor linsen muligvis kan
        forvrænge billedet. Vi kan derfor have skæve linjer og tage
        forkerte beslutninger, for regioner, pga. dette. Endvidere kan
        billedet være taget skævt, således at billedet hælder til den
        ene side. Vi kan selvfølgelig også have at perspektivet i
        billedet er forkert, fordi billedet er taget fra en skæv vinkel.
    \item \textbf{Opdelte malerier}\\
        Nogle store malerier kan være blevet opdelt, da databasen har
        det formål at vise malerierne på en computerskærm, hvor meget
        store billeder kan være svære at betragte. Dette betyder, at
        nogle billeder ikke viser hele maleriet, men blot er et udsnit,
        hvilket påvirker vores muligheder for at sige noget fornuftigt
        om det gyldne snit i maleriet.
\end{itemize}

Nogle stilarter, såsom kalkmalerier og tegninger, har gennem
udokumenterede afprøvninger, vist sig at være besværlige at analysere,
pga. meget svingende farvegengivelse. Endvidere, kan disse være billeder
af en hvælving i en kirke, som ikke egner sig til analyse for det gyldne
snit. Vi har derfor valgt kun at analysere malerier, kendetegnet ved at
de er beskrevet som ``painting'' fra wga.hu.

Alt det ovenstående vil påvirke resultaterne, ved analyse på vores
datasæt.

\subsection{Generelle hypoteser}
Følgende hypoteser er ret generelle. Altså sådan nogenlunde.

\begin{hypotese}
    Hvis det gyldne snit er meget brugt i malerkunsten, må vi have, at
    over halvdelen, af de analyserede malerier, har én eller flere
    regioner liggende i det gyldne snit.
\end{hypotese}

\begin{hypotese}
    Hvis et gyldent snit er specielt æstetisk tiltalende, må vi have, at
    antallet af regioner liggende hvert af de fire snit, som kan
    betragtes som gyldne, ikke afviger mere end $\pm10\%$ fra hinanden.
\end{hypotese}

\begin{hypotese}
    Hvis det gyldne rektangel er et specielt æstetisk tiltalende format,
    må vi have, at mere end en tredjedel malerierne har et lærred, hvis
    dimensioner er lig $\varphi$. \textbf{Problemer med billeder som er
    skåret op.}
\end{hypotese}

\subsection{Antallet af regioner i det gyldne snit}
Følgende hypoteser omhandler antallet af detekterede regioner i det
gyldne snit. Altså sådan nogenlunde.

\begin{hypotese}
    Hvis det gyldne snit er meget brug i malerkunsten, må vi have, at
    antallet af detekterede regioner i det gyldne snit, er skarpt
    større, end antallet i alle andre snit.
\end{hypotese}

\begin{hypotese}
    Hvis det gyldne snit er bevidst brugt af kunstneren, må vi have, at
    antallet regioner liggende i det gyldne snit, er skarpt større, end
    antallet af regioner liggende i snittet ved to tredjedele.
\end{hypotese}

\begin{hypotese}
    Hvis det gyldne snit er bevidst brugt af kunstneren, må vi have, at
    antallet regioner liggende i det gyldne snit, er skarpt større, end
    antallet af regioner liggende det midterste snit.
\end{hypotese}

\subsection{Tidsperiode, kunstner og land}
Følgende hypoteser har noget at gøre med maleriernes ophav og arbejder
ud fra den antagelse, at hvis det gyldne snit \emph{altid} har været
interessant, så skal det altid have været til stede, uanset
nationalitet, årstal og kunstner. [Bør omformuleres].

\begin{hypotese}
    Hvis det gyldne snit altid har været specielt æstetisk tiltalende,
    må vi have, at antallet af regioner liggende i det gyldne snit,
    tidsperioder imellem, højest kan afvige med $\pm10\%$.
\end{hypotese}

\begin{hypotese}
    Hvis det gyldne snit altid er ligeså  æstetisk tiltalende for alle,
    må vi have, at antallet af regioner liggende i det gyldne snit,
    nationaliteter imellem, højest kan afvige med $\pm10\%$.
\end{hypotese}

\begin{hypotese}
    Hvis det gyldne snit er lige æstetisk tiltalende for alle personer,
    må vi have, at antallet af regioner liggende i det gyldne snit, mod
    antallet af regioner liggende i alle andre snit, højst må afgive med
    $\pm10\%$ mellem de enkelte kunstnere. Vi skal altså have, at
    procentdelen af regioner liggende i det gyldne snit, er nogenlunde
    den samme.  \textbf{Meget kontroversiel og nok ikke værd
    at undersøge}.
\end{hypotese}

\subsection{Det gyldne snit mod \emph{the rule of thirds}}
Vi undersøger her, om der virkelig gøres brug af en approksimation til
det gyldne snit, hvor man arbejder ud fra snittet ved to tredjedele.

\begin{hypotese}
    Hvis det gyldne snit er specielt æstetisk tiltalende, og vi har at
    \emph{the rule of thirds} bliver brugt i malerkunsten, som en
    approksimation til det gyldne snit, må vi have, at antallet af
    regioner liggende i det gyldne snit ikke angiver fra antallet af
    regioner i to tredjedele med mere end $15\%$.
\end{hypotese}

}
% vim: set tw=72 spell spelllang=da:


\section{Naiv implementation\label{section_naiv_koersel}}
{
\subsection{Eksperimentsopstilling}
I \ref{chap_afproevning} er de optimale tærskelværdier fundet.
Idet denne hypotese blot kigger på frekvensen for brug af det gyldnesnit
mod andre snit, hvis eneste restriktion er at de skal have et fælles
forhold til det gyldnesnit.
Det er også fordelagtigt at maksimere antallet af andre snit, da det
giver et bedre grundlag for eksperimentet.
Afstanden mellem to snit er begrænset af margin defineret til at være
$2.4\%$\ref{margin}. 
Denne margin skal være tilstede på begge sider af et snit, så derfor vil
hvert snit fylde $(2.4*2)\%$.
Det maksimale antal af snit på et billede må altså være
$100/4.8=20.833$. Hvilket ses på denne figur \ref{snitogmargin}
\begin{figure}[ht]
	\begin{center}
		\includegraphics[scale=0.3]{afsnit/resultater/billeder/20_cuts_med_margin}
	\end{center}
	\caption{Sort:snittene, grøn: margins og rød er midten}
	\label{snitogmargin}
\end{figure}
Yderpunkterne $0.958$ og $0.518$ er problematiske, $0.958$'s margin
løber udover billedet, dvs. at den ren principielt går glip af at
detektere en masse interessante regioner.
$0.518$ lider af det modsatte problem, den kan potentielt fange
interessante regioner, på begge sider af midten.
For at være helt præcis så er det i $0.518$ tilfælde:\\
$1-0.518 = 0.482$
$0.518-0.482=0.036$\\
Hvilket er afstanden mellem de to snit.
Der er altså en stimmel på $0.05-0.036 = 0.014 = 1.4\%$ af billedet,
hvor interessante regioner bliver talt to gange.

Eksperimentet bliver kørt på 17364 billeder.

\newpage
\subsection{Resultater}
Resultaterne strider ikke mod hypotesen, dog tegner \ref{diffratios}
et meget et interessante billede. Det er kun de to snit der ligger
tættere på midten, der indeholder flere interessante regioner.
Antallet af interessante regioner stiger forholdvis konstant mellem
$0.77$ og op til $0.57$.

Med den nuværende algoritme og billedebase burde det gyldnesnit
altså ligge mellem $0.56803398875 +- 2.4\%$.
Og tyder meget på at midten langt mere er stedet kunstrene arbejder
ud fra.
\begin{figure}[ht]
	\begin{minipage}[b]{0.5\linewidth}
		\begin{center}
		\includegraphics[scale=0.4]{afsnit/resultater/billeder/cut0featsperratio.png}
		\caption{Antal af detekterede interessante regioner i det højre
		vertikale snit}
		\label{cut0feats}
		\end{center}
	\end{minipage}
	\hspace{0.5cm}
	\begin{minipage}[b]{0.5\linewidth}
		\begin{center}
		\includegraphics[scale=0.4]{afsnit/resultater/billeder/cut1featsperratio.png}
		\caption{Antal af detekterede interessante regioner i det
		venstre vertikale snit}
		\label{cut1feats}
		\end{center}
	\end{minipage}
\end{figure}
\begin{figure}[ht]
	\begin{minipage}[b]{0.5\linewidth}
		\begin{center}
		\includegraphics[scale=0.4]{afsnit/resultater/billeder/cut2featsperratio.png}
		\caption{Antal af detekterede interessante regioner i det højre
		vertikale snit}
		\label{cut0feats}
		\end{center}
	\end{minipage}
	\hspace{0.5cm}
	\begin{minipage}[b]{0.5\linewidth}
		\begin{center}
		\includegraphics[scale=0.4]{afsnit/resultater/billeder/cut3featsperratio.png}
		\caption{Antal af detekterede interessante regioner i det
		venstre vertikale snit}
		\label{cut1feats}
		\end{center}
	\end{minipage}
\end{figure}

\begin{figure}[h!]
	\begin{center}
		\includegraphics[scale=0.5]{afsnit/resultater/billeder/featsperratio.png}
	\end{center}
	\caption{Antal af detektere interessante regioner på de forskellige snit.}
	\label{diffratios}
\end{figure}




%\begin{verbatim}
%number of features in the golden ratio in different periodes
%
%{'1301-1350\r\n': 151894, '1551-1600\r\n': 184246, '1201-1250\r\n': 419, '1851-1900\r\n': 15092, '1101-1150\r\n': 2817, '1651-1700\r\n': 171119, '1351-1400\r\n': 35464, '1251-1300\r\n': 11864, '1451-1500\r\n': 428338, '1701-1750\r\n': 115703, '1151-1200\r\n': 14688, '1751-1800\r\n': 67703, '1801-1850\r\n': 79182, '1601-1650\r\n': 273832, '1401-1450\r\n': 199989, '1501-1550\r\n': 394100}
%Which golden ration is the most popular, ranging from 0 to 3
%[56092, 57044, 59181, 54152]
%features in the different ratios
%{0.66803398874999997: 222018, 0.86803398875000004: 206899, 0.56803398875: 229650, 0.96803398875000002: 183833, 0.76803398874999995: 213570, 0.91803398874999997: 208340, 0.81803398875: 208081, 0.71803398875000002: 217432, 0.51803398874999995: 230144, 0.61803398875000004: 226462}
%Top 10 cuts, where the most features was found
%[239, 250, 254, 257, 274, 288, 298, 326, 430, 436]
%Top 10 images
%[546, 552, 554, 569, 570, 578, 592, 616, 634, 675]
%Top 10 images, with only the features in the golden feature
%[73, 75, 76, 77, 78, 86, 87, 87, 99, 147]
%Top 10 images, with only features in 2/3 that counts
%[144, 146, 171, 204, 209, 211, 221, 228, 251, 300]
%\end{verbatim}
%TODO:tilføj en ud af hvor mange billeder der var i den periode!
%	og hvilke billeder der er i top 10!
%	og en fordelen af hvor mange features der er i billeder generelt.
%
}
% vim: set tw=72 spell spelllang=da:


}
% vim: set tw=72 spell spelllang=da:


\chapter{Fremtidigt arbejde}
{
{\sffamily Forslag til fremtidigt arbejde. Stub.
}

\begin{itemize}
    \item Bedre udtrækning af regioner
    \item Selv implementere floodfill, så vi kan få gemt nogle flere
    dataer om ragionerne under vejs, så vi ikke behøver at lave gridt
    over billedet.
    \item Ligger der interessante regioner i ``Eye of God''?
    \item Bonus for regioner som ligger i flere gyldne snit.
    \item Unikke regioner i databasen/analysen.
    \item Gyldne snit i de enkelte regioner.
    \item ``Skæve'' snit. Diagonaler.
\end{itemize}

%\section{Titel}
%\input{afsnit/fremtidigt_arbejde/FILE.tex}
}

% vim: set tw=72 spell spelllang=da:


% sort in citation order
\bibliographystyle{unsrt}
\bibliography{litteraturliste}
\addcontentsline{toc}{chapter}{Litteratur}

\appendix
{

\chapter{Programmets struktur\label{appendix_struktur}}
{
{\sffamily Vi giver en kort præsentation af programmets overordnede
struktur. Mappestrukturen er vist i figur \ref{program_struktur}.

\begin{figure}[!h]
    \dirtree{%
    .1 src/ .
    .2 database/ .
    .2 experiments/ .
    .2 lib/ .
    .2 model/ .
    .2 painting/ .
    .2 settings/ .
    .2 tests/ .
    .2 \_\_init\_\_.py .
    .2 pictureresource.db .
    .2 start.py .
    }
    \caption[]{Programmets struktur}
    \label{program_struktur}
\end{figure}

Mappen \texttt{database/} indeholder kode til initialisering af
databasen og udtræk i sammenhæng med resultater. I mappen
\texttt{experiments/} gemmes kørsler tilknyttet de videnskabelige
eksperimenter. \texttt{lib/} indeholder alle metoder vedrørende
billedbehandling, dvs. metoder til udtrækning og bedømmelse af regioner.
I \texttt{model/} er databaseskemaet og hjælpefunktioner til at gemme
specielle datastrukturer defineret. Mapperne \texttt{painting/} og
\texttt{settings/} tilbyder datastrukturer til henholdsvis malerier og
indstillinger. Vi vil i det følgende kigge lidt nærmere på de enkelte
dele af programmet.
}

\subsection{database/}
\dirtree{%
.1 database/ .
.2 \_\_init\_\_.py .
.2 catalog.csv .
.2 parser.py .
.2 udtraek.py .
}
(wga.hu csv-fil, parsing af csv-fil, udtraek)

\subsection{experiments/}
\dirtree{%
.1 experiments/ .
.2 \dots .
}
(dikumentation af eksperimenter)

\subsection{lib/}
\dirtree{%
.1 lib/ .
.2 \_\_init\_\_.py .
.2 edgeDetector.py .
.2 expandedMethod.py .
.2 expandedMethon.py .
.2 featureDetector.py .
.2 goldenLibrary.py .
.2 graphicHelper.py .
.2 grid.py .
.2 lineScanner.py .
.2 marginCalculator.py .
.2 naiveMethod.py .
.2 regionSelector.py .
.2 paintingAnalyzer.py .
.2 statestik.py .
.2 transformations.py .
}
(goldenLibrary - hjælpefunktioner, paintingAnalyser - sammensætter de
øvrige metoder, naiveMethod, expandedMethod, regionSelector)

\subsection{model/}
\dirtree{%
.1 model/ .
.2 \_\_init\_\_.py .
}
(database in/ud)

\subsection{painting/}
\dirtree{%
.1 painting/ .
.2 \_\_init\_\_.py .
}
(container)

\subsection{settings/}
\dirtree{%
.1 settings/ .
.2 \_\_init\_\_.py .
}
(globale indstillinger, indstillinger til kørsel, kan gemmes i database)

}

% vim: set tw=72 spell spelllang=da:


\chapter{Grafer fra Web Gallery of Art\label{appendix_grafer}}
{
Vi supplerer her, de grafer der er givet af \texttt{www.wga.hu}, omhandlende
deres samling af kunstartikler. Graferne er hentet d. 04/01-2010, så de kan
afgive lidt fra den samling vi bruger i vores undersøgelse, da vi hentede alt
på deres hjemmeside medio oktober 2009. Graferne findes på
\href{http://www.wga.hu/database/statisti/index.html}{http://www.wga.hu/database/statisti/index.html}.

\section{Grafer gældende for malerier i samlingen}
\begin{figure}[H]
    \setlength\fboxsep{0pt}
    \setlength\fboxrule{0.5pt}
    \centering
    \fbox{\includegraphics[width=0.8\textwidth]{bilag/billeder/wga.hu/paint_time-frame}}
    \caption[]{Painting Time-frame}
    \label{painting_timeframe}
\end{figure}

\begin{figure}[H]
    \setlength\fboxsep{0pt}
    \setlength\fboxrule{0.5pt}
    \centering
    \fbox{\includegraphics[width=0.8\textwidth]{bilag/billeder/wga.hu/paint_school}}
    \caption[]{Painting School}
    \label{painting_school}
\end{figure}

\begin{figure}[H]
    \setlength\fboxsep{0pt}
    \setlength\fboxrule{0.5pt}
    \centering
    \fbox{\includegraphics[width=0.8\textwidth]{bilag/billeder/wga.hu/paint_style}}
    \caption[]{Painting Technique}
    \label{painting_tech}
\end{figure}

\section{Grafer gældende hele samlingen}
\begin{figure}[H]
    \setlength\fboxsep{0pt}
    \setlength\fboxrule{0.5pt}
    \centering
    \fbox{\includegraphics[width=0.8\textwidth]{bilag/billeder/wga.hu/collecti}}
    \caption[]{Collections}
    \label{collection_collection}
\end{figure}

\begin{figure}[H]
    \setlength\fboxsep{0pt}
    \setlength\fboxrule{0.5pt}
    \centering
    \fbox{\includegraphics[width=0.8\textwidth]{bilag/billeder/wga.hu/subject}}
    \caption[]{Subject}
    \label{collection_subject}
\end{figure}

\begin{figure}[H]
    \setlength\fboxsep{0pt}
    \setlength\fboxrule{0.5pt}
    \centering
    \fbox{\includegraphics[width=0.8\textwidth]{bilag/billeder/wga.hu/artist_bd}}
    \caption[]{Artists active period}
    \label{Collection_born_died}
\end{figure}

}

% vim: set tw=72 spell spelllang=da:



}

% vim: set tw=72 spell spelllang=da:


\end{document}
