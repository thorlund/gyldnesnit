{
Her gives en kort introduktion til hvordan vores program hentes og køres
på en maskine med et Debian-system. Andre systemer vil ikke blive
dækket. Initialisering og kørsel af programmet kræver
programmeringserfaring i Python.

\section{Nødvendige pakker}
De følgende pakker bør være installeret, for at køre programmet:

\begin{itemize}
    \item \texttt{python2.6}
    \item \texttt{python-opencv}
    \item \texttt{opencv-doc}
    \item \texttt{python-numpy}
    \item \texttt{python-sqlobject}
    \item \texttt{sqlite3}
\end{itemize}

Ønsker man at oversætte rapporten, anbefales det at have
\texttt{texlive-full} installeret.

\section{Anskaffelse af programmet}
Vores implementation ligger i et git-repository på GitHub og kan tilgås
ved \url{http://github.com/thorlund/gyldnesnit}. Her kan rapporten også
findes.

Programmet kan anskaffes på to måder: Via \textbf{git} eller ved at
hente et \texttt{tgz} eller \texttt{zip}-arkiv fra
\url{http://github.com/thorlund/gyldnesnit/downloads}. Det anbefales at
hente programmet via \textbf{git}, da arkiverne kan være uddaterede.

Hjælp, til at pakke arkiverne ud, findes ved kommandoerne
\begin{verbatim}
    # > man tar
    # > man unzip
\end{verbatim}

For at hente programmet via \textbf{git}, skal man have
pakken \texttt{git} installeret. Kildekoden hentes ved kommandoen:
\begin{verbatim}
    # > git clone git://github.com/thorlund/gyldnesnit.git
\end{verbatim}
eller, hvis ovenstående fejler:
\begin{verbatim}
    # > git clone http://github.com/thorlund/gyldnesnit.git
\end{verbatim}
Dette opretter mappen \texttt{gyldnesnit/} i den aktuelle mappe, hvor
kildekoden ligger.

\subsection{Initialisering af billeddatabasen}
Lav om i \texttt{start.py} eller sådan noget.

\subsection{Start af en kørsel}
\subsubsection{Oprettelse af \texttt{environment}}

\subsubsection{Start}

\subsection{Udtrækning af resultater}
Udtrækning af resultater kræver erfaring med \emph{SQLObject}, SQL
og/eller Sqlite.

}

% vim: set tw=72 spell spelllang=da:
