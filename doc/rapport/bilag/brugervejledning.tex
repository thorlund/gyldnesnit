{
Her gives en kort introduktion til hvordan vores program hentes og køres
på en maskine med et Debian-system. Andre systemer vil ikke blive
dækket. Initialisering og kørsel af programmet kræver
programmeringserfaring i Python.

\section{Nødvendige pakker}
De følgende pakker bør være installeret, for at køre programmet:

\begin{itemize}
    \item \texttt{python2.6}
    \item \texttt{python-opencv}
    \item \texttt{opencv-doc}
    \item \texttt{python-numpy}
    \item \texttt{python-sqlobject}
    \item \texttt{sqlite3}
\end{itemize}

Ønsker man at oversætte rapporten, anbefales det at have
\texttt{texlive-full} installeret.

\section{Anskaffelse af programmet}
Vores implementation ligger i et git-repository på GitHub og kan tilgås
ved \url{http://github.com/thorlund/gyldnesnit}. Her kan rapporten også
findes.

Programmet kan anskaffes på to måder: Via \textbf{git} eller ved at
hente et \texttt{tgz} eller \texttt{zip}-arkiv fra
\url{http://github.com/thorlund/gyldnesnit/downloads}. Det anbefales at
hente programmet via \textbf{git}, da arkiverne kan være uddaterede.

Hjælp, til at pakke arkiverne ud, findes ved kommandoerne
\begin{verbatim}
    # > man tar
    # > man unzip
\end{verbatim}

For at hente programmet via \textbf{git}, skal man have
pakken \texttt{git} installeret. Kildekoden hentes ved kommandoen:
\begin{verbatim}
    # > git clone git://github.com/thorlund/gyldnesnit.git
\end{verbatim}
eller, hvis ovenstående fejler:
\begin{verbatim}
    # > git clone http://github.com/thorlund/gyldnesnit.git
\end{verbatim}
Dette opretter mappen \texttt{gyldnesnit/} i den aktuelle mappe, hvor
kildekoden ligger.

\subsection{Indstillinger for en kørsel}
Programmet startes ved at køre \texttt{start.py} fra mappen
\texttt{src/} med kommandoen
\begin{verbatim}
    # > ./start.py
\end{verbatim}.
Programmet vil initialisere hele billeddatabasen første gang det
startes. Programmet vil derfor også sætte crawleren til WGA igang med at
hente billeder. Hvis man ikke er interesseret i at hente alle $23,000$
billeder, kan man oprette en mindre testdatabase, hvor der kun vil blive
hentet et begrænset antal billeder. Dette kræver at man sætter de to
følgende variable i \texttt{start.py}.
\begin{verbatim}
    #Set this to true if you want to create a personal database
    testdatabase = True
    #Set this due to how large you want your testdatabase
    count = 10
\end{verbatim}
\label{brugervejl_test_db}
Ovenstående vil begrænse crawleren til at hente 10 billeder fra WGA.
Hele databasen initialiseres stadigvæk med informationerne givet fra
WGA.

Programmet vil efter endt initialisering begynde analysen på databasen.
Analysemetoden som programmet skal bruge sættes i \texttt{start.py} ved
kaldet
\begin{verbatim}
    settings.setMethod('expanded')
\end{verbatim}
hvor \texttt{naive} og \texttt{expanded} er til rådighed. Ved første
kørsel anbefales det at sætte
\begin{verbatim}
    settings.setMethod('naive')
\end{verbatim}.

Selve eksperimentet som skal køres vælges fra mappen
\texttt{experiments/}. Vi har eksperimenterne \texttt{golden} og
\texttt{tenCuts} til rådighed.  \texttt{golden} er meget simpel og
analyserer kun det gyldne snit.  \texttt{tenCuts} er langt mere
avanceret og tager lang tid at køre på hvert billede, da denne analyse
dækker stort set hele billedet. Man kan også lave sine egne
eksperimenter, ved at se på strukturen af de eksisterende. Eksperimentet
sættet ved at importere filen som variablen \texttt{environment} i
\texttt{start.py}. Som standard importeres \texttt{tenCuts}.
\begin{verbatim}
    from experiments import tenCuts as environment
\end{verbatim}
Det anbefales at ændre dette til \texttt{golden}.

\subsection{Udtrækning af resultater}
Udtrækning af resultater kræver erfaring med \emph{SQLObject}, SQL
og/eller Sqlite. Se evt. filen \texttt{src/database/udtraek.py} for
at se hvordan man bruger \emph{SQLObject} til databaseudtræk. Man også
blot køre \texttt{udtraek.py} som giver en del information om
resultaterne.

}

% vim: set tw=72 spell spelllang=da:
