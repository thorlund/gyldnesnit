{
Flere kilder påstår at det gyldne snit bruges i malerier, arkitektur og
musik, da dette forhold har specielt tiltalende æstetiske
egenskaber\cite{GoldenNumber}\cite{RatioArt}\cite{Putz1995}\cite{Stakhov2006490}\cite{Boussora2004}.
Især \cite{GoldenNumber} er særlig ivrig og finder det gyldne snit i alt
lige fra cigaretpakker til skallen fra en nautil. Sidstnævnte bliver
meget ofte brugt som argument for at det gyldne snit findes i naturen i
form af en gylden spiral. Hvis man rent faktisk måler efter viser det
sig at dette ikke er tilfældet. Som argumenteret i \cite{Sharp2002} er
spiralen i nautilskallen rigtigt nok logaritmisk, men ikke med en faktor
$\varPhi$.

Det græske tempel Parthenon spiller også en central rolle i rygterne om
det gyldne snit. Billeder af templet bliver tit vist med et rektangel
tegnet over. Der florerer forskellige udgaver af disse billeder, hvor
der åbenbart ikke tages højde for at dele af templet ikke indgår i
rektanglet. George Markowsky gør i \cite{Markowsky1992} op med mange af
disse vrangforestillinger. Påstande om at Keopspyramiden skulle være
bygget efter det gyldne snit, som angivet i \cite{Stakhov2006490}, at
nogle af Leonardo da Vincis malerier er malet efter det gyldne snit og
at det gyldne rektangel er det mest æstetisk tiltrækkende
format\cite{GoldenNumber}\cite{RatioArt} bliver i hans artikel afvist,
da mange af disse undersøgelser lider under det han kalder \emph{the
Pyramidology Fallacy}. Dette udtrykt henter Markowsky fra
\cite{Gardner1952_2} og beskriver de personer der forsker i
pseudovidenskab såsom HCI\footnote{Right on!!1!  (Undskyld, vi skal nok
fjerne det)} eller pyramider. Her har forskerne ofte mange forskellige
tal at ``jonglere'' med og kan frit vælge netop dem der giver det
ønskede resultat.

Hypotesen om det gyldne rektangels æstetiske egenskaber bliver også
taget op i \cite{Boselie1984} og \cite{Plug1980}, hvor der ikke kan
konkluderes at det gyldne rektangel skulle have nogen æstetisk
signifikans. En lignende hypotese bliver sat på prøve i
\cite{McManus1995}, hvor det undersøges om et billedes geometriske
komposition har indflydelse på dets æstetiske effekt. I. C. McManus
kunne gennem tre eksperimenter ikke finde noget grundlag for at et
billedes geometriske komposition påvirkede testpersonernes bedømmelse.
Det siges i konklusionen:

\quote{Together the results og these experiments throw considerable
doubt upon the hypothesis of the implicit detection of latent
compositional geometry as a major component of aesthetic judgements, at
least for relatively unsophisticated observers[\dots]}

Selvom ovenstående ikke eksplicit nævner det gyldne snit er der
alligevel en klar relevans til billeder opbygget geometrisk efter det
gyldne snit.
}
% vim: set tw=72 spell spelllang=da:
