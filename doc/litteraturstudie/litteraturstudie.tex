% vim: set tw=72 spell spelllang=da:
\documentclass[a4paper, 10pt, danish, final]{article}
%%%%%%%%%%%%%%%%%%%%%%%%%%%%%%%%%%%%%%%%%%%%%%%%%%%%%%%%%%%%%%%%%
% Pakker
%%%%%%%%%%%%%%%%%%%%%%%%%%%%%%%%%%%%%%%%%%%%%%%%%%%%%%%%%%%%%%%%%
%\usepackage{a4wide}
\usepackage[danish]{babel}
\usepackage[utf8]{inputenc}
\usepackage[T1]{fontenc}

\usepackage{charter}
\usepackage{verbatim}
\usepackage{amsfonts}
\usepackage{amsmath}
\usepackage{amssymb}
%\usepackage{mathrsfs}
%\usepackage[mathcal]{euscript}
\usepackage{listings}
\usepackage{graphicx}
\usepackage{multirow}
\usepackage{hyperref}
\usepackage{cite}

%%%%%%%%%%%%%%%%%%%%%%%%%%%%%%%%%%%%%%%%%%%%%%%%%%%%%%%%%%%%%%%%
% Indstillinger
%%%%%%%%%%%%%%%%%%%%%%%%%%%%%%%%%%%%%%%%%%%%%%%%%%%%%%%%%%%%%%%%
\parindent=5pt
\parskip=8pt plus 2pt minus 4pt
\lstset{language=Python, basicstyle=\scriptsize, showstringspaces=false,
numbers=none, stepnumber=1, numberstyle=\tiny}

%%%%%%%%%%%%%%%%%%%%%%%%%%%%%%%%%%%%%%%%%%%%%%%%%%%%%%%%%%%%%%%%
% Kommandoer
%%%%%%%%%%%%%%%%%%%%%%%%%%%%%%%%%%%%%%%%%%%%%%%%%%%%%%%%%%%%%%%%
%\newcommand{\old}[1]{\oldstylenums{#1}}
%\newcommand{\old}[1]{{#1}}
\newcommand{\mailto}[1]{\href{mailto:#1}{#1}}
\newcommand{\link}[1]{\href{#1}{\texttt{#1}}}

%%%%%%%%%%%%%%%%%%%%%%%%%%%%%%%%%%%%%%%%%%%%%%%%%%%%%%%%%%%%%%%
% Titel, forfatter og dato
%%%%%%%%%%%%%%%%%%%%%%%%%%%%%%%%%%%%%%%%%%%%%%%%%%%%%%%%%%%%%%%
\title{Litteraturstudie\\{\Large \emph{Detektion af ``Det Gyldne Snit'' i digitaliserede malerier}}}

\author{Ulrik Bonde - \mailto{bonde@diku.dk}\\
Kasper Steenstrup - \mailto{khsj@diku.dk}\\
Morten Thorlund - \mailto{thorlund@diku.dk}}
\date{\today}

\hypersetup{
colorlinks,%
citecolor=black,%
filecolor=black,%
linkcolor=black,%
urlcolor=black, % 
bookmarksopen=false,
pdftitle={Litteraturstudie - Detektion af Det Gyldne Snit i digitaliserede malerier},
pdfauthor={Ulrik Bonde, Kasper Steenstrup og Morten Thorlund}
}

%%%%%%%%%%%%%%%%%%%%%%%%%%%%%%%%%%%%%%%%%%%%%%%%%%%%%%%%%%%%%%
% Indhold
%%%%%%%%%%%%%%%%%%%%%%%%%%%%%%%%%%%%%%%%%%%%%%%%%%%%%%%%%%%%%%
\begin{document}
\maketitle
\thispagestyle{empty}
%\pagestyle{headings}

{
{\sffamily Dette dokument er den endelige rapport udarbejdet i forbindelse med
kurset ``Bachelorprojekt'' som udbydes på Datalogisk Institut ved
Københavns Universitet. Det forventes at læseren har en basal viden
inden for datalogi og kendskab til begreber der bruges i forbindelse med
billedbehandling.
}
}

%\subsection{Forventninger til læser} Dette opgave omhandler metoder og
%udledninger af akademiske problemer som opstår ved programmering inde for
%feltet billedbehandling, Derfor regner vi med at læseren af denne rapport har
%en basal vide inde for datalogi i retninger af billeders opbygning, det vil
%siger, viden om hvad en pixel er, hvad betjener RGB favre, osv. Dog ligger der
%vægt på at forklaring af vores problemer, Det vi er kommet frem til og hvad man
%kan bruge vores fund til. Ligger på en nivo som en kunst studerende kan
%relatere sig til og forstå.  Vi har tænkt på at lave en lille under afsnit til
%vær del i vores opgave, som forklare det vi laver med meget udpenslet sprog.

% vim: set tw=72 spell spelllang=da:


\section*{Litteratur omhandlende ``Det Gyldne Snit''}
{
Det bliver flere steder påstået, at det gyldne snit bruges i malerier,
arkitektur og musik, da dette forhold har specielt tiltalende æstetiske
egenskaber\cite{GoldenNumber}\cite{RatioArt}\cite{Putz1995}\cite{Stakhov2006490}\cite{Boussora2004}.
Især \cite{GoldenNumber} er særlig ivrig og finder det gyldne snit i alt
lige fra cigaretpakker til skallen fra en nautil. Sidstnævnte bliver
meget ofte brugt som argument for at det gyldne snit findes i naturen i
form af en gylden spiral. Dog bliver det klart at dette ikke er
tilfældet hvis man rent faktisk måler efter, som vist i
\cite{Sharp2002}. Også det græske tempel Parthenon spiller en central
rolle i rygterne om det gyldne snit. Billeder af templet bliver tit vist
med et rektangel tegnet over. Der florerer forskellige udgaver af disse
billeder, hvor der åbenbart ikke tages højde for at dele af templet ikke
indgår i rektanglet. George Markowsky gør i \cite{Markowsky1992} op med
mange af disse vrangforestillinger. Påstande om at Keopspyramiden skulle
være bygget efter det gyldne snit, som angivet i \cite{Stakhov2006490},
at nogle af Leonardo da Vincis malerier er malet efter det gyldne
snit og at det gyldne rektangel er det mest æstetisk tiltrækkende
format\cite{GoldenNumber}\cite{RatioArt} bliver i hans artikel afvist,
da mange af disse undersøgelser lider under det han kalder
``Pyramidology''\footnote{Martin Gardner, ``Fads and Fallacies in the
Name of Science'': En betegnelse for pseudovidenskab, specielt
med henblik på pyramider.}. Hypotesen om det gyldne rektangels æstetiske
egenskaber bliver også taget op i \cite{Boselie1984} og \cite{Plug1980},
hvor der ikke kan konkluderes at det gyldne rektangel skulle have nogen
æstetisk signifikans. En lignende hypotese bliver sat på prøve i
\cite{McManus1995}, hvor det undersøges om et billedes geometriske
komposition har indflydelse på dets æstetiske effekt. I. C. McManus
kunne gennem tre eksperimenter ikke finde noget grundlag for at et billedes
geometriske komposition påvirkede testpersonernes bedømmelse. Det siges
i konklusionen:

\quote{Together the results og these experiments throw considerable
doubt upon the hypothesis of the implicit detection of latent
compositional geometry as a major component of aesthetic judgements, at
least for relatively unsophisticated observers[\dots]}

Selvom ovenstående ikke eksplicit nævner det gyldne snit er der
alligevel en klar relevans til billeder opbygget geometrisk efter det
gyldne snit.
}
% vim: set tw=72 spell spelllang=da:


\section*{Litteratur vedrørende computer vision}
% vim: set tw=72 spell spelllang=da:



%\tableofcontents
%\newpage

\bibliographystyle{plain}
\bibliography{litteraturliste}

%\newpage
%\section*{Bilag} 
%\appendix

%\section{Source code}
%\input{materialer.tex}
%\newpage

\end{document}
