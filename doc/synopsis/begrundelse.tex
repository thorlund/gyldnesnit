Det gyldne snit\footnote{Bliver også omtalt som den gyldne ratio eller det
guddommelige forhold} er et interessant fænomen der kan opleves i naturen og
matematikken. Det gyldne snit er defineret ved tallet
\begin{eqnarray*}
	\varphi	& =	& \frac{\sqrt{5} + 1}{2}\\
		& = 	& 1.6180\ 3398\ 8749\ 8948\ 4820 \dots
\end{eqnarray*}
og bemærker sig blandt andet ved at når man kvadrerer $\varphi$ lægger man
blot 1 til værdien\cite{Bib:MWGold}\cite{Bib:Knott}. Vi kan nu definere
\begin{eqnarray*}
	\phi	& = 		& \frac{1}{\varphi}\\
		& = 		& \varphi - 1\\
		& \thickapprox	& 0.6180339887
\end{eqnarray*}
hvilket er det tal vi vil tage udgangspunkt i. Det begrunder sig i at tallet
$\phi$ bedre lader sig forklare mht. opdelingen af et maleri.

Det gyldne snit fremviser en interessant forbindelse til
Fibonaccis talrække, da forholdet mellem to fibonaccital $F(n)$ og $F(n - 1)$
konvergerer mod $\varphi$ når $n$ nærmer sig uendelig. Mere formelt har vi at
\begin{eqnarray*}
	\varphi	& =	& \lim_{n \rightarrow \infty}{\frac{F(n)}{F(n - 1)}}
\end{eqnarray*}

Fibonaccis talrække viser sig også i naturen ved flere lejligheder, den mest
kendte værende placeringen af kernerne i en
solsikkeblomst\cite{Bib:Dou}. Dette
giver anledning til påstanden om at det gyldne snit er at finde i naturen, ikke
blot i solsikker, men i mange andre biologiske strukturer.

Det gyldne snit er endvidere blevet brugt som et æstetisk virkemiddel i især
grafisk kunst. Den faktiske brug af det gyldne snit i kunsten er dog kun vagt
dokumenteret og lider under en meget svag definition\cite{Bib:Mark}.
Det kan dog tillades at sige, at kompositionen af mange
billeder$^{\textsf{[citation needed]}}$ følger en opdeling der tager
udgangspunkt i det gyldne snit, men der er ikke blevet lavet en præcis analyse
af disse billeder der kunne bevise en egentlig sammenhæng.

Det bemærkes at det gyldne snit, hvor vi nu kigger på tallet $\phi$, ligger
meget tæt op ad $\frac{2}{3}$.  Det er derfor nærliggende at forestille sig, at
inddelingen i kunsten er tættere på $\frac{2}{3}$ end $\phi$, da førstnævnte
er meget lettere at måle op.

Det er derfor interessant at kunne afgøre om der i kunsten virkelig er blevet
tænkt over at komponere malerier efter det gyldne snit. Endvidere er der ikke
før blevet udført en lignende analyse af malerier. En undersøgelse fra 1999 har
lavet statistik på hele 565 malerier, hvor forholdet mellem lærredets
dimensioner er blevet registreret\cite{Bib:Painting}. Undersøgelsen kunne ikke
konkludere at kunstnere foretrækker det gyldne snit i lærredet.

Vi vil gerne kunne analysere titusinder malerier for andet end blot lærredets
dimensioner.
