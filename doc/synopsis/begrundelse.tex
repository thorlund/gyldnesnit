Det Gyldne Snit et et usædvanligt fænomen der kan opleves i naturen og
matematiken[1]. Det Gyldne Snit er endvidere blevet brugt som et æstetisk
virkemiddel i især grafisk kunst. Den faktiske brug af Det Gyldne Snit i
kunsten er dog kun vagt dokumenteret og lider under en meget svag definition
(hvis nogen overhovedet). Kompositionen af mange billeder[2] følger en
opdeling der tager udgangspunkt i Det Gyldne Snit, men der er ikke blevet
lavet en præcis analyse af disse billeder der kunne bevise en egentlig
sammenhæng. Den gyldne ratio[3], eller $\Phi$, ligger meget tæt op ad 2/3. Det
er derfor nærliggende at forestille sig, at inddellingen i kunsten er tættere
på 2/3 end $\Phi$.

[1]: Find eksempler (gode). Fibonacci.
[2]: Citation needed
[3]: ratio/snit, kartoffel/vindmølle, bestem Jer

