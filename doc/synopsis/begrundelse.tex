Det gyldne snit et et usædvanligt\footnote{Nok nærmere super-cook eller
interessant} fænomen der kan opleves i naturen og
matematiken\footnote{Find eksempler (gode). Fibonacci.}. Det gyldne snit
er endvidere blevet brugt som et æstetisk virkemiddel i især grafisk
kunst. Den faktiske brug af det gyldne snit i kunsten er dog kun vagt
dokumenteret og lider under en meget svag definition (hvis nogen
overhovedet). Kompositionen af mange billeder$^{\textsf{[citation
needed]}}$ følger en opdeling der tager udgangspunkt i det gyldne snit,
men der er ikke blevet lavet en præcis analyse af disse billeder der
kunne bevise en egentlig sammenhæng. Den gyldne
ratio\footnote{ratio/snit, kartoffel/vindmølle, bestem jer}, eller
$\Phi$, ligger meget tæt op ad 2/3. Det er derfor nærliggende at
forestille sig, at inddellingen i kunsten er tættere på 2/3 end $\Phi$.

Der er ikke blevet brugt computere til at finde det gyldne snit i
malerier før, selvom det gyldne snit er blevet grund til bla.
ansigtsgenkendelse.
