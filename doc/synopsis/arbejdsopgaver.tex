\newcommand{\opgave}[5]{
\makebox[\textwidth]{#1}\par
%\makebox[\textwidth]{

%\framebox[0.8\width][r]{Bummer,
%I am too wide} \par
%\framebox[1cm][l]{never
%mind, so a
%\emph{#1}
\begin{itemize}
\item{Produkt: #2}
\item{Ressourcekrav:#3}
\item{Afhængigheder:#4}
\item{Belastning: #5}\\
\end{itemize}
}

\opgave{
		Naiv implementation
	}{
		Simpel løsning til problemstillingen i synopsen.
	}{
		Testbilleder
	}{
		Definition af en interessant region\\
		Udtænke testbilleder\\
		Sammenføj kant- og blobdetektion\\
		Konfigurere en database med henblik på statistik
	}{
		80 timer
	 }

\opgave{
		Færdig implementation
	}{
		Færdig løsning til problemstillingen i synopsen.
	}{
		Testbilleder
	}{
		Naiv implementation\\
		Alternativt snit\\
		Kørselskoordinering\\
		Evaluering af statistik\\
		Indkransning af regioner\\
		Fibonacci-spiral\\
		Mængder af regioner\\
		$\frac{2}{3}$-snit mod det gyldne snit\\
		Definere kategorier af interessante regioner
	}{
		100 timer
	 }

\opgave{
		Definition af en interessant region
	}{
		Finde frem til en definition af interessante regioner, der kan tilfredsstille evt. skeptiske holdninger omkring projektet.
	}{
		Kunstakademikere
	}{
		Ingen
	}{
		50 timer
	 }

\opgave{
		Afstand fra snit til interessant region
	}{
		Givet en interessant region, skal der udvikles en algoritme, der finder afstanden fra regionens center og den kant, der er tættest på det gyldne snit.
	}{
		Ingen
	}{
		Definition af en interessant region\\
		Sammenføj kant- og blobdetektion\\
		Interessant-region-detektor
	}{
		35 timer
	}

\opgave{
		Udtænke testbilleder
	}{
		Testbilleder, der skal bruges som rettesnor for alle algoritmer der arbejder med de interessante regioner.
	}{
		Ingen
	}{
		Ingen
	}{
		50 timer
	}

\opgave{
		Sammenføj kant- og blobdetektion
	}{
		Implementere en algoritme som finder interessante regioner i et billede ved hjælp af kant- og blobdetektion.
	}{
		Ingen
	}{
		Ingen
	}{
		100 timer
	}

\opgave{
		Færdiggøre rapport
	}{
		Færdig rapport
	}{
		Ingen
	}{
		Færdig implementation
	}{
		150 timer
}


\opgave{
		Konfigurere en database med henblik på statistik
	}{
		En fungerende database, komplet med billeder og informationer omkring billederne.
	}{
		Opbevaringsplads: \texttt{/vol/projects/disk10/amuse/golddetect}
		Server
	}{
		Ingen
	}{
		80 timer
	}
	
\opgave{
		Alternativt snit
	}{
		Statistisk data om gennemkørsel af et andet snit.
	}{
		Ingen
	}{
		Naiv implementation

	}{
		20 timer
	}

\opgave{
		Kørselskoordinering
	}{
		Opdatering af statistikkoden til pågældende kørsel.
		
	}{
		Ingen
	}{
		Naiv implementation
	}{
		20 timer
	}
	
\opgave{
		Evaluering af statistik af behandlede billeder efter hver iteration
	}{
		Rapportbeskrivelse om interessante statistiske opdagelser efter gennemkørsel af nye programmer.
	}{
		Ingen
	}{
		Konfigurere en database med henblik på statistik\\
		Naiv implementation
	}{
		100 timer
	}


\opgave{
		Interessant-region-detektor
	}{
		Algoritme hvis formål er at lokalisere regioner, der befinder sig i et interval rundt om det gyldne snit.
	}{
		Ingen
	}{
		Definition af en interessant region
	}{
		60 timer
	}
	
\subsection*{Udvidelser}

\opgave{
		Indkransning af regioner
	}{
		Algoritme som indkranser interessante regioner i firkanter, så
		der kan laves ``gyldne snits''-udregninger på regionerne. I
		dette tilfælde behandles interessante regioner som f.eks.
		ansigter, mennesker og større bygningsværker.
	}{
		Ingen
	}{
		Naiv implementation
	}{
		50 timer
	}

\opgave{
		Fibonacci-spiral
	}{
		Program som indtegner Fibonacci-spiralen. 
	}{
		Ingen
	}{
		Naiv implementation
	}{
		50 timer
	}

\opgave{
		Mængder af regioner
	}{
		Funktionalitet som tæller antal regioner i billedet.
	}{
		Ingen
	}{
		Naiv implementation
	}{
		20 timer
	}

\opgave{
		$\frac{2}{3}$ mod det gyldne snit
	}{
		Udregning af $\frac{2}{3}$ i stedet for det gyldne snit. Dette
		ligger meget op af opgaven ``Alternativt snit'', men
		$\frac{2}{3}$ ligger så tæt på det gyldne snit, at der skal
		tages en del forholdsregler.
	}{
		Ingen
	}{
		Naiv implementation
	}{
		70 timer
	}

\opgave{
		Definere kategorier af interessante regioner
	}{
		Implementere en algoritme, der kan kategorisere regioner efter hvor de ligger i forhold til det gyldne snit.
	}{
		Ingen
	}{
		Naiv implementation
	}{
		150 timer
	}
