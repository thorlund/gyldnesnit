Ved brug af metoder fra computer vision, vil vi undersøge et stort antal
digitale malerier for opfyldelse af det gyldne snit. Kan vi, givet en
høj-kvalitets digital gengivelse af et maleri, udvikle et stykke software som
kan afgøre om et maleri opfylder det gyldne snit(, ved at bruge kant- og
blobdetektion)\footnote{Skal det i parentes med eller begrænser det for
meget?}?

Vi siger, at et maleri opfylder det gyldne snit, hvis en interessant region
tangerer\footnote{Vi leder egentlig ikke efter en reel tangent, men noget der
ligner. Der vil blive produceret illustrationer til at vise hvad vi leder
efter.} en linie som opdeler billedet efter det gyldne snit. Vi må som
udgangspunkt antage, at vi ikke kan måle helt præcist, da det gyldne snit
defineres ved det irrationelle tal $\varphi$. Vi må da indregne en lille margen
når vi leder efter træk omkring et gyldent snit.  Derfor ønsker vi at inddele
træk i billedet, der kunne være kandidater til opfyldelse af det gyldne snit, i
kategorier, alt efter hvor langt væk fra snittet dette træk ligger. Vi kan da
komme frem til en vægtning af \emph{hvor meget} et billede opfylder det gyldne
snit.

Billederne hentes fra et online-repository, hvor vi har adgang til metadata om
malerierne. Et interessant spørgsmål er da, givet en intern klassificerering af
et billedes regioner, om man kan finde sammenhænge mellem et maleris
komposition og eksterne faktorer, såsom nationalitet, periode eller kunstner?
Til dette formål skal vi lagre resultater fra billedeanalysen i en database,
hvilket gør videre bearbejdning af metadata ligetil.
