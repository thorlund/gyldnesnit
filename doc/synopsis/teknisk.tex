Vi har valgt at udvikle vores implementation i Python. Valget begrunder
sig hovedsageligt i at Python er et godt værktøj til at lave prototyper
i. Vi undgår herved at bruge alt for meget tid på at definere diverse
datastrukturer, fremfor at koncentrere os om kernen i projektet. Det der
kan argumentere imod valget af Python er at det kan være langsomt. Vores
vurdering er dog at dette ikke vil blive et problem, samt at vi i første
omgang ikke går efter at lave en hurtigt implementation, blot en
implementation der virker.

Til at udføre billedmanipulation benytter vi os af et bibliotek skrevet i C der
hedder OpenCV. Biblioteket er udviklet af Intel og tilbyder, udover et solidt
udvalg af algoritmer, bindinger til Python. Endelig er det meget
veldokumenteret og giver referencer til publikationer om bibliotektets
algoritmer. Biblioteket er udviklet med specielt henblik på real-tids
behandling af billeder, f.eks. med et videokamera som kilde, men egner sig også
til brug på enkelte billeder.

Vi ønsker at opbevare resultater fra kørsler i en database. Et endeligt
valg er ikke taget endnu, men vi vælger en let tilgængelig Open
Source-applikation, som let kan integreres med Pyhton, såsom MySQL eller
PostgreSQL.

Vores kode opbevares hos en gratis tjeneste i et git-repository. Koden
vil da til enhver tid kunne ses og hentes fra
\href{http://github.com/thorlund/gyldnesnit}{http://github.com/thorlund/gyldnesnit}.
Synopsis og rapport skrives i \LaTeX{} og kildekoden hertil vil også være
at finde i vores git-repository.
